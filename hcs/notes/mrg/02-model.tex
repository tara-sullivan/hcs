% begin with an Alon-Fershtman slide; review as quicly as possible ``because many people here are familar with this paper, I'll try to keep this short'' Maybe check with david that this is a good idea
% Then jump into my key contribution: how do we think about the priors

%%%%%%%%%%%%%%%%%%%%%%%%%%%%%%%%%%%%%%%%%%%%%%%%%%%%%%%%%%%%%%%%%%%%%%%%%%%%%%%%
%%%%%%%%%%%%%%%%%%%%%%%%%%%%%%%%%%%%%%%%%%%%%%%%%%%%%%%%%%%%%%%%%%%%%%%%%%%%%%%%
\begin{frame}{Model preliminaries}

% Discrete time
% Infinitely lived agents

% unknown success probability in j

% write that with an indicator function

% 

Individuals endowed with:
\begin{itemize}
    \item [$h_{j0}$:] Skill-$j$ specific human capital ($j=0,\dots,J$)
    \item [$\theta_j$:] Unknown probability of success in $j$
    \item [$P_{j0}$:] Prior beliefs about $\theta_j$
\end{itemize}

\vspace{2ex}
At each $t$, individuals can choose to either study or work in one skill-$j$:
\begin{itemize}
    \item Studying accumulates skill-$j$ human capital and reveals information about underlying probability of success in $j$
    % Endogenous enter labor market at time t as a skill j specialist to maximize 
    \item If you work, you receive wage $w_j$
% To keep things simple, I'll assume utility is linear, and that the value of entering the market in period t as a skill k specialist simply depends on lifetime earnings
\end{itemize}
% Time constraint:
% \begin{equation*}
%     \sum_{j=0}^J (s_{jt} + \ell_{jt}) = 1, \quad \quad s_{jt}, \ell_{jt} \in \{0, 1\}
% \end{equation*}

\vspace{2ex}
Enter labor market at time $t$ in skill-$j$ to maximize expected lifetime payoff:
\begin{equation*}
    \frac{\delta^t}{1 - \delta} U_j (w_{j}, h_{jt}) \ell_{jt}
    = \frac{\delta^t}{1 - \delta} w_{j} h_{jt} \ell_{jt}
    % Is it okay to say this? 
\end{equation*}
% \vspace{2ex}


\end{frame}

%%%%%%%%%%%%%%%%%%%%%%%%%%%%%%%%%%%%%%%%%%%%%%%%%%%%%%%%%%%%%%%%%%%%%%%%%%%%%%%%
% \begin{frame}{Student specialization decision}
\begin{frame}{Evolution of human capital accumulation and beliefs}

% Choose probability 

Students studying skill-$j$ at time $t$ pass the course with probability $\theta_j$:
\begin{equation*}
    a_{jt} \sim \text{Bernoulli} (\theta_j)
\end{equation*}
\vspace{-2.5ex}
\begin{itemize}
  \item 
  Accumulate human capital if they pass the course:
  \begin{align*}
  % h_{jt+1} =& H(h_{kt}, a_{kt}), \quad \quad
%   % a_{jt} \sim F_{\theta_j}
    h_{jt+1} = h_{jt} + \nu_{j} a_{jt}
  \end{align*}
  \item 
  Beliefs about $\theta_j$ evolve:
  \begin{equation*}
      \mgreen<2->{P_{j,t+1}} = \Pi_j (\mblue<2->{P_{jt}},a_{jt})
        % P_{j, t+1} 
  \end{equation*}
\end{itemize}

\pause
\vspace{5ex}
\textbf{Key:} How are  \blue<2->{priors} formed, and how are they \green{updated}? 

% Parameter we are trying to understand is probability of success
% Know that we want a prior distribution supported on [0, 1]

\end{frame}

%%%%%%%%%%%%%%%%%%%%%%%%%%%%%%%%%%%%%%%%%%%%%%%%%%%%%%%%%%%%%%%%%%%%%%%%%%%%%%%%
\begin{frame}[t]{Belief distribution}

Initial prior drawn from Beta distribution 
\begin{equation*}
    \mblue{P_{j0}} = \mathcal{B} (\alpha_{j0}, \beta_{j0})
\end{equation*}
% Beta distribution is appropriate 
% need a probability distribution supported on [0,1]
% Also has some desireable analytic properties
Update according to Bayes Rule $\implies$ posterior drawn from Beta distribution:
\begin{equation*}
    \mgreen{P_{j,t+1}} = \mathcal{B} (\alpha_{j,t+1}, \beta_{j,t+1}), \quad \quad 
    (\alpha_{j,t+1}, \beta_{j,t+1}) = 
    \begin{cases} 
        (\alpha_{jt} + 1, \beta_{jt}) &\text{ if } a_{jt} = 1 \\
        (\alpha_{jt}, \beta_{jt} + 1) &\text{ if } a_{jt} = 0
    \end{cases}
\end{equation*}

\vspace{3ex}
\begin{columns}[T] % align columns
\begin{column}{.51\textwidth}

\pause
\vspace{3ex}
Example: $\alpha_0 = 1$, $\beta_0 = 1$
\vspace{1.5ex}
\begin{itemize}
    \item <3-> success at $t=1$ $\implies$ $\alpha_1 = 2$, $\beta_1 = 1$

    \vspace{1.5ex}
    \item <4-> failure at $t=2$ $\implies$ $\alpha_1 = 2$, $\beta_1 = 2$
    
    \vspace{1.5ex}
    \item <5-> success at $t=3$ $\implies$ $\alpha_1 = 3$, $\beta_1 = 2$
\end{itemize}

\end{column}%
% \hfill%
\begin{column}{.39\textwidth}
\begin{figure}
\only<2>{\input{beta_example0.tex}}
\only<3>{\input{beta_example1.tex}}
\only<4>{\input{beta_example2.tex}}
\only<5>{\input{beta_example3.tex}}
\end{figure}
\end{column}%
\end{columns}
% \hypertarget<2>{beta_11_example}{\beamerbutton{I'm on the fourth slide}}
\hypertarget<2>{model_beta_11}{
  \hyperlink{simulate}{\beamerbutton{Return: simulation set-up}}
  \hyperlink{sim_default}{\beamerbutton{Return: baseline simulation}}
}
\hypertarget<4>{model_beta_22}{\hyperlink{sim_beliefs}{\beamerbutton{Return: simulation}}}
% \hyperlink{belief_effect}{\beamerbutton{Return: simulation}}


\end{frame}




%%%%%%%%%%%%%%%%%%%%%%%%%%%%%%%%%%%%%%%%%%%%%%%%%%%%%%%%%%%%%%%%%%%%%%%%%%%%%%%%
\begin{frame}{Group-based parametrization}



Consider group-based beliefs about abilities:
\begin{itemize}
    \item Each individual has a group type: $g \in \{m, f\}$

    \item Students form beliefs, $P_{j0}$, based on previously observed group successes
\end{itemize}

\vspace{2ex}
Simple parameterization:
\begin{itemize}
    \item [$\alpha_{j0}^g$: ] Number of type-$g$ students who have succeeded in $j$ 
    \item [$\beta_{j0}^g$: ] Number of type-$g$ students who have failed in $j$

    \item [$\implies$] Observed success rate:
    \begin{equation*}
    \mu_{j0}^g = 
      \frac{\alpha_{j0}^g}{\alpha_{j0}^g + \beta_{j0}^g}.
\end{equation*}
This average is based on a sample size of type $g$ students:
\begin{equation*}
    n_{j0}^g = \alpha_{j0}^g + \beta_{j0}^g
 \end{equation*}
\end{itemize}

\vspace{2ex}
Group-based prior beliefs about probability of success in skill-$j$ courses, $\theta_j$:
\begin{equation*}
    \mathcal{B} \pr{\alpha_{j0}^g, \beta_{j0}^g} \quad \implies \quad
    \mathcal{B} \pr{\mu_{j0}^g n_{j0}^g, (1 - \mu_{j0}^g) n_{j0}^g}
\end{equation*}



\end{frame}

%%%%%%%%%%%%%%%%%%%%%%%%%%%%%%%%%%%%%%%%%%%%%%%%%%%%%%%%%%%%%%%%%%%%%%%%%%%%%%%%
\begin{frame}{Group-based belief distribution}

% Suppose sample size of men is larger than that of women, but the observed success rate is the same for the two groups:
Suppose there are more men then women in field $j$: 
\begin{equation*}
  n_{j0}^m > n_{j0}^f
\end{equation*}
But the observed success rate is the same for the two groups:
\begin{equation*}
    \mu_{j0} = \mu_{j0}^m = \mu_{j0}^w
\end{equation*}

\begin{figure}
\input{beta_example_gender.tex}
\end{figure}


\end{frame}


%%%%%%%%%%%%%%%%%%%%%%%%%%%%%%%%%%%%%%%%%%%%%%%%%%%%%%%%%%%%%%%%%%%%%%%%%%%%%%%%

% \nts{Frame remaining simple parameterization}
\begin{frame}{Individual problem}


A policy $\pi: (h_t, P_t^g) \to (s_t, \ell_t)$ is optimal if it maximizes:
\begin{align*}
& \mathbb{E}^\pi \sbr{
   \sum_{t=0}^\infty \delta^t 
   \left. \pr{\sum_{j=1}^J h_{jt} w_j \ell_{jt} } \right\vert
   \pr{(h_{10}, P_{10}^g), \dots, (h_{10}, P_{J0}^g)}
} \\
\text{subject to} \quad& h_{jt+1} = h_{jt}+ a_{jt} s_{jt}, \quad \quad a_{jt} = 
   \begin{cases} 
      \nu_j, & \text{with prob. } \theta_j,  \\ 
      0, & \text{with prob. } 1 - \theta_j,
   \end{cases} 
   %\quad \quad h_{j0} = \alpha_{j0} \nu_j, 
   \\
\quad& P_{jt+1}^g = 
  \mathcal{B} (\alpha_{j,t+1}^g, \beta_{j,t+1}^g), 
  \quad \quad \theta_j \sim P_{j,0}^g \equiv \mathcal{B} (\alpha_{j0}^g, \beta_{j0}^g)
  \quad \quad \text{if $j$ selected,} \\
\quad& \sum_{j=1}^J (s_{jt} + \ell_{jt}) = 1, \quad \quad s_{jt}, \ell_{jt} \in \{0,1\} \\
   & h_{j0} \leq \nu \alpha_{j0}
\end{align*}

\end{frame}

%%%%%%%%%%%%%%%%%%%%%%%%%%%%%%%%%%%%%%%%%%%%%%%%%%%%%%%%%%%%%%%%%%%%%%%%%%%%%%%%
\begin{frame}{Optimal policy rule}

Define the skill $j$ index as the expected payoff if you committed to studying $j$:
%\gen{ 
\begin{equation*}
\mathcal{I}_j (h_j, P_j^g) = \sup_{\tau \geq 0} \mathbb{E}^\tau
\ce{
   \sum_{t=0}^\infty \delta^t \pr{h_{jt} w_j \ell_{jt} }}
   {(h_{j0}, P_{j0}^g) = (h_j, P_j^g)
}
% \mathcal{I}_{jt} (h_{jt}, \alpha_{jt}, \beta_{jt}) = 
% \begin{cases}
% \frac{h_{jt}}{1 - \delta} & \text{if } \{\alpha_{jt}, \beta_{jt}\} \in \mathcal{G}_{j}, \\
% \frac{h_{jt}}{1 - \delta} \sbr{
%    \frac{
%       \left\lceil \frac{\delta}{1 - \delta} \right\rceil
%       \delta^{\left\lceil \frac{\delta}{1 - \delta} \right\rceil - c_{jt} - \alpha_{j0} - \beta_{j0}}}
%    {c_{jt} + \alpha_{j0} + \beta_{j0}}
%    } & \text{if } \{\alpha_{jt}, \beta_{jt}\} \notin \mathcal{G}_{j} \\
% \end{cases}
\end{equation*}

Define the graduation region of skill $j$ as: 
\begin{equation*}
% \mathcal{G}_j =  \left\{ \alpha_{jt}, \beta_{jt} \left\vert c_{jt} \geq \left\lceil \frac{\delta}{1 - \delta} \right\rceil - (\alpha_{j0} + \beta_{j0}) \right. \right\}
\mathcal{G}_j = \left\{ (h_j, P_j^g) \left\vert
   \arg \max_{\tau \geq 0} 
   \mathbb{E}^\tau \ce{\sum_{t=0}^\infty \delta^t \pr{h_{jt} w_j \ell_{jt} }}
   {(h_j, P_j^g)} = 0
   \right. \right\}
\end{equation*}

The following policy $\pi: (h_t, P_t^g) \to (s_t, \ell_t)$ is optimal: 
\begin{enumerate}
    \item At each $t \geq 0$, choose skill $j^* = \arg \max_{j \in J} \mathcal{I}_j$, breaking ties according to any rule
    \item If $(h_{j^*}, P_{j^*}^g) \in \mathcal{G}_{j}$, then enter the labor market as a $j^*$ specialist. Otherwise, study $j^*$ for an additional period.  
\end{enumerate}

\end{frame}

%%%%%%%%%%%%%%%%%%%%%%%%%%%%%%%%%%%%%%%%%%%%%%%%%%%%%%%%%%%%%%%%%%%%%%%%%%%%%%%%

%%%%%%%%%%%%%%%%%%%%%%%%%%%%%%%%%%%%%%%%%%%%%%%%%%%%%%%%%%%%%%%%%%%%%%%%%%%%%%%

