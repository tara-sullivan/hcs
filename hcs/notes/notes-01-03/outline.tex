\documentclass[10 pt]{article}

% Format
\usepackage[T1]{fontenc}
\usepackage[utf8]{inputenc}
\usepackage[margin=1in]{geometry} % 1-inch margins
\usepackage[english]{babel} % English hyphenation, etc.	
\usepackage{setspace} % Set spacing 
%\usepackage{enumerate} % Use different types of enumerate options
\usepackage{enumitem}
\usepackage{csquotes} % Block quotes
\usepackage[dvipsnames]{xcolor} % colors: https://en.wikibooks.org/wiki/LaTeX/Colors#The_68_standard_colors_known_to_dvips
% Math
\usepackage{amsmath, mathrsfs, amsfonts, amssymb, amsthm}
% Figures
\usepackage{graphicx} % Include figures
\usepackage{float} % Improved control over floats
\usepackage{tikz} % Draw figures with tikz

% Colors
\definecolor{Indigo}{HTML}{3C6478}
\definecolor{DarkBrown}{HTML}{45281B}
\definecolor{Brown}{HTML}{161402}
\definecolor{DarkGreen}{HTML}{325101}
\definecolor{LeafGreen}{HTML}{4A6F01}
\definecolor{DarkAlice}{HTML}{107896}
\definecolor{Alice}{HTML}{1496BB}
\definecolor{DarkGray}{RGB}{116 118 120}
\definecolor{DarkBlue}{HTML}{022C36}
\definecolor{MainBlue}{HTML}{132881}
\definecolor{Maroon}{HTML}{6A123D}
\definecolor{DarkPurple}{HTML}{2C033A}
\definecolor{Orange}{HTML}{F18312}

% Hyperlinks
\usepackage{hyperref} % Include hyperlinks
\hypersetup{
  colorlinks   = true, %Colours links instead of ugly boxes
  urlcolor     = Maroon, %Colour for external hyperlinks
  linkcolor    = DarkGray, %Colour of internal links
  citecolor   = MainBlue %Colour of citations
}


% Macro Shortcuts
\newcommand{\R}{\mathbb{R}}
\newcommand{\Q}{\mathbb{Q}}
\newcommand{\Z}{\mathbb{Z}}
\newcommand{\N}{\mathbb{N}}
\newcommand{\EE}{\mathbb{E}}
\newcommand{\PP}{\mathbb{P}}
\newcommand{\BB}{\mathscr{B}}
\newcommand{\e}{\text{e}}
\newcommand{\dd}{\text{d}}


% Theorems
\newtheorem{prop}{Proposition}[section]
\newtheorem{thm}{Theorem}
\theoremstyle{remark}
\newtheorem{claim}{Claim}[section]
\newtheorem{remark}{Remark}
\theoremstyle{definition}
\newtheorem{defn}{Definition}[section]
\newtheorem{lemma}{Lemma}
\newtheorem{ass}{Assumption}


\newif\ifnts
\ntstrue % uncomment to show 
% Notes to self 
\ifnts
  \newcommand{\nts}[1]{{\color{gray}#1}}
\else
  \newcommand{\nts}[1]{}
\fi

%%%%%%%%%%
% Sections that have:
%   (A) Roman numerals 
%   (B) fixed width = fixw
%   (C) coloring

% (A) Roman numeral for section and subsection
% \renewcommand{\thesection}{\Roman{section}} 
% \renewcommand{\thesubsection}{\roman{subsection}}

% (B) Each section has fixed width = fixw 

% Define fixw
\newcommand{\fixw}{28pt}
\newcommand{\fixwh}{14pt}

% (C) Define colors
\newcommand{\secc}[1]{{\color{DarkGreen}#1}} % section color
\newcommand{\sectc}{DarkGreen} % section text color
\newcommand{\subsecc}[1]{{\color{LeafGreen}#1}} % subsection color
\newcommand{\subsectc}{LeafGreen} % subsection text color
\newcommand{\numc}{DarkAlice}

% Set each section width and color
\usepackage{titlesec}
\titleformat{\section}{\normalfont\Large\bfseries\color{\sectc}}
	{\makebox[\fixw][l]{\secc{\thesection.}}}{0pt}{} 
\titleformat{\subsection}{\normalfont\large\bfseries\color{\subsectc}}
	{\makebox[{\fixw}][l]{\subsecc{\thesubsection.}}}{0pt}{} 
\titleformat{\subsubsection}{\normalfont\bfseries}
	{}{0pt}{} %{\makebox[{\fixw}][l]{}}{0pt}{} 

% Highlight certain items
\newcommand{\hitem}[2][DarkAlice]{\color{#1} \item #2 \color{black}}

%%%%%%%%%%
% Lists that start at fixw (see section above)
\newlist{outline}{enumerate}{2}
\setlist[outline,1]{label=\arabic*.,left=0pt .. \fixw}
\setlist[outline,2]{label=\alph*.,left=0pt .. \fixw}

\newlist{blist}{itemize}{2}
\setlist[blist,1]{label=\textbullet,left=0pt .. \fixw}
\setlist[blist,2]{label=\textendash,left=0pt .. \fixw}

%%%%%%%%%%
% Enumerate in footnote
\newlist{footcount}{enumerate}{1}
\setlist[footcount]{label=(\alph*),left=0pt .. \fixw}

%%%%%%%%%%
% Foodnote Edits
% Bottom package ensures that footnote won't be above a figure
\usepackage[bottom]{footmisc}

% No indent in footnotes
% NOTHING SEEMS TO WORK
% \usepackage{scrextend}
% \deffootnote[\fixw]{\fixw}{.195in}{\makebox[\fixw][r]{\thefootnotemark.\hspace{.2in}}}
% \usepackage[flushmargin, hang]{footmisc} % flush footnote mark to left margin
% \renewcommand{\footnotelayout}{\doublespacing\raggedright}
% \usepackage[flushmargin,hang]{footmisc}
% \usepackage[hang, flushmargin]{footmisc}
% \setlength{\footnotemargin}{0.5in}


% % \usepackage[marginal]{footmisc}
% \setlength\footnotemargin{0pt}  % default value: 1.8em

% \usepackage[flushmargin,hang]{footmisc}
% % \setlength{\footnotemargin}{1em} % just to show clearly equal output

% % \usepackage[marginal]{footmisc}
% \setlength{\footnotemargin}{10em} % just to show clearly equal output

% \renewcommand{\footnotelayout}{\raggedright}


%%%%%%%%%%
% Skip line between paragraphs, set indent to \fixw
\usepackage[parfill, indent=\fixw]{parskip}

%%%%%%%%%%
% format caption
% get rid of 'Figure: ' in caption
\usepackage{caption}
% \captionsetup[table]{labelsep=space}
\captionsetup{%
    % labelformat=empty,
    % font=small,
    labelsep=quad,
    tableposition=top,
    labelsep=period,
    margin=\fixw,
}



% General problem, not bernoulli-beta example
\newif\ifgen
% \gentrue % uncomment to show 
\ifgen
  \newcommand{\gen}[1]{#1}
\else
  \newcommand{\gen}[1]{}
\fi

% Include introduction or not
\newif\ifintro
\introtrue % uncomment to show 
\ifintro
  \newcommand{\intro}[1]{#1}
\else
  \newcommand{\intro}[1]{}
\fi

% Include math notes or not
\newif\ifmathnotes
\mathnotestrue % uncomment to show 
\ifmathnotes
  \newcommand{\mathnotes}[1]{#1}
\else
  \newcommand{\mathnotes}[1]{}
\fi

\newcommand{\cc}{\mathbf{c}}
\newcommand{\xx}{\mathbf{x}}

\newcommand{\br}[1]{\left\{ #1 \right\}}
\newcommand{\sbr}[1]{\left[ #1 \right]}
\newcommand{\pr}[1]{\left( #1 \right)}
\newcommand{\ce}[2]{\left[\left. #1 \right\vert #2 \right]}

% Bibliography
% Include introduction or not
\usepackage[backend=biber,style=authoryear,citestyle=authoryear,maxcitenames=3,minbibnames=8,maxbibnames=15,uniquename=false]{biblatex}
\addbibresource{../bibliography.bib}

\newif\ifnobib
%\nobibtrue % uncomment to show 
\ifnobib
  \newcommand{\nobib}[1]{}
\else
  \newcommand{\nobib}[1]{#1}
\fi

% remove space 
\usepackage[font=small,skip=0pt]{caption}

% To find size of textwidth, part 1. 
%\usepackage{layouts}
\begin{document}
% To find size of textwidth, part 2 (6.50127in)
%\printinunitsof{in}\prntlen{\textwidth}



\title{Notes on Human Capital Specialization}
\author{Tara Sullivan}

\maketitle
\onehalfspacing

\noindent\nts{Please note that gray text are notes/comments}

\intro{

%%%%%%%%%%%%%%%%%%%%%%%%%%%%%%%%%%%%%%%%%%%%%%%%%%%%%%%%%%%%%%%%%%%%%%%%%%%%%%%%
\section{Introduction}
%%%%%%%%%%%%%%%%%%%%%%%%%%%%%%%%%%%%%%%%%%%%%%%%%%%%%%%%%%%%%%%%%%%%%%%%%%%%%%%%

% In a model of human capital specialization with unknown ability 
% priors informed by existin group outcomes
% formalizes self-fulfiling prohpecies 
% provides a framework for quantifiying the inefficiencies associated with human capital specialization 
% 

% uncertainty surrounding the returns to success. 

% a bit of a jargon-y note: "role model" in an academic sense
% Mayor Pete: there isn't someone who they know personally who estifies to the value of education
% This is incorrect

% what if you see that the employment rate is twice as high for white grads 

% chance to fail 



% We often rely on prefererences are so often invoked
% Economists, of all people, shouldn't discount the role of rational decision making based on observable facts

% Can I say this: a model of attrition based on observable data and rational decision making, not assumptions about preferences or 

\begin{outline}

\item Current literature on discrimination/affirmative action in higher education: \nts{(goal is to find a statement that encapsulates the following in a relevant way)}

\begin{blist}

\item Whether or not it helps or hurts the minority students it's meant to help \parencite{AL16}

\item Whether or not employing these policies is biased against students in the majority category\footnote{
   Maybe cite something about the Harvard trial here; we read a nice piece in ECON128 in Week 8. 
}

\item Specifically consider choice of college major and the focus on student preparedness \textcite{AAH16} 

\end{blist}

\item Shortcomings of the literature I want to address: \nts{(still working through this)}

\begin{blist}

%\item %Empirical studies rely on very precise mechanisms; I want something flexble.

\item \nts{I think the focus on student preparedness and outcomes misses a couple of things. First, measuring preparedness by GPA and SAT score, as is often the case, misses a lot. I don't know if I can address this, but I think it's important to mention. Second, misallocation of human capital has aggregate welfare effects. Empirical results can't account for that. This is something I can get at. }

\nts{\item I also don't like assuming that the college process is the same for different groups. The cost function for particular majors may be very different for one group; note that this is where the \textcite{CL93} framework might be useful.}

\end{blist}

\item This paper:

\begin{blist}

\item I consider the relationship between discrimination theories \nts{and existing group outcomes} and human capital specialization
% How do different theories of discrimination impact human capital specialization? 

\item Augment model of gradual human capital specialization in \textcite{AF20}

\item Flexible framework that accommodates mechanisms that are not generally accounted for in this literature. \

\item Existing grout outcomes have an impact on human capital specialization decisions. 

\item Shocks received impact human capital specialization decisions. Key: shocks impact groups differently \nts{(I think)}.

\item Incorporates statistical discrimination 

\item limitations of empirical analyses when evaluating the efficacy of affirmative action programs. 

\item Counterfactual analysis: misallocation of human capital specialization

\item Counterfactual analysis: the effectiveness of affirmative action policies. 

\end{blist}

% Another application: how do peopl specialize after recessions

% prior determined by existing outcomes
% prior determined by statistical discrimination 
% how you interpret shocks 


% One nice thing about this being a partial equilibrium approach: human capital here is really just whether your investment in a class pays off in a higher wage in the future. There could still be frictions between your underlying ability and your wage. 

\end{outline}

%%%%%%%%%%%%%%%%%%%%%%%%%%%%%%%%%%%%%%%%%%%%%%%%%%%%%%%%%%%%%%%%%%%%%%%%%%%%%%%%
\subsection*{Literature}

\begin{blist}

\item Similar to \textcite{D08} in that I'm interested in efficiency loss from human capital misallocation \nts{(there's commentary on this in \textcite{AL16} that's worth reflecting on)}

\end{blist}


} % end of intro flag


%%%%%%%%%%%%%%%%%%%%%%%%%%%%%%%%%%%%%%%%%%%%%%%%%%%%%%%%%%%%%%%%%%%%%%%%%%%%%%%%
\section{Model}
%%%%%%%%%%%%%%%%%%%%%%%%%%%%%%%%%%%%%%%%%%%%%%%%%%%%%%%%%%%%%%%%%%%%%%%%%%%%%%%%

%%%%%%%%%%%%%%%%%%%%%%%%%%%%%%%%%%%%%%%%%%%%%%%%%%%%%%%%%%%%%%%%%%%%%%%%%%%%%%%%
\subsection{Alon and Fershtman (2020)} \nobib{\nocite{AF20}}

%%%%%%%%%%%%%%%%%%%%%%%%%%%%%%%%%%%%%%%%%%%%%%%%%%%%%%%%%%%%%%%%%%%%%%%%%%%%%%%%
\subsubsection{Preliminaries}

\begin{tabular}{@{}lll}
\textbf{General}  & \textbf{Description} \\
\multicolumn{2}{@{}l}{\emph{Endowments}} \\
$t$                                       & indivisible unit of time per period \\
$h_0    = \{ h_{10} \dots h_{N0} \}$      & initial human capital \\
$\theta = \{\theta_1, \dots, \theta_N \}$ & unknown abilities \\
$P_{j0}$                                  & independent initial beliefs on $\theta_j$ \\
\multicolumn{2}{@{}l}{\emph{Actions}} \\
$s_t    = \{ s_{1t}, \dots, s_{Nt} \}$    & study time \\
$\ell_t = \{ l_{1t}, \dots, \ell_{Nt} \}$ & work time \\
\multicolumn{2}{@{}l}{\emph{Technology}} \\
$a_{it}   \sim F_{\theta_i}$      & effective study time \\
$h_{it+1} = H_i (h_{it}, a_{it})$ & accumulation by $t+1$ \\
\multicolumn{2}{@{}l}{\emph{Beliefs}} \\
$P_{it+1} = \Pi (P_{it}, a_{it})$ & evolution of beliefs over $\theta_i$
\end{tabular}

%%%%%%%%%%%%%%%%%%%%%%%%%%%%%%%%%%%%%%%%%%%%%%%%%%%%%%%%%%%%%%%%%%%%%%%%%%%%%%%%
\subsubsection{Problem}

\gen{
A policy $\pi: (h_t, P_t) \to (s_t, \ell_t)$ is optimal if it maximizes: 
\begin{align*}
& \mathbb{E}^\pi \sbr{
   \sum_{t=0}^\infty \delta^t 
   \left. \pr{\sum_{i=1}^N \ell_{it} U_i(w_i, h_{it})} \right\vert
   \pr{(h_{10}, P_{10}), \dots, (h_{10}, P_{N0})}
} \\
\text{subject to} \quad& h_{it+1} = H_i (h_{it}, a_{it}), \quad \quad
P_{it+1} = \Pi (P_{it}, a_{it}), 
\quad \quad \text{if $i$ selected,} \\
\quad& \sum_{i=1}^N (s_{it} + \ell_{it}) = 1, \quad \quad s_{it}, \ell_{it} \in \{0,1\}
\end{align*}
It is often useful to parameterize this problem.}
A policy $\pi: (h_t, P_t) \to (s_t, \ell_t)$ is optimal if it maximizes:
\begin{align*}
& \mathbb{E}^\pi \sbr{
   \sum_{t=0}^\infty \delta^t 
   \left. \pr{\sum_{i=1}^N \ell_{it} w_i h_{it}} \right\vert
   \pr{(h_{10}, P_{10}), \dots, (h_{10}, P_{N0})}
} \\
\text{subject to} \quad& h_{it+1} = h_{it}+ a_{it} s_{it}, \quad \quad a_{it} = 
   \begin{cases} 
      \nu_i, & \text{with prob. } \theta_i,  \\ 
      0, & \text{with prob. } 1 - \theta_i,
   \end{cases} 
   %\quad \quad h_{i0} = \alpha_{i0} \nu_i, 
   \\
\quad& P_{it+1} = \mathcal{B} (\alpha_{i,t+1}, \beta_{i,t+1}), 
   \quad \quad \theta_i \sim P_{i,0} \equiv \mathcal{B} (\alpha_{i0}, \beta_{i0})
   \quad \quad \text{if $i$ selected,} \\
\quad& \sum_{i=1}^N (s_{it} + \ell_{it}) = 1, \quad \quad s_{it}, \ell_{it} \in \{0,1\}
\end{align*}

% Notes/questions
\nts{
\begin{itemize}
\item Shouldn't I either index $\pi$ by $t$, or say that a policy at time $t$ is optimal? 
\item Shouldn't this be if $s_{it} = 1$, meaning that you study $i$ during period $t$, rather than ``if $i$ is selected?''. I'm pretty sure that, because you could choose to work in skill $i$, you need to specify if you're studying.
\item Also, shouldn't the transition laws be $h_{it+1} = H_i (h_{it}, a_{it}, s_{it})$, since it depends on whether you study? This could also be fixed by changing the ``if $i$ selected line.'' I would also call this transition law ``course outcome'' or ``studying outcome.''
\end{itemize}
} 

%%%%%%%%%%%%%%%%%%%%%%%%%%%%%%%%%%%%%%%%%%%%%%%%%%%%%%%%%%%%%%%%%%%%%%%%%%%%%%%%
\subsubsection{Optimal Policy}

Let $\tau$ be an optimal stopping rule defined over $\{ a_{j1}, a_{j2}, \dots \}$. 
%%%%%%%%%%%%%%%
% Skill j index
Define the skill $j$ index as the expected payoff if you committed to studying $j$:
%\gen{ 
\begin{equation}\label{idx}
\mathcal{I}_j (h_j, P_j) = \sup_{\tau \geq 0} \mathbb{E}^\tau
\ce{
   \sum_{t=0}^\infty \delta^t U_j (w_j, h_{jt}) \ell_{jt}}
   {(h_{j0}, P_{j0}) = (h_j, P_j))
}
\end{equation}
%%%%%%%%%%%%%%%%%%%
% Graduation region
Define the graduation region of skill $j$ as: 
\begin{equation}\label{grad}
\mathcal{G}_j = \left\{ (h_j, P_j) \left\vert
   \arg \max_{\tau \geq 0} 
   \mathbb{E}^\tau \ce{\sum_{t=0}^\infty \delta^t U_j (w_j, h_j) \ell_{jt}}
   {(h_j, P_j)} = 0
   \right. \right\}
\end{equation}
%%%%%%%%%%%%%%%%%%%%%%%%
% Optimal policy theorem
The following policy $\pi: (h_t, P_t) \to (s_t, \ell_t)$ is optimal: 
\begin{outline}
	\item At each $t \geq 0$, choose skill $j^* = \arg \max_{i \in N} \mathcal{I}_i$, breaking ties according to any rule
	\item If $(h_{j^*}, P_{j^*}) \in \mathcal{G}_{j}$, then enter the labor market as a $j^*$ specialist. Otherwise, study $j^*$ for an additional period.  
\end{outline}
Assuming $h_{j0} \leq \nu \alpha_{j0}$ means that the stopping problem is monotonic.\nts{\footnote{I need better inutition about what this assumption means.}}
The monotonicity of the stopping problem implies that one-step look-ahead is optimal.\nts{\footnote{
	Should this statement go before or after the statement of the optimal policy? 
	Is this statement fully correct? 
}}

%%%%%%%%%%%%%%%%%%%%%%%%%%%%%%%%%%%%%%%%%%%%%%%%%%%%%%%%%%%%%%%%%%%%%%%%%%%%%%%%
\subsubsection{Parameterizations}

It is helpful to consider several typues of parameterizations of this problem. 

\begin{outline}

%%%%%%%%%%%%%%%%%%%%%
% Most tractable case

\hitem{Assume $h_{j0} = \nu \alpha_{j0}$}

Letting $c_{jt}$ be the number of periods and individual spends studying $j$, the stopping condition in this problem simplifies to:
\begin{equation*}
	\frac{1- \delta}{\delta} \geq \frac{1}{c_{jt} + \alpha_{j0} + \beta_{j0}}
\end{equation*}
This implies that the number of periods the individual spends studying skill $j$ before entering the labor market is deterministic: 
\begin{equation}\label{bb-mj}
	c_j^* = \left\lceil \frac{\delta}{1 - \delta} \right\rceil - (\alpha_{j0} + \beta_{j0})
\end{equation}
Under this assumption, the optimal policy can written analytically:
\begin{align*}
\mathcal{I}_{jt} (h_{jt}, \alpha_{jt}, \beta_{jt}) = &
\begin{cases}
\frac{h_{jt}}{1 - \delta} & \text{if } \{\alpha_{jt}, \beta_{jt}\} \in \mathcal{G}_{j}, \\
\frac{h_{jt}}{1 - \delta} \sbr{
   \frac{
      \left\lceil \frac{\delta}{1 - \delta} \right\rceil
      \delta^{\left\lceil \frac{\delta}{1 - \delta} \right\rceil - c_{jt} - \alpha_{j0} - \beta_{j0}}}
   {c_{jt} + \alpha_{j0} + \beta_{j0}}
   } & \text{if } \{\alpha_{jt}, \beta_{jt}\} \notin \mathcal{G}_{j} \\
\end{cases} \\
   \mathcal{G}_j = & \left\{ \alpha_{jt}, \beta_{jt} \left\vert c_{jt} \geq \left\lceil \frac{\delta}{1 - \delta} \right\rceil - (\alpha_{j0} + \beta_{j0}) \right. \right\}
\end{align*}
% Notes/questions
\nts{
\begin{blist}
   \item Shouldn't it be $\mathcal{G}_{j^*}$?
   \item Shouldn't $\tau$ be indexed by $j$?
   \item Is graduation region based on OSLA or initial conditions? I think that when I have some of my ECON200B notes around, I should think about how I could more cleanly write the optimal stopping rule. May it would be clearer to write $\tau_j$?
\end{blist}
}

%%%%%%%%%%%%%%%%%%%%%
% More reasonable case

\hitem{Assume $h_{j0} \leq \nu \alpha_{j0}$}

The stopping condition in this problem simplifies to:
\begin{equation}\label{stop}
	\frac{1- \delta}{\delta} \geq 
	\frac{\nu \alpha_{j0} + h_{jt} - h_{j0}}{h_{jt}(c_{jt} + \alpha_{j0} + \beta_{j0})}
\end{equation}
The index and graduation region are modified accordingly. 

\end{outline}




%%%%%%%%%%%%%%%%%%%%%%%%%%%%%%%%%%%%%%%%%%%%%%%%%%%%%%%%%%%%%%%%%%%%%%%%%%%%%%%%
\mathnotes{
\subsubsection*{Math notes}

Math theorem: If stopping time problem monotonic, then OSLA (one-step look-ahead) stopping rule optimal. 
\begin{blist}

\item Monotonic stopping time: If the optimal stopping rule for skill $j$ says stop today, then the optimal stopping rule would say stop tomorrow, regardless of the stochastic outcome. This would be built into the human capital problem. 

\item Monotonicity of the stopping problem means that stopping at $t$ implies stopping at $t+1$. Stopping at time $t$ implies: \nts{(I think)}
\begin{equation*}
 U_j(w_j, h_{jt}) > \delta \mathbb{E} U_j(w_j, h_{jt+1})
\end{equation*}
If the stoping problem is monotonic, then this means: 
\begin{equation*}
 U_j(w_j, h_{jt+1}) > \delta \mathbb{E} U_j(w_j, h_{jt+2})
\end{equation*}
This holds for all realizations of $a$. 
This condition then implies the optimality of OSLA.

Given the utility function in the parametric problem, if you are studying skill $j$, you would stop at time $t$ if:
\begin{equation}
h_{jt} > \delta \mathbb{E} \sbr{h_{jt+1}}
\end{equation}
The optimal stopping time is monotonic if the following holds for all $j$, for all realizations of $a_{jt}$:
\begin{equation}
h_{jt+1} > \delta \mathbb{E} \sbr{h_{jt+2}}
\end{equation}
Recall:
\begin{align*}
\mathbb{E} \sbr{h_{jt+1}} = \delta (h_{jt} + \mathbb{E} \sbr{a_{it}} s_{it})
\end{align*}
You stop studying skill $j$ at time $t$ ($s_{jt}=0$) if the utility associated with stopping is greater than the expected utility associated with studying ($s_{jt}=1$):
\begin{align*}
h_{jt} >& \delta \mathbb{E} \sbr{h_{jt+1}} \\
=& \delta (h_{jt} + \mathbb{E} \sbr{a_{jt}} s_{jt}) \\
=& \delta h_{jt} + \delta \pr{\nu_j \mathbb{E} \sbr{\theta_j} + 0 (1 - \mathbb{E} \sbr{\theta_j})} \\
=& \delta h_{jt} + \delta \pr{\nu_j \frac{\alpha_{jt}}{\alpha_{jt} + \beta_{jt}}} \\
\implies h_{jt} >& \frac{\delta}{1 - \delta} \pr{\frac{\nu_j \alpha_{jt}}{\alpha_{jt} + \beta_{jt}}}
\end{align*}
Using the fact that $h_{jt} = h_{j0} + \nu (\alpha_{jt} - \alpha_0)$ and the assumption that $h_{j0} = \alpha_{j0} h_{j0}$, the above implies:
\begin{equation*}
\nu \alpha_{jt} > \frac{\delta}{1 - \delta} \pr{\frac{\nu_j \alpha_{jt}}{\alpha_{jt} + \beta_{jt}}} \iff \frac{1-\delta}{\delta} > \alpha_{jt} + \beta_{jt}
\end{equation*} 
\nts{I'm not finishing this right now. But it's worth noting that, in the notes that Titan gave me, a sufficient condition for monotonicity in the beta-Bernoulli problem is that $h_{j0} = \alpha_{j0} h_{j0}$. So I know that you can prove that, under this assumption, the problem is monotonic. I might need to relax that assumption for my model, but it doesn't need to be a problem.}
\end{blist}
} % end of optional math notes section


%%%%%%%%%%%%%%%%%%%%%%%%%%%%%%%%%%%%%%%%%%%%%%%%%%%%%%%%%%%%%%%%%%%%%%%%%%%%%%%%
\subsection{Augmented model}
%%%%%%%%%%%%%%%%%%%%%%%%%%%%%%%%%%%%%%%%%%%%%%%%%%%%%%%%%%%%%%%%%%%%%%%%%%%%%%%%

%%%%%%%%%%%%%%%%%%%%%%%%%%%%%%%%%%%%%%%%%%%%%%%%%%%%%%%%%%%%%%%%%%%%%%%%%%%%%%%%
\subsubsection{Preliminaries}

\begin{blist}



\item I now consider how a student's human capital specialization decisions are affected by their group.
This involves augmenting the above model so that each student has a group type, $g$.
To keep things simple, assume there are only two groups, men and women ($g \in \{m, f\}$).\nts{\footnote{\nts{Possible extension: countably infinite types. What does this mean if you value diversity? Could you get results like diversity helps students who are not in the targeted group to assist?}}} 
The distributions of underlying abilities, $\theta_j$, are the same for men and women.\nts{\footnote{\nts{
  What is the maintained hypothesis in a Bayesian problem?
  Am I assuming that $\theta_j$ is distributed 
}}}
However, initial beliefs about underlying abilities, $P_{j0}^g \equiv \mathcal{B} \pr{\alpha_{j0}^g, \beta_{j0}^g}$, are different for the two groups.
\nts{Be specific about what this will do.}

\item 
\nts{The following would maybe be in the initial model description:}
The ability parameter $\theta_j$ is the probability that a student successfully improves their human capital in subject $j$ after studying it for one period.
A student's decision to enter a particular field depends in part on how much human capital they can expect to successfully acquire, and therefore depends on $\theta_j$.
It may be helpful to think of $\theta_j$ as the student's probability of success or failure in a particular course in field $j$.\footnote{
   It is worth highlighting that the following assumes that probability of success or failure is independent and stationary throughout time.}
The distribution of beliefs about $\theta_j$ %, $\mathcal{B} \pr{\alpha_{j0}^g, \beta_{j0}^g}$, 
is a distribution of beliefs about probabilities.\nts{\footnote{\nts{For a nice, intuitive understanding of this, see \url{https://stats.stackexchange.com/q/47782}}}}
How these beliefs are formed is central to this paper.

\item 
%A student may use previously observed success rates to their prior about their probability of success may use 
A student may form their belief about their probability of success using existing outcomes for their group.
For this example, suppose a type $g$ student's belief about their own ability is based on previously observed success rates for group $g$. 
Let $\alpha_{j0}^g$ and $\beta_{j0}^g$ denote the number of type $g$ students who have succeeded and failed in field $j$ at time 0, respectively.\footnote{
   I am being purposefully vague about what ``studying'' and ``succeeding'' represent.
   For example, the number of ``successful'' students could be:
   \begin{footcount}
      \item The number of students who graduated into field $j$
      \item The number of students who attain human capital $h_j$
      \item Both (a) and (b)
      \item The number of students who pass an initial course in field $j$
    \end{footcount} 
   Many other examples are possible. The total number of observed students who attempt success would need to account for those who tried going this route and failed. 
  \nts{
  This is tricky, and very important. 
  How this endogenously responds is super key; I may run into complications here regarding how this belief updates. I'm assuming that you form your belief when you enter the field, and then that doesn't change at all. It doesn't allow for leaky pipelines to impact beliefs, but this could probably be changed.}
}  
Then the observed success rate, $\mu_{j}^g$, is given by:
\begin{equation*}
\mu_{j0}^g = 
  \frac{\alpha_{j0}^g}{\alpha_{j0}^g + \beta_{j0}^g}.
  %\mu_{j0} = 
  %\frac{\alpha_{j0}^m}{\alpha_{j0}^m + \beta_{j0}^m} = 
  %\frac{\alpha_{j0}^f}{\alpha_{j0}^f + \beta_{j0}^f}
\end{equation*}
This average is based on a sample of type $g$ students of size $n_{j0}^g = \alpha_{j0}^g + \beta_{j0}^g$.
Note that the prior distribution $\mathcal{B} \pr{\alpha_{j0}^g, \beta_{j0}^g}$ can now be specified using the alternative parameterization, $\mathcal{B} \pr{\mu_{j0}^g n_{j0}^g, (1 - \mu_{j0}^g) n_{j0}^g}$.

\item
Assume the sample size of men is larger than that of women, but the observed success rate is the same for the two groups:
\begin{equation*}
  n_{j0}^m > n_{j0}^f, \quad \quad \mu_{j0} = \mu_{j0}^m = \mu_{j0}^w.
  %n_{j0}^m = \alpha_{j0}^m + \beta_{j0}^m > \alpha_{j0}^f + \beta_{j0}^f = n_{j0}^w
\end{equation*}
%Assuming the prior on $\theta_j$ follows a $\mathcal{B} \pr{\alpha_{j0}^g, \beta_{j0}^g}$ distribution, this yields the following parameterization:
This implies that women have more uncertainty regarding their underlying abilities than men.  
\begin{figure}[htb]
\begin{center}
%% Creator: Matplotlib, PGF backend
%%
%% To include the figure in your LaTeX document, write
%%   \input{<filename>.pgf}
%%
%% Make sure the required packages are loaded in your preamble
%%   \usepackage{pgf}
%%
%% Figures using additional raster images can only be included by \input if
%% they are in the same directory as the main LaTeX file. For loading figures
%% from other directories you can use the `import` package
%%   \usepackage{import}
%% and then include the figures with
%%   \import{<path to file>}{<filename>.pgf}
%%
%% Matplotlib used the following preamble
%%   \usepackage[utf8]{inputenc}
%%   \usepackage[T1]{fontenc}
%%
\begingroup%
\makeatletter%
\begin{pgfpicture}%
\pgfpathrectangle{\pgfpointorigin}{\pgfqpoint{3.900762in}{2.600508in}}%
\pgfusepath{use as bounding box, clip}%
\begin{pgfscope}%
\pgfsetbuttcap%
\pgfsetmiterjoin%
\definecolor{currentfill}{rgb}{1.000000,1.000000,1.000000}%
\pgfsetfillcolor{currentfill}%
\pgfsetlinewidth{0.000000pt}%
\definecolor{currentstroke}{rgb}{1.000000,1.000000,1.000000}%
\pgfsetstrokecolor{currentstroke}%
\pgfsetdash{}{0pt}%
\pgfpathmoveto{\pgfqpoint{0.000000in}{0.000000in}}%
\pgfpathlineto{\pgfqpoint{3.900762in}{0.000000in}}%
\pgfpathlineto{\pgfqpoint{3.900762in}{2.600508in}}%
\pgfpathlineto{\pgfqpoint{0.000000in}{2.600508in}}%
\pgfpathclose%
\pgfusepath{fill}%
\end{pgfscope}%
\begin{pgfscope}%
\pgfsetbuttcap%
\pgfsetmiterjoin%
\definecolor{currentfill}{rgb}{1.000000,1.000000,1.000000}%
\pgfsetfillcolor{currentfill}%
\pgfsetlinewidth{0.000000pt}%
\definecolor{currentstroke}{rgb}{0.000000,0.000000,0.000000}%
\pgfsetstrokecolor{currentstroke}%
\pgfsetstrokeopacity{0.000000}%
\pgfsetdash{}{0pt}%
\pgfpathmoveto{\pgfqpoint{0.487595in}{0.325064in}}%
\pgfpathlineto{\pgfqpoint{3.510686in}{0.325064in}}%
\pgfpathlineto{\pgfqpoint{3.510686in}{2.288447in}}%
\pgfpathlineto{\pgfqpoint{0.487595in}{2.288447in}}%
\pgfpathclose%
\pgfusepath{fill}%
\end{pgfscope}%
\begin{pgfscope}%
\pgfsetbuttcap%
\pgfsetroundjoin%
\definecolor{currentfill}{rgb}{0.000000,0.000000,0.000000}%
\pgfsetfillcolor{currentfill}%
\pgfsetlinewidth{0.803000pt}%
\definecolor{currentstroke}{rgb}{0.000000,0.000000,0.000000}%
\pgfsetstrokecolor{currentstroke}%
\pgfsetdash{}{0pt}%
\pgfsys@defobject{currentmarker}{\pgfqpoint{0.000000in}{-0.048611in}}{\pgfqpoint{0.000000in}{0.000000in}}{%
\pgfpathmoveto{\pgfqpoint{0.000000in}{0.000000in}}%
\pgfpathlineto{\pgfqpoint{0.000000in}{-0.048611in}}%
\pgfusepath{stroke,fill}%
}%
\begin{pgfscope}%
\pgfsys@transformshift{0.625008in}{0.325064in}%
\pgfsys@useobject{currentmarker}{}%
\end{pgfscope}%
\end{pgfscope}%
\begin{pgfscope}%
\definecolor{textcolor}{rgb}{0.000000,0.000000,0.000000}%
\pgfsetstrokecolor{textcolor}%
\pgfsetfillcolor{textcolor}%
\pgftext[x=0.625008in,y=0.227841in,,top]{\color{textcolor}\sffamily\fontsize{10.000000}{12.000000}\selectfont \(\displaystyle 0.0\)}%
\end{pgfscope}%
\begin{pgfscope}%
\pgfsetbuttcap%
\pgfsetroundjoin%
\definecolor{currentfill}{rgb}{0.000000,0.000000,0.000000}%
\pgfsetfillcolor{currentfill}%
\pgfsetlinewidth{0.803000pt}%
\definecolor{currentstroke}{rgb}{0.000000,0.000000,0.000000}%
\pgfsetstrokecolor{currentstroke}%
\pgfsetdash{}{0pt}%
\pgfsys@defobject{currentmarker}{\pgfqpoint{0.000000in}{-0.048611in}}{\pgfqpoint{0.000000in}{0.000000in}}{%
\pgfpathmoveto{\pgfqpoint{0.000000in}{0.000000in}}%
\pgfpathlineto{\pgfqpoint{0.000000in}{-0.048611in}}%
\pgfusepath{stroke,fill}%
}%
\begin{pgfscope}%
\pgfsys@transformshift{1.174661in}{0.325064in}%
\pgfsys@useobject{currentmarker}{}%
\end{pgfscope}%
\end{pgfscope}%
\begin{pgfscope}%
\definecolor{textcolor}{rgb}{0.000000,0.000000,0.000000}%
\pgfsetstrokecolor{textcolor}%
\pgfsetfillcolor{textcolor}%
\pgftext[x=1.174661in,y=0.227841in,,top]{\color{textcolor}\sffamily\fontsize{10.000000}{12.000000}\selectfont \(\displaystyle 0.2\)}%
\end{pgfscope}%
\begin{pgfscope}%
\pgfsetbuttcap%
\pgfsetroundjoin%
\definecolor{currentfill}{rgb}{0.000000,0.000000,0.000000}%
\pgfsetfillcolor{currentfill}%
\pgfsetlinewidth{0.803000pt}%
\definecolor{currentstroke}{rgb}{0.000000,0.000000,0.000000}%
\pgfsetstrokecolor{currentstroke}%
\pgfsetdash{}{0pt}%
\pgfsys@defobject{currentmarker}{\pgfqpoint{0.000000in}{-0.048611in}}{\pgfqpoint{0.000000in}{0.000000in}}{%
\pgfpathmoveto{\pgfqpoint{0.000000in}{0.000000in}}%
\pgfpathlineto{\pgfqpoint{0.000000in}{-0.048611in}}%
\pgfusepath{stroke,fill}%
}%
\begin{pgfscope}%
\pgfsys@transformshift{1.724314in}{0.325064in}%
\pgfsys@useobject{currentmarker}{}%
\end{pgfscope}%
\end{pgfscope}%
\begin{pgfscope}%
\definecolor{textcolor}{rgb}{0.000000,0.000000,0.000000}%
\pgfsetstrokecolor{textcolor}%
\pgfsetfillcolor{textcolor}%
\pgftext[x=1.724314in,y=0.227841in,,top]{\color{textcolor}\sffamily\fontsize{10.000000}{12.000000}\selectfont \(\displaystyle 0.4\)}%
\end{pgfscope}%
\begin{pgfscope}%
\pgfsetbuttcap%
\pgfsetroundjoin%
\definecolor{currentfill}{rgb}{0.000000,0.000000,0.000000}%
\pgfsetfillcolor{currentfill}%
\pgfsetlinewidth{0.803000pt}%
\definecolor{currentstroke}{rgb}{0.000000,0.000000,0.000000}%
\pgfsetstrokecolor{currentstroke}%
\pgfsetdash{}{0pt}%
\pgfsys@defobject{currentmarker}{\pgfqpoint{0.000000in}{-0.048611in}}{\pgfqpoint{0.000000in}{0.000000in}}{%
\pgfpathmoveto{\pgfqpoint{0.000000in}{0.000000in}}%
\pgfpathlineto{\pgfqpoint{0.000000in}{-0.048611in}}%
\pgfusepath{stroke,fill}%
}%
\begin{pgfscope}%
\pgfsys@transformshift{2.273967in}{0.325064in}%
\pgfsys@useobject{currentmarker}{}%
\end{pgfscope}%
\end{pgfscope}%
\begin{pgfscope}%
\definecolor{textcolor}{rgb}{0.000000,0.000000,0.000000}%
\pgfsetstrokecolor{textcolor}%
\pgfsetfillcolor{textcolor}%
\pgftext[x=2.273967in,y=0.227841in,,top]{\color{textcolor}\sffamily\fontsize{10.000000}{12.000000}\selectfont \(\displaystyle 0.6\)}%
\end{pgfscope}%
\begin{pgfscope}%
\pgfsetbuttcap%
\pgfsetroundjoin%
\definecolor{currentfill}{rgb}{0.000000,0.000000,0.000000}%
\pgfsetfillcolor{currentfill}%
\pgfsetlinewidth{0.803000pt}%
\definecolor{currentstroke}{rgb}{0.000000,0.000000,0.000000}%
\pgfsetstrokecolor{currentstroke}%
\pgfsetdash{}{0pt}%
\pgfsys@defobject{currentmarker}{\pgfqpoint{0.000000in}{-0.048611in}}{\pgfqpoint{0.000000in}{0.000000in}}{%
\pgfpathmoveto{\pgfqpoint{0.000000in}{0.000000in}}%
\pgfpathlineto{\pgfqpoint{0.000000in}{-0.048611in}}%
\pgfusepath{stroke,fill}%
}%
\begin{pgfscope}%
\pgfsys@transformshift{2.823620in}{0.325064in}%
\pgfsys@useobject{currentmarker}{}%
\end{pgfscope}%
\end{pgfscope}%
\begin{pgfscope}%
\definecolor{textcolor}{rgb}{0.000000,0.000000,0.000000}%
\pgfsetstrokecolor{textcolor}%
\pgfsetfillcolor{textcolor}%
\pgftext[x=2.823620in,y=0.227841in,,top]{\color{textcolor}\sffamily\fontsize{10.000000}{12.000000}\selectfont \(\displaystyle 0.8\)}%
\end{pgfscope}%
\begin{pgfscope}%
\pgfsetbuttcap%
\pgfsetroundjoin%
\definecolor{currentfill}{rgb}{0.000000,0.000000,0.000000}%
\pgfsetfillcolor{currentfill}%
\pgfsetlinewidth{0.803000pt}%
\definecolor{currentstroke}{rgb}{0.000000,0.000000,0.000000}%
\pgfsetstrokecolor{currentstroke}%
\pgfsetdash{}{0pt}%
\pgfsys@defobject{currentmarker}{\pgfqpoint{0.000000in}{-0.048611in}}{\pgfqpoint{0.000000in}{0.000000in}}{%
\pgfpathmoveto{\pgfqpoint{0.000000in}{0.000000in}}%
\pgfpathlineto{\pgfqpoint{0.000000in}{-0.048611in}}%
\pgfusepath{stroke,fill}%
}%
\begin{pgfscope}%
\pgfsys@transformshift{3.373273in}{0.325064in}%
\pgfsys@useobject{currentmarker}{}%
\end{pgfscope}%
\end{pgfscope}%
\begin{pgfscope}%
\definecolor{textcolor}{rgb}{0.000000,0.000000,0.000000}%
\pgfsetstrokecolor{textcolor}%
\pgfsetfillcolor{textcolor}%
\pgftext[x=3.373273in,y=0.227841in,,top]{\color{textcolor}\sffamily\fontsize{10.000000}{12.000000}\selectfont \(\displaystyle 1.0\)}%
\end{pgfscope}%
\begin{pgfscope}%
\pgfsetbuttcap%
\pgfsetroundjoin%
\definecolor{currentfill}{rgb}{0.000000,0.000000,0.000000}%
\pgfsetfillcolor{currentfill}%
\pgfsetlinewidth{0.803000pt}%
\definecolor{currentstroke}{rgb}{0.000000,0.000000,0.000000}%
\pgfsetstrokecolor{currentstroke}%
\pgfsetdash{}{0pt}%
\pgfsys@defobject{currentmarker}{\pgfqpoint{-0.048611in}{0.000000in}}{\pgfqpoint{0.000000in}{0.000000in}}{%
\pgfpathmoveto{\pgfqpoint{0.000000in}{0.000000in}}%
\pgfpathlineto{\pgfqpoint{-0.048611in}{0.000000in}}%
\pgfusepath{stroke,fill}%
}%
\begin{pgfscope}%
\pgfsys@transformshift{0.487595in}{0.414308in}%
\pgfsys@useobject{currentmarker}{}%
\end{pgfscope}%
\end{pgfscope}%
\begin{pgfscope}%
\definecolor{textcolor}{rgb}{0.000000,0.000000,0.000000}%
\pgfsetstrokecolor{textcolor}%
\pgfsetfillcolor{textcolor}%
\pgftext[x=0.320928in,y=0.364166in,left,base]{\color{textcolor}\sffamily\fontsize{10.000000}{12.000000}\selectfont \(\displaystyle 0\)}%
\end{pgfscope}%
\begin{pgfscope}%
\pgfsetbuttcap%
\pgfsetroundjoin%
\definecolor{currentfill}{rgb}{0.000000,0.000000,0.000000}%
\pgfsetfillcolor{currentfill}%
\pgfsetlinewidth{0.803000pt}%
\definecolor{currentstroke}{rgb}{0.000000,0.000000,0.000000}%
\pgfsetstrokecolor{currentstroke}%
\pgfsetdash{}{0pt}%
\pgfsys@defobject{currentmarker}{\pgfqpoint{-0.048611in}{0.000000in}}{\pgfqpoint{0.000000in}{0.000000in}}{%
\pgfpathmoveto{\pgfqpoint{0.000000in}{0.000000in}}%
\pgfpathlineto{\pgfqpoint{-0.048611in}{0.000000in}}%
\pgfusepath{stroke,fill}%
}%
\begin{pgfscope}%
\pgfsys@transformshift{0.487595in}{0.762399in}%
\pgfsys@useobject{currentmarker}{}%
\end{pgfscope}%
\end{pgfscope}%
\begin{pgfscope}%
\definecolor{textcolor}{rgb}{0.000000,0.000000,0.000000}%
\pgfsetstrokecolor{textcolor}%
\pgfsetfillcolor{textcolor}%
\pgftext[x=0.320928in,y=0.712257in,left,base]{\color{textcolor}\sffamily\fontsize{10.000000}{12.000000}\selectfont \(\displaystyle 1\)}%
\end{pgfscope}%
\begin{pgfscope}%
\pgfsetbuttcap%
\pgfsetroundjoin%
\definecolor{currentfill}{rgb}{0.000000,0.000000,0.000000}%
\pgfsetfillcolor{currentfill}%
\pgfsetlinewidth{0.803000pt}%
\definecolor{currentstroke}{rgb}{0.000000,0.000000,0.000000}%
\pgfsetstrokecolor{currentstroke}%
\pgfsetdash{}{0pt}%
\pgfsys@defobject{currentmarker}{\pgfqpoint{-0.048611in}{0.000000in}}{\pgfqpoint{0.000000in}{0.000000in}}{%
\pgfpathmoveto{\pgfqpoint{0.000000in}{0.000000in}}%
\pgfpathlineto{\pgfqpoint{-0.048611in}{0.000000in}}%
\pgfusepath{stroke,fill}%
}%
\begin{pgfscope}%
\pgfsys@transformshift{0.487595in}{1.110491in}%
\pgfsys@useobject{currentmarker}{}%
\end{pgfscope}%
\end{pgfscope}%
\begin{pgfscope}%
\definecolor{textcolor}{rgb}{0.000000,0.000000,0.000000}%
\pgfsetstrokecolor{textcolor}%
\pgfsetfillcolor{textcolor}%
\pgftext[x=0.320928in,y=1.060349in,left,base]{\color{textcolor}\sffamily\fontsize{10.000000}{12.000000}\selectfont \(\displaystyle 2\)}%
\end{pgfscope}%
\begin{pgfscope}%
\pgfsetbuttcap%
\pgfsetroundjoin%
\definecolor{currentfill}{rgb}{0.000000,0.000000,0.000000}%
\pgfsetfillcolor{currentfill}%
\pgfsetlinewidth{0.803000pt}%
\definecolor{currentstroke}{rgb}{0.000000,0.000000,0.000000}%
\pgfsetstrokecolor{currentstroke}%
\pgfsetdash{}{0pt}%
\pgfsys@defobject{currentmarker}{\pgfqpoint{-0.048611in}{0.000000in}}{\pgfqpoint{0.000000in}{0.000000in}}{%
\pgfpathmoveto{\pgfqpoint{0.000000in}{0.000000in}}%
\pgfpathlineto{\pgfqpoint{-0.048611in}{0.000000in}}%
\pgfusepath{stroke,fill}%
}%
\begin{pgfscope}%
\pgfsys@transformshift{0.487595in}{1.458582in}%
\pgfsys@useobject{currentmarker}{}%
\end{pgfscope}%
\end{pgfscope}%
\begin{pgfscope}%
\definecolor{textcolor}{rgb}{0.000000,0.000000,0.000000}%
\pgfsetstrokecolor{textcolor}%
\pgfsetfillcolor{textcolor}%
\pgftext[x=0.320928in,y=1.408440in,left,base]{\color{textcolor}\sffamily\fontsize{10.000000}{12.000000}\selectfont \(\displaystyle 3\)}%
\end{pgfscope}%
\begin{pgfscope}%
\pgfsetbuttcap%
\pgfsetroundjoin%
\definecolor{currentfill}{rgb}{0.000000,0.000000,0.000000}%
\pgfsetfillcolor{currentfill}%
\pgfsetlinewidth{0.803000pt}%
\definecolor{currentstroke}{rgb}{0.000000,0.000000,0.000000}%
\pgfsetstrokecolor{currentstroke}%
\pgfsetdash{}{0pt}%
\pgfsys@defobject{currentmarker}{\pgfqpoint{-0.048611in}{0.000000in}}{\pgfqpoint{0.000000in}{0.000000in}}{%
\pgfpathmoveto{\pgfqpoint{0.000000in}{0.000000in}}%
\pgfpathlineto{\pgfqpoint{-0.048611in}{0.000000in}}%
\pgfusepath{stroke,fill}%
}%
\begin{pgfscope}%
\pgfsys@transformshift{0.487595in}{1.806673in}%
\pgfsys@useobject{currentmarker}{}%
\end{pgfscope}%
\end{pgfscope}%
\begin{pgfscope}%
\definecolor{textcolor}{rgb}{0.000000,0.000000,0.000000}%
\pgfsetstrokecolor{textcolor}%
\pgfsetfillcolor{textcolor}%
\pgftext[x=0.320928in,y=1.756531in,left,base]{\color{textcolor}\sffamily\fontsize{10.000000}{12.000000}\selectfont \(\displaystyle 4\)}%
\end{pgfscope}%
\begin{pgfscope}%
\pgfsetbuttcap%
\pgfsetroundjoin%
\definecolor{currentfill}{rgb}{0.000000,0.000000,0.000000}%
\pgfsetfillcolor{currentfill}%
\pgfsetlinewidth{0.803000pt}%
\definecolor{currentstroke}{rgb}{0.000000,0.000000,0.000000}%
\pgfsetstrokecolor{currentstroke}%
\pgfsetdash{}{0pt}%
\pgfsys@defobject{currentmarker}{\pgfqpoint{-0.048611in}{0.000000in}}{\pgfqpoint{0.000000in}{0.000000in}}{%
\pgfpathmoveto{\pgfqpoint{0.000000in}{0.000000in}}%
\pgfpathlineto{\pgfqpoint{-0.048611in}{0.000000in}}%
\pgfusepath{stroke,fill}%
}%
\begin{pgfscope}%
\pgfsys@transformshift{0.487595in}{2.154764in}%
\pgfsys@useobject{currentmarker}{}%
\end{pgfscope}%
\end{pgfscope}%
\begin{pgfscope}%
\definecolor{textcolor}{rgb}{0.000000,0.000000,0.000000}%
\pgfsetstrokecolor{textcolor}%
\pgfsetfillcolor{textcolor}%
\pgftext[x=0.320928in,y=2.104622in,left,base]{\color{textcolor}\sffamily\fontsize{10.000000}{12.000000}\selectfont \(\displaystyle 5\)}%
\end{pgfscope}%
\begin{pgfscope}%
\pgfpathrectangle{\pgfqpoint{0.487595in}{0.325064in}}{\pgfqpoint{3.023091in}{1.963384in}}%
\pgfusepath{clip}%
\pgfsetrectcap%
\pgfsetroundjoin%
\pgfsetlinewidth{1.505625pt}%
\definecolor{currentstroke}{rgb}{0.121569,0.466667,0.705882}%
\pgfsetstrokecolor{currentstroke}%
\pgfsetdash{}{0pt}%
\pgfpathmoveto{\pgfqpoint{0.625008in}{0.414308in}}%
\pgfpathlineto{\pgfqpoint{1.472321in}{0.415779in}}%
\pgfpathlineto{\pgfqpoint{1.538345in}{0.419171in}}%
\pgfpathlineto{\pgfqpoint{1.582362in}{0.424281in}}%
\pgfpathlineto{\pgfqpoint{1.618125in}{0.431439in}}%
\pgfpathlineto{\pgfqpoint{1.648386in}{0.440629in}}%
\pgfpathlineto{\pgfqpoint{1.675896in}{0.452374in}}%
\pgfpathlineto{\pgfqpoint{1.700655in}{0.466477in}}%
\pgfpathlineto{\pgfqpoint{1.722664in}{0.482454in}}%
\pgfpathlineto{\pgfqpoint{1.744672in}{0.502270in}}%
\pgfpathlineto{\pgfqpoint{1.766680in}{0.526534in}}%
\pgfpathlineto{\pgfqpoint{1.788688in}{0.555875in}}%
\pgfpathlineto{\pgfqpoint{1.810696in}{0.590911in}}%
\pgfpathlineto{\pgfqpoint{1.832704in}{0.632230in}}%
\pgfpathlineto{\pgfqpoint{1.854712in}{0.680353in}}%
\pgfpathlineto{\pgfqpoint{1.876720in}{0.735705in}}%
\pgfpathlineto{\pgfqpoint{1.901479in}{0.806967in}}%
\pgfpathlineto{\pgfqpoint{1.926239in}{0.887882in}}%
\pgfpathlineto{\pgfqpoint{1.953749in}{0.988855in}}%
\pgfpathlineto{\pgfqpoint{1.984010in}{1.112357in}}%
\pgfpathlineto{\pgfqpoint{2.019773in}{1.272245in}}%
\pgfpathlineto{\pgfqpoint{2.074793in}{1.535215in}}%
\pgfpathlineto{\pgfqpoint{2.127063in}{1.781090in}}%
\pgfpathlineto{\pgfqpoint{2.157324in}{1.909852in}}%
\pgfpathlineto{\pgfqpoint{2.182083in}{2.002460in}}%
\pgfpathlineto{\pgfqpoint{2.204091in}{2.072379in}}%
\pgfpathlineto{\pgfqpoint{2.220597in}{2.115878in}}%
\pgfpathlineto{\pgfqpoint{2.237103in}{2.150875in}}%
\pgfpathlineto{\pgfqpoint{2.250858in}{2.173082in}}%
\pgfpathlineto{\pgfqpoint{2.261862in}{2.186074in}}%
\pgfpathlineto{\pgfqpoint{2.272867in}{2.194687in}}%
\pgfpathlineto{\pgfqpoint{2.283871in}{2.198828in}}%
\pgfpathlineto{\pgfqpoint{2.292124in}{2.198963in}}%
\pgfpathlineto{\pgfqpoint{2.300377in}{2.196532in}}%
\pgfpathlineto{\pgfqpoint{2.311381in}{2.189299in}}%
\pgfpathlineto{\pgfqpoint{2.322385in}{2.177521in}}%
\pgfpathlineto{\pgfqpoint{2.333389in}{2.161254in}}%
\pgfpathlineto{\pgfqpoint{2.347144in}{2.134741in}}%
\pgfpathlineto{\pgfqpoint{2.360899in}{2.101592in}}%
\pgfpathlineto{\pgfqpoint{2.377405in}{2.053505in}}%
\pgfpathlineto{\pgfqpoint{2.396662in}{1.986827in}}%
\pgfpathlineto{\pgfqpoint{2.418670in}{1.898303in}}%
\pgfpathlineto{\pgfqpoint{2.443429in}{1.785757in}}%
\pgfpathlineto{\pgfqpoint{2.476442in}{1.620471in}}%
\pgfpathlineto{\pgfqpoint{2.591984in}{1.027519in}}%
\pgfpathlineto{\pgfqpoint{2.622245in}{0.896009in}}%
\pgfpathlineto{\pgfqpoint{2.647005in}{0.801088in}}%
\pgfpathlineto{\pgfqpoint{2.671764in}{0.718450in}}%
\pgfpathlineto{\pgfqpoint{2.693772in}{0.655443in}}%
\pgfpathlineto{\pgfqpoint{2.715780in}{0.602014in}}%
\pgfpathlineto{\pgfqpoint{2.735037in}{0.562700in}}%
\pgfpathlineto{\pgfqpoint{2.754294in}{0.529806in}}%
\pgfpathlineto{\pgfqpoint{2.773551in}{0.502742in}}%
\pgfpathlineto{\pgfqpoint{2.792808in}{0.480856in}}%
\pgfpathlineto{\pgfqpoint{2.812066in}{0.463477in}}%
\pgfpathlineto{\pgfqpoint{2.831323in}{0.449937in}}%
\pgfpathlineto{\pgfqpoint{2.853331in}{0.438345in}}%
\pgfpathlineto{\pgfqpoint{2.878090in}{0.429216in}}%
\pgfpathlineto{\pgfqpoint{2.905600in}{0.422670in}}%
\pgfpathlineto{\pgfqpoint{2.938612in}{0.418189in}}%
\pgfpathlineto{\pgfqpoint{2.982628in}{0.415515in}}%
\pgfpathlineto{\pgfqpoint{3.062408in}{0.414393in}}%
\pgfpathlineto{\pgfqpoint{3.373273in}{0.414308in}}%
\pgfpathlineto{\pgfqpoint{3.373273in}{0.414308in}}%
\pgfusepath{stroke}%
\end{pgfscope}%
\begin{pgfscope}%
\pgfpathrectangle{\pgfqpoint{0.487595in}{0.325064in}}{\pgfqpoint{3.023091in}{1.963384in}}%
\pgfusepath{clip}%
\pgfsetrectcap%
\pgfsetroundjoin%
\pgfsetlinewidth{1.505625pt}%
\definecolor{currentstroke}{rgb}{1.000000,0.498039,0.054902}%
\pgfsetstrokecolor{currentstroke}%
\pgfsetdash{}{0pt}%
\pgfpathmoveto{\pgfqpoint{0.625008in}{0.414308in}}%
\pgfpathlineto{\pgfqpoint{0.911114in}{0.415851in}}%
\pgfpathlineto{\pgfqpoint{0.999147in}{0.419597in}}%
\pgfpathlineto{\pgfqpoint{1.067922in}{0.425566in}}%
\pgfpathlineto{\pgfqpoint{1.125693in}{0.433567in}}%
\pgfpathlineto{\pgfqpoint{1.177963in}{0.443792in}}%
\pgfpathlineto{\pgfqpoint{1.227481in}{0.456585in}}%
\pgfpathlineto{\pgfqpoint{1.274248in}{0.471817in}}%
\pgfpathlineto{\pgfqpoint{1.321015in}{0.490414in}}%
\pgfpathlineto{\pgfqpoint{1.365032in}{0.511210in}}%
\pgfpathlineto{\pgfqpoint{1.409048in}{0.535345in}}%
\pgfpathlineto{\pgfqpoint{1.453064in}{0.562901in}}%
\pgfpathlineto{\pgfqpoint{1.499831in}{0.595934in}}%
\pgfpathlineto{\pgfqpoint{1.546599in}{0.632746in}}%
\pgfpathlineto{\pgfqpoint{1.596117in}{0.675627in}}%
\pgfpathlineto{\pgfqpoint{1.651137in}{0.727541in}}%
\pgfpathlineto{\pgfqpoint{1.711659in}{0.789019in}}%
\pgfpathlineto{\pgfqpoint{1.788688in}{0.871996in}}%
\pgfpathlineto{\pgfqpoint{1.978508in}{1.078738in}}%
\pgfpathlineto{\pgfqpoint{2.033528in}{1.132990in}}%
\pgfpathlineto{\pgfqpoint{2.080295in}{1.174935in}}%
\pgfpathlineto{\pgfqpoint{2.121561in}{1.207957in}}%
\pgfpathlineto{\pgfqpoint{2.160075in}{1.234823in}}%
\pgfpathlineto{\pgfqpoint{2.195838in}{1.255919in}}%
\pgfpathlineto{\pgfqpoint{2.228850in}{1.271785in}}%
\pgfpathlineto{\pgfqpoint{2.259111in}{1.283062in}}%
\pgfpathlineto{\pgfqpoint{2.289373in}{1.291037in}}%
\pgfpathlineto{\pgfqpoint{2.316883in}{1.295294in}}%
\pgfpathlineto{\pgfqpoint{2.344393in}{1.296605in}}%
\pgfpathlineto{\pgfqpoint{2.371903in}{1.294896in}}%
\pgfpathlineto{\pgfqpoint{2.399413in}{1.290114in}}%
\pgfpathlineto{\pgfqpoint{2.426923in}{1.282232in}}%
\pgfpathlineto{\pgfqpoint{2.454434in}{1.271242in}}%
\pgfpathlineto{\pgfqpoint{2.484695in}{1.255588in}}%
\pgfpathlineto{\pgfqpoint{2.514956in}{1.236261in}}%
\pgfpathlineto{\pgfqpoint{2.545217in}{1.213362in}}%
\pgfpathlineto{\pgfqpoint{2.578229in}{1.184475in}}%
\pgfpathlineto{\pgfqpoint{2.613992in}{1.148853in}}%
\pgfpathlineto{\pgfqpoint{2.652507in}{1.105886in}}%
\pgfpathlineto{\pgfqpoint{2.693772in}{1.055193in}}%
\pgfpathlineto{\pgfqpoint{2.743290in}{0.989165in}}%
\pgfpathlineto{\pgfqpoint{2.806563in}{0.899113in}}%
\pgfpathlineto{\pgfqpoint{2.957869in}{0.681307in}}%
\pgfpathlineto{\pgfqpoint{3.004637in}{0.620369in}}%
\pgfpathlineto{\pgfqpoint{3.045902in}{0.571581in}}%
\pgfpathlineto{\pgfqpoint{3.081665in}{0.533920in}}%
\pgfpathlineto{\pgfqpoint{3.114677in}{0.503480in}}%
\pgfpathlineto{\pgfqpoint{3.144938in}{0.479518in}}%
\pgfpathlineto{\pgfqpoint{3.175200in}{0.459493in}}%
\pgfpathlineto{\pgfqpoint{3.202710in}{0.444750in}}%
\pgfpathlineto{\pgfqpoint{3.230220in}{0.433247in}}%
\pgfpathlineto{\pgfqpoint{3.257730in}{0.424826in}}%
\pgfpathlineto{\pgfqpoint{3.287991in}{0.418786in}}%
\pgfpathlineto{\pgfqpoint{3.321003in}{0.415405in}}%
\pgfpathlineto{\pgfqpoint{3.365020in}{0.414313in}}%
\pgfpathlineto{\pgfqpoint{3.373273in}{0.414308in}}%
\pgfpathlineto{\pgfqpoint{3.373273in}{0.414308in}}%
\pgfusepath{stroke}%
\end{pgfscope}%
\begin{pgfscope}%
\pgfsetrectcap%
\pgfsetmiterjoin%
\pgfsetlinewidth{0.803000pt}%
\definecolor{currentstroke}{rgb}{0.000000,0.000000,0.000000}%
\pgfsetstrokecolor{currentstroke}%
\pgfsetdash{}{0pt}%
\pgfpathmoveto{\pgfqpoint{0.487595in}{0.325064in}}%
\pgfpathlineto{\pgfqpoint{0.487595in}{2.288447in}}%
\pgfusepath{stroke}%
\end{pgfscope}%
\begin{pgfscope}%
\pgfsetrectcap%
\pgfsetmiterjoin%
\pgfsetlinewidth{0.803000pt}%
\definecolor{currentstroke}{rgb}{0.000000,0.000000,0.000000}%
\pgfsetstrokecolor{currentstroke}%
\pgfsetdash{}{0pt}%
\pgfpathmoveto{\pgfqpoint{3.510686in}{0.325064in}}%
\pgfpathlineto{\pgfqpoint{3.510686in}{2.288447in}}%
\pgfusepath{stroke}%
\end{pgfscope}%
\begin{pgfscope}%
\pgfsetrectcap%
\pgfsetmiterjoin%
\pgfsetlinewidth{0.803000pt}%
\definecolor{currentstroke}{rgb}{0.000000,0.000000,0.000000}%
\pgfsetstrokecolor{currentstroke}%
\pgfsetdash{}{0pt}%
\pgfpathmoveto{\pgfqpoint{0.487595in}{0.325064in}}%
\pgfpathlineto{\pgfqpoint{3.510686in}{0.325064in}}%
\pgfusepath{stroke}%
\end{pgfscope}%
\begin{pgfscope}%
\pgfsetrectcap%
\pgfsetmiterjoin%
\pgfsetlinewidth{0.803000pt}%
\definecolor{currentstroke}{rgb}{0.000000,0.000000,0.000000}%
\pgfsetstrokecolor{currentstroke}%
\pgfsetdash{}{0pt}%
\pgfpathmoveto{\pgfqpoint{0.487595in}{2.288447in}}%
\pgfpathlineto{\pgfqpoint{3.510686in}{2.288447in}}%
\pgfusepath{stroke}%
\end{pgfscope}%
\begin{pgfscope}%
\definecolor{textcolor}{rgb}{0.000000,0.000000,0.000000}%
\pgfsetstrokecolor{textcolor}%
\pgfsetfillcolor{textcolor}%
\pgftext[x=1.999141in,y=2.371780in,,base]{\color{textcolor}\sffamily\fontsize{12.000000}{14.400000}\selectfont PDF of Beta distribution}%
\end{pgfscope}%
\begin{pgfscope}%
\pgfsetbuttcap%
\pgfsetmiterjoin%
\definecolor{currentfill}{rgb}{1.000000,1.000000,1.000000}%
\pgfsetfillcolor{currentfill}%
\pgfsetfillopacity{0.800000}%
\pgfsetlinewidth{1.003750pt}%
\definecolor{currentstroke}{rgb}{0.800000,0.800000,0.800000}%
\pgfsetstrokecolor{currentstroke}%
\pgfsetstrokeopacity{0.800000}%
\pgfsetdash{}{0pt}%
\pgfpathmoveto{\pgfqpoint{0.584817in}{1.465670in}}%
\pgfpathlineto{\pgfqpoint{1.998659in}{1.465670in}}%
\pgfpathquadraticcurveto{\pgfqpoint{2.026436in}{1.465670in}}{\pgfqpoint{2.026436in}{1.493448in}}%
\pgfpathlineto{\pgfqpoint{2.026436in}{2.191225in}}%
\pgfpathquadraticcurveto{\pgfqpoint{2.026436in}{2.219003in}}{\pgfqpoint{1.998659in}{2.219003in}}%
\pgfpathlineto{\pgfqpoint{0.584817in}{2.219003in}}%
\pgfpathquadraticcurveto{\pgfqpoint{0.557040in}{2.219003in}}{\pgfqpoint{0.557040in}{2.191225in}}%
\pgfpathlineto{\pgfqpoint{0.557040in}{1.493448in}}%
\pgfpathquadraticcurveto{\pgfqpoint{0.557040in}{1.465670in}}{\pgfqpoint{0.584817in}{1.465670in}}%
\pgfpathclose%
\pgfusepath{stroke,fill}%
\end{pgfscope}%
\begin{pgfscope}%
\pgfsetrectcap%
\pgfsetroundjoin%
\pgfsetlinewidth{1.505625pt}%
\definecolor{currentstroke}{rgb}{0.121569,0.466667,0.705882}%
\pgfsetstrokecolor{currentstroke}%
\pgfsetdash{}{0pt}%
\pgfpathmoveto{\pgfqpoint{0.612595in}{2.018722in}}%
\pgfpathlineto{\pgfqpoint{0.890373in}{2.018722in}}%
\pgfusepath{stroke}%
\end{pgfscope}%
\begin{pgfscope}%
\definecolor{textcolor}{rgb}{0.000000,0.000000,0.000000}%
\pgfsetstrokecolor{textcolor}%
\pgfsetfillcolor{textcolor}%
\pgftext[x=1.001484in,y=2.063163in,left,base]{\color{textcolor}\sffamily\fontsize{10.000000}{12.000000}\selectfont Men }%
\end{pgfscope}%
\begin{pgfscope}%
\definecolor{textcolor}{rgb}{0.000000,0.000000,0.000000}%
\pgfsetstrokecolor{textcolor}%
\pgfsetfillcolor{textcolor}%
\pgftext[x=1.001484in,y=1.911772in,left,base]{\color{textcolor}\sffamily\fontsize{10.000000}{12.000000}\selectfont (\(\displaystyle \mu = \)0.6, \(\displaystyle n = \)40)}%
\end{pgfscope}%
\begin{pgfscope}%
\pgfsetrectcap%
\pgfsetroundjoin%
\pgfsetlinewidth{1.505625pt}%
\definecolor{currentstroke}{rgb}{1.000000,0.498039,0.054902}%
\pgfsetstrokecolor{currentstroke}%
\pgfsetdash{}{0pt}%
\pgfpathmoveto{\pgfqpoint{0.612595in}{1.662889in}}%
\pgfpathlineto{\pgfqpoint{0.890373in}{1.662889in}}%
\pgfusepath{stroke}%
\end{pgfscope}%
\begin{pgfscope}%
\definecolor{textcolor}{rgb}{0.000000,0.000000,0.000000}%
\pgfsetstrokecolor{textcolor}%
\pgfsetfillcolor{textcolor}%
\pgftext[x=1.001484in,y=1.707330in,left,base]{\color{textcolor}\sffamily\fontsize{10.000000}{12.000000}\selectfont Women }%
\end{pgfscope}%
\begin{pgfscope}%
\definecolor{textcolor}{rgb}{0.000000,0.000000,0.000000}%
\pgfsetstrokecolor{textcolor}%
\pgfsetfillcolor{textcolor}%
\pgftext[x=1.001484in,y=1.555939in,left,base]{\color{textcolor}\sffamily\fontsize{10.000000}{12.000000}\selectfont (\(\displaystyle \mu = \)0.6, \(\displaystyle n = \)10)}%
\end{pgfscope}%
\end{pgfpicture}%
\makeatother%
\endgroup%

\end{center}
\caption{Example of Beta distributions}
\label{beta_distribution}
\end{figure}
Figure \ref{beta_distribution} provides a numerical example for how these assumptions affect the priors of men and women. 

% https://www.users.csbsju.edu/~mgass/robert.pdf 
% pg. 82 (70)





\end{blist}

%%%%%%%%%%%%%%%%%%%%%%%%%%%%%%%%%%%%%%%%%%%%%%%%%%%%%%%%%%%%%%%%%%%%%%%%%%%%%%%%
\subsubsection{Results}

\begin{outline}

%%%%%%%%%%%%%%%%%%%%%
% Most tractable case

\hitem{Assume $h_{j0} = \nu \alpha_{j0}$}

This is a strong assumption, and one that I'm not particularly fond of; it implies that $h_{j0}^w < h_{j0}^m$, meaning initial human capital starts out lower for women. 
However, it provides a simplification that highlights useful features of the model.\nts{\footnote{\nts{
  This might not be that bad. Human capital $h$ is how much the market values your success in courses. 
  If the market values women less, they might have more to prove. 
  That view ignores the genuine returns to education, so it would be tricky to make that argument.
}}} 

\begin{blist}

\item First note that, according to \eqref{bb-mj}, $c_j^{m*} \leq c_j^{w*}$. 
Intuitively, women who become $j$-specialists study more than men. 
The key takeaway is that women have more uncertainty regarding their underlying abilities, and therefore need to spend more time studying to build up certainty before entering the labor market in this field.\footnote{
   It is worth noting that this result is somewhat mechanical; assuming that $h_{j0} = \nu \alpha_{j0}$ implies that women begin with lower levels of human capital.
   Greater uncertainty and lower human capital go hand in hand. 
   Nevertheless, this provides a useful starting point.

}   
% This is probably a no, but could this be helpful for understanding convergence in education outcomes? 
% What's interesting for this case is the dynamics in the economy as a whole for this case. 
% Do we often see overeducated women? \nts{Could this feed into maternity outcomes? }


\item The skill-$j$ graduation region in \eqref{grad} can now be written as:
\begin{equation*}
	\mathcal{G}_j^g = \left\{ \alpha_{jt}^g, \beta_{jt}^g \left\vert c_{jt} \geq c_j^* \right. \right\}.
\end{equation*}
Because $c_j^{m*} \leq c_j^{w*}$, there are values of $(h_{jt}, c_{jt})$ such that men will not continue on, but women will. 

%\item Finally, I believe that, given $(h_{jt}, c_{jt})$, the continuation payoff of staying in $j$ is larger for men than for women. \nts{I do want to check my math on this. }

%A man and woman with the same $(h_{jt}, c_{jt})$ got there because the woman succeeded more and/or the man failed more. 

% Looking at python this seems wrong. Oops!


\end{blist}

%%%%%%%%%%%%%%%%%%%%%
% More reasonable case

\hitem{Assume $h_{j0} \leq \nu \alpha_{j0}$}

The stopping condition \eqref{stop} can be written using the alternative parameterization: 
\begin{equation*}
	\frac{1- \delta}{\delta} \geq 
	\frac{\nu n_{j0} \mu_{j0} + h_{jt} - h_{j0}}{h_{jt}(c_{jt} + n_{j0})} 
\end{equation*}
I am interested in cases where a women might quit field $j$ while a man might continue on. 
The primary difference between men and women in this model is that men have a larger $n_{j0}^g$. 
Therefore, I examine how this condition changes for men and women with the same values of $(h_{j0}, h_{jt}, c_{jt})$.

\begin{blist}

\item Consider the case where $c_{jt}^g \mu_{j0} < \frac{h_{jt}^g - h_{j0}^g}{\nu}$.
The LHS of this inequality is the number of successes this student expected to have. 
The RHS of this inequality are your realized human capital gains.
Intuitively, this condition means that the amount of time you expected to succeed in this many courses is less than the amount of time you actually succeeded. 

\item The RHS of the stopping condition is decreasing in $n_{j0}^g$ if $c_{jt} \mu_{j0} < \frac{h_{jt} - h_{j0}}{\nu}$, and therefore is smaller for men than for women. 
This means that there exist men and women with the same $(h_{j0}, h_{jt}, c_{jt})$ for whom the smaller sample size $n_{j0}^w$ will cause women to leave the field, whereas men will continue on. 

\end{blist}

%Intuitively, this means that your expected success rate conditional only on means is less than your observed success rate. Intuitively, marginal women who have been unlucky will exit, whereas men will continue on. 

% Next: consider what this means for continuation values, if anything
% Then: what does it mean in the context of affirmative action papers (maybe start with this first)

\end{outline}


% Next: what does this imply if you fail and you're a woman? What about if you fail and you're a man? 

% Show that after failure, women might quit whereas men continue on

\nts{

%%%%%%%%%%%%%%%%%%%%%%%%%%
\subsubsection*{Aggregate}

\begin{blist}

\item What happens in aggregate? This is what I could use some help on. 

\item Intuition: fewer women select into certain fields 

\end{blist}

%%%%%%%%%%%%%%%%%%%%%%%%%%%%%%%%%%%%%%%%%%%%%%%%%%%%%%%%%%%%%%%%%%%%%%%%%%%%%%%%
\section{Statistical discrimination}
%%%%%%%%%%%%%%%%%%%%%%%%%%%%%%%%%%%%%%%%%%%%%%%%%%%%%%%%%%%%%%%%%%%%%%%%%%%%%%%%

\begin{blist}

\item Discrimination: women with identical abilities are treated less favorably than men in certain fields. 

\item If fewer women select into field $j$ then men, then there could be a difference in means

\end{blist}

%%%%%%%%%%%%%%%%%%%%%%%%%%%%%%%%%%%%%%%%%%%%%%%%%%%%%%%%%%%%%%%%%%%%%%%%%%%%%%%%
\section{Affirmative Action}
%%%%%%%%%%%%%%%%%%%%%%%%%%%%%%%%%%%%%%%%%%%%%%%%%%%%%%%%%%%%%%%%%%%%%%%%%%%%%%%%


%%%%%%%%%%%%%%%%%%%%%%%%%%%%%%%%%%%%%%%%%%%%%%%%%%%%%%%%%%%%%%%%%%%%%%%%%%%%%%%%
\section{Key: how is prior formed}
%%%%%%%%%%%%%%%%%%%%%%%%%%%%%%%%%%%%%%%%%%%%%%%%%%%%%%%%%%%%%%%%%%%%%%%%%%%%%%%%

\begin{blist}
% Endogenously determined prior

\item How prior formed is key, and consistent with a number of stories regarding discrimination

\item One possible story: role model story. then this is consistent with some convergence results between men and women.

\item However, doesn't explain everything. What if the prior is because of other factors?

\end{blist}

% human capital differentiates meaningful returns to labor. 
% if you're uncertain about whether your efforts in school will meaningfully impact your wages, then you may not want to continue on a given path.

% bayes rule requires us to specify liklihood function. This is something i think we can connct to Loury


% Rose talked about Texas A&M data on grades 




} % end nts


\nobib{\printbibliography}

\end{document}
