\documentclass[10 pt]{article}

% Format
\usepackage[T1]{fontenc}
\usepackage[utf8]{inputenc}
\usepackage[margin=1in]{geometry} % 1-inch margins
\usepackage[english]{babel} % English hyphenation, etc.	
\usepackage{setspace} % Set spacing 
%\usepackage{enumerate} % Use different types of enumerate options
\usepackage{enumitem}
\usepackage{csquotes} % Block quotes
\usepackage[dvipsnames]{xcolor} % colors: https://en.wikibooks.org/wiki/LaTeX/Colors#The_68_standard_colors_known_to_dvips
% Math
\usepackage{amsmath, mathrsfs, amsfonts, amssymb, amsthm}
% Figures
\usepackage{graphicx} % Include figures
\usepackage{float} % Improved control over floats
\usepackage{tikz} % Draw figures with tikz

% Colors
\definecolor{Indigo}{HTML}{3C6478}
\definecolor{DarkBrown}{HTML}{45281B}
\definecolor{Brown}{HTML}{161402}
\definecolor{DarkGreen}{HTML}{325101}
\definecolor{LeafGreen}{HTML}{4A6F01}
\definecolor{DarkAlice}{HTML}{107896}
\definecolor{Alice}{HTML}{1496BB}
\definecolor{DarkGray}{RGB}{116 118 120}
\definecolor{DarkBlue}{HTML}{022C36}
\definecolor{MainBlue}{HTML}{132881}
\definecolor{Maroon}{HTML}{6A123D}
\definecolor{DarkPurple}{HTML}{2C033A}
\definecolor{Orange}{HTML}{F18312}

% Hyperlinks
\usepackage{hyperref} % Include hyperlinks
\hypersetup{
  colorlinks   = true, %Colours links instead of ugly boxes
  urlcolor     = Maroon, %Colour for external hyperlinks
  linkcolor    = DarkGray, %Colour of internal links
  citecolor   = MainBlue %Colour of citations
}


% Macro Shortcuts
\newcommand{\R}{\mathbb{R}}
\newcommand{\Q}{\mathbb{Q}}
\newcommand{\Z}{\mathbb{Z}}
\newcommand{\N}{\mathbb{N}}
\newcommand{\EE}{\mathbb{E}}
\newcommand{\PP}{\mathbb{P}}
\newcommand{\BB}{\mathscr{B}}
\newcommand{\e}{\text{e}}
\newcommand{\dd}{\text{d}}


% Theorems
\newtheorem{prop}{Proposition}[section]
\newtheorem{thm}{Theorem}
\theoremstyle{remark}
\newtheorem{claim}{Claim}[section]
\newtheorem{remark}{Remark}
\theoremstyle{definition}
\newtheorem{defn}{Definition}[section]
\newtheorem{lemma}{Lemma}
\newtheorem{ass}{Assumption}


\newif\ifnts
%\ntstrue % uncomment to show 
% Notes to self 
\ifnts
  \newcommand{\nts}[1]{{\color{gray}#1}}
\else
  \newcommand{\nts}[1]{}
\fi

%%%%%%%%%%
% Sections that have:
%   (A) Roman numerals 
%   (B) fixed width = fixw
%   (C) coloring

% (A) Roman numeral for section and subsection
\renewcommand{\thesection}{\Roman{section}} 
\renewcommand{\thesubsection}{\roman{subsection}}

% (B) Each section has fixed width = fixw 

% Define fixw
\newcommand{\fixw}{28pt}
\newcommand{\fixwh}{14pt}

% (C) Define colors
\newcommand{\secc}[1]{{\color{DarkGreen}#1}} % section color
\newcommand{\sectc}{DarkGreen} % section text color
\newcommand{\subsecc}[1]{{\color{LeafGreen}#1}} % subsection color
\newcommand{\subsectc}{LeafGreen} % subsection text color
\newcommand{\numc}{DarkAlice}

% Set each section width and color
\usepackage{titlesec}
\titleformat{\section}{\normalfont\Large\bfseries\color{\sectc}}
	{\makebox[\fixw][l]{\secc{\thesection.}}}{0pt}{} 
\titleformat{\subsection}{\normalfont\large\bfseries\color{\subsectc}}
	{\makebox[{\fixw}][l]{\subsecc{\thesubsection.}}}{0pt}{} 
\titleformat{\subsubsection}{\normalfont\bfseries}
	{}{0pt}{} %{\makebox[{\fixw}][l]{}}{0pt}{} 

% Highlight certain items
\newcommand{\hitem}[2][DarkAlice]{\color{#1} \item #2 \color{black}}

%%%%%%%%%%
% Lists that start at fixw (see section above)
\newlist{outline}{enumerate}{2}
\setlist[outline,1]{label=\arabic*.,left=0pt .. \fixw}
\setlist[outline,2]{label=\alph*.,left=0pt .. \fixw}

\newlist{blist}{itemize}{2}
\setlist[blist,1]{label=\textbullet,left=0pt .. \fixw}
\setlist[blist,2]{label=\textendash,left=0pt .. \fixw}

%%%%%%%%%%
% Enumerate in footnote
\newlist{footcount}{enumerate}{1}
\setlist[footcount]{label=(\alph*),left=0pt .. \fixw}

%%%%%%%%%%
% No indent in footnotes
\usepackage[flushmargin,hang]{footmisc}

%\usepackage[marginal]{footmisc}
%\setlength\footnotemargin{5pt}  % default value: 1.8em


% Bibliography
\usepackage[authordate,backend=biber]{biblatex-chicago}
\addbibresource{../bibliography.bib}

% remove space 
\usepackage[font=small,skip=0pt]{caption}

% To find size of textwidth, part 1. 
%\usepackage{layouts}
\begin{document}
% To find size of textwidth, part 2 (6.50127in)
%\printinunitsof{in}\prntlen{\textwidth}



\title{Discrimination literature}
\author{Tara Sullivan}

\maketitle
\onehalfspacing

\noindent\nts{Please note that gray text are notes/comments}

%%%%%%%%%%%%%%%%%%%%%%%%%%%%%%%%%%%%%%%%%%%%%%%%%%%%%%%%%%%%%%%%%%%%%%%%%%%%%%%%
\section{Terminology}
%%%%%%%%%%%%%%%%%%%%%%%%%%%%%%%%%%%%%%%%%%%%%%%%%%%%%%%%%%%%%%%%%%%%%%%%%%%%%%%%

\begin{blist}

\item \textcite{LL12}: An equilibrium result is discriminatory if, on average, it leaves black people worse off than white people. 

\end{blist}



%%%%%%%%%%%%%%%%%%%%%%%%%%%%%%%%%%%%%%%%%%%%%%%%%%%%%%%%%%%%%%%%%%%%%%%%%%%%%%%%
\section{Models of discrimination}
%%%%%%%%%%%%%%%%%%%%%%%%%%%%%%%%%%%%%%%%%%%%%%%%%%%%%%%%%%%%%%%%%%%%%%%%%%%%%%%%

There are primarily two models of economic discrimination: taste-based and statistical discrimination

\begin{outline}

\hitem{Taste-based discrimination}

In his 1957 book \emph{The Economics of Discrimination}, Becker proposes a model of labor market discrimination whereby prejudiced employers receive disutility from hiring employees belonging to a particular group \parencite{B57}.
Specifically, he assumes white employers experience heterogeneous levels of disutility from hiring black employees. 
This disutility, or discrimination coefficient, causes discriminating employers to act as if wages of black employees are higher than those of their white counterparts.
The equilibrium wage gap between black and white workers is ultimately determined by discrimination coefficient of the marginal discriminating employer.
As a result, black workers sort into less-prejudiced workplaces.\footnote{
For additional information, see \textcite{CG08}, who additionally find empirical evidence for the existence of taste-based discrimination in the U.S. labor market.
} 

A key implication of the Becker model is that discriminatory firms will be less profitable than firms that do not discriminate. 
Thus, long-run neoclassical analysis suggests that discrimination will be driven out of the market place, a point emphasized by \textcite{A72}.
A number of authors have incorporated search frictions into the Becker model to explain the long-run persistence of racial wage gaps.\footnote{
    See \textcite{LL12} for a review of the literature.
    It is worth noting that racial wage gaps cannot persist in a taste-based model with firm entry; discriminatory firms will always be less profitable. \nts{(this is worth checking at some point)}
}

\hitem{Statistical discrimination}

The theory of statistical discrimination grew out of the Arrow critique \nts{(needs cite)} and \textcite{P72}.
The classical analysis in \textcite{AC77} assumes true productivity is normally 
% As self-fulfilling prophecy: Lundberg and Startz (1983), Coate and Loury (1993)

% Linear projection from AC77 uses information to form signal. Is this consistent with bayesian analysis? Or no?
% Altonji and Pierret have dynamic model. Is this consistent with bayesian updating? 

\end{outline}

Alt: %%%%%%%%%%%%%%%%%%%%%%%%%%%%%%%%%%%%%%%%%%%%%%%%%%%%%%%%%%%%%%%%%%%%%%%%%%%%%%%%
\subsubsection*{Statistical discrimination literature}

\toedit{First, note that there are two models of economic discrimination: taste-based and statistical discrimination.}
The canonical model of taste-based discrimination in labor markets, as formulated by \textcite{B57}, assumes prejudiced employers receive disutility from hiring employees belonging to a particular group.\footnote{
    For a review of the Becker model and details on testable implications, see \textcite{CG08}, who find empirical evidence for the existence of taste-based discrimination in the U.S. labor market.
} 
A key implication of the Becker model is that discriminatory firms will be less profitable than firms that do not discriminate. 
Thus, long-run neoclassical analysis suggests that discrimination will be driven out of the market place, an implication that appears incongruous with the persistence of unexplained wage differentials between black and white workers.

The theory of statistical discrimination, as first formulated by \textcite{A72} and \textcite{P72}, grew out of this critique.
Statistical discrimination in labor markets considers that employers possess imperfect information about the productivity of potential employees. 
The classical analysis formulated in \textcite{AC77} assumes job applicant's group type is one variable and employer uses for inference. 
\toedit{Note that much of the canonical discrimination literature primarily focuses on labor market discrimination, whereby employers discriminate offer lower wages to employees of a particular group.}
However, all of these papers note that discrimination theories apply in alternative contexts.
\toedit{Elaborate on these examples.}


%%%%%%%%%%%%%%%%%%%%%%%%%%%%%%%%%%%%%%%%%%%%%%%%%%%%%%%%%%%%%%%%%%%%%%%%%%%%%%%%
\section{Empirical evidence of discrimination}
%%%%%%%%%%%%%%%%%%%%%%%%%%%%%%%%%%%%%%%%%%%%%%%%%%%%%%%%%%%%%%%%%%%%%%%%%%%%%%%%

\begin{blist}

\item \textcite{LL12} reviews empirical and theoretical evidence of racial discrimination in labor market. 

\end{blist}




%%%%%%%%%%%%%%%%%%%%%%%%%%%%%%%%%%%%%%%%%%%%%%%%%%%%%%%%%%%%%%%%%%%%%%%%%%%%%%%%
\section{Miscellaneous}
%%%%%%%%%%%%%%%%%%%%%%%%%%%%%%%%%%%%%%%%%%%%%%%%%%%%%%%%%%%%%%%%%%%%%%%%%%%%%%%%

% Write down math for AC77; connection to Bayesian updating?
% Review LS83 (maybe)
% Review Lang and Manove
% See if my model can balance between LS73 and LM11

\begin{blist}

\item \textcite{LM11} note that black students get more education than white students. 

\item \textcite{L98} discusses the limitations of using market factors alone to explain racial inequality. 
He emphasizes that supply side differences between black and white employees explain a significant portion of the wage gap.
He urges economists to examine sources of the skill disparities, emphasizing that inequality in skill acquisition is still a moral problem.
Mentions that in earlier work he has shown that human capital theory provides a richer context for analyzing group inequality

% I agree with him that human capital theory provides a strong foundation for analyzing inequality. 
% Social networks can be part of this story. 
% But this doesn't happen in a vacuum. 

% Look at Steele 1992

\item Self-fulfilling prophecy: \textcite{LS83}, \textcite{CL93}

\item From \textcite{LL12}: taste-based models do not allow for within-race heterogeneity and wage differentials at different skill levels; statistical discrimination does not address employment gaps. (1) how come childbirth cannot address stat. disc. employment gap? (2) can my model allow for employment gaps? 

\item Is there any paper that tries to find evidence for whether we have switched discrimination types over time? Taste-based in the mid-twentieth century, moving towards statistical in the 21st? 


\end{blist}

\printbibliography

\end{document}
