\documentclass[10 pt]{article}

% Format
\usepackage[T1]{fontenc}
\usepackage[utf8]{inputenc}
\usepackage[margin=1in]{geometry} % 1-inch margins
\usepackage[english]{babel} % English hyphenation, etc.	
\usepackage{setspace} % Set spacing 
%\usepackage{enumerate} % Use different types of enumerate options
\usepackage{enumitem}
\usepackage{csquotes} % Block quotes
\usepackage[dvipsnames]{xcolor} % colors: https://en.wikibooks.org/wiki/LaTeX/Colors#The_68_standard_colors_known_to_dvips
% Math
\usepackage{amsmath, mathrsfs, amsfonts, amssymb, amsthm}
% Figures
\usepackage{graphicx} % Include figures
\usepackage{float} % Improved control over floats
\usepackage{tikz} % Draw figures with tikz

% Colors
\definecolor{Indigo}{HTML}{3C6478}
\definecolor{DarkBrown}{HTML}{45281B}
\definecolor{Brown}{HTML}{161402}
\definecolor{DarkGreen}{HTML}{325101}
\definecolor{LeafGreen}{HTML}{4A6F01}
\definecolor{DarkAlice}{HTML}{107896}
\definecolor{Alice}{HTML}{1496BB}
\definecolor{DarkGray}{RGB}{116 118 120}
\definecolor{DarkBlue}{HTML}{022C36}
\definecolor{MainBlue}{HTML}{132881}
\definecolor{Maroon}{HTML}{6A123D}
\definecolor{DarkPurple}{HTML}{2C033A}
\definecolor{Orange}{HTML}{F18312}

% Hyperlinks
\usepackage{hyperref} % Include hyperlinks
\hypersetup{
  colorlinks   = true, %Colours links instead of ugly boxes
  urlcolor     = Maroon, %Colour for external hyperlinks
  linkcolor    = DarkGray, %Colour of internal links
  citecolor   = MainBlue %Colour of citations
}


% Macro Shortcuts
\newcommand{\R}{\mathbb{R}}
\newcommand{\Q}{\mathbb{Q}}
\newcommand{\Z}{\mathbb{Z}}
\newcommand{\N}{\mathbb{N}}
\newcommand{\EE}{\mathbb{E}}
\newcommand{\PP}{\mathbb{P}}
\newcommand{\BB}{\mathscr{B}}
\newcommand{\e}{\text{e}}
\newcommand{\dd}{\text{d}}


% Theorems
\newtheorem{prop}{Proposition}[section]
\newtheorem{thm}{Theorem}
\theoremstyle{remark}
\newtheorem{claim}{Claim}[section]
\newtheorem{remark}{Remark}
\theoremstyle{definition}
\newtheorem{defn}{Definition}[section]
\newtheorem{lemma}{Lemma}
\newtheorem{ass}{Assumption}


\newif\ifnts
% \ntstrue % uncomment to show 
% Notes to self 
\ifnts
  \newcommand{\nts}[1]{{\color{gray}#1}}
\else
  \newcommand{\nts}[1]{}
\fi

%%%%%%%%%%
% Sections that have:
%   (A) Roman numerals 
%   (B) fixed width = fixw
%   (C) coloring

% (A) Roman numeral for section and subsection
\renewcommand{\thesection}{\Roman{section}} 
\renewcommand{\thesubsection}{\roman{subsection}}

% (B) Each section has fixed width = fixw 

% Define fixw
\newcommand{\fixw}{28pt}
\newcommand{\fixwh}{14pt}

% (C) Define colors
\newcommand{\secc}[1]{{\color{DarkGreen}#1}} % section color
\newcommand{\sectc}{DarkGreen} % section text color
\newcommand{\subsecc}[1]{{\color{LeafGreen}#1}} % subsection color
\newcommand{\subsectc}{LeafGreen} % subsection text color
\newcommand{\numc}{DarkAlice}

% Set each section width and color
\usepackage{titlesec}
\titleformat{\section}{\normalfont\Large\bfseries\color{\sectc}}
	{\makebox[\fixw][l]{\secc{\thesection.}}}{0pt}{} 
\titleformat{\subsection}{\normalfont\large\bfseries\color{\subsectc}}
	{\makebox[{\fixw}][l]{\subsecc{\thesubsection.}}}{0pt}{} 
\titleformat{\subsubsection}{\normalfont\bfseries}
	{}{0pt}{} %{\makebox[{\fixw}][l]{}}{0pt}{} 

% Highlight certain items
\newcommand{\hitem}[2][DarkAlice]{\color{#1} \item #2 \color{black}}

%%%%%%%%%%
% Lists that start at fixw (see section above)
\newlist{outline}{enumerate}{2}
\setlist[outline,1]{label=\arabic*.,left=0pt .. \fixw}
\setlist[outline,2]{label=\alph*.,left=0pt .. \fixw}

\newlist{blist}{itemize}{2}
\setlist[blist,1]{label=\textbullet,left=0pt .. \fixw}
\setlist[blist,2]{label=\textendash,left=0pt .. \fixw}

%%%%%%%%%%
% Enumerate in footnote
\newlist{footcount}{enumerate}{1}
\setlist[footcount]{label=(\alph*),left=0pt .. \fixw}

%%%%%%%%%%
% No indent in footnotes
\usepackage[flushmargin,hang]{footmisc}

%\usepackage[marginal]{footmisc}
%\setlength\footnotemargin{5pt}  % default value: 1.8em


% Bibliography
\usepackage[authordate,backend=biber]{biblatex-chicago}
\addbibresource{../bibliography.bib}

% remove space 
% \usepackage[font=small,skip=0pt]{caption}

% To edit
\newif\iftoedit
\toedittrue % uncomment to show
\iftoedit 
  \newcommand{\toedit}[1]{{\color{gray}#1}}
\else
  \newcommand{\toedit}[1]{#1}
\fi

% Notes to self - footnotes
\newif\iffootnts
\ifnts
    \footntstrue % uncomment to show
\else
\fi
\iffootnts
  \newcommand{\footnts}[1]{\nts{\footnote{\nts{#1}}}}
\else
  \newcommand{\footnts}[1]{}
\fi


% % Nest this in \nts
% \ifnts
%     \footntstrue % uncomment to show
% \else
% \fi

\newcommand{\citeposs}[1]{{\citeauthor{#1}'s (\citeyear{#1})}}

\newcommand{\br}[1]{\left\{ #1 \right\}}
\newcommand{\sbr}[1]{\left[ #1 \right]}
\newcommand{\pr}[1]{\left( #1 \right)}
\newcommand{\ce}[2]{\left[\left. #1 \right\vert #2 \right]}

% plot graphs with pgfplots
\usepackage{pgfplots}
\pgfplotsset{compat=newest}
\usepgfplotslibrary{groupplots}
\usepgfplotslibrary{dateplot}
\pgfplotsset{compat=newest,
    every axis/.style={
        axis y line*=left,
        axis x line*=bottom,
        % allows for multi-line legend entries
        legend style={cells={align=left}},
        % allows for multi-line titles
        title style={align=center},
    },
}

% Seeing if I can include subcaptiongs in groupplots
\usepackage{subcaption}

% path to tex images
\makeatletter
\def\input@path{{../../img/}}
\makeatother

% Externalize pgf plots
% Note: will need to do something like: https://tex.stackexchange.com/questions/40652/references-in-externalized-pgfplots
\usetikzlibrary{external}
\tikzexternalize[prefix=figures/]

% Find text width; comment out when not using to avoid warning
% \usepackage{layouts}

% Variables used in model that i might need to change
\newcommand{\study}{m} % I think i could change this to r if the fact that I use m as an index (in gender) is a problem
\newcommand{\pass}{s}

\usepackage{subcaption}

\begin{document}

% Find text width
% textwidth in cm: \printinunitsof{cm}\prntlen{\textwidth}. 
% textheight in cm: \printinunitsof{cm}\prntlen{\textheight}
% outcome: 6.50127in, 16.50746cm, 469.75502pt

\title{Group-based beliefs and human capital specialization}
\author{Tara Sullivan%\footnote{
    %Thanks to Remy Levin, Daniela Vidart
}
%}

\maketitle
\onehalfspacing

\noindent\nts{Please note that gray text are notes/comments. }\toedit{Please note that gray text are placeholders that need to be edited or checked.}

\begin{abstract}
% The gender gap in postsecondary degree attainment has disappeared in the US over the past forty years; in fact, the overall gap has reversed, as women now earn more Bachelor's degrees than men, and at an increasing rate.  
Although the overall gender gap in postsecondary degree attainment has reversed over the past forty years, significant heterogeneity persists in terms of which fields men and women choose to study.
In this paper, I consider the role of group-based beliefs in explaining differential convergence rates for groups across fields. 
I assume a student forms their initial belief about their probability of success in a particular field based on past outcomes for their group type. 
I then incorporate group-based beliefs into the model of gradual human capital specialization from \textcite{AF20} to show how these differences in priors can drive human capital specialization decisions amongst otherwise similar agents. 
%I plan to calibrate this model to match parameters of the US postsecondary education system in order to separately identify the impact of expected lifetime wages and underlying beliefs on specialization decisions.

% skill heterogeneity
% sequential learning



\end{abstract}

%%%%%%%%%%%%%%%%%%%%%%%%%%%%%%%%%%%%%%%%%%%%%%%%%%%%%%%%%%%%%%%%%%%%%%%%%%%%%%%%
\section{Introduction}
\tikzset{external/figure name={intro_}}
%%%%%%%%%%%%%%%%%%%%%%%%%%%%%%%%%%%%%%%%%%%%%%%%%%%%%%%%%%%%%%%%%%%%%%%%%%%%%%%%

\begin{figure}[b]
\centering
% This file was created by tikzplotlib v0.9.1.
\begin{tikzpicture}

\definecolor{color0}{rgb}{0.12156862745098,0.466666666666667,0.705882352941177}
\definecolor{color1}{rgb}{1,0.498039215686275,0.0549019607843137}

\begin{axis}[
legend cell align={left},
legend style={fill opacity=0.8, draw opacity=1, text opacity=1, at={(0.03,0.97)}, anchor=north west, draw=white!80!black},
tick align=outside,
tick pos=left,
title={Number of Bachelors Degrees awarded (millions)},
x grid style={white!69.0196078431373!black},
xlabel={year},
xmin=1988.6, xmax=2019.4,
xtick style={color=black},
y grid style={white!69.0196078431373!black},
ymin=0.4493241, ymax=1.2484059,
ytick style={color=black}
]
\addplot [semithick, color0]
table {%
1990 0.48564600944519
1991 0.490826010704041
1992 0.517989993095398
1993 0.532243013381958
1994 0.532928943634033
1995 0.528069019317627
1997 0.518990993499756
1998 0.522558927536011
1999 0.522891998291016
2000 0.533735036849976
2001 0.557978987693787
2002 0.579033017158508
2003 0.599171996116638
2004 0.629392027854919
2005 0.649704933166504
2006 0.66592800617218
2007 0.687217950820923
2008 0.703808069229126
2009 0.722702980041504
2010 0.750731945037842
2011 0.779560089111328
2012 0.814333915710449
2013 0.836575031280518
2014 0.850880980491638
2015 0.862040996551514
2016 0.871549010276794
2017 0.886856079101562
2018 0.897544026374817
};
\addlegendentry{Men}
\addplot [semithick, color1]
table {%
1990 0.555091023445129
1991 0.581097006797791
1992 0.614879012107849
1993 0.632375001907349
1994 0.638270020484924
1995 0.636955976486206
1996 0.644475936889648
1997 0.652374982833862
1998 0.666815042495728
1999 0.684229016304016
2000 0.708883047103882
2001 0.745826005935669
2002 0.779849052429199
2003 0.805441975593567
2004 0.849164962768555
2005 0.87644100189209
2006 0.905745983123779
2007 0.927672982215881
2008 0.947183012962341
2009 0.968661069869995
2010 1.00603902339935
2011 1.04807901382446
2012 1.09642803668976
2013 1.12388396263123
2015 1.15306401252747
2016 1.17018795013428
2017 1.19238698482513
2018 1.2120840549469
};
\addlegendentry{Women}
\end{axis}

\end{tikzpicture}

\label{fig:n_degrees}
\end{figure}
\begin{figure}[t]
\centering
\begin{tikzpicture}[align=left,
every node/.style={font=\footnotesize}]
% This file was created by tikzplotlib v0.9.2.
\definecolor{color0}{rgb}{0.266666666666667,0.466666666666667,0.666666666666667}
\definecolor{color1}{rgb}{0.933333333333333,0.4,0.466666666666667}
\definecolor{color2}{rgb}{0.133333333333333,0.533333333333333,0.2}
\definecolor{color3}{rgb}{0.8,0.733333333333333,0.266666666666667}
\definecolor{color4}{rgb}{0.4,0.8,0.933333333333333}
\definecolor{color5}{rgb}{0.666666666666667,0.2,0.466666666666667}

\begin{groupplot}[group style={group size=2 by 1, group name=my plots, vertical sep=2cm, horizontal sep=1.2cm}]
\nextgroupplot[
height=6.376357092455836cm,
tick align=outside,
tick pos=left,
width=9.079103cm,
x grid style={white!69.0196078431373!black},
xmin=1988.6, xmax=2030,
xtick style={color=black},
xtick={1990,1995,2000,2005,2010,2015},
xticklabels={\(\displaystyle 1990\),\(\displaystyle 1995\),\(\displaystyle 2000\),\(\displaystyle 2005\),\(\displaystyle 2010\),\(\displaystyle 2015\)},
ymajorgrids,
ymin=0, ymax=2,
ytick style={color=black},
ytick={0,0.2,0.4,0.6,0.8,1,1.2,1.4,1.6,1.8,2},
yticklabels={\(\displaystyle 0\),\(\displaystyle 0.2\),\(\displaystyle 0.4\),\(\displaystyle 0.6\),\(\displaystyle 0.8\),\(\displaystyle 1\),\(\displaystyle 1.2\),\(\displaystyle 1.4\),\(\displaystyle 1.6\),\(\displaystyle 1.8\),\(\displaystyle 2\)}
]
\addplot [semithick, color0]
table {%
1990 0.889276146888733
1991 0.908068180084229
1992 0.908786058425903
1993 0.911298394203186
1995 0.937413215637207
1996 0.960581421852112
1997 0.962796211242676
1998 0.959450006484985
1999 0.985843420028687
2000 1.00777983665466
2001 1.00261080265045
2002 1.01744496822357
2003 1.02670252323151
2004 1.02088141441345
2005 1.00308656692505
2006 0.996885895729065
2007 0.972628951072693
2008 0.96298623085022
2009 0.959978342056274
2010 0.95364236831665
2011 0.953920364379883
2012 0.931203603744507
2013 0.921860694885254
2014 0.899493217468262
2015 0.900677680969238
2016 0.891088247299194
2017 0.886778354644775
2018 0.88598895072937
};
\addplot [semithick, color1]
table {%
1990 1.22759211063385
1991 1.28919923305511
1992 1.32112109661102
1993 1.3564704656601
1994 1.39542412757874
1995 1.44826233386993
1996 1.49454152584076
1997 1.56208670139313
1998 1.59724080562592
1999 1.65807044506073
2000 1.72764885425568
2001 1.76988768577576
2002 1.76754701137543
2003 1.7330185174942
2004 1.70102310180664
2005 1.68849968910217
2006 1.65692889690399
2007 1.65950012207031
2008 1.62763059139252
2009 1.64037382602692
2010 1.63280272483826
2011 1.62376940250397
2012 1.64566314220428
2013 1.6688095331192
2014 1.67092096805573
2015 1.67829787731171
2016 1.73332345485687
2017 1.76386177539825
2018 1.79705047607422
};
\addplot [semithick, color2]
table {%
1990 0.187528491020203
1991 0.187889814376831
1992 0.18504810333252
1993 0.191165328025818
1994 0.198027372360229
1995 0.209525942802429
1996 0.219993829727173
1997 0.226323843002319
1998 0.230563402175903
1999 0.248593807220459
2000 0.259858369827271
2001 0.253314614295959
2002 0.266838073730469
2003 0.248422026634216
2004 0.258007287979126
2005 0.250089168548584
2006 0.243573904037476
2007 0.228035092353821
2008 0.227158188819885
2009 0.221451997756958
2011 0.231951236724854
2012 0.238436937332153
2013 0.240317106246948
2014 0.248236060142517
2015 0.251599311828613
2016 0.265729546546936
2017 0.274784326553345
2018 0.286244869232178
};
\addplot [semithick, color3]
table {%
1990 1.04078304767609
1991 1.04256737232208
1992 1.07366561889648
1993 1.06894600391388
1994 1.05757308006287
1995 1.10440194606781
1996 1.11996006965637
1997 1.17428719997406
1998 1.23312425613403
1999 1.30454993247986
2000 1.40134906768799
2001 1.46844744682312
2002 1.54577028751373
2003 1.63139522075653
2004 1.64863216876984
2005 1.63168549537659
2006 1.602987408638
2007 1.51491057872772
2008 1.47168242931366
2009 1.46346211433411
2010 1.41208970546722
2011 1.44205784797668
2012 1.42999362945557
2013 1.42106962203979
2014 1.41722071170807
2015 1.44253933429718
2016 1.49894797801971
2017 1.56678104400635
2018 1.64683747291565
};
\addplot [semithick, color4]
table {%
1990 0.429780006408691
1991 0.419558525085449
1993 0.395862579345703
1994 0.4017493724823
1995 0.399999976158142
1996 0.381898045539856
1997 0.373727560043335
1998 0.368845701217651
1999 0.373373627662659
2000 0.390813827514648
2002 0.384417414665222
2003 0.37157928943634
2004 0.335911631584167
2005 0.288139462471008
2006 0.261456251144409
2007 0.22909951210022
2008 0.216255187988281
2009 0.219069957733154
2010 0.223915338516235
2011 0.216599345207214
2012 0.225852251052856
2013 0.219969153404236
2014 0.223478555679321
2015 0.222921013832092
2016 0.235600471496582
2017 0.242461562156677
2018 0.257417798042297
};
\addplot [semithick, color5]
table {%
1990 0.460548996925354
1991 0.464974999427795
1992 0.4886314868927
1993 0.488829135894775
1994 0.512030601501465
1995 0.540966749191284
1996 0.565948724746704
1997 0.602615833282471
1998 0.629949331283569
1999 0.666009306907654
2000 0.686820149421692
2001 0.706353425979614
2002 0.736422896385193
2003 0.7142094373703
2004 0.72872519493103
2005 0.739134073257446
2006 0.725082635879517
2007 0.690871238708496
2008 0.687974691390991
2009 0.686181664466858
2010 0.688009262084961
2011 0.668954968452454
2012 0.67028284072876
2013 0.634251594543457
2014 0.645684003829956
2015 0.62436580657959
2016 0.629948854446411
2017 0.653665065765381
2018 0.67027759552002
};
\addplot [semithick, white!73.3333333333333!black]
table {%
1990 0.86025857925415
1991 0.899104356765747
1992 0.886856079101562
1993 0.900158047676086
1994 0.865336656570435
1995 0.88508152961731
1996 0.85411524772644
1997 0.864051580429077
1998 0.88809061050415
1999 0.93377161026001
2000 0.914776563644409
2001 0.881730198860168
2003 0.793915510177612
2004 0.801473140716553
2005 0.769950985908508
2006 0.776896238327026
2007 0.747289896011353
2008 0.764146089553833
2009 0.725896239280701
2010 0.73568868637085
2011 0.726636171340942
2012 0.726435899734497
2013 0.727647542953491
2014 0.723636150360107
2015 0.720515251159668
2016 0.709809184074402
2017 0.696071982383728
2018 0.709341764450073
};
\addplot [semithick, black]
table {%
1990 1.14299511909485
1991 1.18391644954681
1992 1.18704795837402
1993 1.18813216686249
1994 1.19766426086426
1995 1.2061984539032
1996 1.23111665248871
1997 1.25700640678406
1998 1.2760568857193
1999 1.30854749679565
2000 1.32815539836884
2001 1.33665609359741
2002 1.34681272506714
2003 1.34425842761993
2004 1.34918296337128
2005 1.3489830493927
2006 1.36012601852417
2007 1.34989619255066
2008 1.34579741954803
2009 1.34033071994781
2010 1.34007740020752
2011 1.34444940090179
2012 1.34641063213348
2013 1.34343481063843
2014 1.33796620368958
2015 1.33759760856628
2016 1.34265315532684
2017 1.34451031684875
2018 1.35044527053833
};
\draw (axis cs:2018.5,0.835989010989011) node[
  anchor=base west,
  text=color0,
  rotate=0.0
]{Business};
\draw (axis cs:2018.5,1.74705042050622) node[
  anchor=base west,
  text=color1,
  rotate=0.0
]{Social \\ Sciences};
\draw (axis cs:2018.5,0.286244813278008) node[
  anchor=base west,
  text=color2,
  rotate=0.0
]{Engineering};
\draw (axis cs:2018.5,1.44683752645192) node[
  anchor=base west,
  text=color3,
  rotate=0.0
]{Biological \\ Sciences};
\draw (axis cs:2018.5,0.037417838961352) node[
  anchor=base west,
  text=color4,
  rotate=0.0
]{Computer \\ Services};
\draw (axis cs:2018.5,0.44027762382224) node[
  anchor=base west,
  text=color5,
  rotate=0.0
]{Physical \\ Sciences};
\draw (axis cs:2018.5,0.709341764874964) node[
  anchor=base west,
  text=white!73.3333333333333!black,
  rotate=0.0
]{Math};
\draw (axis cs:2018.5,1.30044521494211) node[
  anchor=base west,
  text=black,
  rotate=0.0
]{All Fields};

\nextgroupplot[
height=6.376357092455836cm,
tick align=outside,
tick pos=left,
width=9.079103cm,
x grid style={white!69.0196078431373!black},
xmin=1988.6, xmax=2030,
xtick style={color=black},
xtick={1990,1995,2000,2005,2010,2015},
xticklabels={\(\displaystyle 1990\),\(\displaystyle 1995\),\(\displaystyle 2000\),\(\displaystyle 2005\),\(\displaystyle 2010\),\(\displaystyle 2015\)},
ymajorgrids,
ymin=0, ymax=4,
ytick style={color=black},
ytick={0,0.5,1,1.5,2,2.5,3,3.5,4},
yticklabels={\(\displaystyle 0\),\(\displaystyle 0.5\),\(\displaystyle 1\),\(\displaystyle 1.5\),\(\displaystyle 2\),\(\displaystyle 2.5\),\(\displaystyle 3\),\(\displaystyle 3.5\),\(\displaystyle 4\)}
]
\addplot [semithick, color0]
table {%
1990 2.5175244808197
1991 2.6520824432373
1992 2.74022936820984
1993 2.7347571849823
1994 2.72362613677979
1995 2.69927859306335
1996 2.70083546638489
1997 2.83463954925537
1998 2.9113347530365
1999 3.01813840866089
2000 3.2591769695282
2001 3.43097972869873
2002 3.43798732757568
2003 3.46344590187073
2004 3.48158049583435
2005 3.48040509223938
2006 3.40909743309021
2007 3.41750979423523
2008 3.35590291023254
2009 3.38099837303162
2010 3.34576916694641
2011 3.32759737968445
2012 3.26378393173218
2013 3.256920337677
2014 3.28145527839661
2015 3.37682676315308
2016 3.45335793495178
2017 3.56493353843689
2018 3.73112607002258
};
\addplot [semithick, color1]
table {%
1990 0.453454494476318
1991 0.434512615203857
1992 0.427694320678711
1993 0.426904916763306
1994 0.419453144073486
1995 0.444806575775146
1996 0.434596300125122
1998 0.464142203330994
1999 0.469876527786255
2000 0.497728705406189
2001 0.528073787689209
2002 0.521972298622131
2003 0.524367094039917
2005 0.490026116371155
2006 0.463130831718445
2007 0.454207539558411
2008 0.453919529914856
2009 0.445038676261902
2011 0.456213235855103
2012 0.440835237503052
2013 0.455652236938477
2014 0.464651584625244
2015 0.468673825263977
2016 0.484934091567993
2017 0.485498905181885
2018 0.491626977920532
};
\addplot [semithick, color2]
table {%
1990 0.687340497970581
1991 0.718239784240723
1992 0.715183019638062
1993 0.720339775085449
1994 0.741496920585632
1995 0.745949029922485
1996 0.771579146385193
1997 0.803107261657715
1998 0.797616720199585
1999 0.866689205169678
2000 0.902498960494995
2001 0.944846391677856
2002 0.949172735214233
2003 0.955212831497192
2004 0.910499453544617
2005 0.908062338829041
2006 0.878212213516235
2007 0.87516725063324
2008 0.845878720283508
2009 0.856622457504272
2010 0.84217095375061
2011 0.810709953308105
2012 0.804367065429688
2013 0.806700944900513
2014 0.798325777053833
2015 0.794228196144104
2016 0.862247109413147
2017 0.911926984786987
2018 0.932909369468689
};
\addplot [semithick, color3]
table {%
1990 2.15509796142578
1991 2.24903225898743
1993 2.16480875015259
1994 2.14018177986145
1995 2.08914875984192
1996 2.10084676742554
1997 2.15451312065125
1998 2.20654559135437
1999 2.31257438659668
2000 2.35462379455566
2001 2.42530727386475
2002 2.48414301872253
2003 2.48934626579285
2004 2.51065826416016
2005 2.44160652160645
2006 2.38261532783508
2007 2.40375828742981
2008 2.33004307746887
2009 2.34272146224976
2010 2.29821467399597
2011 2.32439589500427
2012 2.28757739067078
2013 2.31100392341614
2014 2.24083662033081
2015 2.25853753089905
2016 2.39009857177734
2017 2.41193509101868
2018 2.55055665969849
};
\addplot [semithick, color4]
table {%
1990 1.24622249603271
1991 1.27433001995087
1992 1.28972804546356
1993 1.33748888969421
1994 1.38074350357056
1995 1.43718707561493
1996 1.46975755691528
1997 1.49069452285767
1998 1.63201975822449
1999 1.6412159204483
2000 1.61032199859619
2001 1.68787050247192
2002 1.71428573131561
2004 1.5903924703598
2005 1.63254117965698
2006 1.61800968647003
2007 1.58040750026703
2008 1.52280271053314
2009 1.59942007064819
2010 1.7337794303894
2011 1.75879776477814
2012 1.74759149551392
2013 1.64136123657227
2014 1.60894978046417
2015 1.63310813903809
2016 1.62103271484375
2017 1.69435846805573
2018 1.7236739397049
};
\addplot [semithick, color5]
table {%
1990 1.45081532001495
1991 1.41702950000763
1992 1.35909819602966
1993 1.33774352073669
1994 1.30062794685364
1995 1.31255269050598
1996 1.29666662216187
1997 1.37724554538727
1998 1.40262174606323
1999 1.53457581996918
2000 1.67128205299377
2001 1.6860568523407
2002 1.77517664432526
2003 1.69076919555664
2004 1.69307291507721
2005 1.60994327068329
2006 1.57205748558044
2007 1.58068251609802
2008 1.61680722236633
2009 1.64477932453156
2010 1.62862813472748
2011 1.61050426959991
2012 1.59707045555115
2013 1.523766040802
2014 1.50999295711517
2015 1.5084331035614
2016 1.52219450473785
2017 1.56681549549103
2018 1.53559231758118
};
\addplot [semithick, white!73.3333333333333!black]
table {%
1990 1.74866712093353
1991 1.83746552467346
1992 2.02597403526306
1993 1.80351257324219
1994 1.73820173740387
1995 1.7401157617569
1996 1.74474608898163
1997 1.77926981449127
1998 1.85183620452881
1999 1.8871773481369
2000 2.04300141334534
2001 2.15603137016296
2002 2.18847274780273
2003 2.25258803367615
2004 2.24808692932129
2005 2.30172061920166
2006 2.25455236434937
2007 2.30674362182617
2008 2.26183843612671
2009 2.39688038825989
2010 2.37516474723816
2011 2.36272668838501
2012 2.4907488822937
2013 2.49604296684265
2014 2.55269312858582
2015 2.64027833938599
2016 2.73464822769165
2017 2.68607902526855
2018 2.76014471054077
};
\draw (axis cs:2018.5,3.73112597886414) node[
  anchor=base west,
  text=color0,
  rotate=0.0
]{Psychology};
\draw (axis cs:2018.5,0.491626926915826) node[
  anchor=base west,
  text=color1,
  rotate=0.0
]{Economics};
\draw (axis cs:2018.5,0.932909364532259) node[
  anchor=base west,
  text=color2,
  rotate=0.0
]{Political science};
\draw (axis cs:2018.5,2.55055658627087) node[
  anchor=base west,
  text=color3,
  rotate=0.0
]{Sociology};
\draw (axis cs:2018.5,1.72367399741268) node[
  anchor=base west,
  text=color4,
  rotate=0.0
]{Other};
\draw (axis cs:2018.5,1.53559232991857) node[
  anchor=base west,
  text=color5,
  rotate=0.0
]{Int'l relations};
\draw (axis cs:2018.5,2.76014463640016) node[
  anchor=base west,
  text=white!73.3333333333333!black,
  rotate=0.0
]{Anthropology};
\end{groupplot}



\node [text width=8.25373cm, align=center, anchor=south] at (my plots c1r1.north) {\subcaption{\label{fig:ipeds_a} Ratio of women to men}};
\node [text width=8.25373cm, align=center, anchor=south] at (my plots c2r1.north) {\subcaption{\label{fig:ipeds_b} Ratio of women to men - Social Sciences}};

\end{tikzpicture}

\caption{Ratio of women to men completing Bachelor's degrees in U.S. 4-year colleges. Source: IPEDS.}

\end{figure}

The overall gender gap in postsecondary degree attainment has reversed in the US over the past fifty years.
As seen in figure \ref{fig:n_degrees}, women now earn more Bachelor's degrees than men, and at an increasing rate.
%\footnote{\toedit{This phenomenon is also documented in \textcite{BHM10}.}} 
This convergence in human capital across genders has reduced the gender wage gap \parencite{BK17}, and may have even increased aggregate economic productivity \parencite{HHJK19}. 
% \toedit{Moreover, improved allocation of talent increases overall economic productivity \parencite{HHJK19}.}
% However, gender gaps in paritcular fields of study occasionally if we consider specific fields of study, gender convergence becomes much more unclear. 
However, this pattern is not uniformly observed across fields of study.
Consider figure \ref{fig:ipeds_a}, which plots the ratio of women to men completing Bachelor's degrees \toedit{in historically male-dominated subjects}.\footnts{
    I should be referencing the HEGIS data here. These are all historically male dominated fields. Proving that has been a pain in the ass, though.
    Include overall ratio in black on the graphs. 
    Caption: Ratio of women to men completing bachelor's degrees by field of study. Source: IPEDS.
}
While some fields have increased their gender ratios since 1990, others have remained flat or worsened. 
More generally, aggregations of college majors can easily mask underlying heterogeneity in gender convergence \parencite{BHST08}.
This can be seen by comparing the Social Sciences gender ratio in Figure \ref{fig:ipeds_a} with those of its subfields in \ref{fig:ipeds_b}.
Overall, these differences in major choice appear to matter for labor market outcomes \parencite{SHB19}, but the reasons these differences exist is not well understood. 

This paper addresses this heterogeneity using a model of group-based belief formation and gradual human capital specialization. 
Building on \textcite{AF20}, I assume individuals belonging to a particular group choose to work or study in heterogeneous fields. 
Returns to education are stochastic, and underlying abilities are unknown.
Agents form beliefs about their unknown abilities based on existing group outcomes. 
% I assume that individuals have particular group types, and form beliefs based on existing outcomes for their type.
I use simulations to highlight the different mechanisms of this model, and to specifically show how beliefs can impact decision making.\footnts{
    Good to mention here the types of checks I want to do. For example:
    allowing abilities to be correlated. 
} 

This paper proceeds as follows. After a brief literature review, I outline the model in section \ref{sec:model}. Implications of the model are explored in section \ref{sec:sims}. 

%%%%%%%%%%%%%%%%%%%%%%%%%%%%%%%%%%%%%%%%%%%%%%%%%%%%%%%%%%%%%%%%%%%%%%%%%%%%%%%%
\subsubsection*{Literature}

This paper builds on recent literature in human capital specialization.
In particular, I closely draw on the theoretical work in \textcite{AF20}.
Of particular interest are studies about college major choice \parencite{ABM12,AAM16-education}, and analyses about the role of beliefs in human capital specialization decisions \parencite{AHMR-wp}.
This paper shares several theoretical commonalities with \textcite{AAMR16-wp}, who build a dynamic model of school and work decisions, though they focus on attrition; as such, major choice is broadly characterized as a choice between STEM fields and non-STEM fields. 
% They also assume students are uncertain about underlying abilities and update these beliefs over time.

% This paper related to empirical literature on gender gaps and college choice.
This paper is largely motivated by empirical work on the relationship between gender and college major choice, and how that relationship has changed over time.
Differences in major choice by gender has been documented using both administrative data \parencite{D10} and survey results \parencite{Z13}.
It is worth noting that there has been significant improvement in gender ratios across fields compared to the mid-twentieth century.
However, most of this gender convergence ended by the 1980s, well before parity was achieved \parencite{SHB19,EL06}. 
% The gender convergence in college degrees is a well-studied phenomenon \parencite{BK17}.
% \textcite{BHST08} use survey data to show that college major choice is closely related to wages broadly and the gender wage gap specifically.
% The most useful papers suggest a reason for why gender gaps exist.

%\textcite{SHB19} provides a key empirical motivation for this paper.
%They use American Community Survey data to explore the importance of major choice in explaining labor market outcomes.

The literature suggests several reasons for gender differences in college major choice.
A number of studies estimate the impact of preferences \parencite{Z13,WZ14}, including preferences over lifetime temporal flexibility \parencite{B15,WZ18}.
% Factors such as temporal flexibility may be key in explaining the lifetime gender wage gap \parencite{G14,KLS19}.
Though my model can accommodate field-specific preferences, my paper primarily focuses on the role of beliefs and belief formation.
This approach is consistent with a number of determinants of gender major choice, including the presence of same-gender role models \parencite{PS20,LM20}, and the role of negative feedback \parencite{KTU17}.
% Importantly, wages do not appear to be a driving factor; as noted in \textcite{SHB19}, women often sort into lower paying majors, in addition to sorting into lower paying occupations conditional on their major choice.

% \nts{They also don't focus on how groups influence belief formation, I don't think}

% \textcite{PS20} find experimental evidence that the existence of same-gender role models are an important determinant of major choice.

% It is worth highlighting that my research does not discount the impact of children on the gender wage gap.
% \textcite{KLS19} provide strong evidence that children have a significant impact on lifetime earnings for women.

Finally, the results of this paper will be closely tied to statistical discrimination literature. 
The theory of statistical discrimination, as first formulated by \textcite{A72} and \textcite{P72}, assumes discrimination arises out of imperfect information.
For example, \textcite{AC77} assume that employer's use job applicant's race as one variable when inferring potential productivity.
% However, all of these papers note that discrimination theories apply in alternative contexts.
A key concern associated with statistical discrimination in labor markets are self-fulfilling prophecies \textcite{LS83}. 
\toedit{As discussed in section TBD, this model in this paper is closely tied to the theory of statistical discrimination}.

\nts{Question: is statistical discrimination just Bayesian linear regression?} \nts{How does this fit in with the \textcite{L98} idea that }


%%%%%%%%%%%%%%%%%%%%%%%%%%%%%%%%%%%%%%%%%%%%%%%%%%%%%%%%%%%%%%%%%%%%%%%%%%%%%%%%
\section{Model of human capital specialization}\label{sec:model}
\tikzset{external/figure name={model_}}
%%%%%%%%%%%%%%%%%%%%%%%%%%%%%%%%%%%%%%%%%%%%%%%%%%%%%%%%%%%%%%%%%%%%%%%%%%%%%%%%

This section outlines my theory of group-based beliefs and human capital specialization.
% In this model, infinitely-lived agents with unknown, heterogeneous abilities choose between working and studying in each discrete time period. 
% Individuals belong to different groups, and form beliefs about their own abilities based on previously observed group outcomes.
% Individuals who study accumulate both human capital \toedit{(stochastically)} and information about their underlying abilities.
Please note that the agent's specialization problem and subsequent decision rule follows \citeposs{AF20} model of gradual human capital specialization; as such, I refer to the reader to their original paper for details. 

%%%%%%%%%%%%%%%%%%%%%%%%%%%%%%%%%%%%%%%%%%%%%%%%%%%%%%%%%%%%%%%%%%%%%%%%%%%%%%%%
\subsection{Specialization decision}

Assume infinitely lived agents in discrete time choose to work or study among $J$ fields.
This means that at each time period $t$, individuals decide to either study a field $j$ at a postsecondary institution, or to work in a field $j$, where $j\in \{1, \dots, J\}$.
The variable $\study_{jt}$ is equal to one if an agent matriculates and studies field $j$ at time $t$; otherwise, $m_{jt}$ equals zero. 
Likewise, $\ell_{jt}$ indicates whether an agent works in field $j$ at time $t$.
An individual therefore faces the following time constraint in each period $t$:
\begin{equation*}
    \sum_{j=1}^J \pr{\study_{jt} + \ell_{jt}} = 1.
\end{equation*}


Individuals aim to maximize their expected lifetime utility.
%\footnote{\toedit{More general utility forms are possible.}}
Let $U_j$ denote the expected per-period utility associated with working in field $j$, where $U_j$ is a bounded function that is non-decreasing in field-specific wages and in field-specific human capital ($w_j$ and $h_{jt}$, respectively).\footnts{
    As noted in \textcite{AF20}, this can be expected average per-period payoff, and can evolve after graduation.} 
The agent's expected lifetime payoff can be written as: 
\begin{equation}\label{max_utility}
    \sum_{t=0}^\infty \delta^t \sum_{j=1}^J U_{j} (w_j, h_{j, t}) \ell_{j, t},
    % \max_{j \in {1, \dots, J}}
    % \sum_{t=0}^\infty \delta^t U_j (w_j, h_{jt}) \ell_{jt}
\end{equation}
where $\delta \in (0, 1)$ is the discount rate. \nts{(\textcite{AF20} have $\delta \in (0, 1)$? Is it a problem if I have $[0, 1]$?)}

% Intuitively, this implies that if a student takes a course in field $j$, they only accumulate human capital if they pass the course.

Agents are initially endowed with some level of field-$j$ human capital, $h_{j0}$.
They can stochastically accumulate more field-$j$ human capital by studying that field.
Recall that if $\study_{jt}=1$, an agent matriculates in time period $t$ to take a course in field $j$.
Let $\pass_{jt}$ indicate whether an agent studying $j$ at time $t$ succeeds and passes that course; if the student succeeds, then $\pass_{jt} = 1$, otherwise $\pass_{jt} = 0$.
I assume that whether a student passes or fails a particular field-$j$ course is stochastic. 
Specifically, each student is endowed with some immutable probability of success in field-$j$ courses, denoted $\theta_j$.
A student then passes any field-$j$ course with probability $\theta_j$, implying that $\pass_{jt}$ is a Bernoulli random variable with parameter $\theta_j$.\footnote{
    The variable $s_{jt}$ is assumed to be independent and stationary over time. 
    \nts{Need to check this; I used to say the probability of success or failure is independent and stationary throughout time.}
    \nts{I do need to discuss what happens when the probability of success is correlated across fields. Also need to consider if these initial values are correlated across groups.}
}
Agents only accumulates human capital when they pass courses.
Therefore, an agent's human capital evolves according to:
\begin{equation}\label{eq:hc_accumulation}
    h_{j,t+1} = h_{jt} + \nu_j \pass_{jt} \study_{jt},
    \quad \quad
    s_{jt} \sim \text{Bernoulli} (\theta_j),
\end{equation}
where $\nu_j \geq 0$ is the human capital gain associated with passing the course.\footnote{
    The per-period expected accumulation of human capital additionally must be non-negative and bounded. Regularity conditions imposed in section \ref{sec:optimal_policy} ensure that is the case.
    \nts{May want to mention that this can be considered to be the Mincerian returns for a particular course.}
}

A student's probability of success in a field $j$ course, $\theta_j$, is an ability parameter; students with high values of $\theta_j$ are more likely to pass any given class in field $j$, whereas students with low values of $\theta_j$ are more likely to fail. 
\nts{This ties in with the statistics discrimination literature in some way.}
Students do not know their personal value of $\theta_j$, but they have beliefs about what their value of $\theta_j$ might be. 
Their beliefs about their ability is initially given by the distribution $P_{j0}$. 
As students take courses in field-$j$, they update their belief about what their value of $\theta_j$ may be, according to some updating rule $\Pi_j$:\footnts{
    What additional assumptions do I need on the updating rule? What rules do I need on the decision rule to ensure that my results hold?
}
\begin{equation*}
    P_{j,t+1} = \Pi_j (P_{jt}, \pass_{jt})
\end{equation*}

% \item 
\toedit{It is clear from \eqref{max_utility} that agents will only want to work in the field that yields the highest expected lifetime utility.}
The choice of which field to work in is an individual's \emph{specialization decision}.\footnts{(though I'm not totally sure where in the model it becomes clear that agents will only work in one field.)}
Agents that plan to specialize in field $j$ will study $j$ to accumulate $j$-specific human capital, and will endogenously enter the labor market as a field-$j$ specialist.\footnts{
    Technically, this depends on the parameterization of the problem, but will be true for the examples explored in this paper. (Something I should be specific about)
}
The decision of when to stop studying $j$ and enter the labor market is an agent's \emph{stopping decision}. 
\toedit{Overall, students have two incentives for studying a field $j$.}
First, they can potentially accumulate $j$-specific human capital, which directly increases lifetime utility if they become a field-$j$ specialist.
Second, studying $j$ reveals information about their ability in that field, which is central to the specialization decision.
\nts{
%\item 
Then the stochasticity in the specialization decision comes from expected human capital accumulation; thus, how agents update their beliefs and form their expectation is key to the specialization decision. Though this might not be fully correct, as it also comes from field choice and stopping time.
}

% \end{blist} 


%%%%%%%%%%%%%%%%%%%%%%%%%%%%%%%%%%%%%%%%%%%%%%%%%%%%%%%%%%%%%%%%%%%%%%%%%%%%%%%%
\subsection{Group-based beliefs}\label{sec:group_based_beliefs}

Assume each student has a group type, $g$. 
To simplify the exposition, consider two groups, men and women ($g \in \{m, f\}$).\footnts{Possible extension: countably infinite types. What does this mean if you value diversity? Could you get results like diversity helps students who are not in the targeted group to assist?}
The distributions of underlying abilities, $\theta_j$, are the same for men and women.\nts{\footnote{\nts{
    I think I also need to say that distribution of underlying abilities is conditionally independent of group type, right? 
  What is the maintained hypothesis in a Bayesian problem?
  Am I assuming that $\theta_j$ is distributed 
}}}
However, initial beliefs about underlying abilities, $P_{j0}^g$, are different for the two groups.

\toedit{To make this explicit, let's fully parameterize the belief distribution for a type $g$ student.}
Assume the initial initial beliefs about $\theta_j$ follow a beta distribution with parameters $(\alpha_{j0}^g, \beta_{j0}^g)$, implying $P_{j0}^g = \mathcal{B} (\alpha_{j0}^g, \beta_{j0}^g)$.
To understand why this is a reasonable assumption, recall that our unknown ability parameter, $\theta_j$, is the probability that a student succeeds ($\pass_{jt}^g = 1$) or fails ($\pass_{jt}^g = 0$) a field $j$ course at time $t$.
Because the realizations of $\pass_{jt}$ are independent over time, the sequence of successes and failures for some number of total field $j$ courses taken is a binomial random variable.
The beta distribution is a conjugate prior for the binomial distribution, and is thus a natural and tractable choice for modeling beliefs about $\theta_j$ \parencite[pg. 325]{CB02}.\footnts{
    Also see \emph{Doing Bayesian Analysis} Chapter 6, as I learned from Wikipedia: \url{https://en.wikipedia.org/wiki/Beta_distribution\#Alternative_parametrizations}.
    For a description of why the Beta distribution can be intuitively understood as a probabilistic distribution of probabilities, see \url{https://stats.stackexchange.com/q/47782}.
    Also, might want to highlight that the realizations of $s_{jt}$ are independent and therefore 
}
Therefore, if we assume that a student updates their beliefs about $\theta_j$ using Bayes' rule, their posterior is also a Beta distribution: 
\begin{equation*}
    P_{j,t+1}^g = \mathcal{B} (\alpha_{j,t+1}^g, \beta_{j,t+1}^g), \quad \quad 
    (\alpha_{j,t+1}, \beta_{j,t+1}) = 
    \begin{cases} 
        (\alpha_{jt}^g + 1, \beta_{jt}^g) &\text{ if } \pass_{jt}^g = 1 \\
        (\alpha_{jt}^g, \beta_{jt}^g + 1) &\text{ if } \pass_{jt}^g = 0
    \end{cases}
\end{equation*}

% The key assumption of this analysis is that students may use existing group outcomes to form initial beliefs about their probability of success.
% As an example, suppose a type $g$ student's belief about their own ability is based on previously observed success rates for their group, $g$. 

\toedit{To develop intuition about how beliefs based on existing group outcomes influence specialization decisions, it's helpful to proceed with an illustrative, albeit somewhat contrived, parameterization.}
Let $\alpha_{j0}^g$ and $\beta_{j0}^g$ denote the number of type $g$ students who have succeeded and failed in field $j$ at time 0, respectively.
As an example, suppose a type $g$ student is forming their initial beliefs about their probability of success in field $j$, and therefore asks five type $g$ upperclassmen about their experiences in field $j$.
If three of those type $g$ upperclassmen passed the introductory course in field $j$, while two failed, then the student's initial belief parameters $(\alpha_{j0}^g, \beta_{j0}^g)$ would equal $(3, 2)$.


Using this parameterization, the observed group-$g$ success rate, $\mu_{j0}^g$, is given by:
\begin{equation*}
\mu_{j0}^g = 
  \frac{\alpha_{j0}^g}{\alpha_{j0}^g + \beta_{j0}^g}.
  %\mu_{j0} = 
  %\frac{\alpha_{j0}^m}{\alpha_{j0}^m + \beta_{j0}^m} = 
  %\frac{\alpha_{j0}^f}{\alpha_{j0}^f + \beta_{j0}^f}
\end{equation*}
This average is based on a sample of type $g$ students of size $n_{j0}^g = \alpha_{j0}^g + \beta_{j0}^g$.
Note that the beta distribution parameters, $\alpha_{j0}^g$ and $\beta_{j0}^g$, can be expressed using the average success rate, $\mu_{j0}^g$, and the sample size, $n_{j0}^g$:
\begin{equation*}
     \alpha_{j0}^g = \mu_{j0}^g n_{j0}^g,
     \quad \quad \beta_{j0}^g = (1 - \mu_{j0}^g) n_{j0}^g.
 \end{equation*}
 Therefore, an alternative parameterization of prior is given by $\mathcal{B} \pr{\alpha_{j0}^g, \beta_{j0}^g} = \mathcal{B} \pr{\mu_{j0}^g n_{j0}^g, (1 - \mu_{j0}^g) n_{j0}^g}$.
 
Assume the sample size of men is larger than that of women, but that the observed success rate is the same for the two genders:
\begin{equation*}
  n_{j0}^m > n_{j0}^f, \quad \quad \mu_{j0} = \mu_{j0}^m = \mu_{j0}^w.
  %n_{j0}^m = \alpha_{j0}^m + \beta_{j0}^m > \alpha_{j0}^f + \beta_{j0}^f = n_{j0}^w
\end{equation*}
%Assuming the prior on $\theta_j$ follows a $\mathcal{B} \pr{\alpha_{j0}^g, \beta_{j0}^g}$ distribution, this yields the following parameterization:
\begin{figure}
\centering
\input{beta_example_gender.tex}
\caption{PDF of Beta distribution}
\label{beta_distribution}
\end{figure}
Figure \ref{beta_distribution} provides a numerical example to illustrate how these assumptions affect the priors of men and women. 
Although women and men have the same probability of success in expectation, women have more initial uncertainty regarding their underlying abilities. 
\nts{Elaborate on what this means from the perspective of what beta distributions do.}

In section \ref{sec:sims}, I discuss the implications of this assumption.
Before moving forward, it is helpful to highlight some shortcomings of the above illustrative example.
First, I am being purposefully vague about what ``success'' means when agents form their initial priors, $\alpha_{j0}^g$ and $\beta_{j0}^g$.
The example above, in which students solicit feedback from upperclassmen, is helpful for building intuition, and is consistent with the literature highlighting the importance of role-models in specialization decisions; see \textcite{PS20} for directly relevant empirical evidence.
However, success could mean many things in this model. 
It could be the number of type $g$ students who graduate into field $j$, the number of type $g$ professors, the number of students attaining graduate degrees in field $j$, etc. 
I will discuss this matter further when calibrating the model.\footnts{Make sure I do this!}
\toedit{Additionally, while it is illustrative to use the parameters $\alpha_{j0}^g$ and $\beta_{j0}^g$ to tally the total number of type $g$ successes and failures, it is by no means necessary.}

% need to highlight after going through example that:

%%%%%%%%%%%%%%%%%%%%%%%%%%%%%%%%%%%%%%%%%%%%%%%%%%%%%%%%%%%%%%%%%%%%%%%%%%%%%%%%
\subsection{Optimal policy}\label{sec:optimal_policy}

To summarize the individual's problem, let $h_t^g$, $P_t^g$, $\study_t^g$, $\ell_t^g$ denote the $J \times 1$ vectors of field-specific human capital, beliefs, study decisions, and labor decisions, respectively.
A policy $\pi: (h_{t}^g, P_{t}^g) \to (\study_{t}^g, \ell_t^g)$ is optimal if it maximizes lifetime expected utility: 
% \nts{I think I need to change this so $\theta_j$ is being drawn from a different distribution}
\begin{equation}
    \mathbb{E}^\pi \ce{
        \sum_{t=0}^\infty \delta^t 
        \pr{\sum_{j=1}^J U_j(h_{jt}^g, w_j) \ell_{jt}^g}
    }
    {h_{0}^g, P_0^g},
\end{equation}
given the following time constraint: 
\begin{equation*}
    \sum_{j=1}^J (\study_{jt}^g + \ell_{jt}^g) = 1, 
    \quad \quad 
    \study_{jt}^g, \ell_{jt}^g \in \{0,1\},
\end{equation*}
subject to the human capital accumulation and belief transition laws:\footnts{
    Is it correct to write $\theta_{j} \sim P_{j0}^g (\alpha_{j0}^g, \beta_{j0}^g)$ to denote the fact that initial beliefs about $\theta_j$ are given by the prior $P_{j0}^g$? Or does this notation suggest that the actual distribution is given by $P_{j0}^g$? I don't need to say something about the actual distribution here, do I? 
}
\begin{align*}
    h_{jt+1}^g =& h_{jt}^g + \nu_j s_{jt}^g  \study_{jt}^g, 
    \quad \quad 
    s_{jt}^g \sim \text{Bernoulli} (\theta_j), 
    \quad \quad
    \theta_{j} \sim P_{j0}^g \equiv \mathcal{B} (\alpha_{j0}^g, \beta_{j0}^g),
    \\
    P_{j,t+1}^g =& \mathcal{B} (\alpha_{j,t+1}^g, \beta_{j,t+1}^g), 
    \quad \quad 
    (\alpha_{j,t+1}, \beta_{j,t+1}) = 
    \begin{cases} 
        (\alpha_{jt}^g + 1, \beta_{jt}^g) &\text{ if } \study_{jt}^g = 1 \text{ and } \pass_{jt}^g = 1 \\
        (\alpha_{jt}^g, \beta_{jt}^g + 1) &\text{ if } \study_{jt}^g = 1 \text{ and } \pass_{jt}^g = 0 \\
        (\alpha_{jt}^g, \beta_{jt}^g) &\text{ if } \study_{jt}^g = 0
    \end{cases}
    .
\end{align*}

\textcite{AF20} characterize the optimal policy to the above problem.
\toedit{To apply their results, first assume the following initial condition to ensure optimality:}
\begin{align}
    h_{j0} \leq \alpha_{j0}^g \nu_{j}. \label{eq:h_leq_alpha_v}
\end{align} 
Let $\tau$ denote the optimal stopping rule defined over $\{s_{j1}^g, s_{j2}^g, \dots\}$. 
Define the field-$j$ index as the expected lifetime payoff an agent would receive if they commit to studying field-$j$ given their state $(h_{jt}^g, P_{jt}^g)$: \nts{I need to ask Titan about this; I think the conditional should reflect the whole history of the agent, including number of courses taken.}
\begin{equation*}
\mathcal{I}_{jt} (h_{j}^g, P_{j}^g) = \sup_{\tau \geq 0} \mathbb{E}^\tau
\ce{
   \sum_{t=0}^\infty \delta^t 
   U_j(h_{jt}^g, w_j) \ell_{jt}^g
}{
    (h_{j0}^g, P_{j0}^g) = (h_{j}^g, P_{j}^g)
}
\end{equation*}
Define the graduation region of field $j$ as the states where an agent committed to studying field $j$ would choose to stop studying and enter the labor market: 
\begin{equation*}
\mathcal{G}_j (h_{j}^g, P_{j}^g)  = 
    \left\{
        (h_{j}^g, P_{j}^g) 
        \left\vert
            \arg \max_{\tau \geq 0} 
            \mathbb{E}^\tau 
            \ce{
                \sum_{t=0}^\infty \delta^t 
                U_j(h_{jt}^g, w_j) \ell_{jt}^g
            }{
                (h_j, P_j^g)
            } = 0
   \right. \right\}
\end{equation*}
Then the following policy $\pi: (h_{t}^g, P_{t}^g) \to (\study_{t}^g, \ell_t^g)$ is optimal: 
\begin{enumerate}
    \item At each $t \geq 0$, choose skill $j^* = \arg \max_{j \in J} \mathcal{I}_j$, breaking ties according to any rule
    \item If $(h_{j^*}, P_{j^*}^g) \in \mathcal{G}_{j}$, then enter the labor market as a $j^*$ specialist. Otherwise, study $j^*$ for an additional period.  
\end{enumerate}

% At some point I need to highlight exactly where specialization is occuring. 

%%%%%%%%%%%%%%%%%%%%%%%%%%%%%%%%%%%%%%%%%%%%%%%%%%%%%%%%%%%%%%%%%%%%%%%%%%%%%%%%
\section{Implications of model}\label{sec:sims}
\tikzset{external/figure name={sims_}}
%%%%%%%%%%%%%%%%%%%%%%%%%%%%%%%%%%%%%%%%%%%%%%%%%%%%%%%%%%%%%%%%%%%%%%%%%%%%%%%%

An agent's specialization decision is driven by field, individual, and group characteristics.
% Some of these factors, such as wages,
% are field-specific and shared by all agents. \nts{(This group also includes returns to education, $\nu_j$).}
% Other factors, such as ability, are agent-specific. \nts{(this group also could be divided into known and unknown individual characteristics).}
% Finally, initial beliefs in this model are group-specific. 
To better understand how these factors drive an individual's behavior, I simulate the specialization decision of a student choosing between two completely symmetric fields.
Specifically, I assume a student can choose to work or study in one of two fields, field $X$ or field $Y$. 
Expected utility in field $j \in \{X, Y\}$ is equal to expected income: 
\begin{equation}\label{eq:linear_utility}
    U_j(w_j, h_{jt}^g) \ell_{jt}^g = w_j h_{jt}^g \ell_{jt}^g
\end{equation}
Wages in fields X and Y are equal and will be normalized to 1 ($w_X = w_Y = 1$), as are returns to successfully studying human capital ($\nu_X = \nu_Y = 1$).
The student's underlying abilities in the two fields, $\theta_X$ and $\theta_Y$, are both equal to 0.5. Therefore, the student has a 50\% chance of passing any given field X or field Y course.
Finally, I assume the student's beliefs about their own abilities in field X and Y are equal to the uniform prior:\footnote{
    Note that if $(\alpha, \beta) = (1, 1)$, the beta distribution $\mathcal{B} (\alpha, \beta)$ equals the uniform distribution over $[0, 1]$. 
        \nts{Intuitively, this implies that the student thinks all values of $\theta_j$ are equally likely. }
}
\begin{equation*}
    P_{X,0} = \mathcal{B}(\alpha_{X, 0}, \beta_{X, 0}) = \mathcal{B} (1, 1), 
    \quad \quad 
    P_{Y,0} = \mathcal{B}(\alpha_{Y, 0}, \beta_{Y, 0}) = \mathcal{B} (1, 1), 
\end{equation*}

For tractability, I modify the assumption in equation \eqref{eq:h_leq_alpha_v} as follows:
\begin{equation}\label{eq:h_eq_alpha_v}
    h_{j0}^g = \nu_j \alpha_{j0}^g.
\end{equation}
Equation \eqref{eq:h_eq_alpha_v} is consistent with the human capital accumulation function in equation \eqref{eq:hc_accumulation}, and allows for an analytical solution to the index and graduation region, \toedit{as described in \textcite{AF20}.}\footnts{
    Include formulation of this in the appendix. 
}
It is important to note that under this assumption, the number of periods an agent specializing in field $j$ studies is a deterministic function of the agent's initial beliefs.
This means that an agent who specializes in field $j$ will take exactly $c_{j}^*$ courses in field $j$, where $c_{j}^*$ is a function of an agent's initial field-$j$ beliefs.\footnote{
    Additionally, this assumption, and to a lesser extent the relaxed version in equation \eqref{eq:h_leq_alpha_v}, ties together beliefs and initial levels of human capital. 
    This may have relevant implications when applying this model to specific groups. 
    Specifically, if we make assumptions about initial group-based beliefs in line with section \ref{sec:group_based_beliefs}, equation \eqref{eq:h_eq_alpha_v} implies that (1) women begin with lower levels of initial human capital than men, and (2) women study for more periods than men in the fields in which they are more uncertain. \toedit{The first implication is not out of line with the literature, and the second implication here has some empirical support.} It is worth being aware of these implications moving forward. 
}
% However, this assumption ties together beliefs and initial levels of human capital.
\nts{Future versions of this project will relax this assumption.}
\nts{May also want to add that using a simple version of this stopping problem highlights other mechanisms.}

\begin{figure}[t!]
\centering
% This file was created by tikzplotlib v0.9.2.
\begin{tikzpicture}

\definecolor{color0}{rgb}{0.266666666666667,0.466666666666667,0.666666666666667}
\definecolor{color1}{rgb}{0.933333333333333,0.4,0.466666666666667}

\begin{groupplot}[group style={group size=2 by 3, vertical sep=1.8cm, horizontal sep=1.2cm}]
\nextgroupplot[
height=5.101085673964669cm,
tick align=outside,
tick pos=left,
title={Baseline Simulation},
width=8.25373cm,
x grid style={white!69.0196078431373!black},
xmin=-1.05, xmax=27,
xtick style={color=black},
xtick={0,5,10,15,20},
xticklabels={\(\displaystyle 0\),\(\displaystyle 5\),\(\displaystyle 10\),\(\displaystyle 15\),\(\displaystyle 20\)},
ylabel={Fraction enrolled in field},
ymajorgrids,
ymin=0, ymax=1,
ytick style={color=black},
ytick={0,0.25,0.5,0.75,1},
yticklabels={\(\displaystyle 0\),\(\displaystyle 0.25\),\(\displaystyle 0.5\),\(\displaystyle 0.75\),\(\displaystyle 1\)}
]
\addplot [semithick, color0]
table {%
0 0.505500078201294
1 0.503099918365479
2 0.502699971199036
3 0.498600006103516
4 0.502599954605103
5 0.502599954605103
6 0.503499984741211
7 0.506099939346313
8 0.505100011825562
9 0.505399942398071
10 0.505300045013428
11 0.506900072097778
12 0.506399989128113
13 0.506500005722046
14 0.506999969482422
15 0.506700038909912
16 0.506799936294556
17 0.506500005722046
20 0.506500005722046
21 0.506500005722046
};
\addplot [semithick, color1]
table {%
0 0.494500041007996
1 0.496899962425232
2 0.497300028800964
3 0.501399993896484
4 0.497400045394897
5 0.497400045394897
6 0.496500015258789
7 0.493900060653687
8 0.494899988174438
9 0.494600057601929
10 0.494699954986572
11 0.493099927902222
12 0.493600010871887
13 0.493499994277954
14 0.493000030517578
15 0.493299961090088
16 0.493200063705444
17 0.493499994277954
20 0.493499994277954
21 0.493499994277954
};
\draw (axis cs:21.5,0.4265) node[
  anchor=base west,
  text=color0,
  rotate=0.0
]{Field X};
\draw (axis cs:21.5,0.5235) node[
  anchor=base west,
  text=color1,
  rotate=0.0
]{Field Y};
\draw (axis cs:3,0.03) node[
  anchor=base west,
  text=black,
  rotate=0.0
]{N = 10,000};

\nextgroupplot[
height=5.101085673964669cm,
tick align=outside,
tick pos=left,
title={Baseline Simulation (zoomed in)},
width=8.25373cm,
x grid style={white!69.0196078431373!black},
xmin=-1.05, xmax=27,
xtick style={color=black},
xtick={0,5,10,15,20},
xticklabels={\(\displaystyle 0\),\(\displaystyle 5\),\(\displaystyle 10\),\(\displaystyle 15\),\(\displaystyle 20\)},
ymajorgrids,
ymin=0, ymax=1,
ytick style={color=black},
ytick={0,0.25,0.5,0.75,1},
yticklabels={\(\displaystyle 0\),\(\displaystyle 0.25\),\(\displaystyle 0.5\),\(\displaystyle 0.75\),\(\displaystyle 1\)}
]
\addplot [semithick, color0]
table {%
0 0.5
1 0.319999933242798
2 0.360000014305115
3 0.379999995231628
4 0.360000014305115
5 0.379999995231628
6 0.360000014305115
7 0.379999995231628
8 0.360000014305115
9 0.379999995231628
13 0.379999995231628
14 0.399999976158142
15 0.379999995231628
21 0.379999995231628
};
\addplot [semithick, color1]
table {%
0 0.5
1 0.680000066757202
2 0.639999985694885
3 0.620000004768372
4 0.639999985694885
5 0.620000004768372
6 0.639999985694885
7 0.620000004768372
8 0.639999985694885
9 0.620000004768372
13 0.620000004768372
14 0.600000023841858
15 0.620000004768372
21 0.620000004768372
};
\draw (axis cs:21.5,0.3) node[
  anchor=base west,
  text=color0,
  rotate=0.0
]{Field X};
\draw (axis cs:21.5,0.65) node[
  anchor=base west,
  text=color1,
  rotate=0.0
]{Field Y};
\draw (axis cs:3,0.03) node[
  anchor=base west,
  text=black,
  rotate=0.0
]{N = 50};

\nextgroupplot[
height=5.101085673964669cm,
tick align=outside,
tick pos=left,
title={\(\displaystyle w_{X} = 1\); \(\displaystyle w_{Y} = 1.5\)},
width=8.25373cm,
x grid style={white!69.0196078431373!black},
xmin=-1.05, xmax=27,
xtick style={color=black},
xtick={0,5,10,15,20},
xticklabels={\(\displaystyle 0\),\(\displaystyle 5\),\(\displaystyle 10\),\(\displaystyle 15\),\(\displaystyle 20\)},
ylabel={Fraction enrolled in field},
ymajorgrids,
ymin=-0.05, ymax=1.05,
ytick style={color=black},
ytick={0,0.2,0.4,0.6,0.8,1},
yticklabels={\(\displaystyle 0\),\(\displaystyle 0.2\),\(\displaystyle 0.4\),\(\displaystyle 0.6\),\(\displaystyle 0.8\),\(\displaystyle 1\)}
]
\addplot [semithick, color0]
table {%
0 0
1 0
2 0.250999927520752
3 0.111999988555908
4 0.182000041007996
5 0.148000001907349
6 0.14900004863739
7 0.154000043869019
8 0.149999976158142
9 0.138999938964844
10 0.149999976158142
11 0.146999955177307
12 0.144999980926514
13 0.146999955177307
14 0.146999955177307
15 0.148000001907349
21 0.148000001907349
};
\addplot [semithick, color1]
table {%
0 1
1 1
2 0.749000072479248
3 0.888000011444092
4 0.818000078201294
5 0.851999998092651
6 0.851000070571899
7 0.845999956130981
8 0.850000023841858
9 0.861000061035156
10 0.850000023841858
11 0.852999925613403
12 0.855000019073486
13 0.852999925613403
14 0.852999925613403
15 0.851999998092651
21 0.851999998092651
};
\draw (axis cs:21.5,0.178) node[
  anchor=base west,
  text=color0,
  rotate=0.0
]{Field X};
\draw (axis cs:21.5,0.882) node[
  anchor=base west,
  text=color1,
  rotate=0.0
]{Field Y};
\draw (axis cs:3,0.03) node[
  anchor=base west,
  text=black,
  rotate=0.0
]{N = 1,000};

\nextgroupplot[
height=5.101085673964669cm,
tick align=outside,
tick pos=left,
title={\(\displaystyle (\alpha_{X, 0}, \beta_{X, 0}) = (1, 1)\); \(\displaystyle (\alpha_{Y, 0}, \beta_{Y, 0}) = (1, 1)\)},
width=8.25373cm,
x grid style={white!69.0196078431373!black},
xmin=-0.95, xmax=27,
xtick style={color=black},
xtick={0,5,10,15},
xticklabels={\(\displaystyle 0\),\(\displaystyle 5\),\(\displaystyle 10\),\(\displaystyle 15\)},
ymajorgrids,
ymin=-0.05, ymax=1.05,
ytick style={color=black},
ytick={0,0.2,0.4,0.6,0.8,1},
yticklabels={\(\displaystyle 0\),\(\displaystyle 0.2\),\(\displaystyle 0.4\),\(\displaystyle 0.6\),\(\displaystyle 0.8\),\(\displaystyle 1\)}
]
\addplot [semithick, color0]
table {%
0 0
1 0.48800003528595
2 0.230000019073486
3 0.230000019073486
4 0.253000020980835
5 0.246000051498413
6 0.23199999332428
7 0.230000019073486
8 0.222000002861023
9 0.228000044822693
11 0.22599995136261
12 0.22599995136261
13 0.225000023841858
14 0.22599995136261
16 0.223999977111816
19 0.223999977111816
};
\addplot [semithick, color1]
table {%
0 1
1 0.51200008392334
2 0.769999980926514
3 0.769999980926514
4 0.746999979019165
5 0.753999948501587
6 0.76800000667572
7 0.769999980926514
8 0.777999997138977
9 0.772000074386597
11 0.773999929428101
12 0.773999929428101
13 0.774999976158142
14 0.773999929428101
16 0.776000022888184
19 0.776000022888184
};
\draw (axis cs:19.5,0.224) node[
  anchor=base west,
  text=color0,
  rotate=0.0
]{Field X};
\draw (axis cs:19.5,0.776) node[
  anchor=base west,
  text=color1,
  rotate=0.0
]{Field Y};
\draw (axis cs:3,0.03) node[
  anchor=base west,
  text=black,
  rotate=0.0
]{N = 1,000};

\nextgroupplot[
height=5.101085673964669cm,
tick align=outside,
tick pos=left,
title={\(\displaystyle \theta_{X} = 0.25\); \(\displaystyle \theta_{Y} = 0.75\)},
width=8.25373cm,
x grid style={white!69.0196078431373!black},
xlabel={t},
xmin=-1.05, xmax=27,
xtick style={color=black},
xtick={0,5,10,15,20},
xticklabels={\(\displaystyle 0\),\(\displaystyle 5\),\(\displaystyle 10\),\(\displaystyle 15\),\(\displaystyle 20\)},
ylabel={Fraction enrolled in field},
ymajorgrids,
ymin=0, ymax=1,
ytick style={color=black},
ytick={0,0.2,0.4,0.6,0.8,1},
yticklabels={\(\displaystyle 0\),\(\displaystyle 0.2\),\(\displaystyle 0.4\),\(\displaystyle 0.6\),\(\displaystyle 0.8\),\(\displaystyle 1\)}
]
\addplot [semithick, color0]
table {%
0 0.506999969482422
1 0.261000037193298
2 0.272000074386597
3 0.177000045776367
4 0.172000050544739
5 0.152999997138977
6 0.128000020980835
7 0.118000030517578
8 0.123999953269958
9 0.123999953269958
10 0.121999979019165
11 0.111000061035156
12 0.111999988555908
13 0.111999988555908
14 0.108000040054321
15 0.110000014305115
21 0.110000014305115
};
\addplot [semithick, color1]
table {%
0 0.493000030517578
1 0.739000082015991
2 0.727999925613403
3 0.822999954223633
4 0.828000068664551
5 0.847000002861023
6 0.871999979019165
7 0.881999969482422
8 0.875999927520752
9 0.875999927520752
10 0.878000020980835
11 0.888999938964844
12 0.888000011444092
13 0.888000011444092
14 0.891999959945679
15 0.889999985694885
21 0.889999985694885
};
\draw (axis cs:21.5,0.14) node[
  anchor=base west,
  text=color0,
  rotate=0.0
]{Field X};
\draw (axis cs:21.5,0.92) node[
  anchor=base west,
  text=color1,
  rotate=0.0
]{Field Y};
\draw (axis cs:3,0.03) node[
  anchor=base west,
  text=black,
  rotate=0.0
]{N = 1,000};

\nextgroupplot[
height=5.101085673964669cm,
hide x axis,
hide y axis,
tick align=outside,
tick pos=left,
width=8.25373cm,
x grid style={white!69.0196078431373!black},
xmin=0, xmax=1,
xtick style={color=black},
y grid style={white!69.0196078431373!black},
ymin=0, ymax=1,
ytick style={color=black}
]
\end{groupplot}

\end{tikzpicture}

% \caption{test}
\label{fig:sim_plots}
\end{figure}
Each subplot in figure \ref{fig:sim_plots} plots the fraction of simulated agents choosing to study field X or field Y at each time period $t$.
Recall that agents studying field $j$ at time $t$ will either pass and successfully accumulate human capital ($\pass_{jt}^g$ = 1) or they will fail ($\pass_{jt}^g =0$), where $\pass_{jt}^g \sim \text{Bernoulli} (\theta_j)$.
The student then updates their beliefs about their own underlying ability, $\theta_j$. 
% Simulated agents may switch which field they study as they update their beliefs. 
Line movements in figure \ref{fig:sim_plots} are caused by agents switching fields in response to updated beliefs. 
Eventually, students will specialize in one field and enter the labor market as a field-X or field-Y specialist.
Specialization in figure \ref{fig:sim_plots} is represented by a flattening of the curve; once a student has made their specialization decision, they no longer switch fields. 
For clarity, the line for any field $j$ ends once any agent specializing in $j$ stops studying and becomes a field-$j$ specialist.
Therefore, the length of the lines in figure \ref{fig:sim_plots} denote the minimum amount of time an agents spends studying before becoming a field-$j$ specialist.

The baseline scenarios in figures \ref{fig:sim_a} and \ref{fig:sim_b} illustrate these dynamics.
Figure \ref{fig:sim_a} plots the baseline scenario for 10,000 simulations; figure \ref{fig:sim_b} plots the first 50 of these simulations. 
\toedit{Our first takeaway is that the agent's specialization decision in the baseline is effectively a coin flip.}\footnts{
    I would really like a better way to describe this.
    The randomness in ability is driving choice? Not totally sure. 
}
This is most clearly seen in figure \ref{fig:sim_a}; at all time periods, approximately 50\% of the agents are studying field X and and 50\% are studying field Y. 
This should be expected, as fields X and Y are completely symmetric.

% It would be nice to mention persistence here
Some of the more subtle decision dynamics can only be seen with fewer observations. Therefore, \ref{fig:sim_b} zooms in on the first 50 of these simulations.
Note that the fraction of students studying field X or field Y moves in early periods, but flattens out in later periods. 
This is because students at the beginning of their education will update their beliefs in response to course outcomes.
These updated beliefs may cause students to switch fields, shifting the composition of simulated agents studying X or Y.
% Agents switching fields causes the lines in \ref{fig:sim_b} to move in early periods.
In later periods, simulated agents have made their specialization decision and no longer switch fields. \
This specialization is represented by the flattening of the lines in figure \ref{fig:sim_b}.
As in \ref{fig:sim_a}, approximately 50\% of agents specialize in field X, and 50\% specialize in field Y. 
% Finally, although there is some movement in terms of which fields student study, overall there is a persistence

\begin{figure}[t!]
\centering
\input{beta_example_change.tex}
% \caption{Initial prior $P_{j0} = \mathcal{B} (\alpha_{j0}, \beta_{j0})$}
\label{fig:beta_change}
\end{figure}

The remainder of figure \ref{fig:sim_plots} plots variations of the baseline for $N = 10,000$ simulations.
In figure \ref{fig:sim_c}, wages in field Y are 50\% higher than wages in field X. 
All other variables are identical to the baseline scenario. 
Because the expected lifetime payoff is so much higher, approximately 80\% of agents choose to specialize in Y.
Further, all agents begin their education studying Y.
However, after two periods, a large fraction of agents switch from studying Y to studying X. 
\toedit{This is due to agents (randomly) failing their first two courses in field Y, and switching into field X.}
To see why, consider the evolution of the belief distribution in figure \ref{fig:beta_change}. 
The student's initial belief distribution is plotted in figure \ref{fig:beta_ex_a}.
A student that fails their first course in field Y updates their beliefs about their underlying ability in Y to the distribution plotted in figure \ref{fig:beta_ex_d}; if they fail their second course in Y, they update their beliefs to \ref{fig:beta_ex_g}.
As we can see from figure \ref{fig:beta_ex_g}, a student that fails their first two classes in Y will believe they likely have a lower ability in that field.
If they specialized in Y, they would not expect to successfully accumulate much human capital over the course of their studies, implying a lower expected lifetime payoff.
As a result, these agents switch to studying field X, in spite of the lower wages. 


The higher wages in field Y also causes agents specializing in X to spend more time in school, as seen in figure \ref{fig:sim_c}.
As mentioned above, equation \eqref{eq:h_eq_alpha_v} implies that the number of periods an agents spends studying field X or Y is a deterministic function of initial beliefs. 
Agents' initial beliefs about their abilities in X and Y are the same, and as such, agents specializing in either X or Y will study their chosen discipline for the same number of periods. 
% In this simple version of the model does not allow for complementaries; therefore, switching fields necessarily increases the amount of time a student is in school. 
However, because all agents spend the first two periods studying Field Y, those who specialize in Field X will study for a minimum of two more periods.
% \toedit{Students who update their beliefs about their abilities in field Y}

Figure \ref{fig:sim_d} augments the baseline scenario so agents have a higher ability in field Y. 
Specifically, probability of success in any given field X course, $\theta_X$, equals 0.4, whereas the probability of success in field Y is given by $\theta_Y = 0.6$. 
Unsurprisingly, a higher ability in field Y drives specialization into that field.

We now turn to the impact of differential priors on specialization dynamics, plotted in Figure \ref{fig:sim_e}.
I assume simulated agents are initially more certain about their abilities in field Y relative to field X.
Specifically, I assume a student's initial prior about their ability in Y is given by $P_{Y0} = \mathcal{B} (2, 2)$, which corresponds to the distribution plotted in figure \ref{fig:beta_ex_e}.
Their initial prior about their ability in X continues to equal the uniform distribution, $P_{X0} = \mathcal{B} (1, 1)$, seen in figure \ref{fig:beta_ex_a}.
The prior for Y corresponds to the agent having more certainty about their underlying ability than in field X. 
However, the mean of the initial distribution for Y equals the mean of the initial distribution for X.

The first consequence of this assumption is that agents specializing in field X study for more periods than those specializing in field Y, as shown in figure \ref{fig:sim_e}.
As mentioned above, equation \eqref{eq:h_eq_alpha_v} implies that the number of periods an agent spends studying field $j$ before specializing in that field is a deterministic function of the agent's initial beliefs.
Increased uncertainty about field-$j$ ability implies that agents will study $j$ for more time period before specializing. 
\toedit{The second takeaway from \ref{fig:sim_e} is that all agents begin their education studying field Y.}
Agents know that if they become field Y specialists, they will finish their education earlier, and begin earning an income sooner.
The prospect of ending their education earlier drives agents to initially study field Y.

\toedit{The key takeaway from figure \ref{fig:sim_e} is that increased initial certainty about field Y abilities causes more agents to specialize in field Y.
\footnote{
    It's worth emphasizing that this is not driven by risk-aversion across fields;
    \toedit{assuming linear utility in \eqref{eq:linear_utility} ensures that agents are risk-neutral across fields.}
    Rather, \toedit{concavity due to discounting ensures that agents are risk-averse across time.}
}}
Although agents are equally likely to succeed in fields X and Y, and although the payoffs for specializing in these fields are the same, differential initial beliefs about underlying abilities drives the majority of simulated agents to specialize in field Y.
Thus, initial beliefs play a key role in specialization decisions. 
\toedit{I now consider how those beliefs are formed, and the consequences of forming those beliefs based on existing group outcomes.}


% Discuss: agents are risk-neutral across fields, but because of discounting they are risk averse across time (or something like that, I really wish I wrote that down!)

% Non-binary gender identities -
% do not have power to estimate parameters

%%%%%%%%%%%%%%%%%%%%%%%%%%%%%%%%%%%%%%%%%%%%%%%%%%%%%%%%%%%%%%%%%%%%%%%%%%%%%%%%
\section{Estimation}
%%%%%%%%%%%%%%%%%%%%%%%%%%%%%%%%%%%%%%%%%%%%%%%%%%%%%%%%%%%%%%%%%%%%%%%%%%%%%%%%

%%%%%%%%%%%%%%%%%%%%%%%%%%%%%%%%%%%%%%%%%%%%%%%%%%%%%%%%%%%%%%%%%%%%%%%%%%%%%%%%
\subsubsection*{Data sources}

Characteristics of all U.S. postsecondary institutions are reported in the Integrated Postsecondary Education Data System (IPEDS) data.
These data are collected annually by the National Center for Educational Statistics and describe the universe of institutions that participate in federal student financial aid programs. 
The empirical motivation relies on the IPEDS Completion Surveys, which describe all degrees and certificates awarded at postsecondary institutions by field of study, gender, and race.\footnote{
    For more details on the IPEDS series, please visit \url{https://nces.ed.gov/ipeds/}.
    IPEDS data are available from 1986 until the present, though I begin empirical analysis in 1990 due to changes in how fields of study are classified.
    For earlier data, such as those used in Figure \ref{fig:n_degrees}, I supplement the IPEDS series with data from \textcite{S93}.
    However, it is worth noting that the predecessor to IPEDS series is the Higher Education General Information Survey (HEGIS), available through the International Archive of Education Data at University of Michigan (\url{https://www.icpsr.umich.edu/web/ICPSR/series/00030}). As such, I refer the reader to the HEGIS series for a more detailed portrait of postsecondary education statistics than available in \textcite{S93}.
}
% 

\printbibliography

\end{document}
