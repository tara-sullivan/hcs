\documentclass[compress, 8pt]{beamer}
%\documentclass[handout, 8pt]{beamer}
\usepackage[utf8]{inputenc}

% Notes to self: \nts[color option]{text that is a note}
\newif\ifshowcomments
% \showcommentstrue % uncomment to show


% Standard packages
\usepackage{hyperref} % Include links
\usepackage{setspace} % More control over local line spacing: http://felix11h.github.io/blog/line-spacing-beamer
% Math
\usepackage{amsthm, amsmath, mathtools}
\usepackage{appendixnumberbeamer}
% Multi=row and column
\usepackage{multirow,multicol}
% multiline columns
\usepackage{tabularx}


\usepackage{ifthen}

\usepackage{amsbsy}

%%%%%%%%%%%%%%
% Formatting %
%%%%%%%%%%%%%%

% Outer theme
\useoutertheme[subsection=false]{miniframes}
% Fix to display navigational dots; dots only shown for subsections
% Can comment out to have no dots in presentation
\AtBeginSection[]{\subsection{}}
% Font
\usepackage[default]{cantarell}

% Colors
\usepackage{xcolor}
% UCSD color palette: http://ucpa.ucsd.edu/brand/elements/color-palette/
\definecolor{DarkBlue}{RGB}{24, 43, 73}
\definecolor{MainBlue}{RGB}{0, 106, 150}
\definecolor{LightBlue}{RGB}{0, 198, 215}
\definecolor{Green}{RGB}{110 150 59}
\definecolor{Yellow}{RGB}{255 205 0}
\definecolor{LightYellow}{RGB}{243 29 0}
\definecolor{Gold}{RGB}{198 146 20}
\definecolor{Orange}{RGB}{252 137 0}
\definecolor{LightGrey}{RGB}{182, 177, 169}
\definecolor{DarkGray}{RGB}{116 118 120}
\definecolor{DarkPurple}{RGB}{25 0 51}
\definecolor{Purple}{RGB}{55 24 81}
\definecolor{Maroon}{HTML}{6A123D}


% Hyperlink colors
\hypersetup{
  colorlinks   = true, %Colours links instead of ugly boxes
  urlcolor     = Maroon, %Colour for external hyperlinks
  linkcolor    = DarkGray, %Colour of internal links
  citecolor   = MainBlue %Colour of citations
}


% Learned using: https://ramblingacademic.com/2015/12/how-to-quickly-overhaul-beamer-colors/
% To change additional sections: http://www.cpt.univ-mrs.fr/~masson/latex/Beamer-appearance-cheat-sheet.pdf
\setbeamercolor{palette primary}{bg=DarkBlue,fg=white}
\setbeamercolor{palette secondary}{bg=DarkBlue,fg=white}
\setbeamercolor{palette tertiary}{bg=DarkBlue,fg=white}
\setbeamercolor{palette quaternary}{bg=DarkBlue,fg=white}
\setbeamercolor{structure}{fg=DarkBlue} % itemize, enumerate, etc
\setbeamercolor{section in toc}{fg=DarkBlue} % TOC sections
\setbeamercolor{section in head/foot}{bg=DarkBlue,fg=LightGrey}
%\setbeamercolor{navigation symbols}{bg=Gold, fg=Gold}
\setbeamercolor{alerted text}{fg=Gold}
\beamertemplatenavigationsymbolsempty
%tem
%\setbeamertemplate{itemize items}

% Make bullet size smaller
\setbeamertemplate{itemize item}{\tiny\raise1.4pt\hbox{$\blacktriangleright$}}
\setbeamertemplate{itemize subitem}{--}
% make bullet closer to the text
\setlength{\labelsep}{.8ex}

% get rid of 'Figure: ' in caption
\setbeamertemplate{caption}{\raggedright\insertcaption\par}
% Caption flushed left
\usepackage{caption}
\captionsetup{%
    labelformat=empty,
    font=small,
    singlelinecheck=false,
    tableposition=top
}


%%%%%%%%%%%%%%%%
% New commands %
%%%%%%%%%%%%%%%%

\ifshowcomments
\newcommand{\nts}[2][DarkGray]{\setbeamercolor{taras comment}{fg=#1}{\usebeamercolor[fg]{taras comment}\setbeamercolor{item}{fg = taras comment.fg}\emph{#2}}}
\else 
\newcommand{\nts}[2][DarkGray]{}
\fi

% Timed color highlights
\newcommand<>{\blue}[1]{\textbf{\color#2{MainBlue}#1}}
\newcommand<>{\mblue}[1]{{\color#2{MainBlue}#1}}
\newcommand<>{\green}[1]{\textbf{\color#2{Green}#1}}
\newcommand<>{\mgreen}[1]{{\color#2{Green}#1}}

\newcommand{\EE}{\mathbb{E}}
\newcommand{\PP}{\mathbb{P}}
\newcommand{\llog}[1]{\log \left( #1 \right)}
\newcommand{\Cov}[1]{\text{Cov} \left( #1 \right)}

\newcommand{\br}[1]{\left\{ #1 \right\}}
\newcommand{\sbr}[1]{\left[ #1 \right]}
\newcommand{\pr}[1]{\left( #1 \right)}
\newcommand{\ce}[2]{\left[\left. #1 \right\vert #2 \right]}
\newcommand{\cls}[2]{\left. #1 \right\vert #2}
\newcommand{\crs}[2]{#1 \left\vert #2 \right.}
\newcommand{\ceil}[1]{\left\lceil #1 \right\rceil}

% Beamer buttons
% bottom buttom: \bbutton{name_of_link}{Text to print}
\newcommand{\bbutton}[2]{
    \begin{tikzpicture}[remember picture, overlay]
    \node[shift={(-1.4cm,0.5cm)}]() at (current page.south east){%
    \hyperlink{#1}{\beamergotobutton{#2}}};    
    \end{tikzpicture}
}    
% here button
\newcommand{\hbutton}[2]{
    \hyperlink{#1}{\beamergotobutton{#2}}
}
% Cite command: cc
\newcommand{\nn}[2][DarkGray]{{\small \color{#1} #2}}
\newcommand{\ccdoi}[2]{\href{https://doi.org/#1}{\nn{#2}}}
\newcommand{\ccurl}[2]{\href{#1}{\nn{#2}}}

% change slide width on a single slide
% source: https://tex.stackexchange.com/questions/160825/modifying-margins-for-one-slide/242073
\newcommand\Wider[2][3em]{%
\makebox[\linewidth][c]{%
  \begin{minipage}{\dimexpr\textwidth+#1\relax}
  \raggedright#2
  \end{minipage}%
  }%
}

% To remove sections from appendix
% source: https://tex.stackexchange.com/questions/37127/how-to-remove-some-pages-from-the-navigation-bullets-in-beamer
\makeatletter
\let\beamer@writeslidentry@miniframeson=\beamer@writeslidentry
\def\beamer@writeslidentry@miniframesoff{%
  \expandafter\beamer@ifempty\expandafter{\beamer@framestartpage}{}% does not happen normally
  {%else
    % removed \addtocontents commands
    \clearpage\beamer@notesactions%
  }
}
\newcommand*{\miniframeson}{\let\beamer@writeslidentry=\beamer@writeslidentry@miniframeson}
\newcommand*{\miniframesoff}{\let\beamer@writeslidentry=\beamer@writeslidentry@miniframesoff}
\makeatother

%%%%%%%%%%%%%%%%%%%%
% Figures and tikz %
%%%%%%%%%%%%%%%%%%%%

% \usepackage{pgfplots} % maybe for tikz axes?
\usepackage{graphicx} % Include figures
\usepackage{tikz} % Draw figures with tikz
% \usetikzlibrary{positioning} % to use right=of <node> syntax
% \usetikzlibrary{shapes} % for other node shapes; see http://www.texample.net/tikz/examples/node-shapes/
% \usetikzlibrary{shapes.multipart} % multiline nodes in tikz
% \usetikzlibrary{shapes.arrows} % arrow nodes in tikz
% \usetikzlibrary{shapes.misc} % miscellaneous nodes in tikz; for rounded rectangles
% \usetikzlibrary{shapes.geometric} % geometric nodes in tikz; fortriangles

% Presentation specific tikz
%\usetikzlibrary{tikzmark} % to use tikzmarkto add braces to itemize
% \usetikzlibrary{calc} % for widthof
% \tikzset{
%     above label/.style={
%         above = 5pt,
%         %font=\footnotesize,
%         text height = 1.5ex,
%         text depth = 1ex,
%     },
%     below label/.style={  
%         below=4pt,
%         %font=\footnotesize,
%         text height = 1.5ex,
%         text depth = 1ex
%     },
%     brace label/.style={
%         below = 4pt,
%         font=\footnotesize,
%         text height = 1.5ex,
%         text depth = 1ex
%     },
%     brace/.style={
%         decoration={brace, mirror},
%         decorate
%     }
% }

% For every picture that defines or uses external nodes, you'll have to
% apply the 'remember picture' style. To avoid some typing, we'll apply
% the style to all pictures.
%\tikzstyle{every picture}+=[remember picture]

% To avoid a warning
% \pgfplotsset{compat=1.7}



% Avoid a warning on mac; can introduce a warning (and maybe a break) on windows
% \usepackage{etex}
% Avoid a warning
\let\Tiny=\tiny

%Information to be included in the title page:
\title{Group-based beliefs and human capital specialization}
\author[]{Tara Sullivan}
\institute{
Macro Lunch Presentation \\
Tara Sullivan \\
\textit{tasulliv@ucsd.edu}
}
\date{
\today
\nts{\\
\medskip
Notes/comments are in gray and will be excluded from the presentation.
}
}

% plot graphs with pgfplots
\usepackage{pgfplots}
\pgfplotsset{compat=newest}
\usepgfplotslibrary{groupplots}
\usepgfplotslibrary{dateplot}
\pgfplotsset{compat=newest,
    every axis/.style={
        axis y line*=left,
        axis x line*=bottom,
        % allows for multi-line legend entries
        legend style={cells={align=left}},
        % allows for multi-line titles
        title style={align=center},
    },
}

% path to tex images
\makeatletter
\def\input@path{{../../img/}}
\makeatother

% Externalize pgf plots
\usetikzlibrary{external}
\tikzexternalize[prefix=figures/]

\newcommand{\study}{m} % I think i could change this to r if the fact that I use m as an index (in gender) is a problem
\newcommand{\pass}{s}

%%%%%%%%%%%%%%%%%%%%%%%%%%%%%%%%%%%%%%%%%%%%%%%%%%%%%%%%%%%%%%%%%%%%%%%%%%%%%%%%
\begin{document}    
%\beamertemplatenavigationsymbolsempty


%%%%%%%%%%%%%%%%%%%%%%%%%%%%%%%%%%%%%%%%%%%%%%%%%%%%%%%%%%%%%%%%%%%%%%%%%%%%%%%%
\begin{frame}
\titlepage
\tikzset{external/figure name={intro_}}
\end{frame}

%%%%%%%%%%%%%%%%%%%%%%%%%%%%%%%%%%%%%%%%%%%%%%%%%%%%%%%%%%%%%%%%%%%%%%%%%%%%%%%%
\section*{Introduction}

%%%%%%%%%%%%%%%%%%%%%%%%%%%%%%%%%%%%%%%%%%%%%%%%%%%%%%%%%%%%%%%%%%%%%%%%%%%%%%%%
\begin{frame}{Increased attainment of Bachelor's degrees} 

% Women are more educated; but what they are studying matters
% 450,000 men earned bachelor's degrees in 1970, compared to 340,000 women
% Highlight that we are discussing convergence in specific majors since the 1990s.}

\begin{figure}
% % This file was created by tikzplotlib v0.9.1.
\begin{tikzpicture}

\definecolor{color0}{rgb}{0.12156862745098,0.466666666666667,0.705882352941177}
\definecolor{color1}{rgb}{1,0.498039215686275,0.0549019607843137}

\begin{axis}[
legend cell align={left},
legend style={fill opacity=0.8, draw opacity=1, text opacity=1, at={(0.03,0.97)}, anchor=north west, draw=white!80!black},
tick align=outside,
tick pos=left,
title={Number of Bachelors Degrees awarded (millions)},
x grid style={white!69.0196078431373!black},
xlabel={year},
xmin=1988.6, xmax=2019.4,
xtick style={color=black},
y grid style={white!69.0196078431373!black},
ymin=0.4493241, ymax=1.2484059,
ytick style={color=black}
]
\addplot [semithick, color0]
table {%
1990 0.48564600944519
1991 0.490826010704041
1992 0.517989993095398
1993 0.532243013381958
1994 0.532928943634033
1995 0.528069019317627
1997 0.518990993499756
1998 0.522558927536011
1999 0.522891998291016
2000 0.533735036849976
2001 0.557978987693787
2002 0.579033017158508
2003 0.599171996116638
2004 0.629392027854919
2005 0.649704933166504
2006 0.66592800617218
2007 0.687217950820923
2008 0.703808069229126
2009 0.722702980041504
2010 0.750731945037842
2011 0.779560089111328
2012 0.814333915710449
2013 0.836575031280518
2014 0.850880980491638
2015 0.862040996551514
2016 0.871549010276794
2017 0.886856079101562
2018 0.897544026374817
};
\addlegendentry{Men}
\addplot [semithick, color1]
table {%
1990 0.555091023445129
1991 0.581097006797791
1992 0.614879012107849
1993 0.632375001907349
1994 0.638270020484924
1995 0.636955976486206
1996 0.644475936889648
1997 0.652374982833862
1998 0.666815042495728
1999 0.684229016304016
2000 0.708883047103882
2001 0.745826005935669
2002 0.779849052429199
2003 0.805441975593567
2004 0.849164962768555
2005 0.87644100189209
2006 0.905745983123779
2007 0.927672982215881
2008 0.947183012962341
2009 0.968661069869995
2010 1.00603902339935
2011 1.04807901382446
2012 1.09642803668976
2013 1.12388396263123
2015 1.15306401252747
2016 1.17018795013428
2017 1.19238698482513
2018 1.2120840549469
};
\addlegendentry{Women}
\end{axis}

\end{tikzpicture}

% This file was created by tikzplotlib v0.9.1.
\begin{tikzpicture}

\definecolor{color0}{rgb}{0.12156862745098,0.466666666666667,0.705882352941177}
\definecolor{color1}{rgb}{1,0.498039215686275,0.0549019607843137}

\begin{axis}[
legend cell align={left},
legend style={fill opacity=0.8, draw opacity=1, text opacity=1, at={(0.03,0.97)}, anchor=north west, draw=white!80!black},
tick align=outside,
tick pos=left,
title={Number of Bachelors Degrees awarded (millions)},
x grid style={white!69.0196078431373!black},
xlabel={year},
xmin=1988.6, xmax=2019.4,
xtick style={color=black},
y grid style={white!69.0196078431373!black},
ymin=0.4493241, ymax=1.2484059,
ytick style={color=black}
]
\addplot [semithick, color0]
table {%
1990 0.48564600944519
1991 0.490826010704041
1992 0.517989993095398
1993 0.532243013381958
1994 0.532928943634033
1995 0.528069019317627
1997 0.518990993499756
1998 0.522558927536011
1999 0.522891998291016
2000 0.533735036849976
2001 0.557978987693787
2002 0.579033017158508
2003 0.599171996116638
2004 0.629392027854919
2005 0.649704933166504
2006 0.66592800617218
2007 0.687217950820923
2008 0.703808069229126
2009 0.722702980041504
2010 0.750731945037842
2011 0.779560089111328
2012 0.814333915710449
2013 0.836575031280518
2014 0.850880980491638
2015 0.862040996551514
2016 0.871549010276794
2017 0.886856079101562
2018 0.897544026374817
};
\addlegendentry{Men}
\addplot [semithick, color1]
table {%
1990 0.555091023445129
1991 0.581097006797791
1992 0.614879012107849
1993 0.632375001907349
1994 0.638270020484924
1995 0.636955976486206
1996 0.644475936889648
1997 0.652374982833862
1998 0.666815042495728
1999 0.684229016304016
2000 0.708883047103882
2001 0.745826005935669
2002 0.779849052429199
2003 0.805441975593567
2004 0.849164962768555
2005 0.87644100189209
2006 0.905745983123779
2007 0.927672982215881
2008 0.947183012962341
2009 0.968661069869995
2010 1.00603902339935
2011 1.04807901382446
2012 1.09642803668976
2013 1.12388396263123
2015 1.15306401252747
2016 1.17018795013428
2017 1.19238698482513
2018 1.2120840549469
};
\addlegendentry{Women}
\end{axis}

\end{tikzpicture}

\end{figure}

\end{frame}

%%%%%%%%%%%%%%%%%%%%%%%%%%%%%%%%%%%%%%%%%%%%%%%%%%%%%%%%%%%%%%%%%%%%%%%%%%%%%%%%
\begin{frame}{Gender ratio in different fields}

% When we consider gender convergence across different fields

\begin{figure}
% % This file was created by tikzplotlib v0.9.1.
\begin{tikzpicture}

\definecolor{color0}{rgb}{0.12156862745098,0.466666666666667,0.705882352941177}
\definecolor{color1}{rgb}{1,0.498039215686275,0.0549019607843137}

\begin{axis}[
legend cell align={left},
legend style={fill opacity=0.8, draw opacity=1, text opacity=1, at={(0.03,0.97)}, anchor=north west, draw=white!80!black},
tick align=outside,
tick pos=left,
title={Number of Bachelors Degrees awarded (millions)},
x grid style={white!69.0196078431373!black},
xlabel={year},
xmin=1988.6, xmax=2019.4,
xtick style={color=black},
y grid style={white!69.0196078431373!black},
ymin=0.4493241, ymax=1.2484059,
ytick style={color=black}
]
\addplot [semithick, color0]
table {%
1990 0.48564600944519
1991 0.490826010704041
1992 0.517989993095398
1993 0.532243013381958
1994 0.532928943634033
1995 0.528069019317627
1997 0.518990993499756
1998 0.522558927536011
1999 0.522891998291016
2000 0.533735036849976
2001 0.557978987693787
2002 0.579033017158508
2003 0.599171996116638
2004 0.629392027854919
2005 0.649704933166504
2006 0.66592800617218
2007 0.687217950820923
2008 0.703808069229126
2009 0.722702980041504
2010 0.750731945037842
2011 0.779560089111328
2012 0.814333915710449
2013 0.836575031280518
2014 0.850880980491638
2015 0.862040996551514
2016 0.871549010276794
2017 0.886856079101562
2018 0.897544026374817
};
\addlegendentry{Men}
\addplot [semithick, color1]
table {%
1990 0.555091023445129
1991 0.581097006797791
1992 0.614879012107849
1993 0.632375001907349
1994 0.638270020484924
1995 0.636955976486206
1996 0.644475936889648
1997 0.652374982833862
1998 0.666815042495728
1999 0.684229016304016
2000 0.708883047103882
2001 0.745826005935669
2002 0.779849052429199
2003 0.805441975593567
2004 0.849164962768555
2005 0.87644100189209
2006 0.905745983123779
2007 0.927672982215881
2008 0.947183012962341
2009 0.968661069869995
2010 1.00603902339935
2011 1.04807901382446
2012 1.09642803668976
2013 1.12388396263123
2015 1.15306401252747
2016 1.17018795013428
2017 1.19238698482513
2018 1.2120840549469
};
\addlegendentry{Women}
\end{axis}

\end{tikzpicture}

% This file was created by tikzplotlib v0.9.1.
\begin{tikzpicture}

\definecolor{color0}{rgb}{0.266666666666667,0.466666666666667,0.666666666666667}
\definecolor{color1}{rgb}{0.933333333333333,0.4,0.466666666666667}
\definecolor{color2}{rgb}{0.133333333333333,0.533333333333333,0.2}
\definecolor{color3}{rgb}{0.8,0.733333333333333,0.266666666666667}
\definecolor{color4}{rgb}{0.4,0.8,0.933333333333333}
\definecolor{color5}{rgb}{0.666666666666667,0.2,0.466666666666667}

\begin{axis}[
height=207pt,
tick align=outside,
tick pos=left,
title={Ratio of women to men},
width=300pt,
x grid style={white!69.0196078431373!black},
xmin=1990, xmax=2030,
xtick style={color=black},
xtick={1990,1995,2000,2005,2010,2015},
xticklabels={\(\displaystyle 1990\),\(\displaystyle 1995\),\(\displaystyle 2000\),\(\displaystyle 2005\),\(\displaystyle 2010\),\(\displaystyle 2015\)},
ymajorgrids,
ymin=0.111868877532793, ymax=1.72181142557414,
ytick style={color=black}
]
\addplot [semithick, color0]
table {%
1990 0.187528491020203
1991 0.187889814376831
1992 0.18504810333252
1993 0.191165328025818
1994 0.198027372360229
1995 0.209525942802429
1996 0.219993829727173
1997 0.226323843002319
1998 0.230563402175903
1999 0.248593807220459
2000 0.259858369827271
2001 0.253314614295959
2002 0.266838073730469
2003 0.248422026634216
2004 0.258007287979126
2005 0.250089168548584
2006 0.243573904037476
2007 0.228035092353821
2008 0.227158188819885
2009 0.221451997756958
2011 0.231951236724854
2012 0.238436937332153
2013 0.240317106246948
2014 0.248236060142517
2015 0.251599311828613
2016 0.265729546546936
2017 0.274784326553345
2018 0.286244869232178
};
\addplot [semithick, color1]
table {%
1990 0.429780006408691
1991 0.419558525085449
1993 0.395862579345703
1994 0.4017493724823
1995 0.399999976158142
1996 0.381898045539856
1997 0.373727560043335
1998 0.368845701217651
1999 0.373373627662659
2000 0.390813827514648
2001 0.387295722961426
2002 0.384417414665222
2003 0.37157928943634
2004 0.335911631584167
2005 0.288139462471008
2006 0.261456251144409
2007 0.22909951210022
2008 0.216255187988281
2009 0.219069957733154
2010 0.223915338516235
2011 0.216599345207214
2012 0.225852251052856
2013 0.219969153404236
2014 0.223478555679321
2015 0.222921013832092
2016 0.235600471496582
2017 0.242461562156677
2018 0.257417798042297
};
\addplot [semithick, color2]
table {%
1990 0.460548996925354
1991 0.464974999427795
1992 0.4886314868927
1993 0.488829135894775
1994 0.512030601501465
1995 0.540966749191284
1996 0.565948724746704
1997 0.602615833282471
1998 0.629949331283569
1999 0.666009306907654
2000 0.686820149421692
2001 0.706353425979614
2002 0.736422896385193
2003 0.7142094373703
2004 0.72872519493103
2005 0.739134073257446
2006 0.725082635879517
2007 0.690871238708496
2008 0.687974691390991
2009 0.686181664466858
2010 0.688009262084961
2011 0.668954968452454
2012 0.67028284072876
2013 0.634251594543457
2014 0.645684003829956
2015 0.62436580657959
2016 0.629948854446411
2017 0.653665065765381
2018 0.67027759552002
};
\addplot [semithick, color3]
table {%
1990 0.843912363052368
1991 0.878591060638428
1992 0.892377018928528
1993 0.916429758071899
1994 0.941344499588013
1995 0.972469568252563
1996 1.01399052143097
1997 1.05340433120728
1998 1.07601177692413
1999 1.12696635723114
2000 1.16036009788513
2001 1.19738638401031
2002 1.19550442695618
2003 1.1779510974884
2004 1.14445388317108
2005 1.13158094882965
2006 1.10574865341187
2007 1.10396230220795
2008 1.07783782482147
2009 1.08280813694
2010 1.08092784881592
2011 1.07324647903442
2012 1.07712745666504
2013 1.0801340341568
2014 1.06191837787628
2015 1.04196679592133
2016 1.07565009593964
2017 1.0877937078476
2018 1.10102880001068
};
\addplot [semithick, color4]
table {%
1990 0.86025857925415
1991 0.899104356765747
1992 0.886856079101562
1993 0.900158047676086
1994 0.865336656570435
1995 0.88508152961731
1996 0.85411524772644
1997 0.864051580429077
1998 0.88809061050415
1999 0.93377161026001
2000 0.914776563644409
2001 0.881730198860168
2003 0.793915510177612
2004 0.801473140716553
2005 0.769950985908508
2006 0.776896238327026
2007 0.747289896011353
2008 0.764146089553833
2009 0.725896239280701
2010 0.73568868637085
2011 0.726636171340942
2012 0.726435899734497
2013 0.727647542953491
2014 0.723636150360107
2015 0.720515251159668
2016 0.709809184074402
2017 0.696071982383728
2018 0.709341764450073
};
\addplot [semithick, color5]
table {%
1990 0.889276146888733
1991 0.908068180084229
1992 0.908786058425903
1993 0.911298394203186
1995 0.937413215637207
1996 0.960581421852112
1997 0.962796211242676
1998 0.959450006484985
1999 0.985843420028687
2000 1.00777983665466
2001 1.00261080265045
2002 1.01744496822357
2003 1.02670252323151
2004 1.02088141441345
2005 1.00308656692505
2006 0.996885895729065
2007 0.972628951072693
2008 0.96298623085022
2009 0.959978342056274
2010 0.95364236831665
2011 0.953920364379883
2012 0.931203603744507
2013 0.921860694885254
2014 0.899493217468262
2015 0.900677680969238
2016 0.891088247299194
2017 0.886778354644775
2018 0.88598895072937
};
\addplot [semithick, white!73.3333333333333!black]
table {%
1990 1.04078304767609
1991 1.04256737232208
1992 1.07366561889648
1993 1.06894600391388
1994 1.05757308006287
1995 1.10440194606781
1996 1.11996006965637
1997 1.17428719997406
1998 1.23312425613403
1999 1.30454993247986
2000 1.40134906768799
2001 1.46844744682312
2002 1.54577028751373
2003 1.63139522075653
2004 1.64863216876984
2005 1.63168549537659
2006 1.602987408638
2007 1.51491057872772
2008 1.47168242931366
2009 1.46346211433411
2010 1.41208970546722
2011 1.44205784797668
2012 1.42999362945557
2013 1.42106962203979
2014 1.41722071170807
2015 1.44253933429718
2016 1.49894797801971
2017 1.56678104400635
2018 1.64683747291565
};
\draw (axis cs:2018.5,0.286244813278008) node[
  anchor=base west,
  text=color0,
  rotate=0.0
]{Engineering};
\draw (axis cs:2018.5,0.217417838961352) node[
  anchor=base west,
  text=color1,
  rotate=0.0
]{Computer services};
\draw (axis cs:2018.5,0.63027762382224) node[
  anchor=base west,
  text=color2,
  rotate=0.0
]{Physical sciences};
\draw (axis cs:2018.5,1.10102875591671) node[
  anchor=base west,
  text=color3,
  rotate=0.0
]{Social sciences};
\draw (axis cs:2018.5,0.709341764874964) node[
  anchor=base west,
  text=color4,
  rotate=0.0
]{Math and stats};
\draw (axis cs:2018.5,0.885989010989011) node[
  anchor=base west,
  text=color5,
  rotate=0.0
]{Business};
\draw (axis cs:2018.5,1.64683752645192) node[
  anchor=base west,
  text=white!73.3333333333333!black,
  rotate=0.0
]{Biological sciences};
\end{axis}

\end{tikzpicture}

\end{figure}
% In 1970, females were 30% of biology majors

\end{frame}

%%%%%%%%%%%%%%%%%%%%%%%%%%%%%%%%%%%%%%%%%%%%%%%%%%%%%%%%%%%%%%%%%%%%%%%%%%%%%%%%
\begin{frame}{Social Sciences}\label{intro_social_science_ratio}
\Wider[4em]{

\begin{figure}
\setlength{\abovecaptionskip}{2pt}
\setlength{\belowcaptionskip}{-2pt}
\input{social_science_rat.tex}
\end{figure}

\hyperlink{app_social_science_num}{\beamerbutton{Number degrees completed}}
\hyperlink{app_social_science_cip}{\beamerbutton{Social Science CIP}}
\hyperlink{app_engineering}{\beamerbutton{Engineering}}
\hyperlink{app_business}{\beamerbutton{Business}}
\hyperlink{app_computer_science}{\beamerbutton{Computer Science}}
\hyperlink{app_education}{\beamerbutton{Education}}
\hyperlink{app_science_math}{\beamerbutton{Science and math}}

}
\end{frame}

%%%%%%%%%%%%%%%%%%%%%%%%%%%%%%%%%%%%%%%%%%%%%%%%%%%%%%%%%%%%%%%%%%%%%%%%%%%%%%%%
\begin{frame}{Gender convergence across fields}

Do differences in gender convergence across fields represent misallocation of talent?
\begin{itemize}
    \item If yes, this has macroeconomic consequences 
    

    % \item College major choice matters for future income \ccdoi{10.1146/annurev-economics-080511-110908}{(Altonji, Blom, and Meghir 2012)}
\end{itemize}

% \vspace{2ex}
% Common explanations for why convergence might not occur:
% \begin{itemize}
%     \item Preferences
%     \item Abilities
%     \item Discrimination in labor market
%     \item <2-> Beliefs
% \end{itemize}

\pause
\vspace{3ex} 
\textbf{This paper:} the role of group-based beliefs in human capital specialization decisions

\vspace{3ex}
Model of gradual human capital specialization:
\begin{itemize}
    \item Unknown heterogeneous abilities 
    \item Group-based beliefs about abilities 
    \item Sequential learning and human capital accumulation
    \item [$\implies$] Group-based beliefs play an important role in specialization decisions
\end{itemize}
% present a model of human capital and learning ability
% beliefs result in different course taking characterisitcs

\vspace{3ex}
What are the productivity costs associated with misallocation of talent implied by the model? 
\begin{itemize}
    \item Estimate impact of misallocation on aggregate productivity growth in line with \ccdoi{10.3982/ECTA11427}{Hsieh, Hurst, Jones, and Klenow (2019)} 
\end{itemize}

\end{frame}

%%%%%%%%%%%%%%%%%%%%%%%%%%%%%%%%%%%%%%%%%%%%%%%%%%%%%%%%%%%%%%%%%%%%%%%%%%%%%%%%
\begin{frame}{Limited Literature Review}

1. Human capital specialization
\begin{itemize}
    \item Build on model of gradual specialization from Alon and Fershtman (2019)
\end{itemize}

\vspace{4ex}
2. Gender gaps in college choice
\begin{itemize}
    \item Empirically motivated by Sloan, Hurst, and Black (2020)  % pre-labor market human capital specialization
%     % Use newly expanded ACS data that asks about major choice for college educated individuals
%     % Women choose college majors associated with lower potential wages than men
%     % Need to understand this better!!
% Gender gap in schooling and experience has shrunk
\end{itemize}

\vspace{4ex}
3. Determinants of college major choice, in particular the role of beliefs
\begin{itemize}
    \item Arcidiacono et al. (2015): model of sequential learning and role of beliefs
    % does not easily allow for granual fields
    % no human capital accumulation
    \item Subjective expectations literature (Stinebrickner and Stinebrickner, 2014; Wiswall and Zafar, 2019; Zafar, 2013)
\end{itemize}

\vspace{3ex}
4. Statistical discrimination literature 
\begin{itemize}
    \item Lundberg and Startz (1984): efficiency of equal opportunity laws
    \item Coate and Loury (1997): permanent affirmative action and patronizing equilibria
\end{itemize}
% \begin{itemize}
%     \item Dynamics similar to 
% \end{itemize}

\end{frame}

%%%%%%%%%%%%%%%%%%%%%%%%%%%%%%%%%%%%%%%%%%%%%%%%%%%%%%%%%%%%%%%%%%%%%%%%%%%%%%%
\miniframesoff
\begin{frame}{Outline}
    \tableofcontents
\end{frame}
\miniframeson
%%%%%%%%%%%%%%%%%%%%%%%%%%%%%%%%%%%%%%%%%%%%%%%%%%%%%%%%%%%%%%%%%%%%%%%%%%%%%%%

%%%%%%%%%%%%%%%%%%%%%%%%%%%%%%%%%%%%%%%%%%%%%%%%%%%%%%%%%%%%%%%%%%%%%%%%%%%%%%%%
\section[Model]{Model}

% begin with an Alon-Fershtman slide; review as quicly as possible ``because many people here are familar with this paper, I'll try to keep this short'' Maybe check with david that this is a good idea
% Then jump into my key contribution: how do we think about the priors

%%%%%%%%%%%%%%%%%%%%%%%%%%%%%%%%%%%%%%%%%%%%%%%%%%%%%%%%%%%%%%%%%%%%%%%%%%%%%%%%
%%%%%%%%%%%%%%%%%%%%%%%%%%%%%%%%%%%%%%%%%%%%%%%%%%%%%%%%%%%%%%%%%%%%%%%%%%%%%%%%
\begin{frame}{Model preliminaries}

% Discrete time
% Infinitely lived agents

% unknown success probability in j

% write that with an indicator function

% 

Individuals endowed with:
\begin{itemize}
    \item [$h_{j0}$:] Skill-$j$ specific human capital ($j=0,\dots,J$)
    \item [$\theta_j$:] Unknown probability of success in $j$
    \item [$P_{j0}$:] Prior beliefs about $\theta_j$
\end{itemize}

\vspace{2ex}
At each $t$, individuals can choose to either study or work in one skill-$j$:
\begin{itemize}
    \item Studying accumulates skill-$j$ human capital and reveals information about underlying probability of success in $j$
    % Endogenous enter labor market at time t as a skill j specialist to maximize 
    \item If you work, you receive wage $w_j$
% To keep things simple, I'll assume utility is linear, and that the value of entering the market in period t as a skill k specialist simply depends on lifetime earnings
\end{itemize}
% Time constraint:
% \begin{equation*}
%     \sum_{j=0}^J (s_{jt} + \ell_{jt}) = 1, \quad \quad s_{jt}, \ell_{jt} \in \{0, 1\}
% \end{equation*}

\vspace{2ex}
Enter labor market at time $t$ in skill-$j$ to maximize expected lifetime payoff:
\begin{equation*}
    \frac{\delta^t}{1 - \delta} U_j (w_{j}, h_{jt}) \ell_{jt}
    = \frac{\delta^t}{1 - \delta} w_{j} h_{jt} \ell_{jt}
    % Is it okay to say this? 
\end{equation*}
% \vspace{2ex}


\end{frame}

%%%%%%%%%%%%%%%%%%%%%%%%%%%%%%%%%%%%%%%%%%%%%%%%%%%%%%%%%%%%%%%%%%%%%%%%%%%%%%%%
% \begin{frame}{Student specialization decision}
\begin{frame}{Evolution of human capital accumulation and beliefs}

% Choose probability 

Students studying skill-$j$ at time $t$ pass the course with probability $\theta_j$:
\begin{equation*}
    a_{jt} \sim \text{Bernoulli} (\theta_j)
\end{equation*}
\vspace{-2.5ex}
\begin{itemize}
  \item 
  Accumulate human capital if they pass the course:
  \begin{align*}
  % h_{jt+1} =& H(h_{kt}, a_{kt}), \quad \quad
%   % a_{jt} \sim F_{\theta_j}
    h_{jt+1} = h_{jt} + \nu_{j} a_{jt}
  \end{align*}
  \item 
  Beliefs about $\theta_j$ evolve:
  \begin{equation*}
      \mgreen<2->{P_{j,t+1}} = \Pi_j (\mblue<2->{P_{jt}},a_{jt})
        % P_{j, t+1} 
  \end{equation*}
\end{itemize}

\pause
\vspace{5ex}
\textbf{Key:} How are  \blue<2->{priors} formed, and how are they \green{updated}? 

% Parameter we are trying to understand is probability of success
% Know that we want a prior distribution supported on [0, 1]

\end{frame}

%%%%%%%%%%%%%%%%%%%%%%%%%%%%%%%%%%%%%%%%%%%%%%%%%%%%%%%%%%%%%%%%%%%%%%%%%%%%%%%%
\begin{frame}[t]{Belief distribution}

Initial prior drawn from Beta distribution 
\begin{equation*}
    \mblue{P_{j0}} = \mathcal{B} (\alpha_{j0}, \beta_{j0})
\end{equation*}
% Beta distribution is appropriate 
% need a probability distribution supported on [0,1]
% Also has some desireable analytic properties
Update according to Bayes Rule $\implies$ posterior drawn from Beta distribution:
\begin{equation*}
    \mgreen{P_{j,t+1}} = \mathcal{B} (\alpha_{j,t+1}, \beta_{j,t+1}), \quad \quad 
    (\alpha_{j,t+1}, \beta_{j,t+1}) = 
    \begin{cases} 
        (\alpha_{jt} + 1, \beta_{jt}) &\text{ if } a_{jt} = 1 \\
        (\alpha_{jt}, \beta_{jt} + 1) &\text{ if } a_{jt} = 0
    \end{cases}
\end{equation*}

\vspace{3ex}
\begin{columns}[T] % align columns
\begin{column}{.51\textwidth}

\pause
\vspace{3ex}
Example: $\alpha_0 = 1$, $\beta_0 = 1$
\vspace{1.5ex}
\begin{itemize}
    \item <3-> success at $t=1$ $\implies$ $\alpha_1 = 2$, $\beta_1 = 1$

    \vspace{1.5ex}
    \item <4-> failure at $t=2$ $\implies$ $\alpha_1 = 2$, $\beta_1 = 2$
    
    \vspace{1.5ex}
    \item <5-> success at $t=3$ $\implies$ $\alpha_1 = 3$, $\beta_1 = 2$
\end{itemize}

\end{column}%
% \hfill%
\begin{column}{.39\textwidth}
\begin{figure}
\only<2>{\input{beta_example0.tex}}
\only<3>{\input{beta_example1.tex}}
\only<4>{\input{beta_example2.tex}}
\only<5>{\input{beta_example3.tex}}
\end{figure}
\end{column}%
\end{columns}
% \hypertarget<2>{beta_11_example}{\beamerbutton{I'm on the fourth slide}}
\hypertarget<2>{model_beta_11}{
  \hyperlink{simulate}{\beamerbutton{Return: simulation set-up}}
  \hyperlink{sim_default}{\beamerbutton{Return: baseline simulation}}
}
\hypertarget<4>{model_beta_22}{\hyperlink{sim_beliefs}{\beamerbutton{Return: simulation}}}
% \hyperlink{belief_effect}{\beamerbutton{Return: simulation}}


\end{frame}




%%%%%%%%%%%%%%%%%%%%%%%%%%%%%%%%%%%%%%%%%%%%%%%%%%%%%%%%%%%%%%%%%%%%%%%%%%%%%%%%
\begin{frame}{Group-based parametrization}



Consider group-based beliefs about abilities:
\begin{itemize}
    \item Each individual has a group type: $g \in \{m, f\}$

    \item Students form beliefs, $P_{j0}$, based on previously observed group successes
\end{itemize}

\vspace{2ex}
Simple parameterization:
\begin{itemize}
    \item [$\alpha_{j0}^g$: ] Number of type-$g$ students who have succeeded in $j$ 
    \item [$\beta_{j0}^g$: ] Number of type-$g$ students who have failed in $j$

    \item [$\implies$] Observed success rate:
    \begin{equation*}
    \mu_{j0}^g = 
      \frac{\alpha_{j0}^g}{\alpha_{j0}^g + \beta_{j0}^g}.
\end{equation*}
This average is based on a sample size of type $g$ students:
\begin{equation*}
    n_{j0}^g = \alpha_{j0}^g + \beta_{j0}^g
 \end{equation*}
\end{itemize}

\vspace{2ex}
Group-based prior beliefs about probability of success in skill-$j$ courses, $\theta_j$:
\begin{equation*}
    \mathcal{B} \pr{\alpha_{j0}^g, \beta_{j0}^g} \quad \implies \quad
    \mathcal{B} \pr{\mu_{j0}^g n_{j0}^g, (1 - \mu_{j0}^g) n_{j0}^g}
\end{equation*}



\end{frame}

%%%%%%%%%%%%%%%%%%%%%%%%%%%%%%%%%%%%%%%%%%%%%%%%%%%%%%%%%%%%%%%%%%%%%%%%%%%%%%%%
\begin{frame}{Group-based belief distribution}

% Suppose sample size of men is larger than that of women, but the observed success rate is the same for the two groups:
Suppose there are more men then women in field $j$: 
\begin{equation*}
  n_{j0}^m > n_{j0}^f
\end{equation*}
But the observed success rate is the same for the two groups:
\begin{equation*}
    \mu_{j0} = \mu_{j0}^m = \mu_{j0}^w
\end{equation*}

\begin{figure}
\input{beta_example_gender.tex}
\end{figure}


\end{frame}


%%%%%%%%%%%%%%%%%%%%%%%%%%%%%%%%%%%%%%%%%%%%%%%%%%%%%%%%%%%%%%%%%%%%%%%%%%%%%%%%

% \nts{Frame remaining simple parameterization}
\begin{frame}{Individual problem}


A policy $\pi: (h_t, P_t^g) \to (s_t, \ell_t)$ is optimal if it maximizes:
\begin{align*}
& \mathbb{E}^\pi \sbr{
   \sum_{t=0}^\infty \delta^t 
   \left. \pr{\sum_{j=1}^J h_{jt} w_j \ell_{jt} } \right\vert
   \pr{(h_{10}, P_{10}^g), \dots, (h_{10}, P_{J0}^g)}
} \\
\text{subject to} \quad& h_{jt+1} = h_{jt}+ a_{jt} s_{jt}, \quad \quad a_{jt} = 
   \begin{cases} 
      \nu_j, & \text{with prob. } \theta_j,  \\ 
      0, & \text{with prob. } 1 - \theta_j,
   \end{cases} 
   %\quad \quad h_{j0} = \alpha_{j0} \nu_j, 
   \\
\quad& P_{jt+1}^g = 
  \mathcal{B} (\alpha_{j,t+1}^g, \beta_{j,t+1}^g), 
  \quad \quad \theta_j \sim P_{j,0}^g \equiv \mathcal{B} (\alpha_{j0}^g, \beta_{j0}^g)
  \quad \quad \text{if $j$ selected,} \\
\quad& \sum_{j=1}^J (s_{jt} + \ell_{jt}) = 1, \quad \quad s_{jt}, \ell_{jt} \in \{0,1\} \\
   & h_{j0} \leq \nu \alpha_{j0}
\end{align*}

\end{frame}

%%%%%%%%%%%%%%%%%%%%%%%%%%%%%%%%%%%%%%%%%%%%%%%%%%%%%%%%%%%%%%%%%%%%%%%%%%%%%%%%
\begin{frame}{Optimal policy rule}

Define the skill $j$ index as the expected payoff if you committed to studying $j$:
%\gen{ 
\begin{equation*}
\mathcal{I}_j (h_j, P_j^g) = \sup_{\tau \geq 0} \mathbb{E}^\tau
\ce{
   \sum_{t=0}^\infty \delta^t \pr{h_{jt} w_j \ell_{jt} }}
   {(h_{j0}, P_{j0}^g) = (h_j, P_j^g)
}
% \mathcal{I}_{jt} (h_{jt}, \alpha_{jt}, \beta_{jt}) = 
% \begin{cases}
% \frac{h_{jt}}{1 - \delta} & \text{if } \{\alpha_{jt}, \beta_{jt}\} \in \mathcal{G}_{j}, \\
% \frac{h_{jt}}{1 - \delta} \sbr{
%    \frac{
%       \left\lceil \frac{\delta}{1 - \delta} \right\rceil
%       \delta^{\left\lceil \frac{\delta}{1 - \delta} \right\rceil - c_{jt} - \alpha_{j0} - \beta_{j0}}}
%    {c_{jt} + \alpha_{j0} + \beta_{j0}}
%    } & \text{if } \{\alpha_{jt}, \beta_{jt}\} \notin \mathcal{G}_{j} \\
% \end{cases}
\end{equation*}

Define the graduation region of skill $j$ as: 
\begin{equation*}
% \mathcal{G}_j =  \left\{ \alpha_{jt}, \beta_{jt} \left\vert c_{jt} \geq \left\lceil \frac{\delta}{1 - \delta} \right\rceil - (\alpha_{j0} + \beta_{j0}) \right. \right\}
\mathcal{G}_j = \left\{ (h_j, P_j^g) \left\vert
   \arg \max_{\tau \geq 0} 
   \mathbb{E}^\tau \ce{\sum_{t=0}^\infty \delta^t \pr{h_{jt} w_j \ell_{jt} }}
   {(h_j, P_j^g)} = 0
   \right. \right\}
\end{equation*}

The following policy $\pi: (h_t, P_t^g) \to (s_t, \ell_t)$ is optimal: 
\begin{enumerate}
    \item At each $t \geq 0$, choose skill $j^* = \arg \max_{j \in J} \mathcal{I}_j$, breaking ties according to any rule
    \item If $(h_{j^*}, P_{j^*}^g) \in \mathcal{G}_{j}$, then enter the labor market as a $j^*$ specialist. Otherwise, study $j^*$ for an additional period.  
\end{enumerate}

\end{frame}

%%%%%%%%%%%%%%%%%%%%%%%%%%%%%%%%%%%%%%%%%%%%%%%%%%%%%%%%%%%%%%%%%%%%%%%%%%%%%%%%

%%%%%%%%%%%%%%%%%%%%%%%%%%%%%%%%%%%%%%%%%%%%%%%%%%%%%%%%%%%%%%%%%%%%%%%%%%%%%%%


%%%%%%%%%%%%%%%%%%%%%%%%%%%%%%%%%%%%%%%%%%%%%%%%%%%%%%%%%%%%%%%%%%%%%%%%%%%%%%%

%%%%%%%%%%%%%%%%%%%%%%%%%%%%%%%%%%%%%%%%%%%%%%%%%%%%%%%%%%%%%%%%%%%%%%%%%%%%%%%
\miniframesoff
\section[Simulations]{Model Simulations}
\begin{frame}
    \tableofcontents[currentsection]
\end{frame}
\miniframeson

%%%%%%%%%%%%%%%%%%%%%%%%%%%%%%%%%%%%%%%%%%%%%%%%%%%%%%%%%%%%%%%%%%%%%%%%%%%%%
\begin{frame}{Simulate agent behavior}\label{simulate}

% To consider how my model can explain human capital sepcialization decisions

How can the model explain different specialization outcomes?

\vspace{3ex}
Consider a world with two fields, X and Y
\begin{itemize}
    \item Wages are equal: $w_X = w_Y$
    \item The agent's probabilities of success are equal: $\theta_X = \theta_Y$
    \item Initial beliefs are equal to the uninform prior: \hyperlink{model_beta_11}{\beamerbutton{PDF of beliefs}}
    \begin{equation*}
        (\alpha_{X0}, \beta_{X0}) = (\alpha_{Y0}, \beta_{Y0}) = (1, 1)
    \end{equation*}
    \item Assume $h_{j0} = \nu \alpha_{j0}$ \hyperlink{sim_parameterization}{\beamerbutton{Details}}
\end{itemize}

\vspace{3ex}
Simulate agent's specialization decisions when choosing between X and Y
\begin{itemize}
    \item Model fraction of simulated agents choosing X or Y at time $t$ 
\end{itemize}


\end{frame}

%%%%%%%%%%%%%%%%%%%%%%%%%%%%%%%%%%%%%%%%%%%%%%%%%%%%%%%%%%%%%%%%%%%%%%%%%%%%%%%%
\begin{frame}{Default parameterization}\label{sim_default}

% 
\begin{figure}
\centering
% This file was created by tikzplotlib v0.9.2.
\begin{tikzpicture}

\definecolor{color0}{rgb}{0.266666666666667,0.466666666666667,0.666666666666667}
\definecolor{color1}{rgb}{0.933333333333333,0.4,0.466666666666667}

\begin{axis}[
height=6.6314113761540705cm,
tick align=outside,
tick pos=left,
title={Baseline Simulation},
width=9.904475999999999cm,
x grid style={white!69.0196078431373!black},
xlabel={t},
xmin=-1.05, xmax=27,
xtick style={color=black},
xtick={0,5,10,15,20},
xticklabels={\(\displaystyle 0\),\(\displaystyle 5\),\(\displaystyle 10\),\(\displaystyle 15\),\(\displaystyle 20\)},
ylabel={Fraction students enrolled in field},
ymajorgrids,
ymin=0, ymax=1,
ytick style={color=black},
ytick={0,0.25,0.5,0.75,1},
yticklabels={\(\displaystyle 0\),\(\displaystyle 0.25\),\(\displaystyle 0.5\),\(\displaystyle 0.75\),\(\displaystyle 1\)}
]
\addplot [semithick, color0]
table {%
0 0.505500078201294
1 0.503099918365479
2 0.502699971199036
3 0.498600006103516
4 0.502599954605103
5 0.502599954605103
6 0.503499984741211
7 0.506099939346313
8 0.505100011825562
9 0.505399942398071
10 0.505300045013428
11 0.506900072097778
12 0.506399989128113
13 0.506500005722046
14 0.506999969482422
15 0.506700038909912
16 0.506799936294556
17 0.506500005722046
20 0.506500005722046
21 0.506500005722046
};
\addplot [semithick, color1]
table {%
0 0.494500041007996
1 0.496899962425232
2 0.497300028800964
3 0.501399993896484
4 0.497400045394897
5 0.497400045394897
6 0.496500015258789
7 0.493900060653687
8 0.494899988174438
9 0.494600057601929
10 0.494699954986572
11 0.493099927902222
12 0.493600010871887
13 0.493499994277954
14 0.493000030517578
15 0.493299961090088
16 0.493200063705444
17 0.493499994277954
20 0.493499994277954
21 0.493499994277954
};
\draw (axis cs:21.5,0.4265) node[
  anchor=base west,
  text=color0,
  rotate=0.0
]{Field X};
\draw (axis cs:21.5,0.5235) node[
  anchor=base west,
  text=color1,
  rotate=0.0
]{Field Y};
\draw (axis cs:17,0.03) node[
  anchor=base west,
  text=black,
  rotate=0.0
]{N simulations = 10,000};
\end{axis}

\end{tikzpicture}

\end{figure}

\hyperlink{model_beta_11}{\beamerbutton{PDF of beliefs}}
\end{frame}

%%%%%%%%%%%%%%%%%%%%%%%%%%%%%%%%%%%%%%%%%%%%%%%%%%%%%%%%%%%%%%%%%%%%%%%%%%%%%%%%
\begin{frame}{Zooming in}

% should be symmetric

\begin{figure}
\centering
% This file was created by tikzplotlib v0.9.2.
\begin{tikzpicture}

\definecolor{color0}{rgb}{0.266666666666667,0.466666666666667,0.666666666666667}
\definecolor{color1}{rgb}{0.933333333333333,0.4,0.466666666666667}

\begin{axis}[
height=6.6314113761540705cm,
tick align=outside,
tick pos=left,
title={Baseline Simulation (zoomed in)},
width=9.904475999999999cm,
x grid style={white!69.0196078431373!black},
xlabel={t},
xmin=-1.05, xmax=27,
xtick style={color=black},
xtick={0,5,10,15,20},
xticklabels={\(\displaystyle 0\),\(\displaystyle 5\),\(\displaystyle 10\),\(\displaystyle 15\),\(\displaystyle 20\)},
ylabel={Fraction students enrolled in field},
ymajorgrids,
ymin=0, ymax=1,
ytick style={color=black},
ytick={0,0.25,0.5,0.75,1},
yticklabels={\(\displaystyle 0\),\(\displaystyle 0.25\),\(\displaystyle 0.5\),\(\displaystyle 0.75\),\(\displaystyle 1\)}
]
\addplot [semithick, color0]
table {%
0 0.5
1 0.319999933242798
2 0.360000014305115
3 0.379999995231628
4 0.360000014305115
5 0.379999995231628
6 0.360000014305115
7 0.379999995231628
8 0.360000014305115
9 0.379999995231628
13 0.379999995231628
14 0.399999976158142
15 0.379999995231628
21 0.379999995231628
};
\addplot [semithick, color1]
table {%
0 0.5
1 0.680000066757202
2 0.639999985694885
3 0.620000004768372
4 0.639999985694885
5 0.620000004768372
6 0.639999985694885
7 0.620000004768372
8 0.639999985694885
9 0.620000004768372
13 0.620000004768372
14 0.600000023841858
15 0.620000004768372
21 0.620000004768372
};
\draw (axis cs:21.5,0.3) node[
  anchor=base west,
  text=color0,
  rotate=0.0
]{Field X};
\draw (axis cs:21.5,0.65) node[
  anchor=base west,
  text=color1,
  rotate=0.0
]{Field Y};
\draw (axis cs:18,0.03) node[
  anchor=base west,
  text=black,
  rotate=0.0
]{N simulations = 50};
\end{axis}

\end{tikzpicture}

\end{figure}


\end{frame}

%%%%%%%%%%%%%%%%%%%%%%%%%%%%%%%%%%%%%%%%%%%%%%%%%%%%%%%%%%%%%%%%%%%%%%%%%%%%%%%%
\begin{frame}{Wage effects}

% 

\begin{figure}
\centering
% This file was created by tikzplotlib v0.9.1.
\begin{tikzpicture}

\definecolor{color0}{rgb}{0.266666666666667,0.466666666666667,0.666666666666667}
\definecolor{color1}{rgb}{0.933333333333333,0.4,0.466666666666667}

\begin{axis}[
height=207pt,
tick align=outside,
tick pos=left,
title={Field selection and wages \\ Field X: w = 1.0; Field Y: w = 1.5},
width=240pt,
x grid style={white!69.0196078431373!black},
xlabel={\(\displaystyle t\)},
xmin=-1.05, xmax=25,
xtick style={color=black},
xtick={0,5,10,15,20},
xticklabels={\(\displaystyle 0\),\(\displaystyle 5\),\(\displaystyle 10\),\(\displaystyle 15\),\(\displaystyle 20\)},
ylabel={Fraction students enrolled in field},
ymajorgrids,
ymin=0, ymax=1,
ytick style={color=black},
ytick={0,0.2,0.4,0.6,0.8,1},
yticklabels={\(\displaystyle 0\),\(\displaystyle 0.2\),\(\displaystyle 0.4\),\(\displaystyle 0.6\),\(\displaystyle 0.8\),\(\displaystyle 1\)}
]
\addplot [semithick, color0, mark=x, mark size=3, mark options={solid}]
table {%
0 0
1 0
2 0.28
3 0.16
4 0.22
5 0.18
6 0.19
7 0.19
8 0.19
9 0.17
10 0.18
11 0.16
12 0.16
13 0.16
14 0.17
15 0.17
16 0.17
17 0.17
18 0.17
19 0.17
20 0.17
21 0.17
};
\addplot [semithick, color1, mark=x, mark size=3, mark options={solid}]
table {%
0 1
1 1
2 0.72
3 0.84
4 0.78
5 0.82
6 0.81
7 0.81
8 0.81
9 0.83
10 0.82
11 0.84
12 0.84
13 0.84
14 0.83
15 0.83
16 0.83
17 0.83
18 0.83
19 0.83
20 0.83
21 0.83
};
\draw (axis cs:20.5,0.2) node[
  anchor=base west,
  text=color0,
  rotate=0.0
]{Field X};
\draw (axis cs:20.5,0.86) node[
  anchor=base west,
  text=color1,
  rotate=0.0
]{Field Y};
\draw (axis cs:14,0.03) node[
  anchor=base west,
  text=black,
  rotate=0.0
]{N simulations = 100};
\end{axis}

\end{tikzpicture}

\end{figure}

\end{frame}

%%%%%%%%%%%%%%%%%%%%%%%%%%%%%%%%%%%%%%%%%%%%%%%%%%%%%%%%%%%%%%%%%%%%%%%%%%%%%%%%
\begin{frame}{Belief effects}\label{sim_beliefs}

% risk aspect is on human capital accumulation
% your expected probability of gaining human capital is the same in both of these
% there's risk in timing - risk aversion over time; probably from discounting with linear utility function

%%%%%%%%%%%%
% Start going into y because you have more information
% more quickly will converge into what you go at 

% If you'er going to switch, you know sooner

% In expectation 
% What drives length to degree?
% AFter two periods you have sunk benefit


% Inforamtion value outweight human capital
% risk neutral is human capital 

% spend more in uncertain major 

% benefit 

% why do you spend less time studying when you  have more certainty

% if distribution is tighter, initial 

% any signal makes you more likely to switch if you start in more dispersed prior
% swtiching is bad; want to minimize time you spend in school
% go with option for which you 

% c_j depends on alpha0 and beta0

% poteriro vs posterior predictive

% Y provides more 
% you want to accumulate the most human capital in the shortest amount of time
% want to choose the field to get you to 10 suc

\begin{figure}
\centering
% This file was created by tikzplotlib v0.9.2.
\begin{tikzpicture}

\definecolor{color0}{rgb}{0.266666666666667,0.466666666666667,0.666666666666667}
\definecolor{color1}{rgb}{0.933333333333333,0.4,0.466666666666667}

\begin{axis}[
height=5.101085673964669cm,
tick align=outside,
tick pos=left,
title={Field selection and initial beliefs \\ Field X: \(\displaystyle (\alpha_0, \beta_0)=(1, 1)\); Field Y: \(\displaystyle (\alpha_0, \beta_0)=(2, 2)\)},
width=8.25373cm,
x grid style={white!69.0196078431373!black},
xmin=-0.95, xmax=23.75,
xtick style={color=black},
xtick={0,5,10,15},
xticklabels={\(\displaystyle 0\),\(\displaystyle 5\),\(\displaystyle 10\),\(\displaystyle 15\)},
ymajorgrids,
ymin=-0.05, ymax=1.05,
ytick style={color=black},
ytick={0,0.2,0.4,0.6,0.8,1},
yticklabels={\(\displaystyle 0\),\(\displaystyle 0.2\),\(\displaystyle 0.4\),\(\displaystyle 0.6\),\(\displaystyle 0.8\),\(\displaystyle 1\)}
]
\addplot [semithick, color0]
table {%
0 0
1 0.483
2 0.25
3 0.25
4 0.281
5 0.266
6 0.249
7 0.236
8 0.233
9 0.247
10 0.241
11 0.24
12 0.238
13 0.237
14 0.237
15 0.237
16 0.236
17 0.237
18 0.237
19 0.237
};
\addplot [semithick, color1]
table {%
0 1
1 0.517
2 0.75
3 0.75
4 0.719
5 0.734
6 0.751
7 0.764
8 0.767
9 0.753
10 0.759
11 0.76
12 0.762
13 0.763
14 0.763
15 0.763
16 0.764
17 0.763
18 0.763
19 0.763
};
\draw (axis cs:19.5,0.237) node[
  anchor=base west,
  text=color0,
  rotate=0.0
]{Field X};
\draw (axis cs:19.5,0.763) node[
  anchor=base west,
  text=color1,
  rotate=0.0
]{Field Y};
\end{axis}

\end{tikzpicture}

\end{figure}
\hyperlink{model_beta_22}{\beamerbutton{PDF of beliefs}}
\hyperlink{sim_parameterization}{\beamerbutton{Parameterization}}
\hyperlink{app_ability_v_effect}{\beamerbutton{
    Let $\alpha_{X0} \nu_X = \alpha_{Y0} \nu_Y$
}}

\end{frame}

%%%%%%%%%%%%%%%%%%%%%%%%%%%%%%%%%%%%%%%%%%%%%%%%%%%%%%%%%%%%%%%%%%%%%%%%%%%%%%%%
\begin{frame}{Ability to succeed}\label{sim_ability}

\begin{figure}
\centering
% This file was created by tikzplotlib v0.9.2.
\begin{tikzpicture}

\definecolor{color0}{rgb}{0.266666666666667,0.466666666666667,0.666666666666667}
\definecolor{color1}{rgb}{0.933333333333333,0.4,0.466666666666667}

\begin{axis}[
height=5.101085673964669cm,
tick align=outside,
tick pos=left,
title={Field selection and ability to succeed \\ Field X: \(\displaystyle \theta=0.25\); Field Y: \(\displaystyle \theta=0.75\)},
width=8.25373cm,
x grid style={white!69.0196078431373!black},
xmin=-1.05, xmax=26.25,
xtick style={color=black},
xtick={0,5,10,15,20},
xticklabels={\(\displaystyle 0\),\(\displaystyle 5\),\(\displaystyle 10\),\(\displaystyle 15\),\(\displaystyle 20\)},
ymajorgrids,
ymin=0, ymax=1,
ytick style={color=black},
ytick={0,0.2,0.4,0.6,0.8,1},
yticklabels={\(\displaystyle 0\),\(\displaystyle 0.2\),\(\displaystyle 0.4\),\(\displaystyle 0.6\),\(\displaystyle 0.8\),\(\displaystyle 1\)}
]
\addplot [semithick, color0]
table {%
0 0.484
1 0.23
2 0.228
3 0.151
4 0.138
5 0.134
6 0.118
7 0.104
8 0.105
9 0.109
10 0.105
11 0.097
12 0.1
13 0.1
14 0.096
15 0.096
16 0.096
17 0.095
18 0.096
19 0.096
20 0.096
21 0.096
};
\addplot [semithick, color1]
table {%
0 0.516
1 0.77
2 0.772
3 0.849
4 0.862
5 0.866
6 0.882
7 0.896
8 0.895
9 0.891
10 0.895
11 0.903
12 0.9
13 0.9
14 0.904
15 0.904
16 0.904
17 0.905
18 0.904
19 0.904
20 0.904
21 0.904
};
\draw (axis cs:21.5,0.126) node[
  anchor=base west,
  text=color0,
  rotate=0.0
]{Field X};
\draw (axis cs:21.5,0.934) node[
  anchor=base west,
  text=color1,
  rotate=0.0
]{Field Y};
\end{axis}

\end{tikzpicture}

\end{figure}
\hyperlink{app_v_effects}{\beamerbutton{$\nu$ simulation}}

\end{frame}
%%%%%%%%%%%%%%%%%%%%%%%%%%%%%%%%%%%%%%%%%%%%%%%%%%%%%%%%%%%%%%%%%%%%%%%%%%%%%%%

%%%%%%%%%%%%%%%%%%%%%%%%%%%%%%%%%%%%%%%%%%%%%%%%%%%%%%%%%%%%%%%%%%%%%%%%%%%%%%%
\miniframesoff
\section[Calibration]{Calibration}
\begin{frame}
    \tableofcontents[currentsection]
\end{frame}
\miniframeson
%%%%%%%%%%%%%%%%%%%%%%%%%%%%%%%%%%%%%%%%%%%%%%%%%%%%%%%%%%%%%%%%%%%%%%%%%%%%%%%

%%%%%%%%%%%%%%%%%%%%%%%%%%%%%%%%%%%%%%%%%%%%%%%%%%%%%%%%%%%%%%%%%%%%%%%%%%%%%%%%
\begin{frame}{Identification problem}\label{identification}

Identification problem: how to identify $(\alpha_{j0}^g, \beta_{j0}^g)$?
\begin{itemize}
    \item Parameters that characterize prior beliefs about probability of success ($\theta_j$)
\end{itemize}

% \vspace{2ex} 
% Transcript data available via BPS surveys or NLSY97
% \begin{itemize}
%     \item 
% \end{itemize}

\vspace{2ex}
The following state variables are observable using transcript data (BPS or NLSY97):
\begin{align*}
    \overline{m}_{jt} &= \sum_{r=0}^{t-1} m_{jr}, \quad \quad \overline{s}_{jt} = \sum_{r=0}^{t-1} s_{jr}
\end{align*}
Goal: find likelihood function to estimate ($\alpha_{j0}^g, \beta_{j0}^g$).
\begin{itemize}
    \item Determine the best way to incorporate population heterogeneity (random utility? Semi-nonparametric approaches?)
    \item Balance endogeneity concerns with taking advantage of the recursive structure
\end{itemize}

\vspace{2ex}
Possible example: conditional logit approach:
\begin{equation*}
    \log \mathcal{L} = \log \sum_{i = 1}^n \sum_{j=1}^J m_{ijt} \log P(m_{ijt} = 1 \vert \overline{m}_{ijt}, \overline{s}_{ijt}, \alpha_{j0}, \beta_{j0}, \theta_{ji})
\end{equation*}

\end{frame}

%%%%%%%%%%%%%%%%%%%%%%%%%%%%%%%%%%%%%%%%%%%%%%%%%%%%%%%%%%%%%%%%%%%%%%%%%%%%%%%%
\begin{frame}{Useful model notes}\label{id_model_notes}

Under the parametric assumption, agents study a field for a deterministic number of periods: \ \hyperlink{sim_parameterization}{\beamerbutton{Details}}
\begin{equation*}
    m_{j}^* = \left\lceil{\frac{\delta}{1 - \delta}}\right\rceil - \alpha_{j0} - \beta_{j0}
\end{equation*}
Can analytically characterize Index in field $j$:
\begin{equation*}
    \mathcal{I}_{jt} (\alpha_{j0}, \beta_{j0}, \overline{m}_{jt}, \overline{s}_{jt}) = \begin{cases}
    \frac{w_j h_{jt}}{1 - \delta} &\text{ if } (\alpha_{j0}, \beta_{j0}) \in \mathcal{G}_j \\
    \frac{w_j}{1 - \delta} \delta^{m_j^* - \overline{m}_{jt}} \mathbb{E}_t \sbr{h_{j, t + m_j^* - \overline{m}_{jt}}} &\text{ otherwise}.
    \end{cases}
\end{equation*}
Probability an agent chooses $j$ at time $t$ depends on the index:
\begin{align*}
    G_j (x) = P \pr{\mathcal{I}_{jt} < x \vert \cdot} 
    &= P \pr{\condl{
        \delta^{-(\alpha_{j0} + \beta_{j0})} 
        \frac{\alpha_{j0} + \overline{s}_{jt}}{\alpha_{j0} + \beta_{j0} + \overline{m}_{jt}}
        < c_j x
    }
    \overline{m}_{jt}, \overline{s}_{jt}, \alpha_{j0}, \beta_{j0}
    }
\end{align*}
Under conditional independence given all states:
\begin{align*}
    P(m_{jt} = 1 \vert \cdot) &= P(\mathcal{I}_{jt} > \mathcal{I}_{kt} \forall k \neq j \vert \cdot) \\
    &= \int \prod_{k \neq j} G_k (x) d G_j (x)
\end{align*}
\hyperlink{model_optimal_rule}{\beamerbutton{Index definition}}

\end{frame}


% %%%%%%%%%%%%%%%%%%%%%%%%%%%%%%%%%%%%%%%%%%%%%%%%%%%%%%%%%%%%%%%%%%%%%%%%%%%%%%%%
% \begin{frame}{Calibration}

% \begin{table}
% \centering
% \begin{tabular}{lll}
% \hline \hline
% \textbf{Parameter} & \textbf{Description} & \textbf{Source} \\
% \hline
% $J$ & Number of fields & ACS and IPEDS data \\
% \hline
% ($\alpha_{j0}^g$, $\beta_{j0}^g$) & Beta distribution parameters \\
% \hline \hline
% \end{tabular}
% \end{table}
% \begin{itemize}
%     \item Normalizing $\nu_j$ and $w_j$ to one
%     \item Assuming $h_{j0} = \nu_j \alpha_{j0}^g$
% \end{itemize}

% Moments:
% \begin{itemize}
%     \item 
% \end{itemize}

% \end{frame}

%%%%%%%%%%%%%%%%%%%%%%%%%%%%%%%%%%%%%%%%%%%%%%%%%%%%%%%%%%%%%%%%%%%%%%%%%%%%%%%
\miniframesoff
\section[Conclusion]{Conclusion and next steps}
\begin{frame}
    \tableofcontents[currentsection]
\end{frame}
\miniframeson
%%%%%%%%%%%%%%%%%%%%%%%%%%%%%%%%%%%%%%%%%%%%%%%%%%%%%%%%%%%%%%%%%%%%%%%%%%%%%%%

%%%%%%%%%%%%%%%%%%%%%%%%%%%%%%%%%%%%%%%%%%%%%%%%%%%%%%%%%%%%%%%%%%%%%%%%%%%%%%%%
\begin{frame}{Conclusion and next steps}

Model suggests that group-based beliefs play an important role in specialization decisions

\vspace{4ex}
Next steps:
\begin{itemize}
    \item Identification of model parameters
    \item Explore counterfactuals: productivity differences if we remove misallocation of talent?
\end{itemize}

\vspace{4ex}
Additional counterfactual exercise: affirmative action 
\begin{itemize}
    \item If we remove discrimination, how long would it take for women's beliefs to converge?
    \item Can affirmative action address these biases?
\end{itemize}

\end{frame}


% %%%%%%%%%%%%%%%%%%%%%%%%%%%%%%%%%%%%%%%%%%%%%%%%%%%%%%%%%%%%%%%%%%%%%%%%%%%%%%%%
% \begin{frame}{Note on discrimination}

% Are these differential outcomes the result of discrimination? Depends on your definition:
% \begin{itemize}
%     \item Lundberg and Startz (1984): ``Economic discrimination exists when groups with equal average initial endowments of productive ability do not receive equal average compensation in equilibrium''
% \end{itemize}

% \end{frame}


%%%%%%%%%%%%%%%%%%%%%%%%%%%%%%%%%%%%%%%%%%%%%%%%%%%%%%%%%%%%%%%%%%%%%%%%%%%%%%%%
% Appendix
%%%%%%%%%%%%%%%%%%%%%%%%%%%%%%%%%%%%%%%%%%%%%%%%%%%%%%%%%%%%%%%%%%%%%%%%%%%%%%%%
\miniframesoff
\begin{frame}[noframenumbering]
\begin{beamercolorbox}[sep=11pt,center]{title}
Appendix
\end{beamercolorbox}
\end{frame}
\tikzset{external/figure name={appendix_}}
%%%%%%%%%%%%%%%%%%%%%%%%%%%%%%%%%%%%%%%%%%%%%%%%%%%%%%%%%%%%%%%%%%%%%%%%%%%%%%%

%%%%%%%%%%%%%%%%%%%%%%%%%%%%%%%%%%%%%%%%%%%%%%%%%%%%%%%%%%%%%%%%%%%%%%%%%%%%%%%%
\begin{frame}{}\label{app_social_science_num}
\Wider[4em]{

\begin{figure}
\input{social_science_area.tex}
\end{figure}
\hyperlink{intro_social_science_ratio}{\beamerbutton{Return: Social science ratio}}
}
\end{frame}

%%%%%%%%%%%%%%%%%%%%%%%%%%%%%%%%%%%%%%%%%%%%%%%%%%%%%%%%%%%%%%%%%%%%%%%%%%%%%%%%
\begin{frame}{Social Sciences}\label{app_social_science_cip}
\Wider[4em]{

\begin{figure}
\setlength{\abovecaptionskip}{2pt}
\setlength{\belowcaptionskip}{-2pt}
\input{cip45.tex}
\end{figure}
\hyperlink{intro_social_science_ratio}{\beamerbutton{Return: Social science ratio}}
}
\end{frame}

%%%%%%%%%%%%%%%%%%%%%%%%%%%%%%%%%%%%%%%%%%%%%%%%%%%%%%%%%%%%%%%%%%%%%%%%%%%%%%%%
\begin{frame}{Engineering}\label{app_engineering}
\Wider[4em]{

\begin{figure}
\setlength{\abovecaptionskip}{2pt}
\setlength{\belowcaptionskip}{-2pt}
\input{cip14.tex}
\end{figure}
\hyperlink{intro_social_science_ratio}{\beamerbutton{Return: Social science ratio}}
}
\end{frame}

%%%%%%%%%%%%%%%%%%%%%%%%%%%%%%%%%%%%%%%%%%%%%%%%%%%%%%%%%%%%%%%%%%%%%%%%%%%%%%%%
\begin{frame}{Business}\label{app_business}
\Wider[4em]{

\begin{figure}
\setlength{\abovecaptionskip}{2pt}
\setlength{\belowcaptionskip}{-2pt}
\input{cip52.tex}
\end{figure}
\hyperlink{intro_social_science_ratio}{\beamerbutton{Return: Social science ratio}}
}
\end{frame}

%%%%%%%%%%%%%%%%%%%%%%%%%%%%%%%%%%%%%%%%%%%%%%%%%%%%%%%%%%%%%%%%%%%%%%%%%%%%%%%%
\begin{frame}{Computer Science}\label{app_computer_science}
\Wider[4em]{

\begin{figure}
\setlength{\abovecaptionskip}{2pt}
\setlength{\belowcaptionskip}{-2pt}
% This file was created by tikzplotlib v0.9.2.
\begin{tikzpicture}

\definecolor{color0}{rgb}{0.266666666666667,0.466666666666667,0.666666666666667}
\definecolor{color1}{rgb}{0.933333333333333,0.4,0.466666666666667}
\definecolor{color2}{rgb}{0.133333333333333,0.533333333333333,0.2}
\definecolor{color3}{rgb}{0.8,0.733333333333333,0.266666666666667}
\definecolor{color4}{rgb}{0.4,0.8,0.933333333333333}
\definecolor{color5}{rgb}{0.666666666666667,0.2,0.466666666666667}

\begin{axis}[
height=140pt,
tick align=outside,
tick pos=left,
title={Ratio of women to men},
unbounded coords=jump,
width=300pt,
x grid style={white!69.0196078431373!black},
xmin=1988.6, xmax=2031,
xtick style={color=black},
xtick={1990,1995,2000,2005,2010,2015},
xticklabels={\(\displaystyle 1990\),\(\displaystyle 1995\),\(\displaystyle 2000\),\(\displaystyle 2005\),\(\displaystyle 2010\),\(\displaystyle 2015\)},
ymajorgrids,
ymin=0, ymax=1,
ytick style={color=black},
ytick={0,0.1,0.2,0.3,0.4,0.5,0.6,0.7,0.8,0.9,1},
yticklabels={\(\displaystyle 0\),\(\displaystyle 0.1\),\(\displaystyle 0.2\),\(\displaystyle 0.3\),\(\displaystyle 0.4\),\(\displaystyle 0.5\),\(\displaystyle 0.6\),\(\displaystyle 0.7\),\(\displaystyle 0.8\),\(\displaystyle 0.9\),\(\displaystyle 1\)}
]
\addplot [semithick, color0]
table {%
1990 0.394967913627625
1991 0.37271773815155
1992 0.370465278625488
1993 0.345574140548706
1994 0.365840792655945
1995 0.351528167724609
1996 0.344272255897522
1997 0.3332200050354
1998 0.321436882019043
1999 0.320170640945435
2000 0.343914747238159
2001 0.338396310806274
2002 0.337104439735413
2003 0.332112669944763
2004 0.320477366447449
2005 0.272923469543457
2006 0.251452445983887
2007 0.225056648254395
2008 0.206436157226562
2009 0.225293636322021
2010 0.219744563102722
2011 0.207880258560181
2012 0.208938837051392
2013 0.217817306518555
2014 0.214048385620117
2015 0.221825838088989
2016 0.223389029502869
2017 0.226261734962463
2018 0.240228176116943
};
\addplot [semithick, color1]
table {%
1990 nan
1991 nan
1992 0.3265465935787
1993 0.302248126561199
1994 0.2999222999223
1995 0.304198473282443
1996 0.278148148148148
1997 0.276795380728979
1998 0.283663704716336
1999 0.287935502447452
2000 0.283360408447436
2001 0.276973281664287
2002 0.289675114815445
2003 0.270382165605096
2004 0.242696416753333
2005 0.207340962985875
2006 0.182779629842511
2007 0.155608214849921
2008 0.13874859708193
2009 0.14136500214623
2010 0.146981988175443
2011 0.145420088439671
2012 0.155357142857143
2013 0.15297619047619
2014 0.173902111967818
2015 0.1828334168516
2016 0.205791482401405
2017 0.2134888957118
2018 0.231927175843694
};
\addplot [semithick, color2]
table {%
1990 0.612352168199737
1991 0.632269348491474
1992 0.601075750784402
1993 0.669402644778842
1994 0.580039920159681
1995 0.66078753076292
1996 0.590057361376673
1997 0.591881116346502
1998 0.586007702182285
1999 0.599660729431722
2000 0.615333035315154
2001 0.570116566584246
2002 0.560919540229885
2003 0.552210724365005
2004 0.508236165093467
2005 0.431290064102564
2006 0.383183709218305
2007 0.338664635694339
2008 0.311026615969582
2009 0.309437854564208
2010 0.316780821917808
2011 0.321109328443731
2012 0.309195912927588
2013 0.279330792037272
2014 0.293806030969845
2015 0.291763791763792
2016 0.319848771266541
2017 0.335242527295507
2018 0.340254305488492
};
\addplot [semithick, color3]
table {%
1990 nan
1991 nan
1992 0.757575757575758
1993 0.676470588235294
1994 0.92
1995 0.619047619047619
1996 0.257142857142857
1997 0.8
1998 0.515151515151515
1999 0.394736842105263
2000 0.142857142857143
2001 0.655172413793103
2002 0.456521739130435
2003 0.622327790973872
2004 0.366598778004073
2005 0.26284751474305
2006 0.234259259259259
2007 0.212081418253447
2008 0.240674501788452
2009 0.20728441349759
2010 0.24040404040404
2011 0.192297111416781
2012 0.213741987179487
2013 0.203204661325564
2014 0.190322580645161
2015 0.198224852071006
2016 0.193636363636364
2017 0.198331193838254
2018 0.209817893903405
};
\addplot [semithick, color4]
table {%
1990 nan
1991 nan
1992 nan
1993 nan
1994 nan
1995 nan
1996 nan
1997 nan
1998 nan
1999 nan
2000 nan
2001 nan
2002 nan
2003 0.453731343283582
2004 0.48406862745098
2005 0.52530779753762
2006 0.540609137055838
2007 0.338213762811127
2008 0.324987963408763
2009 0.334102445777573
2010 0.335807050092764
2011 0.385281385281385
2012 0.495648734177215
2013 0.508133230054222
2014 0.511384845091452
2015 0.514325974635979
2016 0.573572120038722
2017 0.684032802701399
2018 0.766576819407008
};
\addplot [semithick, color5]
table {%
1990 0.496227510156703
1991 0.559025133282559
1992 0.573730862207897
1993 0.585096596136155
1994 0.581833761782348
1995 0.555163283318623
1996 0.556741028128031
1997 0.532289628180039
1998 0.498251748251748
1999 0.57484076433121
2000 0.55296343001261
2001 0.571029529130088
2002 0.514033427940713
2003 0.465948777648428
2004 0.418907007492287
2005 0.37811320754717
2006 0.318900343642612
2007 0.268292682926829
2008 0.230987246102976
2009 0.21584440227704
2010 0.220114689016321
2011 0.226400613967767
2012 0.244788164088769
2013 0.206656101426307
2014 0.227162209100382
2015 0.204568700988749
2016 0.227431770468859
2017 0.202842873607376
2018 0.221844453888653
};
\addplot [semithick, white!73.3333333333333!black]
table {%
1990 nan
1991 nan
1992 0.333333333333333
1993 0.609375
1994 0.417910447761194
1995 0.371794871794872
1996 0.308411214953271
1997 0.290123456790123
1998 0.441913439635535
1999 0.318181818181818
2000 0.358090185676393
2001 0.359615384615385
2002 0.35969387755102
2003 0.326538931920098
2004 0.284086112283664
2005 0.241577335375191
2006 0.243598862019915
2007 0.215324927255092
2008 0.176638176638177
2009 0.113035551504102
2010 0.137533274179237
2011 0.119799139167862
2012 0.113329040566645
2013 0.110236220472441
2014 0.114686951433587
2015 0.107761027359017
2016 0.130511463844797
2017 0.125265392781316
2018 0.181153533712429
};
\draw (axis cs:2018.5,0.300228192859772) node[
  anchor=base west,
  text=color0,
  rotate=0.0
]{General};
\draw (axis cs:2018.5,0.231927175843694) node[
  anchor=base west,
  text=color1,
  rotate=0.0
]{Computer science};
\draw (axis cs:2018.5,0.420254305488492) node[
  anchor=base west,
  text=color2,
  rotate=0.0
]{Information science};
\draw (axis cs:2018.5,0.109817893903405) node[
  anchor=base west,
  text=color3,
  rotate=0.0
]{IT};
\draw (axis cs:2018.5,0.766576819407008) node[
  anchor=base west,
  text=color4,
  rotate=0.0
]{Media applications};
\draw (axis cs:2018.5,0.171844453888653) node[
  anchor=base west,
  text=color5,
  rotate=0.0
]{Other};
\draw (axis cs:2018.5,0.0411535337124289) node[
  anchor=base west,
  text=white!73.3333333333333!black,
  rotate=0.0
]{Systems network};
\end{axis}

\end{tikzpicture}

\vspace{0.1cm}
\begin{tikzpicture}
% This file was created by tikzplotlib v0.9.2.
\definecolor{color0}{rgb}{0.266666666666667,0.466666666666667,0.666666666666667}
\definecolor{color1}{rgb}{0.933333333333333,0.4,0.466666666666667}
\definecolor{color2}{rgb}{0.133333333333333,0.533333333333333,0.2}
\definecolor{color3}{rgb}{0.8,0.733333333333333,0.266666666666667}
\definecolor{color4}{rgb}{0.4,0.8,0.933333333333333}
\definecolor{color5}{rgb}{0.666666666666667,0.2,0.466666666666667}

\begin{groupplot}[group style={group size=2 by 1, group name=my plots, horizontal sep=0.8cm}]
\nextgroupplot[
height=90pt, width=160pt,
legend style={at={(2.02, 0.5)},anchor=west,},
reverse legend,
tick align=outside,
tick pos=left,
x grid style={white!69.0196078431373!black},
xlabel={Women},
xmin=1988.6, xmax=2019.4,
xtick style={color=black},
xtick={1990,2000,2010},
xticklabels={\(\displaystyle 1990\),\(\displaystyle 2000\),\(\displaystyle 2010\)},
ymajorgrids,
ymin=0, ymax=17.94555,
ytick style={color=black}
]
\path [draw=color0, fill=color0]
(axis cs:1990,6.028)
--(axis cs:1990,0)
--(axis cs:1991,0)
--(axis cs:1992,0)
--(axis cs:1993,0)
--(axis cs:1994,0)
--(axis cs:1995,0)
--(axis cs:1996,0)
--(axis cs:1997,0)
--(axis cs:1998,0)
--(axis cs:1999,0)
--(axis cs:2000,0)
--(axis cs:2001,0)
--(axis cs:2002,0)
--(axis cs:2003,0)
--(axis cs:2004,0)
--(axis cs:2005,0)
--(axis cs:2006,0)
--(axis cs:2007,0)
--(axis cs:2008,0)
--(axis cs:2009,0)
--(axis cs:2010,0)
--(axis cs:2011,0)
--(axis cs:2012,0)
--(axis cs:2013,0)
--(axis cs:2014,0)
--(axis cs:2015,0)
--(axis cs:2016,0)
--(axis cs:2017,0)
--(axis cs:2018,0)
--(axis cs:2018,6.527)
--(axis cs:2018,6.527)
--(axis cs:2017,5.483)
--(axis cs:2016,4.85)
--(axis cs:2015,4.291)
--(axis cs:2014,3.736)
--(axis cs:2013,3.423)
--(axis cs:2012,3.048)
--(axis cs:2011,2.944)
--(axis cs:2010,2.976)
--(axis cs:2009,3.203)
--(axis cs:2008,2.489)
--(axis cs:2007,3.679)
--(axis cs:2006,4.501)
--(axis cs:2005,5.566)
--(axis cs:2004,6.955)
--(axis cs:2003,6.439)
--(axis cs:2002,6.913)
--(axis cs:2001,6.461)
--(axis cs:2000,5.612)
--(axis cs:1999,4.354)
--(axis cs:1998,4.152)
--(axis cs:1997,3.922)
--(axis cs:1996,3.955)
--(axis cs:1995,4.083)
--(axis cs:1994,4.164)
--(axis cs:1993,4.15)
--(axis cs:1992,4.34)
--(axis cs:1991,5.328)
--(axis cs:1990,6.028)
--cycle;

\path [draw=color1, fill=color1]
(axis cs:1990,6.028)
--(axis cs:1990,6.028)
--(axis cs:1991,5.328)
--(axis cs:1992,4.34)
--(axis cs:1993,4.15)
--(axis cs:1994,4.164)
--(axis cs:1995,4.083)
--(axis cs:1996,3.955)
--(axis cs:1997,3.922)
--(axis cs:1998,4.152)
--(axis cs:1999,4.354)
--(axis cs:2000,5.612)
--(axis cs:2001,6.461)
--(axis cs:2002,6.913)
--(axis cs:2003,6.439)
--(axis cs:2004,6.955)
--(axis cs:2005,5.566)
--(axis cs:2006,4.501)
--(axis cs:2007,3.679)
--(axis cs:2008,2.489)
--(axis cs:2009,3.203)
--(axis cs:2010,2.976)
--(axis cs:2011,2.944)
--(axis cs:2012,3.048)
--(axis cs:2013,3.423)
--(axis cs:2014,3.736)
--(axis cs:2015,4.291)
--(axis cs:2016,4.85)
--(axis cs:2017,5.483)
--(axis cs:2018,6.527)
--(axis cs:2018,11.75)
--(axis cs:2018,11.75)
--(axis cs:2017,9.655)
--(axis cs:2016,8.247)
--(axis cs:2015,6.845)
--(axis cs:2014,5.811)
--(axis cs:2013,4.965)
--(axis cs:2012,4.44)
--(axis cs:2011,4.095)
--(axis cs:2010,4.045)
--(axis cs:2009,4.191)
--(axis cs:2008,3.478)
--(axis cs:2007,4.861)
--(axis cs:2006,6.091)
--(axis cs:2005,7.577)
--(axis cs:2004,9.522)
--(axis cs:2003,8.986)
--(axis cs:2002,8.616)
--(axis cs:2001,7.819)
--(axis cs:2000,6.833)
--(axis cs:1999,5.354)
--(axis cs:1998,4.982)
--(axis cs:1997,4.689)
--(axis cs:1996,4.706)
--(axis cs:1995,4.88)
--(axis cs:1994,4.936)
--(axis cs:1993,4.876)
--(axis cs:1992,5.174)
--(axis cs:1991,5.328)
--(axis cs:1990,6.028)
--cycle;

\path [draw=color2, fill=color2]
(axis cs:1990,7.426)
--(axis cs:1990,6.028)
--(axis cs:1991,5.328)
--(axis cs:1992,5.174)
--(axis cs:1993,4.876)
--(axis cs:1994,4.936)
--(axis cs:1995,4.88)
--(axis cs:1996,4.706)
--(axis cs:1997,4.689)
--(axis cs:1998,4.982)
--(axis cs:1999,5.354)
--(axis cs:2000,6.833)
--(axis cs:2001,7.819)
--(axis cs:2002,8.616)
--(axis cs:2003,8.986)
--(axis cs:2004,9.522)
--(axis cs:2005,7.577)
--(axis cs:2006,6.091)
--(axis cs:2007,4.861)
--(axis cs:2008,3.478)
--(axis cs:2009,4.191)
--(axis cs:2010,4.045)
--(axis cs:2011,4.095)
--(axis cs:2012,4.44)
--(axis cs:2013,4.965)
--(axis cs:2014,5.811)
--(axis cs:2015,6.845)
--(axis cs:2016,8.247)
--(axis cs:2017,9.655)
--(axis cs:2018,11.75)
--(axis cs:2018,13.864)
--(axis cs:2018,13.864)
--(axis cs:2017,11.528)
--(axis cs:2016,9.939)
--(axis cs:2015,8.347)
--(axis cs:2014,7.253)
--(axis cs:2013,6.284)
--(axis cs:2012,5.832)
--(axis cs:2011,5.496)
--(axis cs:2010,5.34)
--(axis cs:2009,5.391)
--(axis cs:2008,4.705)
--(axis cs:2007,6.195)
--(axis cs:2006,7.841)
--(axis cs:2005,9.73)
--(axis cs:2004,12.268)
--(axis cs:2003,13.095)
--(axis cs:2002,12.276)
--(axis cs:2001,11.047)
--(axis cs:2000,9.586)
--(axis cs:1999,7.475)
--(axis cs:1998,6.808)
--(axis cs:1997,6.322)
--(axis cs:1996,6.249)
--(axis cs:1995,6.491)
--(axis cs:1994,6.389)
--(axis cs:1993,6.344)
--(axis cs:1992,6.515)
--(axis cs:1991,6.774)
--(axis cs:1990,7.426)
--cycle;

\path [draw=color3, fill=color3]
(axis cs:1990,7.426)
--(axis cs:1990,7.426)
--(axis cs:1991,6.774)
--(axis cs:1992,6.515)
--(axis cs:1993,6.344)
--(axis cs:1994,6.389)
--(axis cs:1995,6.491)
--(axis cs:1996,6.249)
--(axis cs:1997,6.322)
--(axis cs:1998,6.808)
--(axis cs:1999,7.475)
--(axis cs:2000,9.586)
--(axis cs:2001,11.047)
--(axis cs:2002,12.276)
--(axis cs:2003,13.095)
--(axis cs:2004,12.268)
--(axis cs:2005,9.73)
--(axis cs:2006,7.841)
--(axis cs:2007,6.195)
--(axis cs:2008,4.705)
--(axis cs:2009,5.391)
--(axis cs:2010,5.34)
--(axis cs:2011,5.496)
--(axis cs:2012,5.832)
--(axis cs:2013,6.284)
--(axis cs:2014,7.253)
--(axis cs:2015,8.347)
--(axis cs:2016,9.939)
--(axis cs:2017,11.528)
--(axis cs:2018,13.864)
--(axis cs:2018,14.924)
--(axis cs:2018,14.924)
--(axis cs:2017,12.455)
--(axis cs:2016,10.791)
--(axis cs:2015,9.419)
--(axis cs:2014,8.315)
--(axis cs:2013,7.4)
--(axis cs:2012,6.899)
--(axis cs:2011,6.195)
--(axis cs:2010,5.935)
--(axis cs:2009,5.778)
--(axis cs:2008,5.647)
--(axis cs:2007,6.518)
--(axis cs:2006,8.094)
--(axis cs:2005,10.042)
--(axis cs:2004,12.448)
--(axis cs:2003,13.357)
--(axis cs:2002,12.297)
--(axis cs:2001,11.066)
--(axis cs:2000,9.591)
--(axis cs:1999,7.49)
--(axis cs:1998,6.825)
--(axis cs:1997,6.338)
--(axis cs:1996,6.258)
--(axis cs:1995,6.504)
--(axis cs:1994,6.412)
--(axis cs:1993,6.367)
--(axis cs:1992,6.54)
--(axis cs:1991,6.774)
--(axis cs:1990,7.426)
--cycle;

\path [draw=color4, fill=color4]
(axis cs:1990,7.426)
--(axis cs:1990,7.426)
--(axis cs:1991,6.774)
--(axis cs:1992,6.54)
--(axis cs:1993,6.367)
--(axis cs:1994,6.412)
--(axis cs:1995,6.504)
--(axis cs:1996,6.258)
--(axis cs:1997,6.338)
--(axis cs:1998,6.825)
--(axis cs:1999,7.49)
--(axis cs:2000,9.591)
--(axis cs:2001,11.066)
--(axis cs:2002,12.297)
--(axis cs:2003,13.357)
--(axis cs:2004,12.448)
--(axis cs:2005,10.042)
--(axis cs:2006,8.094)
--(axis cs:2007,6.518)
--(axis cs:2008,5.647)
--(axis cs:2009,5.778)
--(axis cs:2010,5.935)
--(axis cs:2011,6.195)
--(axis cs:2012,6.899)
--(axis cs:2013,7.4)
--(axis cs:2014,8.315)
--(axis cs:2015,9.419)
--(axis cs:2016,10.791)
--(axis cs:2017,12.455)
--(axis cs:2018,14.924)
--(axis cs:2018,16.346)
--(axis cs:2018,16.346)
--(axis cs:2017,13.873)
--(axis cs:2016,11.976)
--(axis cs:2015,10.514)
--(axis cs:2014,9.685)
--(axis cs:2013,8.712)
--(axis cs:2012,8.152)
--(axis cs:2011,7.174)
--(axis cs:2010,6.84)
--(axis cs:2009,6.502)
--(axis cs:2008,6.322)
--(axis cs:2007,6.98)
--(axis cs:2006,8.52)
--(axis cs:2005,10.426)
--(axis cs:2004,12.843)
--(axis cs:2003,13.661)
--(axis cs:2002,12.297)
--(axis cs:2001,11.066)
--(axis cs:2000,9.591)
--(axis cs:1999,7.49)
--(axis cs:1998,6.825)
--(axis cs:1997,6.338)
--(axis cs:1996,6.258)
--(axis cs:1995,6.504)
--(axis cs:1994,6.412)
--(axis cs:1993,6.367)
--(axis cs:1992,6.54)
--(axis cs:1991,6.774)
--(axis cs:1990,7.426)
--cycle;

\path [draw=color5, fill=color5]
(axis cs:1990,8.281)
--(axis cs:1990,7.426)
--(axis cs:1991,6.774)
--(axis cs:1992,6.54)
--(axis cs:1993,6.367)
--(axis cs:1994,6.412)
--(axis cs:1995,6.504)
--(axis cs:1996,6.258)
--(axis cs:1997,6.338)
--(axis cs:1998,6.825)
--(axis cs:1999,7.49)
--(axis cs:2000,9.591)
--(axis cs:2001,11.066)
--(axis cs:2002,12.297)
--(axis cs:2003,13.661)
--(axis cs:2004,12.843)
--(axis cs:2005,10.426)
--(axis cs:2006,8.52)
--(axis cs:2007,6.98)
--(axis cs:2008,6.322)
--(axis cs:2009,6.502)
--(axis cs:2010,6.84)
--(axis cs:2011,7.174)
--(axis cs:2012,8.152)
--(axis cs:2013,8.712)
--(axis cs:2014,9.685)
--(axis cs:2015,10.514)
--(axis cs:2016,11.976)
--(axis cs:2017,13.873)
--(axis cs:2018,16.346)
--(axis cs:2018,16.868)
--(axis cs:2018,16.868)
--(axis cs:2017,14.401)
--(axis cs:2016,12.626)
--(axis cs:2015,11.114)
--(axis cs:2014,10.339)
--(axis cs:2013,9.364)
--(axis cs:2012,8.88)
--(axis cs:2011,7.764)
--(axis cs:2010,7.339)
--(axis cs:2009,6.957)
--(axis cs:2008,6.811)
--(axis cs:2007,7.651)
--(axis cs:2006,9.448)
--(axis cs:2005,11.929)
--(axis cs:2004,14.744)
--(axis cs:2003,15.262)
--(axis cs:2002,13.927)
--(axis cs:2001,12.497)
--(axis cs:2000,10.468)
--(axis cs:1999,8.212)
--(axis cs:1998,7.395)
--(axis cs:1997,6.882)
--(axis cs:1996,6.832)
--(axis cs:1995,7.133)
--(axis cs:1994,7.091)
--(axis cs:1993,7.003)
--(axis cs:1992,7.252)
--(axis cs:1991,7.508)
--(axis cs:1990,8.281)
--cycle;

\path [draw=white!73.3333333333333!black, fill=white!73.3333333333333!black]
(axis cs:1990,8.281)
--(axis cs:1990,8.281)
--(axis cs:1991,7.508)
--(axis cs:1992,7.252)
--(axis cs:1993,7.003)
--(axis cs:1994,7.091)
--(axis cs:1995,7.133)
--(axis cs:1996,6.832)
--(axis cs:1997,6.882)
--(axis cs:1998,7.395)
--(axis cs:1999,8.212)
--(axis cs:2000,10.468)
--(axis cs:2001,12.497)
--(axis cs:2002,13.927)
--(axis cs:2003,15.262)
--(axis cs:2004,14.744)
--(axis cs:2005,11.929)
--(axis cs:2006,9.448)
--(axis cs:2007,7.651)
--(axis cs:2008,6.811)
--(axis cs:2009,6.957)
--(axis cs:2010,7.339)
--(axis cs:2011,7.764)
--(axis cs:2012,8.88)
--(axis cs:2013,9.364)
--(axis cs:2014,10.339)
--(axis cs:2015,11.114)
--(axis cs:2016,12.626)
--(axis cs:2017,14.401)
--(axis cs:2018,16.868)
--(axis cs:2018,17.091)
--(axis cs:2018,17.091)
--(axis cs:2017,14.578)
--(axis cs:2016,12.848)
--(axis cs:2015,11.307)
--(axis cs:2014,10.535)
--(axis cs:2013,9.546)
--(axis cs:2012,9.056)
--(axis cs:2011,7.931)
--(axis cs:2010,7.494)
--(axis cs:2009,7.081)
--(axis cs:2008,7.059)
--(axis cs:2007,8.095)
--(axis cs:2006,10.133)
--(axis cs:2005,12.56)
--(axis cs:2004,15.417)
--(axis cs:2003,16.063)
--(axis cs:2002,14.491)
--(axis cs:2001,12.871)
--(axis cs:2000,10.738)
--(axis cs:1999,8.408)
--(axis cs:1998,7.589)
--(axis cs:1997,6.976)
--(axis cs:1996,6.865)
--(axis cs:1995,7.162)
--(axis cs:1994,7.119)
--(axis cs:1993,7.042)
--(axis cs:1992,7.278)
--(axis cs:1991,7.508)
--(axis cs:1990,8.281)
--cycle;

\addplot [semithick, color0]
table {%
1990 6.02799987792969
1991 5.32800006866455
1992 4.34000015258789
1993 4.15000009536743
1994 4.16400003433228
1995 4.08300018310547
1996 3.95499992370605
1997 3.92199993133545
1998 4.15199995040894
1999 4.35400009155273
2000 5.61199998855591
2001 6.46099996566772
2002 6.91300010681152
2003 6.43900012969971
2004 6.95499992370605
2005 5.56599998474121
2006 4.50099992752075
2007 3.67899990081787
2008 2.48900008201599
2009 3.20300006866455
2010 2.9760000705719
2011 2.94400000572205
2012 3.04800009727478
2013 3.42300009727478
2014 3.73600006103516
2016 4.84999990463257
2017 5.48299980163574
2018 6.52699995040894
};
\addplot [semithick, color1]
table {%
1990 6.02799987792969
1991 5.32800006866455
1992 5.17399978637695
1993 4.87599992752075
1994 4.93599987030029
1995 4.88000011444092
1996 4.70599985122681
1997 4.68900012969971
1998 4.98199987411499
1999 5.35400009155273
2000 6.83300018310547
2001 7.81899976730347
2002 8.61600017547607
2003 8.98600006103516
2004 9.52200031280518
2005 7.5770001411438
2006 6.09100008010864
2007 4.86100006103516
2008 3.4779999256134
2009 4.19099998474121
2010 4.04500007629395
2011 4.09499979019165
2012 4.44000005722046
2013 4.96500015258789
2014 5.81099987030029
2015 6.84499979019165
2017 9.65499973297119
2018 11.75
};
\addplot [semithick, color2]
table {%
1990 7.42600011825562
1991 6.77400016784668
1992 6.5149998664856
1993 6.3439998626709
1994 6.38899993896484
1995 6.49100017547607
1996 6.24900007247925
1997 6.32200002670288
1998 6.80800008773804
1999 7.47499990463257
2000 9.58600044250488
2001 11.0469999313354
2002 12.2760000228882
2003 13.0950002670288
2004 12.2679996490479
2005 9.72999954223633
2006 7.84100008010864
2007 6.19500017166138
2008 4.70499992370605
2009 5.39099979400635
2010 5.34000015258789
2011 5.49599981307983
2012 5.83199977874756
2013 6.28399991989136
2014 7.25299978256226
2015 8.34700012207031
2017 11.5279998779297
2018 13.8640003204346
};
\addplot [semithick, color3]
table {%
1990 7.42600011825562
1991 6.77400016784668
1992 6.53999996185303
1993 6.36700010299683
1994 6.41200017929077
1995 6.50400018692017
1996 6.25799989700317
1997 6.33799982070923
1998 6.82499980926514
1999 7.48999977111816
2000 9.59099960327148
2001 11.0659999847412
2002 12.2969999313354
2003 13.3570003509521
2004 12.4479999542236
2005 10.0419998168945
2006 8.0939998626709
2007 6.51800012588501
2008 5.64699983596802
2009 5.77799987792969
2010 5.93499994277954
2011 6.19500017166138
2012 6.89900016784668
2013 7.40000009536743
2014 8.3149995803833
2015 9.41899967193604
2016 10.7910003662109
2017 12.4549999237061
2018 14.923999786377
};
\addplot [semithick, color4]
table {%
1990 7.42600011825562
1991 6.77400016784668
1992 6.53999996185303
1993 6.36700010299683
1994 6.41200017929077
1995 6.50400018692017
1996 6.25799989700317
1997 6.33799982070923
1998 6.82499980926514
1999 7.48999977111816
2000 9.59099960327148
2001 11.0659999847412
2002 12.2969999313354
2003 13.66100025177
2004 12.8430004119873
2005 10.4259996414185
2006 8.52000045776367
2007 6.98000001907349
2008 6.32200002670288
2009 6.5019998550415
2011 7.17399978637695
2012 8.15200042724609
2013 8.71199989318848
2014 9.6850004196167
2015 10.5139999389648
2016 11.9759998321533
2017 13.8730001449585
2018 16.3460006713867
};
\addplot [semithick, color5]
table {%
1990 8.2810001373291
1991 7.50799989700317
1992 7.2519998550415
1993 7.00299978256226
1994 7.09100008010864
1995 7.13299989700317
1996 6.83199977874756
1997 6.88199996948242
1998 7.39499998092651
1999 8.21199989318848
2000 10.4680004119873
2001 12.4969997406006
2002 13.9270000457764
2003 15.2620000839233
2004 14.7440004348755
2005 11.9289999008179
2006 9.44799995422363
2007 7.65100002288818
2008 6.81099987030029
2009 6.95699977874756
2010 7.33900022506714
2011 7.76399993896484
2012 8.88000011444092
2013 9.36400032043457
2014 10.33899974823
2015 11.1140003204346
2016 12.6260004043579
2017 14.4010000228882
2018 16.8680000305176
};
\addplot [semithick, white!73.3333333333333!black]
table {%
1990 8.2810001373291
1991 7.50799989700317
1993 7.04199981689453
1994 7.11899995803833
1995 7.16200017929077
1996 6.86499977111816
1997 6.97599983215332
1998 7.58900022506714
1999 8.40799999237061
2000 10.7379999160767
2001 12.871000289917
2002 14.4910001754761
2003 16.0629997253418
2004 15.4169998168945
2005 12.5600004196167
2006 10.1330003738403
2007 8.09500026702881
2008 7.05900001525879
2009 7.08099985122681
2010 7.49399995803833
2011 7.93100023269653
2012 9.05599975585938
2013 9.54599952697754
2014 10.5349998474121
2015 11.3070001602173
2016 12.8479995727539
2017 14.5780000686646
2018 17.0909996032715
};

\nextgroupplot[
height=90pt, width=160pt,
legend style={at={(2.02, 0.5)},anchor=west,},
reverse legend,
tick align=outside,
tick pos=left,
x grid style={white!69.0196078431373!black},
xlabel={Men},
xmin=1988.6, xmax=2019.4,
xtick style={color=black},
xtick={1990,2000,2010},
xticklabels={\(\displaystyle 1990\),\(\displaystyle 2000\),\(\displaystyle 2010\)},
ymajorgrids,
ymin=0, ymax=69.7137,
ytick style={color=black}
]
\path [draw=color0, fill=color0]
(axis cs:1990,15.262)
--(axis cs:1990,0)
--(axis cs:1991,0)
--(axis cs:1992,0)
--(axis cs:1993,0)
--(axis cs:1994,0)
--(axis cs:1995,0)
--(axis cs:1996,0)
--(axis cs:1997,0)
--(axis cs:1998,0)
--(axis cs:1999,0)
--(axis cs:2000,0)
--(axis cs:2001,0)
--(axis cs:2002,0)
--(axis cs:2003,0)
--(axis cs:2004,0)
--(axis cs:2005,0)
--(axis cs:2006,0)
--(axis cs:2007,0)
--(axis cs:2008,0)
--(axis cs:2009,0)
--(axis cs:2010,0)
--(axis cs:2011,0)
--(axis cs:2012,0)
--(axis cs:2013,0)
--(axis cs:2014,0)
--(axis cs:2015,0)
--(axis cs:2016,0)
--(axis cs:2017,0)
--(axis cs:2018,0)
--(axis cs:2018,27.17)
--(axis cs:2018,27.17)
--(axis cs:2017,24.233)
--(axis cs:2016,21.711)
--(axis cs:2015,19.344)
--(axis cs:2014,17.454)
--(axis cs:2013,15.715)
--(axis cs:2012,14.588)
--(axis cs:2011,14.162)
--(axis cs:2010,13.543)
--(axis cs:2009,14.217)
--(axis cs:2008,12.057)
--(axis cs:2007,16.347)
--(axis cs:2006,17.9)
--(axis cs:2005,20.394)
--(axis cs:2004,21.702)
--(axis cs:2003,19.388)
--(axis cs:2002,20.507)
--(axis cs:2001,19.093)
--(axis cs:2000,16.318)
--(axis cs:1999,13.599)
--(axis cs:1998,12.917)
--(axis cs:1997,11.77)
--(axis cs:1996,11.488)
--(axis cs:1995,11.615)
--(axis cs:1994,11.382)
--(axis cs:1993,12.009)
--(axis cs:1992,11.715)
--(axis cs:1991,14.295)
--(axis cs:1990,15.262)
--cycle;

\path [draw=color1, fill=color1]
(axis cs:1990,15.262)
--(axis cs:1990,15.262)
--(axis cs:1991,14.295)
--(axis cs:1992,11.715)
--(axis cs:1993,12.009)
--(axis cs:1994,11.382)
--(axis cs:1995,11.615)
--(axis cs:1996,11.488)
--(axis cs:1997,11.77)
--(axis cs:1998,12.917)
--(axis cs:1999,13.599)
--(axis cs:2000,16.318)
--(axis cs:2001,19.093)
--(axis cs:2002,20.507)
--(axis cs:2003,19.388)
--(axis cs:2004,21.702)
--(axis cs:2005,20.394)
--(axis cs:2006,17.9)
--(axis cs:2007,16.347)
--(axis cs:2008,12.057)
--(axis cs:2009,14.217)
--(axis cs:2010,13.543)
--(axis cs:2011,14.162)
--(axis cs:2012,14.588)
--(axis cs:2013,15.715)
--(axis cs:2014,17.454)
--(axis cs:2015,19.344)
--(axis cs:2016,21.711)
--(axis cs:2017,24.233)
--(axis cs:2018,27.17)
--(axis cs:2018,49.69)
--(axis cs:2018,49.69)
--(axis cs:2017,43.775)
--(axis cs:2016,38.218)
--(axis cs:2015,33.313)
--(axis cs:2014,29.386)
--(axis cs:2013,25.795)
--(axis cs:2012,23.548)
--(axis cs:2011,22.077)
--(axis cs:2010,20.816)
--(axis cs:2009,21.206)
--(axis cs:2008,19.185)
--(axis cs:2007,23.943)
--(axis cs:2006,26.599)
--(axis cs:2005,30.093)
--(axis cs:2004,32.279)
--(axis cs:2003,28.808)
--(axis cs:2002,26.386)
--(axis cs:2001,23.996)
--(axis cs:2000,20.627)
--(axis cs:1999,17.072)
--(axis cs:1998,15.843)
--(axis cs:1997,14.541)
--(axis cs:1996,14.188)
--(axis cs:1995,14.235)
--(axis cs:1994,13.956)
--(axis cs:1993,14.411)
--(axis cs:1992,14.269)
--(axis cs:1991,14.295)
--(axis cs:1990,15.262)
--cycle;

\path [draw=color2, fill=color2]
(axis cs:1990,17.545)
--(axis cs:1990,15.262)
--(axis cs:1991,14.295)
--(axis cs:1992,14.269)
--(axis cs:1993,14.411)
--(axis cs:1994,13.956)
--(axis cs:1995,14.235)
--(axis cs:1996,14.188)
--(axis cs:1997,14.541)
--(axis cs:1998,15.843)
--(axis cs:1999,17.072)
--(axis cs:2000,20.627)
--(axis cs:2001,23.996)
--(axis cs:2002,26.386)
--(axis cs:2003,28.808)
--(axis cs:2004,32.279)
--(axis cs:2005,30.093)
--(axis cs:2006,26.599)
--(axis cs:2007,23.943)
--(axis cs:2008,19.185)
--(axis cs:2009,21.206)
--(axis cs:2010,20.816)
--(axis cs:2011,22.077)
--(axis cs:2012,23.548)
--(axis cs:2013,25.795)
--(axis cs:2014,29.386)
--(axis cs:2015,33.313)
--(axis cs:2016,38.218)
--(axis cs:2017,43.775)
--(axis cs:2018,49.69)
--(axis cs:2018,55.903)
--(axis cs:2018,55.903)
--(axis cs:2017,49.362)
--(axis cs:2016,43.508)
--(axis cs:2015,38.461)
--(axis cs:2014,34.294)
--(axis cs:2013,30.517)
--(axis cs:2012,28.05)
--(axis cs:2011,26.44)
--(axis cs:2010,24.904)
--(axis cs:2009,25.084)
--(axis cs:2008,23.13)
--(axis cs:2007,27.882)
--(axis cs:2006,31.166)
--(axis cs:2005,35.085)
--(axis cs:2004,37.682)
--(axis cs:2003,36.249)
--(axis cs:2002,32.911)
--(axis cs:2001,29.658)
--(axis cs:2000,25.101)
--(axis cs:1999,20.609)
--(axis cs:1998,18.959)
--(axis cs:1997,17.3)
--(axis cs:1996,16.803)
--(axis cs:1995,16.673)
--(axis cs:1994,16.461)
--(axis cs:1993,16.604)
--(axis cs:1992,16.5)
--(axis cs:1991,16.582)
--(axis cs:1990,17.545)
--cycle;

\path [draw=color3, fill=color3]
(axis cs:1990,17.545)
--(axis cs:1990,17.545)
--(axis cs:1991,16.582)
--(axis cs:1992,16.5)
--(axis cs:1993,16.604)
--(axis cs:1994,16.461)
--(axis cs:1995,16.673)
--(axis cs:1996,16.803)
--(axis cs:1997,17.3)
--(axis cs:1998,18.959)
--(axis cs:1999,20.609)
--(axis cs:2000,25.101)
--(axis cs:2001,29.658)
--(axis cs:2002,32.911)
--(axis cs:2003,36.249)
--(axis cs:2004,37.682)
--(axis cs:2005,35.085)
--(axis cs:2006,31.166)
--(axis cs:2007,27.882)
--(axis cs:2008,23.13)
--(axis cs:2009,25.084)
--(axis cs:2010,24.904)
--(axis cs:2011,26.44)
--(axis cs:2012,28.05)
--(axis cs:2013,30.517)
--(axis cs:2014,34.294)
--(axis cs:2015,38.461)
--(axis cs:2016,43.508)
--(axis cs:2017,49.362)
--(axis cs:2018,55.903)
--(axis cs:2018,60.955)
--(axis cs:2018,60.955)
--(axis cs:2017,54.036)
--(axis cs:2016,47.908)
--(axis cs:2015,43.869)
--(axis cs:2014,39.874)
--(axis cs:2013,36.009)
--(axis cs:2012,33.042)
--(axis cs:2011,30.075)
--(axis cs:2010,27.379)
--(axis cs:2009,26.951)
--(axis cs:2008,27.044)
--(axis cs:2007,29.405)
--(axis cs:2006,32.246)
--(axis cs:2005,36.272)
--(axis cs:2004,38.173)
--(axis cs:2003,36.67)
--(axis cs:2002,32.957)
--(axis cs:2001,29.687)
--(axis cs:2000,25.136)
--(axis cs:1999,20.647)
--(axis cs:1998,18.992)
--(axis cs:1997,17.32)
--(axis cs:1996,16.838)
--(axis cs:1995,16.694)
--(axis cs:1994,16.486)
--(axis cs:1993,16.638)
--(axis cs:1992,16.533)
--(axis cs:1991,16.582)
--(axis cs:1990,17.545)
--cycle;

\path [draw=color4, fill=color4]
(axis cs:1990,17.545)
--(axis cs:1990,17.545)
--(axis cs:1991,16.582)
--(axis cs:1992,16.533)
--(axis cs:1993,16.638)
--(axis cs:1994,16.486)
--(axis cs:1995,16.694)
--(axis cs:1996,16.838)
--(axis cs:1997,17.32)
--(axis cs:1998,18.992)
--(axis cs:1999,20.647)
--(axis cs:2000,25.136)
--(axis cs:2001,29.687)
--(axis cs:2002,32.957)
--(axis cs:2003,36.67)
--(axis cs:2004,38.173)
--(axis cs:2005,36.272)
--(axis cs:2006,32.246)
--(axis cs:2007,29.405)
--(axis cs:2008,27.044)
--(axis cs:2009,26.951)
--(axis cs:2010,27.379)
--(axis cs:2011,30.075)
--(axis cs:2012,33.042)
--(axis cs:2013,36.009)
--(axis cs:2014,39.874)
--(axis cs:2015,43.869)
--(axis cs:2016,47.908)
--(axis cs:2017,54.036)
--(axis cs:2018,60.955)
--(axis cs:2018,62.81)
--(axis cs:2018,62.81)
--(axis cs:2017,56.109)
--(axis cs:2016,49.974)
--(axis cs:2015,45.998)
--(axis cs:2014,42.553)
--(axis cs:2013,38.591)
--(axis cs:2012,35.57)
--(axis cs:2011,32.616)
--(axis cs:2010,30.074)
--(axis cs:2009,29.118)
--(axis cs:2008,29.121)
--(axis cs:2007,30.771)
--(axis cs:2006,33.034)
--(axis cs:2005,37.003)
--(axis cs:2004,38.989)
--(axis cs:2003,37.34)
--(axis cs:2002,32.957)
--(axis cs:2001,29.687)
--(axis cs:2000,25.136)
--(axis cs:1999,20.647)
--(axis cs:1998,18.992)
--(axis cs:1997,17.32)
--(axis cs:1996,16.838)
--(axis cs:1995,16.694)
--(axis cs:1994,16.486)
--(axis cs:1993,16.638)
--(axis cs:1992,16.533)
--(axis cs:1991,16.582)
--(axis cs:1990,17.545)
--cycle;

\path [draw=color5, fill=color5]
(axis cs:1990,19.268)
--(axis cs:1990,17.545)
--(axis cs:1991,16.582)
--(axis cs:1992,16.533)
--(axis cs:1993,16.638)
--(axis cs:1994,16.486)
--(axis cs:1995,16.694)
--(axis cs:1996,16.838)
--(axis cs:1997,17.32)
--(axis cs:1998,18.992)
--(axis cs:1999,20.647)
--(axis cs:2000,25.136)
--(axis cs:2001,29.687)
--(axis cs:2002,32.957)
--(axis cs:2003,37.34)
--(axis cs:2004,38.989)
--(axis cs:2005,37.003)
--(axis cs:2006,33.034)
--(axis cs:2007,30.771)
--(axis cs:2008,29.121)
--(axis cs:2009,29.118)
--(axis cs:2010,30.074)
--(axis cs:2011,32.616)
--(axis cs:2012,35.57)
--(axis cs:2013,38.591)
--(axis cs:2014,42.553)
--(axis cs:2015,45.998)
--(axis cs:2016,49.974)
--(axis cs:2017,56.109)
--(axis cs:2018,62.81)
--(axis cs:2018,65.163)
--(axis cs:2018,65.163)
--(axis cs:2017,58.712)
--(axis cs:2016,52.832)
--(axis cs:2015,48.931)
--(axis cs:2014,45.432)
--(axis cs:2013,41.746)
--(axis cs:2012,38.544)
--(axis cs:2011,35.222)
--(axis cs:2010,32.341)
--(axis cs:2009,31.226)
--(axis cs:2008,31.238)
--(axis cs:2007,33.272)
--(axis cs:2006,35.944)
--(axis cs:2005,40.978)
--(axis cs:2004,43.527)
--(axis cs:2003,40.776)
--(axis cs:2002,36.128)
--(axis cs:2001,32.193)
--(axis cs:2000,26.722)
--(axis cs:1999,21.903)
--(axis cs:1998,20.136)
--(axis cs:1997,18.342)
--(axis cs:1996,17.869)
--(axis cs:1995,17.827)
--(axis cs:1994,17.653)
--(axis cs:1993,17.725)
--(axis cs:1992,17.774)
--(axis cs:1991,17.895)
--(axis cs:1990,19.268)
--cycle;

\path [draw=white!73.3333333333333!black, fill=white!73.3333333333333!black]
(axis cs:1990,19.268)
--(axis cs:1990,19.268)
--(axis cs:1991,17.895)
--(axis cs:1992,17.774)
--(axis cs:1993,17.725)
--(axis cs:1994,17.653)
--(axis cs:1995,17.827)
--(axis cs:1996,17.869)
--(axis cs:1997,18.342)
--(axis cs:1998,20.136)
--(axis cs:1999,21.903)
--(axis cs:2000,26.722)
--(axis cs:2001,32.193)
--(axis cs:2002,36.128)
--(axis cs:2003,40.776)
--(axis cs:2004,43.527)
--(axis cs:2005,40.978)
--(axis cs:2006,35.944)
--(axis cs:2007,33.272)
--(axis cs:2008,31.238)
--(axis cs:2009,31.226)
--(axis cs:2010,32.341)
--(axis cs:2011,35.222)
--(axis cs:2012,38.544)
--(axis cs:2013,41.746)
--(axis cs:2014,45.432)
--(axis cs:2015,48.931)
--(axis cs:2016,52.832)
--(axis cs:2017,58.712)
--(axis cs:2018,65.163)
--(axis cs:2018,66.394)
--(axis cs:2018,66.394)
--(axis cs:2017,60.125)
--(axis cs:2016,54.533)
--(axis cs:2015,50.722)
--(axis cs:2014,47.141)
--(axis cs:2013,43.397)
--(axis cs:2012,40.097)
--(axis cs:2011,36.616)
--(axis cs:2010,33.468)
--(axis cs:2009,32.323)
--(axis cs:2008,32.642)
--(axis cs:2007,35.334)
--(axis cs:2006,38.756)
--(axis cs:2005,43.59)
--(axis cs:2004,45.896)
--(axis cs:2003,43.229)
--(axis cs:2002,37.696)
--(axis cs:2001,33.233)
--(axis cs:2000,27.476)
--(axis cs:1999,22.519)
--(axis cs:1998,20.575)
--(axis cs:1997,18.666)
--(axis cs:1996,17.976)
--(axis cs:1995,17.905)
--(axis cs:1994,17.72)
--(axis cs:1993,17.789)
--(axis cs:1992,17.852)
--(axis cs:1991,17.895)
--(axis cs:1990,19.268)
--cycle;

\addplot [semithick, color0]
table {%
1990 15.2620000839233
1991 14.2950000762939
1992 11.7150001525879
1993 12.0089998245239
1994 11.3819999694824
1995 11.6149997711182
1996 11.4879999160767
1997 11.7700004577637
1998 12.9169998168945
1999 13.5989999771118
2000 16.318000793457
2001 19.0930004119873
2002 20.5069999694824
2003 19.3880004882812
2004 21.7019996643066
2005 20.3939990997314
2006 17.8999996185303
2007 16.3470001220703
2008 12.0570001602173
2009 14.2170000076294
2010 13.5430002212524
2011 14.1619997024536
2012 14.5880002975464
2013 15.7150001525879
2014 17.4540004730225
2015 19.3439998626709
2016 21.7110004425049
2017 24.2329998016357
2018 27.1700000762939
};
\addplot [semithick, color1]
table {%
1990 15.2620000839233
1991 14.2950000762939
1992 14.2690000534058
1993 14.41100025177
1994 13.956000328064
1995 14.2349996566772
1996 14.1879997253418
1997 14.5410003662109
1998 15.8430004119873
1999 17.07200050354
2000 20.6270008087158
2001 23.996000289917
2002 26.3859996795654
2003 28.8080005645752
2004 32.2789993286133
2005 30.0930004119873
2006 26.5990009307861
2007 23.943000793457
2008 19.1849994659424
2009 21.2059993743896
2010 20.8159999847412
2011 22.0769996643066
2012 23.5480003356934
2013 25.7950000762939
2014 29.3859996795654
2015 33.3129997253418
2016 38.2179985046387
2017 43.7750015258789
2018 49.689998626709
};
\addplot [semithick, color2]
table {%
1990 17.5450000762939
1991 16.5820007324219
1992 16.5
1993 16.6040000915527
1994 16.4610004425049
1995 16.6730003356934
1996 16.80299949646
1997 17.2999992370605
1999 20.6089992523193
2000 25.1009998321533
2001 29.6580009460449
2002 32.9109992980957
2003 36.2490005493164
2004 37.681999206543
2005 35.0849990844727
2006 31.1660003662109
2007 27.8819999694824
2008 23.1299991607666
2009 25.0839996337891
2010 24.9039993286133
2011 26.4400005340576
2012 28.0499992370605
2013 30.5170001983643
2014 34.2939987182617
2015 38.4609985351562
2016 43.507999420166
2017 49.3619995117188
2018 55.9029998779297
};
\addplot [semithick, color3]
table {%
1990 17.5450000762939
1991 16.5820007324219
1992 16.5330009460449
1993 16.6380004882812
1994 16.4860000610352
1995 16.6940002441406
1996 16.8379993438721
1997 17.3199996948242
1998 18.992000579834
1999 20.6469993591309
2000 25.1359996795654
2001 29.6870002746582
2002 32.9570007324219
2003 36.6699981689453
2004 38.1730003356934
2005 36.2719993591309
2006 32.2459983825684
2007 29.4050006866455
2008 27.0440006256104
2009 26.951000213623
2010 27.378999710083
2011 30.0750007629395
2013 36.0089988708496
2014 39.8740005493164
2015 43.8689994812012
2016 47.9080009460449
2017 54.0359992980957
2018 60.9550018310547
};
\addplot [semithick, color4]
table {%
1990 17.5450000762939
1991 16.5820007324219
1992 16.5330009460449
1993 16.6380004882812
1994 16.4860000610352
1995 16.6940002441406
1996 16.8379993438721
1997 17.3199996948242
1998 18.992000579834
1999 20.6469993591309
2000 25.1359996795654
2001 29.6870002746582
2002 32.9570007324219
2003 37.3400001525879
2004 38.9889984130859
2005 37.0029983520508
2006 33.0340003967285
2007 30.7709999084473
2008 29.121000289917
2009 29.1180000305176
2010 30.0739994049072
2011 32.6160011291504
2012 35.5699996948242
2013 38.5909996032715
2014 42.5530014038086
2015 45.9980010986328
2016 49.9739990234375
2017 56.109001159668
2018 62.810001373291
};
\addplot [semithick, color5]
table {%
1990 19.2679996490479
1991 17.8950004577637
1992 17.7740001678467
1993 17.7250003814697
1994 17.6529998779297
1995 17.8269996643066
1996 17.8689994812012
1997 18.3419990539551
1998 20.1359996795654
1999 21.9029998779297
2000 26.7220001220703
2001 32.193000793457
2002 36.1279983520508
2003 40.7760009765625
2004 43.5270004272461
2005 40.9780006408691
2006 35.9440002441406
2007 33.2719993591309
2008 31.238000869751
2009 31.2259998321533
2010 32.3409996032715
2011 35.2220001220703
2012 38.5439987182617
2013 41.7459983825684
2014 45.431999206543
2015 48.9309997558594
2016 52.8320007324219
2017 58.7120018005371
2018 65.1630020141602
};
\addplot [semithick, white!73.3333333333333!black]
table {%
1990 19.2679996490479
1991 17.8950004577637
1992 17.8519992828369
1994 17.7199993133545
1995 17.9050006866455
1996 17.9759998321533
1997 18.6660003662109
1998 20.5750007629395
1999 22.5189990997314
2000 27.4759998321533
2001 33.2330017089844
2002 37.6959991455078
2003 43.2290000915527
2004 45.8959999084473
2005 43.5900001525879
2006 38.7560005187988
2007 35.3339996337891
2008 32.6419982910156
2009 32.3230018615723
2010 33.4679985046387
2011 36.6160011291504
2012 40.0970001220703
2013 43.3969993591309
2014 47.140998840332
2015 50.7220001220703
2016 54.5330009460449
2017 60.125
2018 66.3939971923828
};
\end{groupplot}




\end{tikzpicture}
\caption{Number Bachelor's degrees awarded (thousands). Source: IPEDS.}

\end{figure}
\hyperlink{intro_social_science_ratio}{\beamerbutton{Return: Social science ratio}}
}
\end{frame}

%%%%%%%%%%%%%%%%%%%%%%%%%%%%%%%%%%%%%%%%%%%%%%%%%%%%%%%%%%%%%%%%%%%%%%%%%%%%%%%%
\begin{frame}{Education}\label{app_education}
\Wider[4em]{

\begin{figure}
\setlength{\abovecaptionskip}{2pt}
\setlength{\belowcaptionskip}{-2pt}
\input{cip13.tex}
\end{figure}
\hyperlink{intro_social_science_ratio}{\beamerbutton{Return: Social science ratio}}
}
\end{frame}

%%%%%%%%%%%%%%%%%%%%%%%%%%%%%%%%%%%%%%%%%%%%%%%%%%%%%%%%%%%%%%%%%%%%%%%%%%%%%%%%
\begin{frame}{Biological and Physical Sciences and Mathematics}\label{app_science_math}
\Wider[4em]{

\begin{figure}
\setlength{\abovecaptionskip}{2pt}
\setlength{\belowcaptionskip}{-2pt}
\input{science_math.tex}
\end{figure}
\hyperlink{intro_social_science_ratio}{\beamerbutton{Return: Social science ratio}}
\hyperlink{app_physical_science_math}{\beamerbutton{No biology}}
}
\end{frame}

%%%%%%%%%%%%%%%%%%%%%%%%%%%%%%%%%%%%%%%%%%%%%%%%%%%%%%%%%%%%%%%%%%%%%%%%%%%%%%%%
\begin{frame}{Physical Sciences and math}\label{app_physical_science_math}
\Wider[4em]{

\begin{figure}
\setlength{\abovecaptionskip}{2pt}
\setlength{\belowcaptionskip}{-2pt}
\input{physical_science_math.tex}
\end{figure}
\hyperlink{app_science_math}{\beamerbutton{Return: Science and math}}
\hyperlink{intro_social_science_ratio}{\beamerbutton{Return: Social science ratio}}
}
\end{frame}


\end{document}