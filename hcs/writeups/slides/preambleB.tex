

% Standard packages
\usepackage{hyperref} % Include links
\usepackage{setspace} % More control over local line spacing: http://felix11h.github.io/blog/line-spacing-beamer
% Math
\usepackage{amsthm, amsmath, mathtools}
\usepackage{appendixnumberbeamer}
% Multi=row and column
\usepackage{multirow,multicol}
% multiline columns
\usepackage{tabularx}


\usepackage{ifthen}

\usepackage{amsbsy}

%%%%%%%%%%%%%%
% Formatting %
%%%%%%%%%%%%%%

% Outer theme
\useoutertheme[subsection=false]{miniframes}
% Fix to display navigational dots; dots only shown for subsections
% Can comment out to have no dots in presentation
\AtBeginSection[]{\subsection{}}
% Font
\usepackage[default]{cantarell}

% Colors
\usepackage{xcolor}
% UCSD color palette: http://ucpa.ucsd.edu/brand/elements/color-palette/
\definecolor{DarkBlue}{RGB}{24, 43, 73}
\definecolor{MainBlue}{RGB}{0, 106, 150}
\definecolor{LightBlue}{RGB}{0, 198, 215}
\definecolor{Green}{RGB}{110 150 59}
\definecolor{Yellow}{RGB}{255 205 0}
\definecolor{LightYellow}{RGB}{243 29 0}
\definecolor{Gold}{RGB}{198 146 20}
\definecolor{Orange}{RGB}{252 137 0}
\definecolor{LightGrey}{RGB}{182, 177, 169}
\definecolor{DarkGray}{RGB}{116 118 120}
\definecolor{DarkPurple}{RGB}{25 0 51}
\definecolor{Purple}{RGB}{55 24 81}
\definecolor{Maroon}{HTML}{6A123D}


% Hyperlink colors
\hypersetup{
  colorlinks   = true, %Colours links instead of ugly boxes
  urlcolor     = Maroon, %Colour for external hyperlinks
  linkcolor    = DarkGray, %Colour of internal links
  citecolor   = MainBlue %Colour of citations
}


% Learned using: https://ramblingacademic.com/2015/12/how-to-quickly-overhaul-beamer-colors/
% To change additional sections: http://www.cpt.univ-mrs.fr/~masson/latex/Beamer-appearance-cheat-sheet.pdf
\setbeamercolor{palette primary}{bg=DarkBlue,fg=white}
\setbeamercolor{palette secondary}{bg=DarkBlue,fg=white}
\setbeamercolor{palette tertiary}{bg=DarkBlue,fg=white}
\setbeamercolor{palette quaternary}{bg=DarkBlue,fg=white}
\setbeamercolor{structure}{fg=DarkBlue} % itemize, enumerate, etc
\setbeamercolor{section in toc}{fg=DarkBlue} % TOC sections
\setbeamercolor{section in head/foot}{bg=DarkBlue,fg=LightGrey}
%\setbeamercolor{navigation symbols}{bg=Gold, fg=Gold}
\setbeamercolor{alerted text}{fg=Gold}
\beamertemplatenavigationsymbolsempty
%tem
%\setbeamertemplate{itemize items}

% Make bullet size smaller
\setbeamertemplate{itemize item}{\tiny\raise1.4pt\hbox{$\blacktriangleright$}}
\setbeamertemplate{itemize subitem}{--}
% make bullet closer to the text
\setlength{\labelsep}{.8ex}

% get rid of 'Figure: ' in caption
\setbeamertemplate{caption}{\raggedright\insertcaption\par}
% Caption flushed left
\usepackage{caption}
\captionsetup{%
    labelformat=empty,
    font=small,
    singlelinecheck=false,
    tableposition=top
}


%%%%%%%%%%%%%%%%
% New commands %
%%%%%%%%%%%%%%%%

\ifshowcomments
\newcommand{\nts}[2][DarkGray]{\setbeamercolor{taras comment}{fg=#1}{\usebeamercolor[fg]{taras comment}\setbeamercolor{item}{fg = taras comment.fg}\emph{#2}}}
\else 
\newcommand{\nts}[2][DarkGray]{}
\fi

% Timed color highlights
\newcommand<>{\blue}[1]{\textbf{\color#2{MainBlue}#1}}
\newcommand<>{\mblue}[1]{{\color#2{MainBlue}#1}}
\newcommand<>{\green}[1]{\textbf{\color#2{Green}#1}}
\newcommand<>{\mgreen}[1]{{\color#2{Green}#1}}

\newcommand{\EE}[1]{\mathbb{E} \left[ #1 \right]}
\newcommand{\llog}[1]{\log \left( #1 \right)}
\newcommand{\Cov}[1]{\text{Cov} \left( #1 \right)}

\newcommand{\br}[1]{\left\{ #1 \right\}}
\newcommand{\sbr}[1]{\left[ #1 \right]}
\newcommand{\pr}[1]{\left( #1 \right)}
\newcommand{\ce}[2]{\left[\left. #1 \right\vert #2 \right]}
\newcommand{\condl}[1]{\left. #1 \right\vert}

% Beamer buttons
% bottom buttom: \bbutton{name_of_link}{Text to print}
\newcommand{\bbutton}[2]{
    \begin{tikzpicture}[remember picture, overlay]
    \node[shift={(-1.4cm,0.5cm)}]() at (current page.south east){%
    \hyperlink{#1}{\beamergotobutton{#2}}};    
    \end{tikzpicture}
}    
% here button
\newcommand{\hbutton}[2]{
    \hyperlink{#1}{\beamergotobutton{#2}}
}
% Cite command: cc
\newcommand{\nn}[2][DarkGray]{{\small \color{#1} #2}}
\newcommand{\ccdoi}[2]{\href{https://doi.org/#1}{\nn{#2}}}

% change slide width on a single slide
% source: https://tex.stackexchange.com/questions/160825/modifying-margins-for-one-slide/242073
\newcommand\Wider[2][3em]{%
\makebox[\linewidth][c]{%
  \begin{minipage}{\dimexpr\textwidth+#1\relax}
  \raggedright#2
  \end{minipage}%
  }%
}

% To remove sections from appendix
% source: https://tex.stackexchange.com/questions/37127/how-to-remove-some-pages-from-the-navigation-bullets-in-beamer
\makeatletter
\let\beamer@writeslidentry@miniframeson=\beamer@writeslidentry
\def\beamer@writeslidentry@miniframesoff{%
  \expandafter\beamer@ifempty\expandafter{\beamer@framestartpage}{}% does not happen normally
  {%else
    % removed \addtocontents commands
    \clearpage\beamer@notesactions%
  }
}
\newcommand*{\miniframeson}{\let\beamer@writeslidentry=\beamer@writeslidentry@miniframeson}
\newcommand*{\miniframesoff}{\let\beamer@writeslidentry=\beamer@writeslidentry@miniframesoff}
\makeatother

%%%%%%%%%%%%%%%%%%%%
% Figures and tikz %
%%%%%%%%%%%%%%%%%%%%

% \usepackage{pgfplots} % maybe for tikz axes?
\usepackage{graphicx} % Include figures
\usepackage{tikz} % Draw figures with tikz
% \usetikzlibrary{positioning} % to use right=of <node> syntax
% \usetikzlibrary{shapes} % for other node shapes; see http://www.texample.net/tikz/examples/node-shapes/
% \usetikzlibrary{shapes.multipart} % multiline nodes in tikz
% \usetikzlibrary{shapes.arrows} % arrow nodes in tikz
% \usetikzlibrary{shapes.misc} % miscellaneous nodes in tikz; for rounded rectangles
% \usetikzlibrary{shapes.geometric} % geometric nodes in tikz; fortriangles

% Presentation specific tikz
%\usetikzlibrary{tikzmark} % to use tikzmarkto add braces to itemize
% \usetikzlibrary{calc} % for widthof
% \tikzset{
%     above label/.style={
%         above = 5pt,
%         %font=\footnotesize,
%         text height = 1.5ex,
%         text depth = 1ex,
%     },
%     below label/.style={  
%         below=4pt,
%         %font=\footnotesize,
%         text height = 1.5ex,
%         text depth = 1ex
%     },
%     brace label/.style={
%         below = 4pt,
%         font=\footnotesize,
%         text height = 1.5ex,
%         text depth = 1ex
%     },
%     brace/.style={
%         decoration={brace, mirror},
%         decorate
%     }
% }

% For every picture that defines or uses external nodes, you'll have to
% apply the 'remember picture' style. To avoid some typing, we'll apply
% the style to all pictures.
%\tikzstyle{every picture}+=[remember picture]

% To avoid a warning
% \pgfplotsset{compat=1.7}