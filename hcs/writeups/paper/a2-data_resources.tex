%!TEX root = outline.tex
%%%%%%%%%%%%%%%%%%%%%%%%%%%%%%%%%%%%%%%%%%%%%%%%%%%%%%%%%%%%%%%%%%%%%%%%%%%%%%%%
% \subsubsection*{Data sources}
\subsection{Data sources}\label{sec:data_sources}

I have utilized a number of data resources for this project thus far. Several of them are outlined below. 

Characteristics of all U.S. postsecondary institutions are reported in the Integrated Postsecondary Education Data System (IPEDS) data.
These data are collected annually by the National Center for Educational Statistics and describe the universe of institutions that participate in federal student financial aid programs. 
The empirical motivation for this analysis relies on the IPEDS Completion Surveys, which describe all degrees and certificates awarded at postsecondary institutions by field of study, gender, and race.\footnote{
    For more details on the IPEDS series, please visit \url{https://nces.ed.gov/ipeds/}.
    IPEDS data are available from 1986 until the present, though I begin empirical analysis in 1990 due to changes in how fields of study are classified.
    For earlier data, such as those used in Figure \ref{fig:n_degrees}, I supplement the IPEDS series with data from \textcite{S93}.
    However, it is worth noting that the predecessor to IPEDS series is the Higher Education General Information Survey (HEGIS), available through the International Archive of Education Data at University of Michigan (\url{https://www.icpsr.umich.edu/web/ICPSR/series/00030}). As such, I refer the reader to the HEGIS series for a more detailed portrait of postsecondary education statistics than available in \textcite{S93}.
}
University-level graduation rates can be estimated using the IPEDS graduation surveys.
% Details on IPEDS graduation surveys are in the Appendix. 

The American Community Survey (ACS), conducted by the U.S. Census Bureau, collects detailed information on American households. 
These data include information on employment and income, demographic information, and, crucially, educational attainment. 
Specifically, beginning in 2009, the ACS began recording up to two undergraduate fields of study for household members.
This has allowed papers such as \textcite{SHB19} to examine the dynamics of human capital specialization decisions across cohorts. 
For detailed information on using ACS data, see \textcite{IPUMS}.
% Individual-level data on degree completions are from the American Community Survey (ACS), as accessed using \textcite{IPUMS}.
% I follow the data selection procedure outlined in Appendix A of \textcite{SHB19}.

Individual student data on beliefs and abilities are more difficult to come by.
An ideal resource is the 2012/17 Beginning Postsecondary Students Longitudinal Study, conducted by the National Center for Education Statistics, which can be used to identify initial levels of human capital and group-level beliefs. 
This longitudinal study collects education and employment data from a nationally representative sample of first-time beginning postsecondary students.
Respondents are initially surveyed in 2011-2012, the beginning of their postsecondary studies. 
Follow-up surveys were conducted three and six years after they began their studies.\footnote{
    For additional information, see \url{https://nces.ed.gov/pubsearch/pubsinfo.asp?pubid=2020504}.
    Restricted-use licenses are required for access to BPS microdata. 
    However, data aggregates and simple regression analysis are available through the NCES PowerStats DataLab. 
}
Key for this analysis are the BPS transcript studies, which contain information on first-year major choice, high school performance, and outcomes. 
An application for restricted-use BPS data is pending. 
% Used by: \textcite{SS14}

The National Longitudinal Survey of Youth (NLS97) provides additional data that can be used to identify student characteristic.
These data consist of a representative sample of approximately 9,000 U.S. men and women born between 1980 and 1984.
Respondent's educational and labor market experiences are recorded over time. 
In particular, these data contain transcript information on respondents.
Though BPS data would be preferable, as the data are more recent and more representative, NLS97 data provides a strong basis for beginning this analysis.  

