\documentclass[11 pt]{article}

% General tex preamble
% Format
\usepackage[T1]{fontenc}
\usepackage[utf8]{inputenc}
\usepackage[margin=1in]{geometry} % 1-inch margins
\usepackage[english]{babel} % English hyphenation, etc.	
\usepackage{setspace} % Set spacing 
%\usepackage{enumerate} % Use different types of enumerate options
\usepackage{enumitem}
\usepackage{csquotes} % Block quotes
\usepackage[dvipsnames]{xcolor} % colors: https://en.wikibooks.org/wiki/LaTeX/Colors#The_68_standard_colors_known_to_dvips
% Math
\usepackage{amsmath, mathrsfs, amsfonts, amssymb, amsthm}
% Figures
\usepackage{graphicx} % Include figures
\usepackage{float} % Improved control over floats
\usepackage{tikz} % Draw figures with tikz

% Colors
\definecolor{Indigo}{HTML}{3C6478}
\definecolor{DarkBrown}{HTML}{45281B}
\definecolor{Brown}{HTML}{161402}
\definecolor{DarkGreen}{HTML}{325101}
\definecolor{LeafGreen}{HTML}{4A6F01}
\definecolor{DarkAlice}{HTML}{107896}
\definecolor{Alice}{HTML}{1496BB}
\definecolor{DarkGray}{RGB}{116 118 120}
\definecolor{DarkBlue}{HTML}{022C36}
\definecolor{MainBlue}{HTML}{132881}
\definecolor{Maroon}{HTML}{6A123D}
\definecolor{DarkPurple}{HTML}{2C033A}
\definecolor{Orange}{HTML}{F18312}

% Hyperlinks
\usepackage{hyperref} % Include hyperlinks
\hypersetup{
  colorlinks   = true, %Colours links instead of ugly boxes
  urlcolor     = Maroon, %Colour for external hyperlinks
  linkcolor    = DarkGray, %Colour of internal links
  citecolor   = MainBlue %Colour of citations
}


% Macro Shortcuts
\newcommand{\R}{\mathbb{R}}
\newcommand{\Q}{\mathbb{Q}}
\newcommand{\Z}{\mathbb{Z}}
\newcommand{\N}{\mathbb{N}}
\newcommand{\EE}{\mathbb{E}}
\newcommand{\PP}{\mathbb{P}}
\newcommand{\BB}{\mathscr{B}}
\newcommand{\e}{\text{e}}
\newcommand{\dd}{\text{d}}


% Theorems
\newtheorem{prop}{Proposition}[section]
\newtheorem{thm}{Theorem}
\theoremstyle{remark}
\newtheorem{claim}{Claim}[section]
\newtheorem{remark}{Remark}
\theoremstyle{definition}
\newtheorem{defn}{Definition}[section]
\newtheorem{lemma}{Lemma}
\newtheorem{ass}{Assumption}

% Spacing and colors
% Notes to self 
\ifnts
  \newcommand{\nts}[1]{{\color{gray}#1}}
\else
  \newcommand{\nts}[1]{}
\fi

%%%%%%%%%%
% Sections that have:
%   (A) Roman numerals 
%   (B) fixed width = fixw
%   (C) coloring

% (A) Roman numeral for section and subsection
% \renewcommand{\thesection}{\Roman{section}} 
% \renewcommand{\thesubsection}{\roman{subsection}}

% (B) Each section has fixed width = fixw 

% Define fixw
\newcommand{\fixw}{28pt}
\newcommand{\fixwh}{14pt}

% (C) Define colors
\newcommand{\secc}[1]{{\color{DarkGreen}#1}} % section color
\newcommand{\sectc}{DarkGreen} % section text color
\newcommand{\subsecc}[1]{{\color{LeafGreen}#1}} % subsection color
\newcommand{\subsectc}{LeafGreen} % subsection text color
\newcommand{\numc}{DarkAlice}

% Set each section width and color
\usepackage{titlesec}
\titleformat{\section}{\normalfont\Large\bfseries\color{\sectc}}
	{\makebox[\fixw][l]{\secc{\thesection.}}}{0pt}{} 
\titleformat{\subsection}{\normalfont\large\bfseries\color{\subsectc}}
	{\makebox[{\fixw}][l]{\subsecc{\thesubsection.}}}{0pt}{} 
\titleformat{\subsubsection}{\normalfont\bfseries}
	{}{0pt}{} %{\makebox[{\fixw}][l]{}}{0pt}{} 

% Highlight certain items
\newcommand{\hitem}[2][DarkAlice]{\color{#1} \item #2 \color{black}}

%%%%%%%%%%
% Lists that start at fixw (see section above)
\newlist{outline}{enumerate}{2}
\setlist[outline,1]{label=\arabic*.,left=0pt .. \fixw}
\setlist[outline,2]{label=\alph*.,left=0pt .. \fixw}

\newlist{blist}{itemize}{2}
\setlist[blist,1]{label=\textbullet,left=0pt .. \fixw}
\setlist[blist,2]{label=\textendash,left=0pt .. \fixw}

%%%%%%%%%%
% Enumerate in footnote
\newlist{footcount}{enumerate}{1}
\setlist[footcount]{label=(\alph*),left=0pt .. \fixw}

%%%%%%%%%%
% Foodnote Edits
% Bottom package ensures that footnote won't be above a figure
\usepackage[bottom]{footmisc}

% No indent in footnotes
% NOTHING SEEMS TO WORK
% \usepackage{scrextend}
% \deffootnote[\fixw]{\fixw}{.195in}{\makebox[\fixw][r]{\thefootnotemark.\hspace{.2in}}}
% \usepackage[flushmargin, hang]{footmisc} % flush footnote mark to left margin
% \renewcommand{\footnotelayout}{\doublespacing\raggedright}
% \usepackage[flushmargin,hang]{footmisc}
% \usepackage[hang, flushmargin]{footmisc}
% \setlength{\footnotemargin}{0.5in}


% % \usepackage[marginal]{footmisc}
% \setlength\footnotemargin{0pt}  % default value: 1.8em

% \usepackage[flushmargin,hang]{footmisc}
% % \setlength{\footnotemargin}{1em} % just to show clearly equal output

% % \usepackage[marginal]{footmisc}
% \setlength{\footnotemargin}{10em} % just to show clearly equal output

% \renewcommand{\footnotelayout}{\raggedright}


%%%%%%%%%%
% Skip line between paragraphs, set indent to \fixw
\usepackage[parfill, indent=\fixw]{parskip}

%%%%%%%%%%
% format caption
% get rid of 'Figure: ' in caption
\usepackage{caption}
% \captionsetup[table]{labelsep=space}
\captionsetup{%
    % labelformat=empty,
    % font=small,
    labelsep=quad,
    tableposition=top,
    labelsep=period,
    margin=\fixw,
}


% Figures & bibliography
%!TEX root = outline.tex
%%%%%%%%%%%%%%%%%%%%%%%%%%%%%%%%%%%%%%%%%%%%%%%%%%%%%%%%%%%%%%%%%%%%%%%%%%%%%%
% Bibliography

% Import biblatex
\usepackage[authordate,backend=biber]{biblatex-chicago}
% \addbibresource{../bibliography.bib}
\addbibresource{bibliography.bib}
% Ensure that the cite year command includes a hyperlink
% https://tex.stackexchange.com/a/476849
\DeclareCiteCommand{\citeyear}
    {\usebibmacro{prenote}}
    {\bibhyperref{\printfield{year}}\bibhyperref{\printfield{extrayear}}}
    {\multicitedelim}
    {\usebibmacro{postnote}}

% possessive citation command
\newcommand{\citeposs}[1]{{\citeauthor{#1}'s (\citeyear{#1})}}

% remove space 
% \usepackage[font=small,skip=0pt]{caption}

%%%%%%%%%%%%%%%%%%%%%%%%%%%%%%%%%%%%%%%%%%%%%%%%%%%%%%%%%%%%%%%%%%%%%%%%%%%%%%
% plot graphs with pgfplots
\usepackage{pgfplots}
\pgfplotsset{compat=newest}
\usepgfplotslibrary{groupplots}
\usepgfplotslibrary{dateplot}
\pgfplotsset{compat=newest,
    every axis/.style={
        axis y line*=left,
        axis x line*=bottom,
        % allows for multi-line legend entries
        legend style={cells={align=left}},
        % allows for multi-line titles
        title style={align=center},
    },
}

% Include subcaptiongs in groupplots
\usepackage{subcaption}

% for \toprule or \bottomrule in tables
\usepackage{booktabs}
% for long tables
\usepackage{longtable}

% path to tex images
\makeatletter
\def\input@path{{../../img/}}
\makeatother

% Externalize pgf plots
% Note: will need to do something like: https://tex.stackexchange.com/questions/40652/references-in-externalized-pgfplots
\usetikzlibrary{external}
\tikzexternalize[prefix=figures/]

\usepackage{subcaption}

% Prevents placing floats after a new section
% Comment out for most versions
% \usepackage[section]{placeins}

% Find text width; comment out when not using to avoid warning
% \usepackage{layouts}

% Find text width
% textwidth in cm: \printinunitsof{cm}\prntlen{\textwidth}. 
% textheight in cm: \printinunitsof{cm}\prntlen{\textheight}
% outcome: 6.50127in, 16.50746cm, 469.75502pt

%%%%%%%%%%%%%%%%%%%%%%%%%%%%%%%%%%%%%%%%%%%%%%%%%%%%%%%%%%%%%%%%%%%%%%%%%%%%%%
% Toggles

% Include notes-to-self in this build; gray when true, invisible otherwise
\newif\ifnts
% \ntstrue % uncomment to show
% Notes to self 
\ifnts
  \newcommand{\nts}[1]{{\color{gray}#1}}
\else
  \newcommand{\nts}[1]{}
\fi

% Notes to self as a footnote
\newif\iffootnts
\ifnts
    \footntstrue % uncomment to show
\else
\fi
\iffootnts
  \newcommand{\footnts}[1]{\nts{\footnote{\nts{#1}}}}
\else
  \newcommand{\footnts}[1]{}
\fi


% Things I may want to edit; gray when true, black otherwise
\newif\iftoedit
% \toedittrue % uncomment to show
\iftoedit 
  \newcommand{\toedit}[1]{{\color{gray}#1}}
\else
  \newcommand{\toedit}[1]{#1}
\fi

%%%%%%%%%%%%%%%%%%%%%%%%%%%%%%%%%%%%%%%%%%%%%%%%%%%%%%%%%%%%%%%%%%%%%%%%%%%%%%
% Shorthand

% Containers
\newcommand{\br}[1]{\left\{ #1 \right\}} % curly brackets {}
\newcommand{\sbr}[1]{\left[ #1 \right]} % square bracksets []
\newcommand{\pr}[1]{\left( #1 \right)} % parentheses ()
\newcommand{\ce}[2]{\left[\left. #1 \right\vert #2 \right]} % conditional []
\newcommand{\cls}[2]{\left. #1 \right\vert #2} % conditional; find size based on left side
\newcommand{\crs}[2]{#1 \left\vert #2 \right.} % conditional; find size based on right side
\newcommand{\ceil}[1]{\left\lceil #1 \right\rceil} % ceiling function
\newcommand{\floor}[1]{\left\lfloor #1 \right\rfloor} % floor function


\newcommand{\study}{m} % I think i could change this to r if the fact that I use m as an index (in gender) is a problem
\newcommand{\pass}{s}
% state variables
\newcommand{\states}{\tilde{\study}_{jt}, \tilde{\pass}_{jt}, \alpha_{j0}, \beta_{j0}, h_{j0}}
\newcommand{\pstates}{\tilde{\study}_{jt}, \tilde{\pass}_{jt}, \psi_{j0}}
\newcommand{\ddelta}{\left\lceil \frac{\delta}{1 - \delta} \right\rceil}
%ceiling minus floor
\newcommand{\cmf}[1]{\left\lceil #1 \right\rceil - \left\lfloor #1 \right\rfloor}
% k successes
\newcommand*{\ks}[1][t]{\tilde{\pass}_{j #1}}
% \newcommand{\ks}[1]{\tilde{\pass}_{jn}}

%%%%%%%%%%%%%%%%%%%%%%%%%%%%%%%%%%%%%%%%%%%%%%%%%%%%%%%%%%%%%%%%%%%%%%%%%%%%%%
\begin{document}



\title{Group-based beliefs and human capital specialization}
\author{Tara Sullivan%\footnote{
    %Thanks to 
    %Valerie Ramey, Titan Alon, Prashant Bhradawaj, Sally Sadoff, Leo Porter;
    %Remy Levin, Daniela Vidart
}
%}

\maketitle
\onehalfspacing

\ifnts
    Please note that gray text are notes/comments. }
\fi
\iftoedit
   \toedit{Please note that gray text are placeholders that need to be edited or checked.}
\fi

\begin{abstract}
Although the overall gender gap in postsecondary degree attainment has reversed over the past forty years, significant heterogeneity persists in terms of which fields men and women choose to study.
In this paper, I consider the role of group-based beliefs in explaining these differential convergence rates across fields. 
I assume a student forms their initial belief about their probability of success in a particular field based on past outcomes for their group type. 
I then incorporate group-based beliefs into the model of gradual human capital specialization from \textcite{AF20} to show how these differences in priors can drive human capital specialization decisions amongst otherwise similar agents.
I conclude by discussing how I will use this framework in future research.
\end{abstract}

%%%%%%%%%%%%%%%%%%%%%%%%%%%%%%%%%%%%%%%%%%%%%%%%%%%%%%%%%%%%%%%%%%%%%%%%%%%%%%%%
\section{Introduction}\label{sec:intro}
\tikzset{external/figure name={intro_}}
%%%%%%%%%%%%%%%%%%%%%%%%%%%%%%%%%%%%%%%%%%%%%%%%%%%%%%%%%%%%%%%%%%%%%%%%%%%%%%%%

%!TEX root = outline.tex
\begin{figure}[b!] %htb!
\centering
% This file was created by tikzplotlib v0.9.1.
\begin{tikzpicture}

\definecolor{color0}{rgb}{0.12156862745098,0.466666666666667,0.705882352941177}
\definecolor{color1}{rgb}{1,0.498039215686275,0.0549019607843137}

\begin{axis}[
legend cell align={left},
legend style={fill opacity=0.8, draw opacity=1, text opacity=1, at={(0.03,0.97)}, anchor=north west, draw=white!80!black},
tick align=outside,
tick pos=left,
title={Number of Bachelors Degrees awarded (millions)},
x grid style={white!69.0196078431373!black},
xlabel={year},
xmin=1988.6, xmax=2019.4,
xtick style={color=black},
y grid style={white!69.0196078431373!black},
ymin=0.4493241, ymax=1.2484059,
ytick style={color=black}
]
\addplot [semithick, color0]
table {%
1990 0.48564600944519
1991 0.490826010704041
1992 0.517989993095398
1993 0.532243013381958
1994 0.532928943634033
1995 0.528069019317627
1997 0.518990993499756
1998 0.522558927536011
1999 0.522891998291016
2000 0.533735036849976
2001 0.557978987693787
2002 0.579033017158508
2003 0.599171996116638
2004 0.629392027854919
2005 0.649704933166504
2006 0.66592800617218
2007 0.687217950820923
2008 0.703808069229126
2009 0.722702980041504
2010 0.750731945037842
2011 0.779560089111328
2012 0.814333915710449
2013 0.836575031280518
2014 0.850880980491638
2015 0.862040996551514
2016 0.871549010276794
2017 0.886856079101562
2018 0.897544026374817
};
\addlegendentry{Men}
\addplot [semithick, color1]
table {%
1990 0.555091023445129
1991 0.581097006797791
1992 0.614879012107849
1993 0.632375001907349
1994 0.638270020484924
1995 0.636955976486206
1996 0.644475936889648
1997 0.652374982833862
1998 0.666815042495728
1999 0.684229016304016
2000 0.708883047103882
2001 0.745826005935669
2002 0.779849052429199
2003 0.805441975593567
2004 0.849164962768555
2005 0.87644100189209
2006 0.905745983123779
2007 0.927672982215881
2008 0.947183012962341
2009 0.968661069869995
2010 1.00603902339935
2011 1.04807901382446
2012 1.09642803668976
2013 1.12388396263123
2015 1.15306401252747
2016 1.17018795013428
2017 1.19238698482513
2018 1.2120840549469
};
\addlegendentry{Women}
\end{axis}

\end{tikzpicture}

\label{fig:n_degrees}
\end{figure}
\begin{figure}[t!] %htb!
\centering
\begin{tikzpicture}[align=left,
every node/.style={font=\footnotesize}]
% This file was created by tikzplotlib v0.9.2.
\definecolor{color0}{rgb}{0.266666666666667,0.466666666666667,0.666666666666667}
\definecolor{color1}{rgb}{0.933333333333333,0.4,0.466666666666667}
\definecolor{color2}{rgb}{0.133333333333333,0.533333333333333,0.2}
\definecolor{color3}{rgb}{0.8,0.733333333333333,0.266666666666667}
\definecolor{color4}{rgb}{0.4,0.8,0.933333333333333}
\definecolor{color5}{rgb}{0.666666666666667,0.2,0.466666666666667}

\begin{groupplot}[group style={group size=2 by 1, group name=my plots, vertical sep=2cm, horizontal sep=1.2cm}]
\nextgroupplot[
height=6.376357092455836cm,
tick align=outside,
tick pos=left,
width=9.079103cm,
x grid style={white!69.0196078431373!black},
xmin=1988.6, xmax=2030,
xtick style={color=black},
xtick={1990,1995,2000,2005,2010,2015},
xticklabels={\(\displaystyle 1990\),\(\displaystyle 1995\),\(\displaystyle 2000\),\(\displaystyle 2005\),\(\displaystyle 2010\),\(\displaystyle 2015\)},
ymajorgrids,
ymin=0, ymax=2,
ytick style={color=black},
ytick={0,0.2,0.4,0.6,0.8,1,1.2,1.4,1.6,1.8,2},
yticklabels={\(\displaystyle 0\),\(\displaystyle 0.2\),\(\displaystyle 0.4\),\(\displaystyle 0.6\),\(\displaystyle 0.8\),\(\displaystyle 1\),\(\displaystyle 1.2\),\(\displaystyle 1.4\),\(\displaystyle 1.6\),\(\displaystyle 1.8\),\(\displaystyle 2\)}
]
\addplot [semithick, color0]
table {%
1990 0.889276146888733
1991 0.908068180084229
1992 0.908786058425903
1993 0.911298394203186
1995 0.937413215637207
1996 0.960581421852112
1997 0.962796211242676
1998 0.959450006484985
1999 0.985843420028687
2000 1.00777983665466
2001 1.00261080265045
2002 1.01744496822357
2003 1.02670252323151
2004 1.02088141441345
2005 1.00308656692505
2006 0.996885895729065
2007 0.972628951072693
2008 0.96298623085022
2009 0.959978342056274
2010 0.95364236831665
2011 0.953920364379883
2012 0.931203603744507
2013 0.921860694885254
2014 0.899493217468262
2015 0.900677680969238
2016 0.891088247299194
2017 0.886778354644775
2018 0.88598895072937
};
\addplot [semithick, color1]
table {%
1990 1.22759211063385
1991 1.28919923305511
1992 1.32112109661102
1993 1.3564704656601
1994 1.39542412757874
1995 1.44826233386993
1996 1.49454152584076
1997 1.56208670139313
1998 1.59724080562592
1999 1.65807044506073
2000 1.72764885425568
2001 1.76988768577576
2002 1.76754701137543
2003 1.7330185174942
2004 1.70102310180664
2005 1.68849968910217
2006 1.65692889690399
2007 1.65950012207031
2008 1.62763059139252
2009 1.64037382602692
2010 1.63280272483826
2011 1.62376940250397
2012 1.64566314220428
2013 1.6688095331192
2014 1.67092096805573
2015 1.67829787731171
2016 1.73332345485687
2017 1.76386177539825
2018 1.79705047607422
};
\addplot [semithick, color2]
table {%
1990 0.187528491020203
1991 0.187889814376831
1992 0.18504810333252
1993 0.191165328025818
1994 0.198027372360229
1995 0.209525942802429
1996 0.219993829727173
1997 0.226323843002319
1998 0.230563402175903
1999 0.248593807220459
2000 0.259858369827271
2001 0.253314614295959
2002 0.266838073730469
2003 0.248422026634216
2004 0.258007287979126
2005 0.250089168548584
2006 0.243573904037476
2007 0.228035092353821
2008 0.227158188819885
2009 0.221451997756958
2011 0.231951236724854
2012 0.238436937332153
2013 0.240317106246948
2014 0.248236060142517
2015 0.251599311828613
2016 0.265729546546936
2017 0.274784326553345
2018 0.286244869232178
};
\addplot [semithick, color3]
table {%
1990 1.04078304767609
1991 1.04256737232208
1992 1.07366561889648
1993 1.06894600391388
1994 1.05757308006287
1995 1.10440194606781
1996 1.11996006965637
1997 1.17428719997406
1998 1.23312425613403
1999 1.30454993247986
2000 1.40134906768799
2001 1.46844744682312
2002 1.54577028751373
2003 1.63139522075653
2004 1.64863216876984
2005 1.63168549537659
2006 1.602987408638
2007 1.51491057872772
2008 1.47168242931366
2009 1.46346211433411
2010 1.41208970546722
2011 1.44205784797668
2012 1.42999362945557
2013 1.42106962203979
2014 1.41722071170807
2015 1.44253933429718
2016 1.49894797801971
2017 1.56678104400635
2018 1.64683747291565
};
\addplot [semithick, color4]
table {%
1990 0.429780006408691
1991 0.419558525085449
1993 0.395862579345703
1994 0.4017493724823
1995 0.399999976158142
1996 0.381898045539856
1997 0.373727560043335
1998 0.368845701217651
1999 0.373373627662659
2000 0.390813827514648
2002 0.384417414665222
2003 0.37157928943634
2004 0.335911631584167
2005 0.288139462471008
2006 0.261456251144409
2007 0.22909951210022
2008 0.216255187988281
2009 0.219069957733154
2010 0.223915338516235
2011 0.216599345207214
2012 0.225852251052856
2013 0.219969153404236
2014 0.223478555679321
2015 0.222921013832092
2016 0.235600471496582
2017 0.242461562156677
2018 0.257417798042297
};
\addplot [semithick, color5]
table {%
1990 0.460548996925354
1991 0.464974999427795
1992 0.4886314868927
1993 0.488829135894775
1994 0.512030601501465
1995 0.540966749191284
1996 0.565948724746704
1997 0.602615833282471
1998 0.629949331283569
1999 0.666009306907654
2000 0.686820149421692
2001 0.706353425979614
2002 0.736422896385193
2003 0.7142094373703
2004 0.72872519493103
2005 0.739134073257446
2006 0.725082635879517
2007 0.690871238708496
2008 0.687974691390991
2009 0.686181664466858
2010 0.688009262084961
2011 0.668954968452454
2012 0.67028284072876
2013 0.634251594543457
2014 0.645684003829956
2015 0.62436580657959
2016 0.629948854446411
2017 0.653665065765381
2018 0.67027759552002
};
\addplot [semithick, white!73.3333333333333!black]
table {%
1990 0.86025857925415
1991 0.899104356765747
1992 0.886856079101562
1993 0.900158047676086
1994 0.865336656570435
1995 0.88508152961731
1996 0.85411524772644
1997 0.864051580429077
1998 0.88809061050415
1999 0.93377161026001
2000 0.914776563644409
2001 0.881730198860168
2003 0.793915510177612
2004 0.801473140716553
2005 0.769950985908508
2006 0.776896238327026
2007 0.747289896011353
2008 0.764146089553833
2009 0.725896239280701
2010 0.73568868637085
2011 0.726636171340942
2012 0.726435899734497
2013 0.727647542953491
2014 0.723636150360107
2015 0.720515251159668
2016 0.709809184074402
2017 0.696071982383728
2018 0.709341764450073
};
\addplot [semithick, black]
table {%
1990 1.14299511909485
1991 1.18391644954681
1992 1.18704795837402
1993 1.18813216686249
1994 1.19766426086426
1995 1.2061984539032
1996 1.23111665248871
1997 1.25700640678406
1998 1.2760568857193
1999 1.30854749679565
2000 1.32815539836884
2001 1.33665609359741
2002 1.34681272506714
2003 1.34425842761993
2004 1.34918296337128
2005 1.3489830493927
2006 1.36012601852417
2007 1.34989619255066
2008 1.34579741954803
2009 1.34033071994781
2010 1.34007740020752
2011 1.34444940090179
2012 1.34641063213348
2013 1.34343481063843
2014 1.33796620368958
2015 1.33759760856628
2016 1.34265315532684
2017 1.34451031684875
2018 1.35044527053833
};
\draw (axis cs:2018.5,0.835989010989011) node[
  anchor=base west,
  text=color0,
  rotate=0.0
]{Business};
\draw (axis cs:2018.5,1.74705042050622) node[
  anchor=base west,
  text=color1,
  rotate=0.0
]{Social \\ Sciences};
\draw (axis cs:2018.5,0.286244813278008) node[
  anchor=base west,
  text=color2,
  rotate=0.0
]{Engineering};
\draw (axis cs:2018.5,1.44683752645192) node[
  anchor=base west,
  text=color3,
  rotate=0.0
]{Biological \\ Sciences};
\draw (axis cs:2018.5,0.037417838961352) node[
  anchor=base west,
  text=color4,
  rotate=0.0
]{Computer \\ Services};
\draw (axis cs:2018.5,0.44027762382224) node[
  anchor=base west,
  text=color5,
  rotate=0.0
]{Physical \\ Sciences};
\draw (axis cs:2018.5,0.709341764874964) node[
  anchor=base west,
  text=white!73.3333333333333!black,
  rotate=0.0
]{Math};
\draw (axis cs:2018.5,1.30044521494211) node[
  anchor=base west,
  text=black,
  rotate=0.0
]{All Fields};

\nextgroupplot[
height=6.376357092455836cm,
tick align=outside,
tick pos=left,
width=9.079103cm,
x grid style={white!69.0196078431373!black},
xmin=1988.6, xmax=2030,
xtick style={color=black},
xtick={1990,1995,2000,2005,2010,2015},
xticklabels={\(\displaystyle 1990\),\(\displaystyle 1995\),\(\displaystyle 2000\),\(\displaystyle 2005\),\(\displaystyle 2010\),\(\displaystyle 2015\)},
ymajorgrids,
ymin=0, ymax=4,
ytick style={color=black},
ytick={0,0.5,1,1.5,2,2.5,3,3.5,4},
yticklabels={\(\displaystyle 0\),\(\displaystyle 0.5\),\(\displaystyle 1\),\(\displaystyle 1.5\),\(\displaystyle 2\),\(\displaystyle 2.5\),\(\displaystyle 3\),\(\displaystyle 3.5\),\(\displaystyle 4\)}
]
\addplot [semithick, color0]
table {%
1990 2.5175244808197
1991 2.6520824432373
1992 2.74022936820984
1993 2.7347571849823
1994 2.72362613677979
1995 2.69927859306335
1996 2.70083546638489
1997 2.83463954925537
1998 2.9113347530365
1999 3.01813840866089
2000 3.2591769695282
2001 3.43097972869873
2002 3.43798732757568
2003 3.46344590187073
2004 3.48158049583435
2005 3.48040509223938
2006 3.40909743309021
2007 3.41750979423523
2008 3.35590291023254
2009 3.38099837303162
2010 3.34576916694641
2011 3.32759737968445
2012 3.26378393173218
2013 3.256920337677
2014 3.28145527839661
2015 3.37682676315308
2016 3.45335793495178
2017 3.56493353843689
2018 3.73112607002258
};
\addplot [semithick, color1]
table {%
1990 0.453454494476318
1991 0.434512615203857
1992 0.427694320678711
1993 0.426904916763306
1994 0.419453144073486
1995 0.444806575775146
1996 0.434596300125122
1998 0.464142203330994
1999 0.469876527786255
2000 0.497728705406189
2001 0.528073787689209
2002 0.521972298622131
2003 0.524367094039917
2005 0.490026116371155
2006 0.463130831718445
2007 0.454207539558411
2008 0.453919529914856
2009 0.445038676261902
2011 0.456213235855103
2012 0.440835237503052
2013 0.455652236938477
2014 0.464651584625244
2015 0.468673825263977
2016 0.484934091567993
2017 0.485498905181885
2018 0.491626977920532
};
\addplot [semithick, color2]
table {%
1990 0.687340497970581
1991 0.718239784240723
1992 0.715183019638062
1993 0.720339775085449
1994 0.741496920585632
1995 0.745949029922485
1996 0.771579146385193
1997 0.803107261657715
1998 0.797616720199585
1999 0.866689205169678
2000 0.902498960494995
2001 0.944846391677856
2002 0.949172735214233
2003 0.955212831497192
2004 0.910499453544617
2005 0.908062338829041
2006 0.878212213516235
2007 0.87516725063324
2008 0.845878720283508
2009 0.856622457504272
2010 0.84217095375061
2011 0.810709953308105
2012 0.804367065429688
2013 0.806700944900513
2014 0.798325777053833
2015 0.794228196144104
2016 0.862247109413147
2017 0.911926984786987
2018 0.932909369468689
};
\addplot [semithick, color3]
table {%
1990 2.15509796142578
1991 2.24903225898743
1993 2.16480875015259
1994 2.14018177986145
1995 2.08914875984192
1996 2.10084676742554
1997 2.15451312065125
1998 2.20654559135437
1999 2.31257438659668
2000 2.35462379455566
2001 2.42530727386475
2002 2.48414301872253
2003 2.48934626579285
2004 2.51065826416016
2005 2.44160652160645
2006 2.38261532783508
2007 2.40375828742981
2008 2.33004307746887
2009 2.34272146224976
2010 2.29821467399597
2011 2.32439589500427
2012 2.28757739067078
2013 2.31100392341614
2014 2.24083662033081
2015 2.25853753089905
2016 2.39009857177734
2017 2.41193509101868
2018 2.55055665969849
};
\addplot [semithick, color4]
table {%
1990 1.24622249603271
1991 1.27433001995087
1992 1.28972804546356
1993 1.33748888969421
1994 1.38074350357056
1995 1.43718707561493
1996 1.46975755691528
1997 1.49069452285767
1998 1.63201975822449
1999 1.6412159204483
2000 1.61032199859619
2001 1.68787050247192
2002 1.71428573131561
2004 1.5903924703598
2005 1.63254117965698
2006 1.61800968647003
2007 1.58040750026703
2008 1.52280271053314
2009 1.59942007064819
2010 1.7337794303894
2011 1.75879776477814
2012 1.74759149551392
2013 1.64136123657227
2014 1.60894978046417
2015 1.63310813903809
2016 1.62103271484375
2017 1.69435846805573
2018 1.7236739397049
};
\addplot [semithick, color5]
table {%
1990 1.45081532001495
1991 1.41702950000763
1992 1.35909819602966
1993 1.33774352073669
1994 1.30062794685364
1995 1.31255269050598
1996 1.29666662216187
1997 1.37724554538727
1998 1.40262174606323
1999 1.53457581996918
2000 1.67128205299377
2001 1.6860568523407
2002 1.77517664432526
2003 1.69076919555664
2004 1.69307291507721
2005 1.60994327068329
2006 1.57205748558044
2007 1.58068251609802
2008 1.61680722236633
2009 1.64477932453156
2010 1.62862813472748
2011 1.61050426959991
2012 1.59707045555115
2013 1.523766040802
2014 1.50999295711517
2015 1.5084331035614
2016 1.52219450473785
2017 1.56681549549103
2018 1.53559231758118
};
\addplot [semithick, white!73.3333333333333!black]
table {%
1990 1.74866712093353
1991 1.83746552467346
1992 2.02597403526306
1993 1.80351257324219
1994 1.73820173740387
1995 1.7401157617569
1996 1.74474608898163
1997 1.77926981449127
1998 1.85183620452881
1999 1.8871773481369
2000 2.04300141334534
2001 2.15603137016296
2002 2.18847274780273
2003 2.25258803367615
2004 2.24808692932129
2005 2.30172061920166
2006 2.25455236434937
2007 2.30674362182617
2008 2.26183843612671
2009 2.39688038825989
2010 2.37516474723816
2011 2.36272668838501
2012 2.4907488822937
2013 2.49604296684265
2014 2.55269312858582
2015 2.64027833938599
2016 2.73464822769165
2017 2.68607902526855
2018 2.76014471054077
};
\draw (axis cs:2018.5,3.73112597886414) node[
  anchor=base west,
  text=color0,
  rotate=0.0
]{Psychology};
\draw (axis cs:2018.5,0.491626926915826) node[
  anchor=base west,
  text=color1,
  rotate=0.0
]{Economics};
\draw (axis cs:2018.5,0.932909364532259) node[
  anchor=base west,
  text=color2,
  rotate=0.0
]{Political science};
\draw (axis cs:2018.5,2.55055658627087) node[
  anchor=base west,
  text=color3,
  rotate=0.0
]{Sociology};
\draw (axis cs:2018.5,1.72367399741268) node[
  anchor=base west,
  text=color4,
  rotate=0.0
]{Other};
\draw (axis cs:2018.5,1.53559232991857) node[
  anchor=base west,
  text=color5,
  rotate=0.0
]{Int'l relations};
\draw (axis cs:2018.5,2.76014463640016) node[
  anchor=base west,
  text=white!73.3333333333333!black,
  rotate=0.0
]{Anthropology};
\end{groupplot}



\node [text width=8.25373cm, align=center, anchor=south] at (my plots c1r1.north) {\subcaption{\label{fig:ipeds_a} Ratio of women to men}};
\node [text width=8.25373cm, align=center, anchor=south] at (my plots c2r1.north) {\subcaption{\label{fig:ipeds_b} Ratio of women to men - Social Sciences}};

\end{tikzpicture}

\caption{Ratio of women to men completing Bachelor's degrees in U.S. 4-year colleges. Source: IPEDS.}

\end{figure}

The gender gap in postsecondary degree attainment has reversed in the US over the past fifty years, as seen in figure \ref{fig:n_degrees}.
%\footnote{\toedit{This phenomenon is also documented in \textcite{BHM10}.}} 
The overall gender convergence in human capital has \toedit{been related to a number of macroeconomic benefits, including a reduction in the gender wage gap \parencite{BK17} and increased aggregate economic productivity \parencite{HHJK19}.} 
% \toedit{Moreover, improved allocation of talent increases overall economic productivity \parencite{HHJK19}.}
% However, gender gaps in paritcular fields of study occasionally if we consider specific fields of study, gender convergence becomes much more unclear. 
However, this pattern is not uniformly observed across fields of study.
Consider figure \ref{fig:ipeds_a}, which plots the ratio of women to men completing Bachelor's degrees \toedit{in several subjects}.\footnts{
    I may want to use HEGIS data to emphasize that these are historically male dominated fields.
}
While the gender ratios of some fields have increased since 1990, others have remained flat or worsened. 
More generally, aggregations of college majors can easily mask heterogeneity in gender convergence across fields \parencite{BHST08}.
This can be seen when comparing the overall Social Sciences gender ratio from Figure \ref{fig:ipeds_a} with those of its subfields in \ref{fig:ipeds_b}.
% Although these differences in major choice across genders appear to matter for labor market outcomes \parencite{SHB19}, the reasons these differences exist is not well understood. 
Because differences in major choice across genders appear to matter for labor market outcomes \parencite{SHB19}, it is worth examining why these differences persist.

This paper addresses this heterogeneity using a model of group-based belief formation and gradual human capital specialization. 
Building on \textcite{AF20}, I assume individuals belonging to a particular group choose to work or study in heterogeneous fields. 
Returns to education are stochastic, and underlying abilities are unknown.
Agents form beliefs about their unknown abilities based on existing group outcomes, and update these beliefs as they study. 
I can use this model to analytically characterize the dynamics of student's belief distribution as their education proceeds. 

I then use simulations to highlight the different mechanisms of this model.
% Agent specialization decisions may be motivated by wages, abilities, or beliefs. 
In particular, I highlight the important role that beliefs play in impacting specialization decisions.\footnts{
    Good to mention here the types of checks I want to do. For example:
    allowing abilities to be correlated. 
}
I use these results to motivate how we can view the model through the lens of statistical discrimination. 
Future work will identify model parameters and conduct counterfactual exercises. 
This draft will briefly review possible identification methods and data sources. 
I will then discuss in the conclusion several possible quantitative exercises this model can be used for. 


This paper proceeds as follows. After a brief literature review, I outline the model in section \ref{sec:model}. Analytical results are derived in section \ref{sec:analytic_results}, and implications of the model are explored in section \ref{sec:sims}. The connection between this model and the theory of statistical discrimination are outlined in section \ref{sec:stat_discrim}. Next steps, including identification and quantitative analysis, are discussed in section \ref{sec:identification}.

%%%%%%%%%%%%%%%%%%%%%%%%%%%%%%%%%%%%%%%%%%%%%%%%%%%%%%%%%%%%%%%%%%%%%%%%%%%%%%%%
\subsubsection*{Literature}

This paper builds upon the extensive literature on human capital formation \parencite{B62,B67,M74,R83}. 
I expand on \citeposs{AF20} theoretical model of gradual specialization, a recent contribution to the literature. 
Their framework closely relates to two classical papers from this field.
\toedit{The first is the seminal \textcite{M74} model of the returns to education. 
The second is the \textcite{R51} model of occupational choice and skill heterogeneity.
% Their paper effectively generalizes the mincerean model to include skill heterogeneity 
\citeposs{AF20} model, and by extension my own, can be viewed as a generalization of the mincerian model of human capital accumulation to include a dynamic Roy model with unknown heterogeneous abilities and sequential learning.}

This paper focuses on pre-labor market specialization decisions, specifically college major choice. 
As such, this project is closely tied to the theoretical and empirical literature on education decisions and college major choice.
\citeposs{A93} seminal work on the role of uncertainty in sequential education decisions provides an important theoretical antecedent for this paper. 
The empirical literature on the determinants of educational choices and returns to education is too extensive to fully outline here; as such, I refer the reader to \textcite{PWZ20} for an excellent overview of the determinants of college major choice.\footnote{
    Additional literature reviews are found in \textcite{ABM12} and \textcite{AAM16-education}.
    % Following the terminology from \textcite{PWZ20}, the determinants of college major choice can be broadly categorized into six categories: ``expected earnings and ability sorting, learning, subjective expectations, non-pecuniary considerations, peer and family effects, and supply side factors.'' 
}
Although major-specific wages are an important determinant of college major choice in this model, non-pecuniary motivations are central to this analysis.
These include sorting on ability \parencite{A04}, preferences \parencite{Z13}, peer effects, \parencite{PS20}, and beliefs \parencite{AHMR-wp}.
Of particular relevance is work on how these determinants differ by gender. 


% This paper related to empirical literature on gender gaps and college choice.
% Overall, the literature on college major choice often considers differences in specialization decisions by gender. 
This paper is largely motivated by empirical work documenting the relationship between gender and college major choice.
The gender convergence in overall college degree attainment is a well-studied phenomenon, and thoroughly reviewed in \textcite{BK17}.
However, most of this gender convergence ended by the 1980s, well before parity was achieved \parencite{SHB19,EL06}.
% The literature suggests several reasons for gender differences in college major choice.
% A number of studies estimate the impact of preferences \parencite{Z13,WZ14}, including preferences over lifetime temporal flexibility \parencite{B15,WZ18}.
% % Factors such as temporal flexibility may be key in explaining the lifetime gender wage gap \parencite{G14,KLS19}.
% Though my model can accommodate field-specific preferences, my paper primarily focuses on the role of beliefs and belief formation.
% This approach is consistent with a number of determinants of gender major choice, including the presence of same-gender role models \parencite{PS20,LM20}, and the role of negative feedback \parencite{KTU17}.
% % Importantly, wages do not appear to be a driving factor; as noted in \textcite{SHB19}, women often sort into lower paying majors, in addition to sorting into lower paying occupations conditional on their major choice.



% Differences in major choice by gender has been documented using both administrative data \parencite{D10} and survey results \parencite{Z13}.
% \textcite{BHST08} use survey data to show that college major choice is closely related to wages broadly and the gender wage gap specifically.
% The most useful papers suggest a reason for why gender gaps exist.

%\textcite{SHB19} provides a key empirical motivation for this paper.
%They use American Community Survey data to explore the importance of major choice in explaining labor market outcomes.


% As noted in \toedit{\textcite{Z13}}, much of the literature on nonpecuniary determinants focus on the primacy of preferences. 
% \nts{What did Titan and I say about preferences}

% \begin{blist}
% \item Some experimentation is optimal in this paper (I think); this is different than what you get from a static choice model like \textcite{Z13} (and, I think, \textcite{A93}).
% \item  Also, as mentioned in \textcite{Z13}, ability is important \toedit{(and may have differential impacts on ex-post payoffs by gender)}
% \item \textcite{A04}: ability matters; sequential uncertainty
% \end{blist}

% The role of non-pecuniary job attributes in determining college major choice.
% These papers focus on the role of preferences.
% how non-pecuniary job characteristics influence major choice.


% My papers follows the tradition of papers that consider the role of uncertainty in education decisions (Zafar 2013)

% This paper builds on recent literature in human capital specialization.
% In particular, I draw on work about college major choice and occupational choice \parencite{ABM12,AAM16-education}.
% Of particular interest is research on the role of beliefs in human capital specialization decisions \parencite{AHMR-wp}.
% This paper shares several theoretical commonalities with \textcite{AAMR16-wp}, who build a dynamic model of school and work decisions, though they focus on attrition; as such, major choice is broadly characterized as a choice between STEM fields and non-STEM fields. 
% They also assume students are uncertain about underlying abilities and update these beliefs over time.




% \nts{They also don't focus on how groups influence belief formation, I don't think}

% \textcite{PS20} find experimental evidence that the existence of same-gender role models are an important determinant of major choice.

% It is worth highlighting that my research does not discount the impact of children on the gender wage gap.
% \textcite{KLS19} provide strong evidence that children have a significant impact on lifetime earnings for women.

Finally, the results of this paper will be closely tied to statistical discrimination literature. 
The theoretical connection between the model outlined in this paper and the theory of statistical discrimination will be discussed in detail in section \ref{sec:stat_discrim}; relevant literature will be reviewed there.


%%%%%%%%%%%%%%%%%%%%%%%%%%%%%%%%%%%%%%%%%%%%%%%%%%%%%%%%%%%%%%%%%%%%%%%%%%%%%%%%
\section{Model of human capital specialization}\label{sec:model}
\tikzset{external/figure name={model_}}
%%%%%%%%%%%%%%%%%%%%%%%%%%%%%%%%%%%%%%%%%%%%%%%%%%%%%%%%%%%%%%%%%%%%%%%%%%%%%%%%

% begin with an Alon-Fershtman slide; review as quicly as possible ``because many people here are familar with this paper, I'll try to keep this short'' Maybe check with david that this is a good idea
% Then jump into my key contribution: how do we think about the priors

%%%%%%%%%%%%%%%%%%%%%%%%%%%%%%%%%%%%%%%%%%%%%%%%%%%%%%%%%%%%%%%%%%%%%%%%%%%%%%%%
%%%%%%%%%%%%%%%%%%%%%%%%%%%%%%%%%%%%%%%%%%%%%%%%%%%%%%%%%%%%%%%%%%%%%%%%%%%%%%%%
\begin{frame}{Model preliminaries}

% Discrete time
% Infinitely lived agents

% unknown success probability in j

% write that with an indicator function

% 

Individuals endowed with:
\begin{itemize}
    \item [$h_{j0}$:] Skill-$j$ specific human capital ($j=0,\dots,J$)
    \item [$\theta_j$:] Unknown probability of success in $j$
    \item [$P_{j0}$:] Prior beliefs about $\theta_j$
\end{itemize}

\vspace{2ex}
At each $t$, individuals can choose to either study or work in one skill-$j$:
\begin{itemize}
    \item Studying accumulates skill-$j$ human capital and reveals information about underlying probability of success in $j$
    % Endogenous enter labor market at time t as a skill j specialist to maximize 
    \item If you work, you receive wage $w_j$
% To keep things simple, I'll assume utility is linear, and that the value of entering the market in period t as a skill k specialist simply depends on lifetime earnings
\end{itemize}
% Time constraint:
% \begin{equation*}
%     \sum_{j=0}^J (s_{jt} + \ell_{jt}) = 1, \quad \quad s_{jt}, \ell_{jt} \in \{0, 1\}
% \end{equation*}

\vspace{2ex}
Enter labor market at time $t$ in skill-$j$ to maximize expected lifetime payoff:
\begin{equation*}
    \frac{\delta^t}{1 - \delta} U_j (w_{j}, h_{jt}) \ell_{jt}
    = \frac{\delta^t}{1 - \delta} w_{j} h_{jt} \ell_{jt}
    % Is it okay to say this? 
\end{equation*}
% \vspace{2ex}


\end{frame}

%%%%%%%%%%%%%%%%%%%%%%%%%%%%%%%%%%%%%%%%%%%%%%%%%%%%%%%%%%%%%%%%%%%%%%%%%%%%%%%%
% \begin{frame}{Student specialization decision}
\begin{frame}{Evolution of human capital accumulation and beliefs}

% Choose probability 

Students studying skill-$j$ at time $t$ pass the course with probability $\theta_j$:
\begin{equation*}
    a_{jt} \sim \text{Bernoulli} (\theta_j)
\end{equation*}
\vspace{-2.5ex}
\begin{itemize}
  \item 
  Accumulate human capital if they pass the course:
  \begin{align*}
  % h_{jt+1} =& H(h_{kt}, a_{kt}), \quad \quad
%   % a_{jt} \sim F_{\theta_j}
    h_{jt+1} = h_{jt} + \nu_{j} a_{jt}
  \end{align*}
  \item 
  Beliefs about $\theta_j$ evolve:
  \begin{equation*}
      \mgreen<2->{P_{j,t+1}} = \Pi_j (\mblue<2->{P_{jt}},a_{jt})
        % P_{j, t+1} 
  \end{equation*}
\end{itemize}

\pause
\vspace{5ex}
\textbf{Key:} How are  \blue<2->{priors} formed, and how are they \green{updated}? 

% Parameter we are trying to understand is probability of success
% Know that we want a prior distribution supported on [0, 1]

\end{frame}

%%%%%%%%%%%%%%%%%%%%%%%%%%%%%%%%%%%%%%%%%%%%%%%%%%%%%%%%%%%%%%%%%%%%%%%%%%%%%%%%
\begin{frame}[t]{Belief distribution}

Initial prior drawn from Beta distribution 
\begin{equation*}
    \mblue{P_{j0}} = \mathcal{B} (\alpha_{j0}, \beta_{j0})
\end{equation*}
% Beta distribution is appropriate 
% need a probability distribution supported on [0,1]
% Also has some desireable analytic properties
Update according to Bayes Rule $\implies$ posterior drawn from Beta distribution:
\begin{equation*}
    \mgreen{P_{j,t+1}} = \mathcal{B} (\alpha_{j,t+1}, \beta_{j,t+1}), \quad \quad 
    (\alpha_{j,t+1}, \beta_{j,t+1}) = 
    \begin{cases} 
        (\alpha_{jt} + 1, \beta_{jt}) &\text{ if } a_{jt} = 1 \\
        (\alpha_{jt}, \beta_{jt} + 1) &\text{ if } a_{jt} = 0
    \end{cases}
\end{equation*}

\vspace{3ex}
\begin{columns}[T] % align columns
\begin{column}{.51\textwidth}

\pause
\vspace{3ex}
Example: $\alpha_0 = 1$, $\beta_0 = 1$
\vspace{1.5ex}
\begin{itemize}
    \item <3-> success at $t=1$ $\implies$ $\alpha_1 = 2$, $\beta_1 = 1$

    \vspace{1.5ex}
    \item <4-> failure at $t=2$ $\implies$ $\alpha_1 = 2$, $\beta_1 = 2$
    
    \vspace{1.5ex}
    \item <5-> success at $t=3$ $\implies$ $\alpha_1 = 3$, $\beta_1 = 2$
\end{itemize}

\end{column}%
% \hfill%
\begin{column}{.39\textwidth}
\begin{figure}
\only<2>{% This file was created by tikzplotlib v0.9.1.
\begin{tikzpicture}

\definecolor{color0}{rgb}{0.12156862745098,0.466666666666667,0.705882352941177}

\begin{axis}[
height=120pt,
tick align=outside,
tick pos=left,
title={Beliefs \(\displaystyle p(\theta | \alpha, \beta)\)},
width=150pt,
x grid style={white!69.0196078431373!black},
xmin=0, xmax=1,
xtick style={color=black},
y grid style={white!69.0196078431373!black},
ymin=0, ymax=2.25,
ytick style={color=black}
]
\addplot [very thick, color0]
table {%
0 1
1 1
};
\end{axis}

\end{tikzpicture}
}
\only<3>{% This file was created by tikzplotlib v0.9.1.
\begin{tikzpicture}

\definecolor{color0}{rgb}{0.12156862745098,0.466666666666667,0.705882352941177}

\begin{axis}[
height=120pt,
tick align=outside,
tick pos=left,
title={Beliefs \(\displaystyle p(\theta | \alpha, \beta)\)},
width=150pt,
x grid style={white!69.0196078431373!black},
xmin=0, xmax=1,
xtick style={color=black},
y grid style={white!69.0196078431373!black},
ymin=0, ymax=2.25,
ytick style={color=black}
]
\addplot [semithick, black, opacity=0.5]
table {%
0 1
1 1
};
\addplot [very thick, color0]
table {%
0 0
1 2
};
\end{axis}

\end{tikzpicture}
}
\only<4>{% This file was created by tikzplotlib v0.9.2.
\begin{tikzpicture}

\definecolor{color0}{rgb}{0.12156862745098,0.466666666666667,0.705882352941177}

\begin{axis}[
height=120pt,
tick align=outside,
tick pos=left,
title={Beliefs \(\displaystyle p(\theta | \alpha, \beta)\)},
width=150pt,
x grid style={white!69.0196078431373!black},
xmin=0, xmax=1,
xtick style={color=black},
y grid style={white!69.0196078431373!black},
ymin=0, ymax=2.25,
ytick style={color=black}
]
\addplot [semithick, black, opacity=0.3]
table {%
0 1
1 1
};
\addplot [semithick, black, opacity=0.5]
table {%
0 0
1 2
};
\addplot [very thick, color0]
table {%
0 0
0.0204081535339355 0.11995005607605
0.0408163070678711 0.234902143478394
0.0612244606018066 0.344856262207031
0.0816326141357422 0.449812650680542
0.102040767669678 0.549770951271057
0.122448921203613 0.644731283187866
0.142857193946838 0.734693884849548
0.163265347480774 0.819658517837524
0.183673501014709 0.899625182151794
0.204081654548645 0.974593877792358
0.224489808082581 1.04456472396851
0.244897961616516 1.10953772068024
0.265306115150452 1.16951274871826
0.285714268684387 1.22448980808258
0.306122422218323 1.27446901798248
0.326530575752258 1.31945025920868
0.346938848495483 1.35943353176117
0.367347002029419 1.39441895484924
0.387755155563354 1.4244065284729
0.40816330909729 1.44939613342285
0.428571462631226 1.4693877696991
0.448979616165161 1.48438155651093
0.469387769699097 1.49437737464905
0.489795923233032 1.49937522411346
0.510204076766968 1.49937522411346
0.530612230300903 1.49437737464905
0.551020383834839 1.48438155651093
0.571428537368774 1.4693877696991
0.59183669090271 1.44939613342285
0.612244844436646 1.4244065284729
0.632652997970581 1.39441895484924
0.653061151504517 1.35943353176117
0.673469305038452 1.31945025920868
0.693877577781677 1.27446901798248
0.714285731315613 1.22448980808258
0.734693884849548 1.16951274871826
0.755102038383484 1.10953772068024
0.775510191917419 1.04456472396851
0.795918345451355 0.974593877792358
0.81632661819458 0.899625182151794
0.836734771728516 0.819658517837524
0.857142925262451 0.734693884849548
0.877551078796387 0.644731283187866
0.897959232330322 0.549770951271057
0.918367385864258 0.449812650680542
0.938775539398193 0.344856262207031
0.959183692932129 0.234902143478394
0.979591846466064 0.11995005607605
1 0
};
\end{axis}

\end{tikzpicture}
}
\only<5>{% This file was created by tikzplotlib v0.9.2.
\begin{tikzpicture}

\definecolor{color0}{rgb}{0.12156862745098,0.466666666666667,0.705882352941177}

\begin{axis}[
height=120pt,
tick align=outside,
tick pos=left,
title={Beliefs \(\displaystyle p(\theta | \alpha, \beta)\)},
width=150pt,
x grid style={white!69.0196078431373!black},
xmin=0, xmax=1,
xtick style={color=black},
y grid style={white!69.0196078431373!black},
ymin=0, ymax=2.25,
ytick style={color=black}
]
\addplot [semithick, black, opacity=0.1]
table {%
0 1
1 1
};
\addplot [semithick, black, opacity=0.3]
table {%
0 0
1 2
};
\addplot [semithick, black, opacity=0.5]
table {%
0 0
0.0204081535339355 0.11995005607605
0.0408163070678711 0.234902143478394
0.0612244606018066 0.344856262207031
0.0816326141357422 0.449812650680542
0.102040767669678 0.549770951271057
0.122448921203613 0.644731283187866
0.142857193946838 0.734693884849548
0.163265347480774 0.819658517837524
0.183673501014709 0.899625182151794
0.204081654548645 0.974593877792358
0.224489808082581 1.04456472396851
0.244897961616516 1.10953772068024
0.265306115150452 1.16951274871826
0.285714268684387 1.22448980808258
0.306122422218323 1.27446901798248
0.326530575752258 1.31945025920868
0.346938848495483 1.35943353176117
0.367347002029419 1.39441895484924
0.387755155563354 1.4244065284729
0.40816330909729 1.44939613342285
0.428571462631226 1.4693877696991
0.448979616165161 1.48438155651093
0.469387769699097 1.49437737464905
0.489795923233032 1.49937522411346
0.510204076766968 1.49937522411346
0.530612230300903 1.49437737464905
0.551020383834839 1.48438155651093
0.571428537368774 1.4693877696991
0.59183669090271 1.44939613342285
0.612244844436646 1.4244065284729
0.632652997970581 1.39441895484924
0.653061151504517 1.35943353176117
0.673469305038452 1.31945025920868
0.693877577781677 1.27446901798248
0.714285731315613 1.22448980808258
0.734693884849548 1.16951274871826
0.755102038383484 1.10953772068024
0.775510191917419 1.04456472396851
0.795918345451355 0.974593877792358
0.81632661819458 0.899625182151794
0.836734771728516 0.819658517837524
0.857142925262451 0.734693884849548
0.877551078796387 0.644731283187866
0.897959232330322 0.549770951271057
0.918367385864258 0.449812650680542
0.938775539398193 0.344856262207031
0.959183692932129 0.234902143478394
0.979591846466064 0.11995005607605
1 0
};
\addplot [very thick, color0]
table {%
0 0
0.0204081535339355 0.00489592552185059
0.0408163070678711 0.01917564868927
0.0612244606018066 0.0422272682189941
0.0816326141357422 0.0734387636184692
0.102040767669678 0.112198114395142
0.122448921203613 0.157893419265747
0.142857193946838 0.209912538528442
0.163265347480774 0.267643570899963
0.183673501014709 0.330474615097046
0.204081654548645 0.397793412208557
0.224489808082581 0.468988299369812
0.244897961616516 0.543447017669678
0.265306115150452 0.62055778503418
0.285714268684387 0.699708461761475
0.326530575752258 0.861681699752808
0.367347002029419 1.02447104454041
0.387755155563354 1.10464179515839
0.40816330909729 1.18318045139313
0.428571462631226 1.25947523117065
0.448979616165161 1.33291399478912
0.469387769699097 1.40288484096527
0.489795923233032 1.46877574920654
0.510204076766968 1.52997469902039
0.530612230300903 1.58586978912354
0.551020383834839 1.63584899902344
0.571428537368774 1.67930030822754
0.59183669090271 1.71561169624329
0.612244844436646 1.74417126178741
0.632652997970581 1.76436686515808
0.653061151504517 1.77558672428131
0.673469305038452 1.77721869945526
0.693877577781677 1.76865077018738
0.714285731315613 1.74927115440369
0.734693884849548 1.71846759319305
0.755102038383484 1.67562830448151
0.775510191917419 1.62014126777649
0.795918345451355 1.55139434337616
0.81632661819458 1.46877574920654
0.836734771728516 1.3716733455658
0.857142925262451 1.25947523117065
0.877551078796387 1.13156938552856
0.897959232330322 0.987343788146973
0.918367385864258 0.826186418533325
0.938775539398193 0.647485256195068
0.959183692932129 0.450628519058228
0.979591846466064 0.235004186630249
1 0
};
\end{axis}

\end{tikzpicture}
}
\end{figure}
\end{column}%
\end{columns}
% \hypertarget<2>{beta_11_example}{\beamerbutton{I'm on the fourth slide}}
\hypertarget<2>{model_beta_11}{
  \hyperlink{simulate}{\beamerbutton{Return: simulation set-up}}
  \hyperlink{sim_default}{\beamerbutton{Return: baseline simulation}}
}
\hypertarget<4>{model_beta_22}{\hyperlink{sim_beliefs}{\beamerbutton{Return: simulation}}}
% \hyperlink{belief_effect}{\beamerbutton{Return: simulation}}


\end{frame}




%%%%%%%%%%%%%%%%%%%%%%%%%%%%%%%%%%%%%%%%%%%%%%%%%%%%%%%%%%%%%%%%%%%%%%%%%%%%%%%%
\begin{frame}{Group-based parametrization}



Consider group-based beliefs about abilities:
\begin{itemize}
    \item Each individual has a group type: $g \in \{m, f\}$

    \item Students form beliefs, $P_{j0}$, based on previously observed group successes
\end{itemize}

\vspace{2ex}
Simple parameterization:
\begin{itemize}
    \item [$\alpha_{j0}^g$: ] Number of type-$g$ students who have succeeded in $j$ 
    \item [$\beta_{j0}^g$: ] Number of type-$g$ students who have failed in $j$

    \item [$\implies$] Observed success rate:
    \begin{equation*}
    \mu_{j0}^g = 
      \frac{\alpha_{j0}^g}{\alpha_{j0}^g + \beta_{j0}^g}.
\end{equation*}
This average is based on a sample size of type $g$ students:
\begin{equation*}
    n_{j0}^g = \alpha_{j0}^g + \beta_{j0}^g
 \end{equation*}
\end{itemize}

\vspace{2ex}
Group-based prior beliefs about probability of success in skill-$j$ courses, $\theta_j$:
\begin{equation*}
    \mathcal{B} \pr{\alpha_{j0}^g, \beta_{j0}^g} \quad \implies \quad
    \mathcal{B} \pr{\mu_{j0}^g n_{j0}^g, (1 - \mu_{j0}^g) n_{j0}^g}
\end{equation*}



\end{frame}

%%%%%%%%%%%%%%%%%%%%%%%%%%%%%%%%%%%%%%%%%%%%%%%%%%%%%%%%%%%%%%%%%%%%%%%%%%%%%%%%
\begin{frame}{Group-based belief distribution}

% Suppose sample size of men is larger than that of women, but the observed success rate is the same for the two groups:
Suppose there are more men then women in field $j$: 
\begin{equation*}
  n_{j0}^m > n_{j0}^f
\end{equation*}
But the observed success rate is the same for the two groups:
\begin{equation*}
    \mu_{j0} = \mu_{j0}^m = \mu_{j0}^w
\end{equation*}

\begin{figure}
% This file was created by tikzplotlib v0.9.2.
\begin{tikzpicture}

\definecolor{color0}{rgb}{0.266666666666667,0.466666666666667,0.666666666666667}
\definecolor{color1}{rgb}{0.933333333333333,0.4,0.466666666666667}

\begin{axis}[
height=6.376357092455836cm,
legend cell align={left},
legend style={fill opacity=0.8, draw opacity=1, text opacity=1, at={(0.03,0.97)}, anchor=north west, draw=none},
tick align=outside,
tick pos=left,
width=10.317162499999998cm,
x grid style={white!69.0196078431373!black},
xmin=-0.05, xmax=1.05,
xtick style={color=black},
y grid style={white!69.0196078431373!black},
ymin=0, ymax=2.66141549653429,
ytick style={color=black}
]
\addplot [semithick, color0]
table {%
0 0
0.0580580234527588 0.000277876853942871
0.0750750303268433 0.000951051712036133
0.0880880355834961 0.00202715396881104
0.0990991592407227 0.00352215766906738
0.108108162879944 0.0052802562713623
0.117117166519165 0.00764262676239014
0.125125169754028 0.0103511810302734
0.132132172584534 0.0132688283920288
0.139139175415039 0.0167677402496338
0.146146178245544 0.0209178924560547
0.15315318107605 0.0257914066314697
0.160160183906555 0.0314624309539795
0.166166186332703 0.0370153188705444
0.17217218875885 0.0432579517364502
0.178178191184998 0.0502384901046753
0.184184193611145 0.0580054521560669
0.190190196037292 0.0666069984436035
0.19619619846344 0.0760910511016846
0.203203201293945 0.0883336067199707
0.210210204124451 0.101914048194885
0.217217206954956 0.116902947425842
0.224224209785461 0.133368015289307
0.231231212615967 0.151373624801636
0.238238215446472 0.170979976654053
0.245245218276978 0.192243218421936
0.252252221107483 0.215214252471924
0.260260224342346 0.243616461753845
0.268268227577209 0.274367570877075
0.276276350021362 0.307515382766724
0.284284353256226 0.343096613883972
0.292292356491089 0.381135225296021
0.301301240921021 0.426879644393921
0.310310363769531 0.475743412971497
0.319319248199463 0.527699708938599
0.329329371452332 0.588993549346924
0.339339375495911 0.65393853187561
0.350350379943848 0.729425430297852
0.361361384391785 0.808918356895447
0.37337338924408 0.899857521057129
0.386386394500732 1.00282371044159
0.401401400566101 1.12650215625763
0.419419407844543 1.28014957904816
0.448448419570923 1.53373908996582
0.473473429679871 1.75053000450134
0.489489555358887 1.88434946537018
0.50250244140625 1.98835706710815
0.513513565063477 2.07205963134766
0.523523569107056 2.14405512809753
0.532532453536987 2.2050347328186
0.541541576385498 2.2619833946228
0.549549579620361 2.30890417098999
0.556556582450867 2.34688472747803
0.563563585281372 2.38181519508362
0.56956958770752 2.40920257568359
0.575575590133667 2.43412804603577
0.581581592559814 2.45649647712708
0.586586594581604 2.47311806678772
0.591591596603394 2.48785185813904
0.596596598625183 2.50065088272095
0.600600600242615 2.50946736335754
0.604604601860046 2.51699638366699
0.608608603477478 2.52321815490723
0.611611604690552 2.527015209198
0.614614605903625 2.53005909919739
0.617617607116699 2.53234314918518
0.620620608329773 2.53386044502258
0.623623609542847 2.53460574150085
0.62662661075592 2.53457283973694
0.629629611968994 2.53375744819641
0.632632613182068 2.53215456008911
0.635635614395142 2.52976059913635
0.638638615608215 2.52657175064087
0.641641616821289 2.52258539199829
0.644644618034363 2.51779890060425
0.648648619651794 2.51016926765442
0.652652740478516 2.50111126899719
0.656656742095947 2.49062418937683
0.660660743713379 2.47870826721191
0.665665626525879 2.46180844306946
0.670670747756958 2.44268989562988
0.675675630569458 2.42136645317078
0.681681632995605 2.39289450645447
0.687687635421753 2.36131596565247
0.6936936378479 2.32668232917786
0.700700759887695 2.28249788284302
0.707707643508911 2.23435544967651
0.714714765548706 2.18238949775696
0.722722768783569 2.11851692199707
0.730730772018433 2.05011296272278
0.739739656448364 1.9681077003479
0.749749660491943 1.87126696109772
0.76076078414917 1.75860500335693
0.772772789001465 1.62956130504608
0.787787795066833 1.46146321296692
0.810810804367065 1.19596135616302
0.834834814071655 0.920849561691284
0.848848819732666 0.767037749290466
0.859859943389893 0.652013540267944
0.869869947433472 0.553140640258789
0.878878831863403 0.469607830047607
0.886886835098267 0.400230884552002
0.89489483833313 0.335862994194031
0.901901960372925 0.283928394317627
0.908908843994141 0.236297965049744
0.914914846420288 0.199018955230713
0.920920848846436 0.165092349052429
0.926926851272583 0.134564399719238
0.931931972503662 0.111733078956604
0.936936855316162 0.0912656784057617
0.941941976547241 0.0731372833251953
0.946946859359741 0.0573046207427979
0.950950980186462 0.0462501049041748
0.954954981803894 0.0365835428237915
0.958958983421326 0.0282543897628784
0.962962985038757 0.021202564239502
0.966966986656189 0.0153579711914062
0.970970988273621 0.0106403827667236
0.974974989891052 0.00695860385894775
0.978978991508484 0.00420975685119629
0.982982993125916 0.00227940082550049
0.986986994743347 0.00104022026062012
0.990990996360779 0.000352263450622559
0.996996998786926 1.34706497192383e-05
1 0
};
\addlegendentry{Men \\ ($\mu = $0.6, $n = $10)}
\addplot [semithick, color1]
table {%
0 0
0.00500500202178955 0.00029909610748291
0.0100100040435791 0.0011904239654541
0.0150150060653687 0.00266480445861816
0.0200200080871582 0.00471329689025879
0.0250250101089478 0.00732696056365967
0.0310310125350952 0.011196494102478
0.0370370149612427 0.0158512592315674
0.0430430173873901 0.021275520324707
0.0490490198135376 0.0274536609649658
0.0550550222396851 0.0343701839447021
0.0620620250701904 0.0433518886566162
0.0690690279006958 0.0532925128936768
0.0760760307312012 0.0641672611236572
0.0840840339660645 0.0777077674865723
0.0920920372009277 0.0923991203308105
0.101101160049438 0.110256433486938
0.11011016368866 0.129470825195312
0.119119167327881 0.149989724159241
0.12912917137146 0.174254298210144
0.139139175415039 0.199992060661316
0.150150179862976 0.229919075965881
0.161161184310913 0.261445164680481
0.173173189163208 0.297548055648804
0.186186194419861 0.338533163070679
0.200200200080872 0.384672880172729
0.21521520614624 0.436192035675049
0.231231212615967 0.493253231048584
0.249249219894409 0.559686422348022
0.269269227981567 0.635787844657898
0.293293237686157 0.729499101638794
0.327327370643616 0.864867448806763
0.379379391670227 1.07190155982971
0.403403401374817 1.16504085063934
0.423423409461975 1.24047493934631
0.441441416740417 1.30615937709808
0.457457423210144 1.36243712902069
0.472472429275513 1.41312110424042
0.486486434936523 1.45839333534241
0.499499559402466 1.49849700927734
0.511511564254761 1.5337210893631
0.522522449493408 1.56438684463501
0.533533573150635 1.59340107440948
0.543543577194214 1.61826372146606
0.553553581237793 1.64160966873169
0.562562584877014 1.66126477718353
0.571571588516235 1.67958033084869
0.580580592155457 1.69650363922119
0.58858859539032 1.71033525466919
0.596596598625183 1.72298836708069
0.603603601455688 1.73306393623352
0.610610604286194 1.74218416213989
0.617617607116699 1.75032413005829
0.623623609542847 1.75650227069855
0.629629611968994 1.76192653179169
0.635635614395142 1.76658129692078
0.641641616821289 1.77045083045959
0.646646618843079 1.773064494133
0.651651620864868 1.77511298656464
0.656656742095947 1.7765873670578
0.661661624908447 1.77747869491577
0.666666746139526 1.77777779102325
0.671671628952026 1.77747571468353
0.676676750183105 1.77656328678131
0.681681632995605 1.77503180503845
0.686686754226685 1.77287185192108
0.691691637039185 1.77007472515106
0.696696758270264 1.76663112640381
0.701701641082764 1.76253223419189
0.706706762313843 1.75776898860931
0.71271276473999 1.75116336345673
0.718718767166138 1.743572473526
0.724724769592285 1.73498058319092
0.730730772018433 1.72537219524384
0.73673677444458 1.71473157405853
0.742742776870728 1.70304334163666
0.749749660491943 1.68806207180023
0.756756782531738 1.67160880565643
0.763763785362244 1.65365862846375
0.770770788192749 1.63418686389923
0.777777791023254 1.61316871643066
0.785785794258118 1.58722269535065
0.793793797492981 1.55918765068054
0.801801800727844 1.5290265083313
0.809809803962708 1.49670231342316
0.817817807197571 1.46217811107635
0.826826810836792 1.42066276073456
0.835835814476013 1.37626349925995
0.844844818115234 1.32892775535583
0.853853940963745 1.27860295772552
0.862862825393677 1.22523641586304
0.872872829437256 1.16230869293213
0.882882833480835 1.09548842906952
0.892892837524414 1.02470362186432
0.902902841567993 0.949881792068481
0.912912845611572 0.870950937271118
0.923923969268799 0.779294729232788
0.934934854507446 0.682483077049255
0.945945978164673 0.580419778823853
0.95695698261261 0.473008632659912
0.967967987060547 0.360153555870056
0.978978991508484 0.241758584976196
0.990990996360779 0.106168985366821
1 0
};
\addlegendentry{Women \\ ($\mu = $0.6, $n = $5)}
\addplot [semithick, white!73.3333333333333!black, dotted, forget plot]
table {%
0.600000023841858 0
0.600000023841858 2.66141557693481
};
\end{axis}

\end{tikzpicture}

\end{figure}


\end{frame}


%%%%%%%%%%%%%%%%%%%%%%%%%%%%%%%%%%%%%%%%%%%%%%%%%%%%%%%%%%%%%%%%%%%%%%%%%%%%%%%%

% \nts{Frame remaining simple parameterization}
\begin{frame}{Individual problem}


A policy $\pi: (h_t, P_t^g) \to (s_t, \ell_t)$ is optimal if it maximizes:
\begin{align*}
& \mathbb{E}^\pi \sbr{
   \sum_{t=0}^\infty \delta^t 
   \left. \pr{\sum_{j=1}^J h_{jt} w_j \ell_{jt} } \right\vert
   \pr{(h_{10}, P_{10}^g), \dots, (h_{10}, P_{J0}^g)}
} \\
\text{subject to} \quad& h_{jt+1} = h_{jt}+ a_{jt} s_{jt}, \quad \quad a_{jt} = 
   \begin{cases} 
      \nu_j, & \text{with prob. } \theta_j,  \\ 
      0, & \text{with prob. } 1 - \theta_j,
   \end{cases} 
   %\quad \quad h_{j0} = \alpha_{j0} \nu_j, 
   \\
\quad& P_{jt+1}^g = 
  \mathcal{B} (\alpha_{j,t+1}^g, \beta_{j,t+1}^g), 
  \quad \quad \theta_j \sim P_{j,0}^g \equiv \mathcal{B} (\alpha_{j0}^g, \beta_{j0}^g)
  \quad \quad \text{if $j$ selected,} \\
\quad& \sum_{j=1}^J (s_{jt} + \ell_{jt}) = 1, \quad \quad s_{jt}, \ell_{jt} \in \{0,1\} \\
   & h_{j0} \leq \nu \alpha_{j0}
\end{align*}

\end{frame}

%%%%%%%%%%%%%%%%%%%%%%%%%%%%%%%%%%%%%%%%%%%%%%%%%%%%%%%%%%%%%%%%%%%%%%%%%%%%%%%%
\begin{frame}{Optimal policy rule}

Define the skill $j$ index as the expected payoff if you committed to studying $j$:
%\gen{ 
\begin{equation*}
\mathcal{I}_j (h_j, P_j^g) = \sup_{\tau \geq 0} \mathbb{E}^\tau
\ce{
   \sum_{t=0}^\infty \delta^t \pr{h_{jt} w_j \ell_{jt} }}
   {(h_{j0}, P_{j0}^g) = (h_j, P_j^g)
}
% \mathcal{I}_{jt} (h_{jt}, \alpha_{jt}, \beta_{jt}) = 
% \begin{cases}
% \frac{h_{jt}}{1 - \delta} & \text{if } \{\alpha_{jt}, \beta_{jt}\} \in \mathcal{G}_{j}, \\
% \frac{h_{jt}}{1 - \delta} \sbr{
%    \frac{
%       \left\lceil \frac{\delta}{1 - \delta} \right\rceil
%       \delta^{\left\lceil \frac{\delta}{1 - \delta} \right\rceil - c_{jt} - \alpha_{j0} - \beta_{j0}}}
%    {c_{jt} + \alpha_{j0} + \beta_{j0}}
%    } & \text{if } \{\alpha_{jt}, \beta_{jt}\} \notin \mathcal{G}_{j} \\
% \end{cases}
\end{equation*}

Define the graduation region of skill $j$ as: 
\begin{equation*}
% \mathcal{G}_j =  \left\{ \alpha_{jt}, \beta_{jt} \left\vert c_{jt} \geq \left\lceil \frac{\delta}{1 - \delta} \right\rceil - (\alpha_{j0} + \beta_{j0}) \right. \right\}
\mathcal{G}_j = \left\{ (h_j, P_j^g) \left\vert
   \arg \max_{\tau \geq 0} 
   \mathbb{E}^\tau \ce{\sum_{t=0}^\infty \delta^t \pr{h_{jt} w_j \ell_{jt} }}
   {(h_j, P_j^g)} = 0
   \right. \right\}
\end{equation*}

The following policy $\pi: (h_t, P_t^g) \to (s_t, \ell_t)$ is optimal: 
\begin{enumerate}
    \item At each $t \geq 0$, choose skill $j^* = \arg \max_{j \in J} \mathcal{I}_j$, breaking ties according to any rule
    \item If $(h_{j^*}, P_{j^*}^g) \in \mathcal{G}_{j}$, then enter the labor market as a $j^*$ specialist. Otherwise, study $j^*$ for an additional period.  
\end{enumerate}

\end{frame}

%%%%%%%%%%%%%%%%%%%%%%%%%%%%%%%%%%%%%%%%%%%%%%%%%%%%%%%%%%%%%%%%%%%%%%%%%%%%%%%%

%%%%%%%%%%%%%%%%%%%%%%%%%%%%%%%%%%%%%%%%%%%%%%%%%%%%%%%%%%%%%%%%%%%%%%%%%%%%%%%



%%%%%%%%%%%%%%%%%%%%%%%%%%%%%%%%%%%%%%%%%%%%%%%%%%%%%%%%%%%%%%%%%%%%%%%%%%%%%%%%
\section{Analytical solution to model}\label{sec:analytic_results}
%%%%%%%%%%%%%%%%%%%%%%%%%%%%%%%%%%%%%%%%%%%%%%%%%%%%%%%%%%%%%%%%%%%%%%%%%%%%%%%%

%!TEX root = outline.tex
The optimal policy outlined in \ref{sec:optimal_policy} is characterized by two objects: the index \eqref{eq:index_general}, which summarizes the agent's expected lifetime payoff if they commit to one field and ignore all others; and the graduation region \eqref{eq:graduation_general}, which defines the states where an agent would choose to stop studying a particular field and enter the labor market.
\toedit{This section derives an analytical solution to these objects under the initial condition assumption \eqref{eq:h_leq_alpha_v}.}

Section \ref{sec:comment_state_vars} begins by introducing alternative notation to characterize state variables.
In section \ref{sec:initial_condition}, I discuss the intuition behind the initial condition assumption \eqref{eq:h_leq_alpha_v}, and its implications for the optimal stopping problem.
This motivates a tractable solution to the graduation region. 
Section \ref{sec:evaluating_index} uses the results from \ref{sec:initial_condition} to derive a simplified version of the index. 
Fully computing the index involves evaluating agent expectations over possible stopping times. 
The solution to these expectations are derived in section \ref{sec:solving_index}.
A concise summary of how to compute this solution is presented in section \ref{sec:computing}.

%%%%%%%%%%%%%%%%%%%%%%%%%%%%%%%%%%%%%%%%%%%%%%%%%%%%%%%%%%%%%%%%%%%%%%%%%%%%%%%%
\subsubsection*{Relation to parametric results from \textcite{AF20}}
%%%%%%%%%%%%%%%%%%%%%%%%%%%%%%%%%%%%%%%%%%%%%%%%%%%%%%%%%%%%%%%%%%%%%%%%%%%%%%%%

A key advantage of \citeposs{AF20} model is its computability under the stronger initial condition assumption $h_{j0} = \alpha_{j0} \nu_j$.
This assumption is not out of line with the human capital accumulation function \eqref{eq:hc_accumulation}, and may be reasonable for simulation exercises when all parameters of the problem are known.
However, this assumption presents both theoretical and empirical objections.
The goal of this paper is to assess how beliefs impact specialization decisions.
In the context of gender, this may involve considering whether a man and woman with similar initial human capital levels but different beliefs make different specialization choices; in the example outlined in section \ref{sec:group_based_beliefs}, this involves assessing whether men and women with similar $h_{j0}$ make different specialization choices when $\alpha_{j0}^m > \alpha_{j0}^w$.
However, assuming $h_{j0} = \alpha_{j0}^g \nu_j$ implies that women begin with lower levels of human capital than men in a particular field $j$, complicating this type of counterfactual analysis.
More practically, when bringing the model to the data,  I want to be able to control for initial human capital levels when agents make initial specialization choices. 
Assuming $h_{j0} = \alpha_{j0}^g \nu_j$ effectively eliminates the variable $h_{j0}$.
% Below I focus on the more general results when the initial monotonic 
For this reason, the sections below focus on the more general monotonic initial condition \eqref{eq:h_leq_alpha_v}.
I present the tractable results outlined in \textcite{AF20} in the context of a more general solution.
% A detailed treatment of the assumption $h_{j0} = \alpha_{j0}^g \nu_j$ versus the monotonic initial condition $h_{j0} \leq \alpha_{j0}^g \nu_j$ is warranted given the particular application of this paper.
% To see why, consider the case where $h_{j0} = \alpha_{j0}^g \nu_j$ and $\alpha_{j0}^m > \alpha_{j0}^f$.

%%%%%%%%%%%%%%%%%%%%%%%%%%%%%%%%%%%%%%%%%%%%%%%%%%%%%%%%%%%%%%%%%%%%%%%%%%%%%%%%
\subsection{Comment on state variables}\label{sec:comment_state_vars}
%%%%%%%%%%%%%%%%%%%%%%%%%%%%%%%%%%%%%%%%%%%%%%%%%%%%%%%%%%%%%%%%%%%%%%%%%%%%%%%%

State variables in this model are given by an agent's vector of human capital, $h_{jt}$, and their beliefs, $P_{jt}$.
This section presents an alternative characterization of the agent's state variables that simplifies the analytical solution.

Define $\tilde{\study}_{jt}$ as the total number of times a student has chosen to matriculate in field $j$ by time $t$, and define $\tilde{s}_{jt}$ as the total number of times a student has passed their field $j$ courses:
\begin{equation}\label{eq:def_totals}
     \tilde{\study}_{jt} = \sum_{n=0}^{t-1} \study_{jn}, \quad \quad
     \tilde{\pass}_{jt} = \sum_{n=0}^{t-1} \pass_{jn}.
\end{equation} 
The individual's state variables at time $t$ are $(h_{jt}, \alpha_{jt}, \beta_{jt})$. 
Using some simple algebraic transformations,\footnote{
    Specifically, note that (1) $\tilde{\study}_{jt} + \alpha_{j0} + \beta_{j0} = \alpha_{jt} + \beta_{jt}$; (2) $\alpha_{jt} = \tilde{\pass}_{jt} + \alpha_{j0}$; and (3) $h_{jt} = \nu_j \tilde{s}_{jt} + h_{j0}$.
} we can now characterize the states at time $t$ using $(\alpha_{j0}, \beta_{j0}, h_{j0}, \tilde{\study}_{jt}, \tilde{\pass}_{jt})$.
In words, the agent's state variables at time $t$ are the initial belief parameters $\alpha_{j0}$ and $\beta_{j0}$, initial human capital $h_{j0}$, the endogenous number of field-$j$ courses $\tilde{\study}_{jt}$, and the stochastic number of times an agent passed their field-$j$ courses, $\ks$.
Given the structure of the problem, there is no need to directly track the evolution of $(\alpha_{jt}, \beta_{jt}, h_{jt})$ over time, because (1) all information about the evolution of beliefs is captured by initial beliefs $(\alpha_{j0}, \beta_{j0})$, course choices ($\tilde{\study}_{jt}$), and course outcomes ($\tilde{\pass}_{jt}$); and (2) all information about human capital evolution is characterized by initial human capital endowments ($h_{j0}$), course choices ($\tilde{\study}_{jt}$), and course outcomes ($\tilde{\pass}_{jt}$). 


%%%%%%%%%%%%%%%%%%%%%%%%%%%%%%%%%%%%%%%%%%%%%%%%%%%%%%%%%%%%%%%%%%%%%%%%%%%%%%%%
\subsection{Initial condition assumption and optimal stopping time}\label{sec:initial_condition}
%%%%%%%%%%%%%%%%%%%%%%%%%%%%%%%%%%%%%%%%%%%%%%%%%%%%%%%%%%%%%%%%%%%%%%%%%%%%%%%%


The optimal policy from section \ref{sec:optimal_policy} is contingent on assuming equation \eqref{eq:h_leq_alpha_v}, which states that $h_{j0} \leq \nu_j \alpha_{j0}$. 
% As mentioned above, the optimality of the policy outlined in section \ref{sec:optimal_policy} relies on this condition.
Assuming that $h_{j0} \leq \nu_j \alpha_{j0}$ ensures that the stopping problem is monotonic, which in turn implies the optimality the policy outlined in section \ref{sec:optimal_policy}.
As such, I will occasionally refer to \eqref{eq:h_leq_alpha_v} as the \emph{monotonic initial condition} or the \emph{monotonicity assumption}. 

\toedit{It may be helpful to briefly outline why the monotonic initial condition implies optimality of the above policy.}
Recall that the graduation index, \eqref{eq:graduation_general}, characterizes the states where an individual would stop studying field-$j$ and enter the labor market as a field-$j$ specialist, ignoring all other fields.
Therefore, this object characterizes the field-specific stopping problem facing an individual.
Under the monotonic initial condition \eqref{eq:h_leq_alpha_v}, the stopping problem for a given field is monotone.
Intuitively, monotonicity of the stopping problem means that an agent who wants to stop studying $j$ at time $t$ would also want to stop studying $j$ at time $t+1$ if they continued on, independent of stochastic outcomes.
Therefore, an agent's stopping decision can be reduced to a comparison between their current expected lifetime payoff in $j$ and their expected payoff the following period.
In other words, monotonicity implies the optimality of a one-step-look-ahead comparison for the field-specific stopping problem.


% \toedit{
Therefore, the monotonicity condition \eqref{eq:h_leq_alpha_v} ensures that an agent evaluating field $j$ at time $t$ will stop studying $j$ if their expected lifetime payoff in the current period exceeds their expected lifetime payoff in $t+1$:
\begin{equation*}
    \frac{1}{1 - \delta} w_j h_{jt} 
    \geq 
    \frac{\delta}{1 - \delta} w_j \EE_t \left[\left. h_{j,t+1} \right\vert h_{jt}, \alpha_{jt}, \beta_{jt} \right]
    % = \frac{1}{1 - \delta} w_j \EE [h_{jt} + \nu_j \study_{jt}].
\end{equation*}
This equation can be simplified using the human capital accumulation function \eqref{eq:hc_accumulation}:
\begin{equation*}
    h_{jt} \geq \delta (h_{jt} + \nu_j \EE_t \left[\left. s_{jt} \right\vert h_{jt}, \alpha_{jt}, \beta_{jt}\right]).
\end{equation*}
Recalling that the course outcome $s_{jt}$ is a $\text{Bernoulli} (\theta_j)$ random variable, this can be written using an agent's beliefs about $\theta_j$ at time $t$:
\begin{equation}\label{eq:stop_general}
    \frac{1 - \delta}{\delta} \geq \frac{\nu_j \alpha_{jt}}{h_{jt} (\alpha_{jt} + \beta_{jt})}
\end{equation}
Using the definitions of $\tilde{\study}_{jt}$ and $\tilde{\pass}_{jt}$ from equation \eqref{eq:def_totals}, the stopping condition \eqref{eq:stop_general} becomes:\footnote{
    To replicate this derivation, note $\nu_j \alpha_{jt} = h_{jt} - h_{j0} + \nu_j \alpha_{j0}$. 
    Then \eqref{eq:stop_general} implies:
    \begin{equation*}
        h_{jt} \pr{\alpha_{jt} + \beta_{jt} - \frac{\delta}{1 - \delta}} 
        \geq \
        \frac{\delta}{1 - \delta} \pr{\nu_j \alpha_{j0} - h_{j0}}
    \end{equation*}
    Using the fact that $\tilde{\study}_{jt} + \alpha_{j0} + \beta_{j0} = \alpha_{jt} + \beta_{jt}$:
    \begin{equation*}
        h_{jt} \tilde{\study_{jt}}
        + h_{jt} \pr{\alpha_{j0} + \beta_{j0}} 
        \geq \
        \frac{\delta}{1 - \delta} \pr{\nu_j \alpha_{j0} - h_{j0} + h_{jt}}
    \end{equation*}
    The simplified stopping condition under monotonicity \eqref{eq:stop_montonicity} follows from the fact that $h_{jt} - h_{j0} = \nu_j \tilde{\pass}_{jt}$
}
\begin{equation}\label{eq:stop_montonicity}
    \tilde{\study}_{jt} \geq \frac{\delta}{1 - \delta}
    \pr{
        \frac{\nu_j \alpha_{j0} + \nu_j \tilde{\pass}_{jt}}{h_{j0} + \nu_j \tilde{\pass}_{jt}}
    } - \alpha_{j0} - \beta_{j0}
\end{equation}
% We can refer to equation \eqref{eq:stop_montonicity} as the \emph{simplified stopping condition under monotonicity}.
Intuitively, this stopping condition says that an agent will stop studying a field $j$ once their total number of completed field-$j$ courses exceeds the right-hand-side inequality.
\toedit{The graduation index \eqref{eq:graduation_general} can now be written to reflect the stopping condition \eqref{eq:stop_montonicity}:}
\begin{equation}\label{eq:graduation_monotonicity}
     \mathcal{G}_j = 
     \left\{ \states \left\vert
     \tilde{\study}_{jt} \geq \frac{\delta}{1 - \delta}
    \pr{
        \frac{\nu_j \alpha_{j0} + \nu_j \ks}{h_{j0} + \nu_j \ks}
    } - \alpha_{j0} - \beta_{j0}
     \right. \right\}
 \end{equation} 
\toedit{Before proceeding, it is helpful to discuss some properties of this object.}
An agent in this model begins with the initial states $(h_{j0}, \alpha_{j0}, \beta_{j0})$. They have taken zero courses in $j$, implying $\tilde{\study}_{j0} = 0$, and thus have passed zero courses in $j$, meaning $\ks = 0$.
Under the initial monotonicity condition \eqref{eq:h_leq_alpha_v}, the fraction $\frac{\nu_j \alpha_{j0} + \nu_j \ks}{h_{j0} + \nu_j \ks}$ is always greater than or equal to 1, and approaches 1 as $\ks$ increases.
This fact is important for bounding the number of periods an agent spends in school.


%%%%%%%%%%%%%%%%%%%%%%%%%%%%%%%%%%%%%%%%%%%%%%%%%%%%%%%%%%%%%%%%%%%%%%%%%%%%%%%
\subsection{Simplified index}\label{sec:evaluating_index}
%%%%%%%%%%%%%%%%%%%%%%%%%%%%%%%%%%%%%%%%%%%%%%%%%%%%%%%%%%%%%%%%%%%%%%%%%%%%%%%

The goal of this section is to derive a simplified version of the index \eqref{eq:index_general}.
% Recall that the agent's state when evaluating field $j$ at time $t$ is determined by their states, $\pr{\states}$.
Recall that the index $\mathcal{I}_j$ from equation \eqref{eq:index_general} characterizes the expected lifetime payoffs associated with specializing in $j$, ignoring other fields. 
If the the simplified stopping condition under monotonicity holds at time $t$ (i.e. $\pr{\states} \in \mathcal{G}_j$), then the agent would expect to enter the labor market (ignoring other fields).
Their expected lifetime payoff in $j$ equals their expected lifetime earnings given their current levels of human capital:
\begin{equation*}
    \frac{1}{1 - \delta} w_j h_{jt} 
    = 
    \frac{1}{1 - \delta} w_j 
    \pr{h_{j0} + \nu_j \tilde{\pass}_{jt}}
\end{equation*}
If $\pr{\states} \notin \mathcal{G}_j$, then the agent would plan on continuing their studies in $j$. 
Their expected lifetime payoff depends on how much human capital they expect to accumulate in $j$. 
To make this concrete, let $\study_{j}^*$ denote the total number of periods an agent expects to study field $j$ before entering the labor market.
Then the agent expects to be in school for $\study_j^* - \tilde{\study}_{jt}$ more periods.
Because the agent is not earning an income while they are in school, their expected lifetime payoff will be discounted by $\delta^{\study_j^* - \tilde{\study}_{jt}}$.
They then expect to enter the labor market at time $t + \study_j^* - \tilde{\study}_{jt}$ with some level of human capital, given by $h_{j,t + \study_j^* - \tilde{\study}_{jt}}$.
The index when $(\states) \notin \mathcal{G}_j$ is then given by:
\begin{equation*}
    \frac{1}{1 - \delta} w_j \EE_t \ce{
        \delta^{\study_j^* - \tilde{\study}_{jt}}
        h_{j,t + (\study_j^* - \tilde{\study}_{jt})}
    }{\pstates} 
\end{equation*}
Therefore, the index \eqref{eq:index_general} is characterized by:
\begin{multline}
    \label{eq:index_monotonicity_general}
    \mathcal{I}_j %(\states) 
    = 
    \begin{cases}
    \frac{w_j h_{jt}}{1 - \delta}
    &\text{if } (\states) \in \mathcal{G}_j,
    \\
    \frac{w_j}{1 - \delta} \EE_t \ce{
        \delta^{\study_j^* - \tilde{\study}_{jt}}
        h_{j,t + (\study_j^* - \tilde{\study}_{jt})}
    }{\states}
    &\text{otherwise.}
    \end{cases} 
\end{multline}
The method for evaluation the conditional expectation in equation \eqref{eq:index_monotonicity_general} is described in section \ref{sec:solving_index}.
A summary of these results are in section \ref{sec:computing}.




%%%%%%%%%%%%%%%%%%%%%%%%%%%%%%%%%%%%%%%%%%%%%%%%%%%%%%%%%%%%%%%%%%%%%%%%%%%%%%%%
\subsection{Computing agent behavior}\label{sec:computing}

This section summarizes how to compute an agent's behavior given states $(\states) = (\pstates)$ and assuming the initial monotonicity assumption. 
Recall that the graduation region under the initial monotonicity assumption is given by \eqref{eq:graduation_monotonicity}: 
\begin{equation*}
     \mathcal{G}_j = 
     \left\{ \states \left\vert
     \tilde{\study}_{jt} \geq \frac{\delta}{1 - \delta}
    \pr{
        \frac{\nu_j \alpha_{j0} + \nu_j \tilde{\pass}_{jt}}{h_{j0} + \nu_j \tilde{\pass}_{jt}}
    } - \alpha_{j0} - \beta_{j0}
     \right. \right\}
 \end{equation*} 
The index \eqref{eq:index_monotonicity_general} under the initial monotonicity assumption \eqref{eq:h_leq_alpha_v} is given by:
\begin{alignat*}{4}
    \label{eq:index_monotonicity}
    &\mathcal{I}_j (\pstates)
    =&&
    \begin{cases}
    \begin{array}{l}
    \frac{w_j}{1 - \delta} 
    h_{jt}
    \end{array}
    &(\pstates) \in \mathcal{G}_j,
    \\
    \begin{array}{l}
    \frac{w_j}{1 - \delta} 
    \EE_t \ce{
        \delta^{\study_j^* - \tilde{\study}_{jt}}
        h_{j,t + (\study_j^* - \tilde{\study}_{jt})}
    }{\pstates}
    \end{array}
    % \begin{array}{l}
    % \frac{w_j}{1 - \delta}
    % \left(
    %     \EE_t \ce{\delta^{\study_j^* - \tilde{\study}_{jt}}}{\pstates} 
    %     \pr{h_{j0} + \nu_j \tilde{\pass}_{jt}}
    % \right.
    % \\
    % \left.
    % \ \ 
    % + \nu_j \EE_t \ce{
    %     \delta^{\study_j^* - \tilde{\study}_{jt}}
    %     \pr{\study_j^* - \tilde{\study}_{jt}}
    % }{\pstates}   
    % \frac{\alpha_{j0} + \tilde{\pass}_{jt}}{\alpha_{j0} + \beta_{j0} + \tilde{\study}_{jt}}
    % \right)
    % \end{array}
    &\text{otherwise,}
    \end{cases} 
\end{alignat*}


\begin{outline}

\item To find the index for all fields $j$, first determine whether an agent is in their graduation region \eqref{eq:graduation_monotonicity}. 

\begin{outline}
    \item If the agent is in their graduation region for field $j$, set the $j$-index equal to $\frac{w_{jt} h_{jt}}{1 - \delta}$. \textbf{Skip to step 6.}

    \item If the agent is not in their graduation region, the index must be calculated using their expected accumulation of human capital, which is a function of expected time remaining in $j$:
\begin{multline*}
% Beginning of expected human capital accumulation
    \EE_t \ce{
        \delta^{\study_j^* - \tilde{\study}_{jt}}
        h_{j,t + (\study_j^* - \tilde{\study}_{jt})}
    }{\pstates}
    = 
    \EE_t 
    \ce{
        g\pr{\study_j^* - \tilde{\study}_{jt}}
    }{\pstates}
    \\
    =
    \EE_t 
    \ce{
        \delta^{\study_j^* - \tilde{\study}_{jt}}
        \pr{
            h_{j0} + \nu_j \ks
            + \pr{\study_j^* - \tilde{\study}_{jt}}
            \frac{\alpha_{j0} + \tilde{\pass}_{jt}}{\alpha_{j0} + \beta_{j0} + \tilde{\study}_{jt}}
        }
    }{\pstates}
\end{multline*}
    Proceed to the next step.
\end{outline}

\item Re-index the problem to simplify analysis:
\begin{alignat*}{3}
    \hat{\alpha}_{j0} =& \alpha_{j0} + \tilde{\pass}_{jt},
    \quad \quad
    &\hat{\alpha}_{j0} + \hat{\beta}_{j0} 
    =& \alpha_{j0} + \beta_{j0} + \tilde{\study}_{jt},
    \\
    \hat{h}_{j0} 
    =& h_{j0} + \nu_j \tilde{\pass}_{jt}, 
    \quad \quad
    &\hat{\psi}_{j0} 
    =& 
    \pr{\hat{\alpha}_{j0}, \hat{\beta}_{j0}, \hat{h}_{j0}},
\end{alignat*}
This way, we are considering how many courses an agent is expecting to study from time $t=0$, instead of how many remaining courses an agent expects to study at an arbitrary $t$.

\item Let $N = \study_{jt}^* - \tilde{\study}_{jt}$ denote the number of periods an agent expects to study, and let $\underline{n}$ and $\overline{n}$ denote the lower and upper bound of $N$, respectively. As described in section \ref{sec:solving_index}, these bounds are given by:

\begin{align*}
    \underline{n} 
    = 
    \underline{\study_{jt}^* - \tilde{\study}_{jt}} 
    =& \ceil{\frac{\delta}{1 - \delta}} - \hat{\alpha}_{j0} - \hat{\beta}_{j0}
    \\
    \overline{n}
    =
    \overline{\study_{jt}^* - \tilde{\study}_{jt}}
    =&
    \begin{cases}
        \underline{n}
        &\text{ if } 1 \leq \frac{\hat{\alpha}_{j0} \nu_j}{\hat{h}_{j0}} \leq \frac{\ddelta}{\frac{\delta}{1 - \delta}}
        \\
        \ceil{\frac{\delta}{1 - \delta} \frac{\hat{\alpha}_{j0} \nu_j}{\hat{h}_{j0}}} - \hat{\alpha}_{j0} - \hat{\beta}_{j0},
        &\text{ otherwise.}
    \end{cases}
\end{align*}

\item Find the probability distribution of stopping between $\underline{n}$ and $\overline{n}$ conditional on $\pr{\hat{\alpha}_{j0}, \hat{\beta}_{j0}, \hat{h}_{j0}}$.
First evaluate the probability that $N = \underline{n}$:
.\begin{align*}
    \PP_0 = \PP \pr{\crs{
        N = \underline{n}
    }{\hat{\alpha}_{j0}, \hat{\beta}_{j0}, \hat{h}_{j0}}}
    =&
    \begin{cases}
    1 &\text{ if } 1 \leq \frac{\alpha_{j0} \nu_j}{h_{j0}} \leq \frac{\ddelta}{\frac{\delta}{1 - \delta}}
    \\
    0 &\text{otherwise}.
    \end{cases}
\end{align*}
Next evaluate the probability of stopping at $N = \underline{n} + 1$.
\begin{align*}
    \PP_1 
    =&
    \PP \pr{\cls{N = \underline{n} + 1}{\hat{\psi}_{j0}}}
    \\
    =&
    \PP \pr{\cls{N = \underline{n} + 1}{N \neq \underline{n}, \psi_{j0}}} 
    \pr{1 - \PP_0},
    % \\
    % =&
\end{align*}
where $\PP \pr{\cls{N = \underline{n} + 1}{N \neq \underline{n}, \psi_{j0}}}$ is the probability of a binomial random variable.
% This value is an expectation of a 
For any integer $x > 1$ such that $\underline{n} + x < \overline{n}$, the probability of stopping at $\underline{n} + x$ is given by:
\begin{align*}
    \PP_x
    =&
    \PP \pr{\cls{N = \underline{n} + x}{\hat{\psi}_{j0}}}
    \\
    =&
    \PP \pr{\cls{N = \underline{n} + x}{N \neq \underline{n} + x - 1, \hat{\psi}_{j0}}}
    \pr{1 - \sum_{n = 0}^{x - 1} \PP_n},
\end{align*}
where $\PP \pr{\cls{N = \underline{n} + x}{N \neq \underline{n} + x - 1, \hat{\psi}_{j0}}}$
is a conditional sum of one-to-one functions of a Bernoulli and a binomial random variable. 
Finally, the probability of stopping at the upper bound $\overline{n}$ is given by:
\begin{align*}
    \PP_{\overline{n} - \underline{n}} = \PP \pr{\crs{
        N = \overline{n}
    }{\hat{\alpha}_{j0}, \hat{\beta}_{j0}, \hat{h}_{j0}}}
    =&
    1 - \sum_{n = 0}^{\overline{n} - \underline{n} - 1} \PP_n
\end{align*}
See section \ref{sec:solving_index} for details and proofs. 

\item Compute expected discounted accumulation of human capital
\begin{align*}
    \EE_t \ce{
        g\pr{N}
    }{\hat{\psi}_{j0}} 
    &=
    \sum_{n = \underline{n}}^{\overline{n}}
    g\pr{N}
    \mathbb{P}
    \pr{\crs{
        N = n
    }{\hat{\psi}_{j0}}
    }
\end{align*}
And set the index equal to $\frac{w_j}{1 - \delta} \EE_t \ce{g\pr{N}}{\hat{\psi}_{j0}} $.

\item Follow the optimal policy outlined in section \ref{sec:optimal_policy}. If the agent chooses to study this period, return to step one in the next period. 

\end{outline}



%%%%%%%%%%%%%%%%%%%%%%%%%%%%%%%%%%%%%%%%%%%%%%%%%%%%%%%%%%%%%%%%%%%%%%%%%%%%%%%%
\section{Implications of the model}\label{sec:sims}
\tikzset{external/figure name={sims_}}
%%%%%%%%%%%%%%%%%%%%%%%%%%%%%%%%%%%%%%%%%%%%%%%%%%%%%%%%%%%%%%%%%%%%%%%%%%%%%%%%

%!TEX root = outline.tex
An agent's specialization decision is impacted by individual, group, and field characteristics.
This section illustrates how these factors motivate an individual's behavior using a simplified version of the model.
Section \ref{sec:sims_prelims} develops a version of the model where an agent chooses between two completely symmetric fields. 
Simulations are explored in section \ref{sec:sims_plots} to illustrate how different factors influence decision making.
In particular, I emphasize the role that beliefs play in an agent's specialization decision. 

%%%%%%%%%%%%%%%%%%%%%%%%%%%%%%%%%%%%%%%%%%%%%%%%%%%%%%%%%%%%%%%%%%%%%%%%%%%%%%%%
\subsection{Choice between symmetric fields}\label{sec:sims_prelims}


Assume a student can choose to work or study in one of two fields, field $X$ or field $Y$. 
Utility in field $j \in \{X, Y\}$ at time $t$ is equal to income: 
\begin{equation}\label{eq:linear_utility}
    U_j(w_j, h_{jt}^g) \ell_{jt}^g = w_j h_{jt}^g \ell_{jt}^g
\end{equation}
Wages in fields X and Y are equal and are normalized to 1 ($w_X = w_Y = 1$), as are returns to successfully studying human capital ($\nu_X = \nu_Y = 1$).
The student's underlying abilities in the two fields, $\theta_X$ and $\theta_Y$, are both equal to 0.5. Therefore, the student has a 50\% chance of passing any given field X or field Y course.
Finally, I assume the student's beliefs about their own abilities in fields X and Y are equal to the uniform prior:\footnote{
    Note that if $(\alpha, \beta) = (1, 1)$, the beta distribution $\mathcal{B} (\alpha, \beta)$ equals the uniform distribution over $[0, 1]$. This distribution can be seen graphically in figure \ref{fig:beta_ex_a}.
        \nts{Intuitively, this implies that the student thinks all values of $\theta_j$ are equally likely. }
}
\begin{equation*}
    P_{X,0} = \mathcal{B}(\alpha_{X, 0}, \beta_{X, 0}) = \mathcal{B} (1, 1), 
    \quad \quad 
    P_{Y,0} = \mathcal{B}(\alpha_{Y, 0}, \beta_{Y, 0}) = \mathcal{B} (1, 1), 
\end{equation*}

For tractability, I modify the assumption \eqref{eq:h_leq_alpha_v} as follows:
\begin{equation}\label{eq:h_eq_alpha_v}
    h_{j0}^g = \nu_j \alpha_{j0}^g.
\end{equation}
Equation \eqref{eq:h_eq_alpha_v} is consistent with the human capital accumulation function in equation \eqref{eq:hc_accumulation}. 
Further, this assumption {}ensures that the number of periods an agent studies in school is a deterministic function of initial beliefs, as discussed in section \ref{sec:solving_index}.\footnote{
    This is specifically addressed while finding lower and upper bounds for stopping times in section \ref{sec:solving_index}.
    Assuming $h_{j0} = \nu_j \alpha_{j0}$ implies that all agents specializing in $j$ with initial beliefs $(\alpha_{j0}, \beta_{j0})$ will take exactly $\ddelta - \alpha_{j0} - \beta_{j0}$ courses in $j$ before entering the labor force.  
}
The role of beliefs can be more clearly seen in these simulations because all agents who specialize in field $j$ with the same initial beliefs will take the same number of courses in $j$.%\footnote{
%     It is worth emphasizing that this assumption ties together initial beliefs and initial human capital in a way that may not be desirable for counterfactual exercises. See the discussion at the start of section \ref{sec:analytic_results} for details. 
% }


The point at which an agent has ``specialized'' in a particular field $j$ is not clearly defined in the model. 
I intuitively describe what specialization looks like in the simulations below, but it is helpful to provide some concrete definition of specialization. 
In the figures below, an agent has ``specialized'' in a field $j$ if they would choose to continue to specialize in $j$ if they failed all of their remaining courses in that field.
Mathematically, this is represented by the following condition, letting $\study_j^*$ denote the number of courses an agent with beliefs $(\alpha_{j0}, \beta_{j0})$ would take in $j$ before specializing:
\begin{equation*}
    \frac{1}{1 - \delta} 
    \delta^{\study_j^* - \tilde{\study}_{jt}}
    w_j h_{jt}
    >
    \mathcal{I}_{k} (\tilde{\study}_{kt}, \tilde{\pass}_{kt}, \alpha_{k0}, \beta_{k0}, h_{k0}), \quad \forall k \neq j.
\end{equation*}
The left-hand side of this inequality is the agent's lifetime payoff of specializing in field $j$ if they fail all of their remaining courses in that field.
This would imply they do not accumulate any more human capital in field $j$, so their lifetime payoff is based on their current levels of human capital, $h_{jt}$, discounted according to the number of periods they expect to study.
The right-hand side of this inequality is the expected lifetime payoff associated will all other fields. 

%%%%%%%%%%%%%%%%%%%%%%%%%%%%%%%%%%%%%%%%%%%%%%%%%%%%%%%%%%%%%%%%%%%%%%%%%%%%%%%%
\subsection{Simulations}\label{sec:sims_plots}

\begin{figure}[t!]
\centering
\begin{tikzpicture}[every node/.style={font=\small}]
% This file was created by tikzplotlib v0.9.2.
\definecolor{color0}{rgb}{0.266666666666667,0.466666666666667,0.666666666666667}
\definecolor{color1}{rgb}{0.933333333333333,0.4,0.466666666666667}

\begin{groupplot}[group style={group size=2 by 3, vertical sep=2cm, group name=my plots, horizontal sep=1.2cm}]
\nextgroupplot[
height=5.101085673964669cm,
tick pos=left,
width=8.25373cm,
x grid style={white!69.0196078431373!black},
xlabel style={at={(ticklabel* cs:1.00)}, anchor=north east, font=\normalsize},
xlabel={\(\displaystyle t\)},
xmin=-1.05, xmax=30,
xtick style={color=black},
xtick={0,5,10,15,20},
xticklabels={\(\displaystyle 0\),\(\displaystyle 5\),\(\displaystyle 10\),\(\displaystyle 15\),\(\displaystyle 20\)},
ylabel={Fraction enrolled in field},
ymajorgrids,
ymin=0, ymax=1,
ytick style={color=black},
ytick={0,0.25,0.5,0.75,1},
yticklabels={\(\displaystyle 0\),\(\displaystyle 0.25\),\(\displaystyle 0.5\),\(\displaystyle 0.75\),\(\displaystyle 1\)}
]
\addplot [thick, color0, mark=x, mark size=3, mark options={solid}]
table {%
0 0.4937
1 0.5042
2 0.5024
3 0.5021
4 0.5
5 0.5003
6 0.4991
7 0.4992
8 0.4979
9 0.4984
10 0.4979
11 0.4977
12 0.4976
13 0.498
14 0.498
15 0.4977
16 0.4979
17 0.4977
18 0.4976
19 0.4976
20 0.4976
21 0.4976
};
\addplot [thick, color1, mark=x, mark size=3, mark options={solid}]
table {%
0 0.5063
1 0.4958
2 0.4976
3 0.4979
4 0.5
5 0.4997
6 0.5009
7 0.5008
8 0.5021
9 0.5016
10 0.5021
11 0.5023
12 0.5024
13 0.502
14 0.502
15 0.5023
16 0.5021
17 0.5023
18 0.5024
19 0.5024
20 0.5024
21 0.5024
};
\addplot [semithick, color0, opacity=0.5, dashed]
table {%
10 -4.44089209850063e-16
10 0.999999999999999
};
\addplot [semithick, color1, opacity=0.5, dashed]
table {%
10 -4.44089209850063e-16
10 0.999999999999999
};
\draw (axis cs:21.5,0.4176) node[
  anchor=base west,
  text=color0,
  rotate=0.0
]{Field X};
\draw (axis cs:21.5,0.5324) node[
  anchor=base west,
  text=color1,
  rotate=0.0
]{Field Y};

\nextgroupplot[
height=5.101085673964669cm,
tick pos=left,
width=8.25373cm,
x grid style={white!69.0196078431373!black},
xlabel style={at={(ticklabel* cs:1.00)}, anchor=north east, font=\normalsize},
xlabel={\(\displaystyle t\)},
xmin=-1.05, xmax=30,
xtick style={color=black},
xtick={0,5,10,15,20},
xticklabels={\(\displaystyle 0\),\(\displaystyle 5\),\(\displaystyle 10\),\(\displaystyle 15\),\(\displaystyle 20\)},
ymajorgrids,
ymin=0, ymax=1,
ytick style={color=black},
ytick={0,0.25,0.5,0.75,1},
yticklabels={\(\displaystyle 0\),\(\displaystyle 0.25\),\(\displaystyle 0.5\),\(\displaystyle 0.75\),\(\displaystyle 1\)}
]
\addplot [thick, color0, mark=x, mark size=3, mark options={solid}]
table {%
0 0.46
1 0.4
2 0.42
3 0.5
4 0.42
5 0.48
6 0.46
7 0.44
8 0.46
9 0.48
10 0.46
11 0.48
12 0.48
13 0.48
14 0.48
15 0.48
16 0.48
17 0.48
18 0.48
19 0.48
20 0.48
21 0.48
};
\addplot [thick, color1, mark=x, mark size=3, mark options={solid}]
table {%
0 0.54
1 0.6
2 0.58
3 0.5
4 0.58
5 0.52
6 0.54
7 0.56
8 0.54
9 0.52
10 0.54
11 0.52
12 0.52
13 0.52
14 0.52
15 0.52
16 0.52
17 0.52
18 0.52
19 0.52
20 0.52
21 0.52
};
\addplot [semithick, color0, opacity=0.5, dashed]
table {%
10 -4.44089209850063e-16
10 0.999999999999999
};
\addplot [semithick, color1, opacity=0.5, dashed]
table {%
10 -4.44089209850063e-16
10 0.999999999999999
};
\draw (axis cs:21.5,0.4) node[
  anchor=base west,
  text=color0,
  rotate=0.0
]{Field X};
\draw (axis cs:21.5,0.55) node[
  anchor=base west,
  text=color1,
  rotate=0.0
]{Field Y};

\nextgroupplot[
height=5.101085673964669cm,
tick pos=left,
unbounded coords=jump,
width=8.25373cm,
x grid style={white!69.0196078431373!black},
xlabel style={at={(ticklabel* cs:1.00)}, anchor=north east, font=\normalsize},
xlabel={\(\displaystyle t\)},
xmin=-1.15, xmax=30,
xtick style={color=black},
xtick={0,5,10,15,20},
xticklabels={\(\displaystyle 0\),\(\displaystyle 5\),\(\displaystyle 10\),\(\displaystyle 15\),\(\displaystyle 20\)},
ylabel={Fraction enrolled in field},
ymajorgrids,
ymin=-0.05, ymax=1.05,
ytick style={color=black},
ytick={0,0.2,0.4,0.6,0.8,1},
yticklabels={\(\displaystyle 0\),\(\displaystyle 0.2\),\(\displaystyle 0.4\),\(\displaystyle 0.6\),\(\displaystyle 0.8\),\(\displaystyle 1\)}
]
\addplot [thick, color0, mark=x, mark size=3, mark options={solid}]
table {%
0 0
1 0
2 0.2477
3 0.1273
4 0.1875
5 0.1583
6 0.1714
7 0.1705
8 0.1665
9 0.1589
10 0.1632
11 0.1599
12 0.16
13 0.1602
14 0.1613
15 0.1606
16 0.1607
17 0.1606
18 0.1606
19 0.1605
20 0.1605
21 0.1605
22 0.1605
23 0.1605
};
\addplot [thick, color1, mark=x, mark size=3, mark options={solid}]
table {%
0 1
1 1
2 0.7523
3 0.8727
4 0.8125
5 0.8417
6 0.8286
7 0.8295
8 0.8335
9 0.8411
10 0.8368
11 0.8401
12 0.84
13 0.8398
14 0.8387
15 0.8394
16 0.8393
17 0.8394
18 0.8394
19 0.8395
20 0.8395
21 0.8395
22 nan
23 nan
};
\addplot [semithick, color0, opacity=0.5, dashed]
table {%
12 -0.0499999999999998
12 1.05
};
\addplot [semithick, color1, opacity=0.5, dashed]
table {%
8 -0.0499999999999998
8 1.05
};
\draw (axis cs:23.5,0.1905) node[
  anchor=base west,
  text=color0,
  rotate=0.0
]{Field X};
\draw (axis cs:21.5,0.8695) node[
  anchor=base west,
  text=color1,
  rotate=0.0
]{Field Y};

\nextgroupplot[
height=5.101085673964669cm,
tick pos=left,
width=8.25373cm,
x grid style={white!69.0196078431373!black},
xlabel style={at={(ticklabel* cs:1.00)}, anchor=north east, font=\normalsize},
xlabel={\(\displaystyle t\)},
xmin=-1.05, xmax=30,
xtick style={color=black},
xtick={0,5,10,15,20},
xticklabels={\(\displaystyle 0\),\(\displaystyle 5\),\(\displaystyle 10\),\(\displaystyle 15\),\(\displaystyle 20\)},
ymajorgrids,
ymin=0, ymax=1,
ytick style={color=black},
ytick={0,0.2,0.4,0.6,0.8,1},
yticklabels={\(\displaystyle 0\),\(\displaystyle 0.2\),\(\displaystyle 0.4\),\(\displaystyle 0.6\),\(\displaystyle 0.8\),\(\displaystyle 1\)}
]
\addplot [thick, color0, mark=x, mark size=3, mark options={solid}]
table {%
0 0.4957
1 0.4002
2 0.4038
3 0.3522
4 0.3549
5 0.3476
6 0.3399
7 0.3254
8 0.3293
9 0.329
10 0.3268
11 0.3219
12 0.3236
13 0.3236
14 0.3209
15 0.3219
16 0.3219
17 0.3219
18 0.322
19 0.3222
20 0.3222
21 0.3222
};
\addplot [thick, color1, mark=x, mark size=3, mark options={solid}]
table {%
0 0.5043
1 0.5998
2 0.5962
3 0.6478
4 0.6451
5 0.6524
6 0.6601
7 0.6746
8 0.6707
9 0.671
10 0.6732
11 0.6781
12 0.6764
13 0.6764
14 0.6791
15 0.6781
16 0.6781
17 0.6781
18 0.678
19 0.6778
20 0.6778
21 0.6778
};
\addplot [semithick, color0, opacity=0.5, dashed]
table {%
12 0
12 1
};
\addplot [semithick, color1, opacity=0.5, dashed]
table {%
9 0
9 1
};
\draw (axis cs:21.5,0.3522) node[
  anchor=base west,
  text=color0,
  rotate=0.0
]{Field X};
\draw (axis cs:21.5,0.7078) node[
  anchor=base west,
  text=color1,
  rotate=0.0
]{Field Y};

\nextgroupplot[
height=5.101085673964669cm,
tick pos=left,
unbounded coords=jump,
width=8.25373cm,
x grid style={white!69.0196078431373!black},
xlabel style={at={(ticklabel* cs:1.00)}, anchor=north east, font=\normalsize},
xlabel={\(\displaystyle t\)},
xmin=-1.1, xmax=30,
xtick style={color=black},
xtick={0,5,10,15,20},
xticklabels={\(\displaystyle 0\),\(\displaystyle 5\),\(\displaystyle 10\),\(\displaystyle 15\),\(\displaystyle 20\)},
ylabel={Fraction enrolled in field},
ymajorgrids,
ymin=-0.05, ymax=1.05,
ytick style={color=black},
ytick={0,0.2,0.4,0.6,0.8,1},
yticklabels={\(\displaystyle 0\),\(\displaystyle 0.2\),\(\displaystyle 0.4\),\(\displaystyle 0.6\),\(\displaystyle 0.8\),\(\displaystyle 1\)}
]
\addplot [thick, color0, mark=x, mark size=3, mark options={solid}]
table {%
0 0
1 0.4978
2 0.2521
3 0.2521
4 0.2767
5 0.2819
6 0.2649
7 0.2576
8 0.255
9 0.2662
10 0.2602
11 0.2602
12 0.2603
13 0.2593
14 0.2595
15 0.2592
16 0.2595
17 0.2595
18 0.2595
19 0.2595
20 0.2595
21 0.2595
22 0.2596
};
\addplot [thick, color1, mark=x, mark size=3, mark options={solid}]
table {%
0 1
1 0.5022
2 0.7479
3 0.7479
4 0.7233
5 0.7181
6 0.7351
7 0.7424
8 0.745
9 0.7338
10 0.7398
11 0.7398
12 0.7397
13 0.7407
14 0.7405
15 0.7408
16 0.7405
17 0.7405
18 0.7405
19 0.7405
20 nan
21 nan
22 nan
};
\addplot [semithick, color0, opacity=0.5, dashed]
table {%
11 -0.05
11 1.05
};
\addplot [semithick, color1, opacity=0.5, dashed]
table {%
9 -0.05
9 1.05
};
\draw (axis cs:22.5,0.2596) node[
  anchor=base west,
  text=color0,
  rotate=0.0
]{Field X};
\draw (axis cs:19.5,0.7405) node[
  anchor=base west,
  text=color1,
  rotate=0.0
]{Field Y};

\nextgroupplot[
height=5.101085673964669cm,
hide x axis,
hide y axis,
tick align=outside,
tick pos=left,
width=8.25373cm,
x grid style={white!69.0196078431373!black},
xlabel style={at={(ticklabel* cs:1.00)}, anchor=north east, font=\normalsize},
xmin=0, xmax=1,
xtick style={color=black},
y grid style={white!69.0196078431373!black},
ymin=0, ymax=1,
ytick style={color=black}
]
\end{groupplot}



\node [text width=7.428356999999999cm, align=center, anchor=south] at (my plots c1r1.north) {\subcaption{\label{fig:sim_a} Baseline simulation}};
\node [text width=7.428356999999999cm, align=center, anchor=south] at (my plots c2r1.north) {\subcaption{\label{fig:sim_b} Baseline simulation (zoomed in)}};
\node [text width=7.428356999999999cm, align=center, anchor=south] at (my plots c1r2.north) {\subcaption{\label{fig:sim_c} Wages }};
\node [text width=7.428356999999999cm, align=center, anchor=south] at (my plots c2r2.north) {\subcaption{\label{fig:sim_d} Ability to succeed}};
\node [text width=7.428356999999999cm, align=center, anchor=south] at (my plots c1r3.north) {\subcaption{\label{fig:sim_e} Initial beliefs}};

\end{tikzpicture}

\caption{Simulations of simple version of model. Figure (a) presents the baseline for $N = 10,000$ simulations; figure (b) does the same for the first 50 simulations. The remaining figures have $N = 10,000$ simulations. Figure (c) repeats the simulations for $w_{X} = 1$ and $w_{Y} = 1.5$. Figure (d) repeats the simulations when $\theta_{X} = 0.4$ and $\theta_{Y} = 0.6$. Figure (e) repeats the simulations when $(\alpha_{X0}, \beta_{X0}) = (1, 1)$ and $(\alpha_{Y0}, \beta_{Y0}) = (2, 2)$.}

% \caption{test}
\label{fig:sim_plots}
\end{figure}
Each subplot in figure \ref{fig:sim_plots} plots the fraction of simulated agents choosing to study field X or field Y at each time period $t$.
Recall that agents studying field $j$ at time $t$ will either pass and successfully accumulate human capital ($\pass_{jt}^g$ = 1) or they will fail ($\pass_{jt}^g =0$), where $\pass_{jt}^g \sim \text{Bernoulli} (\theta_j)$.
The student then updates their beliefs about their own underlying ability, $\theta_j$. 
% Simulated agents may switch which field they study as they update their beliefs. 
Line movements in figure \ref{fig:sim_plots} are caused by agents switching fields in response to updated beliefs. 
Eventually, students will specialize in one field and enter the labor market as a field-X or field-Y specialist.
The line for any field $j$ ends once any agent specializing in $j$ stops studying and enters the labor market.
Therefore, the length of the lines in figure \ref{fig:sim_plots} denote the minimum amount of time an agents spends studying before becoming a field-$j$ specialist.
Specialization in figure \ref{fig:sim_plots} is generally represented by a flattening of the curve; once a student has made their specialization decision, they no longer switch fields. 
To make this explicit, I use the definition of specialization from section \ref{sec:sims_prelims}, and define specialization as the point when an agent could fail all of their remaining field $j$ courses, and would still choose to specialize in that field.
The median point by which simulated agents have made their specialization decision is represented by the dashed vertical line in each plot. 

The baseline scenarios in figures \ref{fig:sim_a} and \ref{fig:sim_b} illustrate these dynamics.
Figure \ref{fig:sim_a} plots the baseline scenario for 10,000 simulations; figure \ref{fig:sim_b} plots the first 50 of these simulations. 
\toedit{Our first takeaway is that the agent's specialization decision in the baseline is effectively a coin flip.}\footnts{
    I would really like a better way to describe this.
    The randomness in ability is driving choice? Not totally sure. 
}
This is most clearly seen in figure \ref{fig:sim_a}; at all time periods, approximately 50\% of the agents are studying field X and and 50\% are studying field Y. 
This should be expected, as fields X and Y are completely symmetric.

% It would be nice to mention persistence here
Some of the more subtle decision dynamics can only be seen with fewer observations. Therefore, \ref{fig:sim_b} zooms in on the first 50 of these simulations.
Note that the fraction of students studying field X or field Y moves in early periods, but flattens out in later periods. 
This is because students at the beginning of their education will update their beliefs in response to course outcomes.
These updated beliefs may cause students to switch fields, shifting the composition of simulated agents studying X or Y.
% Agents switching fields causes the lines in \ref{fig:sim_b} to move in early periods.
In later periods, simulated agents have made their specialization decision and no longer switch fields. \
This specialization is represented by the flattening of the lines in figure \ref{fig:sim_b}.
As in \ref{fig:sim_a}, approximately 50\% of agents specialize in field X, and 50\% specialize in field Y. 
% Finally, although there is some movement in terms of which fields student study, overall there is a persistence

\begin{figure}[t!]
\centering
\begin{tikzpicture}[every node/.style={font=\small}]
% This file was created by tikzplotlib v0.9.2.
\definecolor{color0}{rgb}{0.266666666666667,0.466666666666667,0.666666666666667}

\begin{groupplot}[group style={group size=3 by 3, group name=my plots, horizontal sep=0.75cm}]
\nextgroupplot[
height=3.570759971775268cm,
scaled x ticks=manual:{}{\pgfmathparse{#1}},
tick pos=left,
width=5.777610999999999cm,
x grid style={white!69.0196078431373!black},
xmin=-0.05, xmax=1.05,
xtick style={color=black},
xticklabels={},
y grid style={white!69.0196078431373!black},
ymin=-0.15, ymax=3.15,
ytick style={color=black}
]
\addplot [thick, color0]
table {%
0 1
1 1
};

\nextgroupplot[
height=3.570759971775268cm,
scaled x ticks=manual:{}{\pgfmathparse{#1}},
scaled y ticks=manual:{}{\pgfmathparse{#1}},
tick pos=left,
width=5.777610999999999cm,
x grid style={white!69.0196078431373!black},
xmin=-0.05, xmax=1.05,
xtick style={color=black},
xticklabels={},
y grid style={white!69.0196078431373!black},
ymin=-0.15, ymax=3.15,
ytick style={color=black},
yticklabels={}
]
\addplot [thick, color0]
table {%
0 0
1 2
};

\nextgroupplot[
height=3.570759971775268cm,
scaled x ticks=manual:{}{\pgfmathparse{#1}},
scaled y ticks=manual:{}{\pgfmathparse{#1}},
tick pos=left,
width=5.777610999999999cm,
x grid style={white!69.0196078431373!black},
xmin=-0.05, xmax=1.05,
xtick style={color=black},
xticklabels={},
y grid style={white!69.0196078431373!black},
ymin=-0.15, ymax=3.15,
ytick style={color=black},
yticklabels={}
]
\addplot [thick, color0]
table {%
0 0
0.0204081535339355 0.00124943256378174
0.0408163070678711 0.00499796867370605
0.0612244606018066 0.0112453699111938
0.0816326141357422 0.0199916362762451
0.102040767669678 0.031237006187439
0.122448921203613 0.0449812412261963
0.142857193946838 0.0612244606018066
0.163265347480774 0.07996666431427
0.183673501014709 0.101207852363586
0.204081654548645 0.124947905540466
0.224489808082581 0.151186943054199
0.244897961616516 0.179924964904785
0.265306115150452 0.211161971092224
0.285714268684387 0.244897961616516
0.306122422218323 0.281132817268372
0.326530575752258 0.31986665725708
0.346938848495483 0.361099481582642
0.367347002029419 0.404831290245056
0.387755155563354 0.451062083244324
0.40816330909729 0.499791741371155
0.428571462631226 0.551020383834839
0.448979616165161 0.604748010635376
0.469387769699097 0.660974621772766
0.489795923233032 0.71970009803772
0.510204076766968 0.780924558639526
0.530612230300903 0.844648122787476
0.551020383834839 0.910870552062988
0.571428537368774 0.979591846466064
0.59183669090271 1.05081212520599
0.612244844436646 1.12453138828278
0.632652997970581 1.20074963569641
0.653061151504517 1.2794668674469
0.673469305038452 1.36068308353424
0.693877577781677 1.44439816474915
0.714285731315613 1.5306122303009
0.734693884849548 1.61932528018951
0.755102038383484 1.71053731441498
0.775510191917419 1.80424821376801
0.795918345451355 1.90045809745789
0.81632661819458 1.99916696548462
0.836734771728516 2.1003749370575
0.857142925262451 2.20408153533936
0.877551078796387 2.31028747558594
0.897959232330322 2.4189920425415
0.918367385864258 2.53019571304321
0.938775539398193 2.64389848709106
0.959183692932129 2.7600998878479
0.979591846466064 2.87880039215088
1 3
};

\nextgroupplot[
height=3.570759971775268cm,
scaled x ticks=manual:{}{\pgfmathparse{#1}},
tick pos=left,
width=5.777610999999999cm,
x grid style={white!69.0196078431373!black},
xmin=-0.05, xmax=1.05,
xtick style={color=black},
xticklabels={},
y grid style={white!69.0196078431373!black},
ymin=-0.15, ymax=3.15,
ytick style={color=black}
]
\addplot [thick, color0]
table {%
0 2
1 0
};

\nextgroupplot[
height=3.570759971775268cm,
scaled x ticks=manual:{}{\pgfmathparse{#1}},
scaled y ticks=manual:{}{\pgfmathparse{#1}},
tick pos=left,
width=5.777610999999999cm,
x grid style={white!69.0196078431373!black},
xmin=-0.05, xmax=1.05,
xtick style={color=black},
xticklabels={},
y grid style={white!69.0196078431373!black},
ymin=-0.15, ymax=3.15,
ytick style={color=black},
yticklabels={}
]
\addplot [thick, color0]
table {%
0 0
0.0204081535339355 0.11995005607605
0.0408163070678711 0.234902143478394
0.0612244606018066 0.344856262207031
0.0816326141357422 0.449812650680542
0.102040767669678 0.549770951271057
0.122448921203613 0.644731283187866
0.142857193946838 0.734693884849548
0.163265347480774 0.819658517837524
0.183673501014709 0.899625182151794
0.204081654548645 0.974593877792358
0.224489808082581 1.04456472396851
0.244897961616516 1.10953772068024
0.265306115150452 1.16951274871826
0.285714268684387 1.22448980808258
0.306122422218323 1.27446901798248
0.326530575752258 1.31945025920868
0.346938848495483 1.35943353176117
0.367347002029419 1.39441895484924
0.387755155563354 1.4244065284729
0.40816330909729 1.44939613342285
0.428571462631226 1.4693877696991
0.448979616165161 1.48438155651093
0.469387769699097 1.49437737464905
0.489795923233032 1.49937522411346
0.510204076766968 1.49937522411346
0.530612230300903 1.49437737464905
0.551020383834839 1.48438155651093
0.571428537368774 1.4693877696991
0.59183669090271 1.44939613342285
0.612244844436646 1.4244065284729
0.632652997970581 1.39441895484924
0.653061151504517 1.35943353176117
0.673469305038452 1.31945025920868
0.693877577781677 1.27446901798248
0.714285731315613 1.22448980808258
0.734693884849548 1.16951274871826
0.755102038383484 1.10953772068024
0.775510191917419 1.04456472396851
0.795918345451355 0.974593877792358
0.81632661819458 0.899625182151794
0.836734771728516 0.819658517837524
0.857142925262451 0.734693884849548
0.877551078796387 0.644731283187866
0.897959232330322 0.549770951271057
0.918367385864258 0.449812650680542
0.938775539398193 0.344856262207031
0.959183692932129 0.234902143478394
0.979591846466064 0.11995005607605
1 0
};

\nextgroupplot[
height=3.570759971775268cm,
scaled x ticks=manual:{}{\pgfmathparse{#1}},
scaled y ticks=manual:{}{\pgfmathparse{#1}},
tick pos=left,
width=5.777610999999999cm,
x grid style={white!69.0196078431373!black},
xmin=-0.05, xmax=1.05,
xtick style={color=black},
xticklabels={},
y grid style={white!69.0196078431373!black},
ymin=-0.15, ymax=3.15,
ytick style={color=black},
yticklabels={}
]
\addplot [thick, color0]
table {%
0 0
0.0204081535339355 0.00489592552185059
0.0408163070678711 0.01917564868927
0.0612244606018066 0.0422272682189941
0.0816326141357422 0.0734387636184692
0.102040767669678 0.112198114395142
0.122448921203613 0.157893419265747
0.142857193946838 0.209912538528442
0.163265347480774 0.267643570899963
0.183673501014709 0.330474615097046
0.204081654548645 0.397793412208557
0.224489808082581 0.468988299369812
0.244897961616516 0.543447017669678
0.265306115150452 0.62055778503418
0.285714268684387 0.699708461761475
0.326530575752258 0.861681699752808
0.367347002029419 1.02447104454041
0.387755155563354 1.10464179515839
0.40816330909729 1.18318045139313
0.428571462631226 1.25947523117065
0.448979616165161 1.33291399478912
0.469387769699097 1.40288484096527
0.489795923233032 1.46877574920654
0.510204076766968 1.52997469902039
0.530612230300903 1.58586978912354
0.551020383834839 1.63584899902344
0.571428537368774 1.67930030822754
0.59183669090271 1.71561169624329
0.612244844436646 1.74417126178741
0.632652997970581 1.76436686515808
0.653061151504517 1.77558672428131
0.673469305038452 1.77721869945526
0.693877577781677 1.76865077018738
0.714285731315613 1.74927115440369
0.734693884849548 1.71846759319305
0.755102038383484 1.67562830448151
0.775510191917419 1.62014126777649
0.795918345451355 1.55139434337616
0.81632661819458 1.46877574920654
0.836734771728516 1.3716733455658
0.857142925262451 1.25947523117065
0.877551078796387 1.13156938552856
0.897959232330322 0.987343788146973
0.918367385864258 0.826186418533325
0.938775539398193 0.647485256195068
0.959183692932129 0.450628519058228
0.979591846466064 0.235004186630249
1 0
};

\nextgroupplot[
height=3.570759971775268cm,
tick pos=left,
width=5.777610999999999cm,
x grid style={white!69.0196078431373!black},
xmin=-0.05, xmax=1.05,
xtick style={color=black},
y grid style={white!69.0196078431373!black},
ymin=-0.15, ymax=3.15,
ytick style={color=black}
]
\addplot [thick, color0]
table {%
0 3
0.0204081535339355 2.87880039215088
0.0408163070678711 2.7600998878479
0.0612244606018066 2.64389848709106
0.0816326141357422 2.53019571304321
0.102040767669678 2.4189920425415
0.122448921203613 2.31028747558594
0.142857193946838 2.20408153533936
0.163265347480774 2.1003749370575
0.183673501014709 1.99916696548462
0.204081654548645 1.90045809745789
0.224489808082581 1.80424821376801
0.244897961616516 1.71053731441498
0.265306115150452 1.61932528018951
0.285714268684387 1.5306122303009
0.306122422218323 1.44439816474915
0.326530575752258 1.36068308353424
0.346938848495483 1.2794668674469
0.367347002029419 1.20074963569641
0.387755155563354 1.12453138828278
0.40816330909729 1.05081212520599
0.428571462631226 0.979591846466064
0.448979616165161 0.910870552062988
0.469387769699097 0.844648122787476
0.489795923233032 0.780924558639526
0.510204076766968 0.71970009803772
0.530612230300903 0.660974621772766
0.551020383834839 0.604748010635376
0.571428537368774 0.551020383834839
0.59183669090271 0.499791741371155
0.612244844436646 0.451062083244324
0.632652997970581 0.404831290245056
0.653061151504517 0.361099481582642
0.673469305038452 0.31986665725708
0.693877577781677 0.281132817268372
0.714285731315613 0.244897961616516
0.734693884849548 0.211161971092224
0.755102038383484 0.179924964904785
0.775510191917419 0.151186943054199
0.795918345451355 0.124947905540466
0.81632661819458 0.101207852363586
0.836734771728516 0.07996666431427
0.857142925262451 0.0612244606018066
0.877551078796387 0.0449812412261963
0.897959232330322 0.031237006187439
0.918367385864258 0.0199916362762451
0.938775539398193 0.0112453699111938
0.959183692932129 0.00499796867370605
0.979591846466064 0.00124943256378174
1 0
};

\nextgroupplot[
height=3.570759971775268cm,
scaled y ticks=manual:{}{\pgfmathparse{#1}},
tick pos=left,
width=5.777610999999999cm,
x grid style={white!69.0196078431373!black},
xmin=-0.05, xmax=1.05,
xtick style={color=black},
y grid style={white!69.0196078431373!black},
ymin=-0.15, ymax=3.15,
ytick style={color=black},
yticklabels={}
]
\addplot [thick, color0]
table {%
0 0
0.0204081535339355 0.235004186630249
0.0408163070678711 0.450628519058228
0.0612244606018066 0.647485256195068
0.0816326141357422 0.826186418533325
0.102040767669678 0.987343788146973
0.122448921203613 1.13156938552856
0.142857193946838 1.25947523117065
0.163265347480774 1.3716733455658
0.183673501014709 1.46877574920654
0.204081654548645 1.55139434337616
0.224489808082581 1.62014126777649
0.244897961616516 1.67562830448151
0.265306115150452 1.71846759319305
0.285714268684387 1.74927115440369
0.306122422218323 1.76865077018738
0.326530575752258 1.77721869945526
0.346938848495483 1.77558672428131
0.367347002029419 1.76436686515808
0.387755155563354 1.74417126178741
0.40816330909729 1.71561169624329
0.428571462631226 1.67930030822754
0.448979616165161 1.63584899902344
0.469387769699097 1.58586978912354
0.489795923233032 1.52997469902039
0.510204076766968 1.46877574920654
0.530612230300903 1.40288484096527
0.551020383834839 1.33291399478912
0.571428537368774 1.25947523117065
0.59183669090271 1.18318045139313
0.612244844436646 1.10464179515839
0.653061151504517 0.943280458450317
0.693877577781677 0.780287146568298
0.714285731315613 0.699708461761475
0.734693884849548 0.62055778503418
0.755102038383484 0.543447017669678
0.775510191917419 0.468988299369812
0.795918345451355 0.397793412208557
0.81632661819458 0.330474615097046
0.836734771728516 0.267643570899963
0.857142925262451 0.209912538528442
0.877551078796387 0.157893419265747
0.897959232330322 0.112198114395142
0.918367385864258 0.0734387636184692
0.938775539398193 0.0422272682189941
0.959183692932129 0.01917564868927
0.979591846466064 0.00489592552185059
1 0
};

\nextgroupplot[
height=3.570759971775268cm,
scaled y ticks=manual:{}{\pgfmathparse{#1}},
tick pos=left,
width=5.777610999999999cm,
x grid style={white!69.0196078431373!black},
xmin=-0.05, xmax=1.05,
xtick style={color=black},
y grid style={white!69.0196078431373!black},
ymin=-0.15, ymax=3.15,
ytick style={color=black},
yticklabels={}
]
\addplot [thick, color0]
table {%
0 0
0.0204081535339355 0.011989951133728
0.0408163070678711 0.0459824800491333
0.0612244606018066 0.0991048812866211
0.0816326141357422 0.168609499931335
0.102040767669678 0.251873373985291
0.122448921203613 0.346398830413818
0.142857193946838 0.449812650680542
0.163265347480774 0.559866666793823
0.183673501014709 0.674437880516052
0.244897961616516 1.0258948802948
0.265306115150452 1.13979995250702
0.285714268684387 1.24947941303253
0.306122422218323 1.35355925559998
0.326530575752258 1.45079076290131
0.346938848495483 1.54004967212677
0.367347002029419 1.62033689022064
0.387755155563354 1.69077825546265
0.40816330909729 1.75062417984009
0.428571462631226 1.79925036430359
0.448979616165161 1.83615708351135
0.469387769699097 1.86096966266632
0.489795923233032 1.87343847751617
0.510204076766968 1.87343847751617
0.530612230300903 1.86096966266632
0.551020383834839 1.83615708351135
0.571428537368774 1.79925036430359
0.59183669090271 1.75062417984009
0.612244844436646 1.69077825546265
0.632652997970581 1.62033689022064
0.653061151504517 1.54004967212677
0.673469305038452 1.45079076290131
0.693877577781677 1.35355925559998
0.714285731315613 1.24947941303253
0.734693884849548 1.13979995250702
0.755102038383484 1.0258948802948
0.836734771728516 0.559866666793823
0.857142925262451 0.449812650680542
0.877551078796387 0.346398830413818
0.897959232330322 0.251873373985291
0.918367385864258 0.168609499931335
0.938775539398193 0.0991048812866211
0.959183692932129 0.0459824800491333
0.979591846466064 0.011989951133728
1 0
};
\end{groupplot}



\node [text width=5.3649245cm, align=center, anchor=south] at (my plots c1r1.north) {\subcaption{\label{fig:beta_ex_a} $(\alpha_0, \beta_0) = (1, 1)$}};
\node [text width=5.3649245cm, align=center, anchor=south] at (my plots c2r1.north) {\subcaption{\label{fig:beta_ex_b} $(\alpha_0, \beta_0) = (2, 1)$}};
\node [text width=5.3649245cm, align=center, anchor=south] at (my plots c3r1.north) {\subcaption{\label{fig:beta_ex_c} $(\alpha_0, \beta_0) = (3, 1)$}};
\node [text width=5.3649245cm, align=center, anchor=south] at (my plots c1r2.north) {\subcaption{\label{fig:beta_ex_d} $(\alpha_0, \beta_0) = (1, 2)$}};
\node [text width=5.3649245cm, align=center, anchor=south] at (my plots c2r2.north) {\subcaption{\label{fig:beta_ex_e} $(\alpha_0, \beta_0) = (2, 2)$}};
\node [text width=5.3649245cm, align=center, anchor=south] at (my plots c3r2.north) {\subcaption{\label{fig:beta_ex_f} $(\alpha_0, \beta_0) = (3, 2)$}};
\node [text width=5.3649245cm, align=center, anchor=south] at (my plots c1r3.north) {\subcaption{\label{fig:beta_ex_g} $(\alpha_0, \beta_0) = (1, 3)$}};
\node [text width=5.3649245cm, align=center, anchor=south] at (my plots c2r3.north) {\subcaption{\label{fig:beta_ex_h} $(\alpha_0, \beta_0) = (2, 3)$}};
\node [text width=5.3649245cm, align=center, anchor=south] at (my plots c3r3.north) {\subcaption{\label{fig:beta_ex_i} $(\alpha_0, \beta_0) = (3, 3)$}};

\end{tikzpicture}

\caption{Evolution of the Beta distribution $\mathcal{B} (\alpha_{0}, \beta_{0})$ for different values of $(\alpha_{0}, \beta_{0})$.}

% \caption{Initial prior $P_{j0} = \mathcal{B} (\alpha_{j0}, \beta_{j0})$}
\label{fig:beta_change}
\end{figure}

The remainder of figure \ref{fig:sim_plots} plots variations of the baseline for $N = 10,000$ simulations.
In figure \ref{fig:sim_c}, wages in field Y are 50\% higher than wages in field X. 
All other variables are identical to the baseline scenario. 
Unsurprisingly, higher wages in Y drive specialization into that field.
Because the expected lifetime payoff is so much higher, approximately 80\% of agents choose to specialize in Y.

The field X line in figure \ref{fig:sim_c} is longer than the field Y line, implying that agents who specialize in X spend more time in school.
To understand why this happens, first note that all agents begin their education studying Y because of the higher relative wages. 
However, after two periods, a large fraction of agents switch from studying Y to studying X. 
\toedit{This is due to agents (randomly) failing their first two courses in field Y, and switching into field X.}
The reason agents switch fields can be seen by the evolution of their belief distributions, shown in figure \ref{fig:beta_change}.
The student's initial belief distribution is plotted in figure \ref{fig:beta_ex_a}.
A student that fails their first course in field Y updates their beliefs about their underlying ability in Y to the distribution plotted in figure \ref{fig:beta_ex_d}; if they fail their second course in Y, they update their beliefs to \ref{fig:beta_ex_g}.
As we can see from figure \ref{fig:beta_ex_g}, a student that fails their first two classes in Y will believe they likely have a lower ability in that field.
As such, if they choose to specialize in Y, they would not expect to successfully accumulate much human capital over the course of their studies, implying a lower expected lifetime payoff.
As a result, these agents switch to studying field X, in spite of the lower wages.
% This leads to more overall time in school because, as mentioned above, equation \eqref{eq:h_eq_alpha_v} implies that the number of periods an agents spends studying field X or Y is a deterministic function of initial beliefs. 
This switching leads to more overall time in school; as mentioned above, equation \eqref{eq:h_eq_alpha_v} implies that the number of periods an agents spends studying field X or Y before becoming a specialist is a deterministic function of initial beliefs. 
Agents' initial beliefs about their abilities in X and Y are the same, and as such, agents specializing in either X or Y will study their chosen discipline for the same number of periods. 
% In this simple version of the model does not allow for complementaries; therefore, switching fields necessarily increases the amount of time a student is in school. 
However, because all agents spend their first two periods studying field Y, those who specialize in field X will study for a minimum of two more periods.

% % The higher wages in field Y also cause agents specializing in X to spend more time in school. 
% % The minimum periods a field Y agents spends in school as seen in figure \ref{fig:sim_c}.
% As mentioned above, equation \eqref{eq:h_eq_alpha_v} implies that the number of periods an agents spends studying field X or Y is a deterministic function of initial beliefs. 
% Agents' initial beliefs about their abilities in X and Y are the same, and as such, agents specializing in either X or Y will study their chosen discipline for the same number of periods. 
% % In this simple version of the model does not allow for complementaries; therefore, switching fields necessarily increases the amount of time a student is in school. 
% However, because all agents spend the first two periods studying Field Y, those who specialize in Field X will study for a minimum of two more periods.
% % \toedit{Students who update their beliefs about their abilities in field Y}

% Further, all agents begin their education studying Y.
% However, after two periods, a large fraction of agents switch from studying Y to studying X. 
% \toedit{This is due to agents (randomly) failing their first two courses in field Y, and switching into field X.}
% To see why, consider the evolution of the belief distribution in figure \ref{fig:beta_change}. 
% The student's initial belief distribution is plotted in figure \ref{fig:beta_ex_a}.
% A student that fails their first course in field Y updates their beliefs about their underlying ability in Y to the distribution plotted in figure \ref{fig:beta_ex_d}; if they fail their second course in Y, they update their beliefs to \ref{fig:beta_ex_g}.
% As we can see from figure \ref{fig:beta_ex_g}, a student that fails their first two classes in Y will believe they likely have a lower ability in that field.
% If they specialized in Y, they would not expect to successfully accumulate much human capital over the course of their studies, implying a lower expected lifetime payoff.
% As a result, these agents switch to studying field X, in spite of the lower wages. 


Figure \ref{fig:sim_d} augments the baseline scenario so agents have a higher ability in field Y. 
Specifically, probability of success in any given field X course, $\theta_X$, equals 0.4, whereas the probability of success in field Y is given by $\theta_Y = 0.6$. 
Unsurprisingly, a higher ability in field Y drives specialization into that field.

We now turn to the impact of differential priors on specialization dynamics, plotted in figure \ref{fig:sim_e}.
I assume simulated agents are initially more certain about their abilities in field Y relative to field X.
Specifically, I assume a student's initial prior about their ability in Y is given by $P_{Y0} = \mathcal{B} (2, 2)$; this corresponds to the distribution plotted in figure \ref{fig:beta_ex_e}.
Their initial prior about their ability in X continues to equal the uniform distribution, $P_{X0} = \mathcal{B} (1, 1)$.
Note that agents have the same belief about their probability of success in X and Y in expectation. 
However, the variances of the initial distributions suggest that agents have more certainty about their underlying ability in field Y than in field X.

The first consequence of this assumption is that agents specializing in field X study for more periods than those specializing in field Y, as shown in figure \ref{fig:sim_e}.
As mentioned above, equation \eqref{eq:h_eq_alpha_v} implies that the number of periods an agent spends studying $j$ before specializing in that field is a deterministic function of the agent's initial beliefs.
Agents have more initial uncertainty about their abilities in X than in Y, and therefore they will study X for more periods before specializing in that field.
% Increased initial uncertainty about field-$j$ ability implies that agents will study $j$ for more time period before specializing. 
\toedit{The second takeaway from \ref{fig:sim_e} is that all agents begin their education studying field Y.}
Agents know that if they become field Y specialists, they will finish their education earlier, and begin earning an income sooner.
The prospect of ending their education earlier drives agents to initially study field Y.

\toedit{The key takeaway from figure \ref{fig:sim_e} is that increased initial certainty about field Y abilities causes more agents to specialize in field Y.
\footnote{
    It's worth emphasizing that this is not driven by risk aversion across fields;
    \toedit{assuming linear utility in \eqref{eq:linear_utility} ensures that agents are risk neutral across fields.}
    Rather, \toedit{concavity due to discounting ensures that agents are risk averse across time.} 
    \nts{Note that \textcite{A93} assumes first order stochastic dominance. It might be helpful to be able to communicate why my assumptions are different.}
}}
Although agents are equally likely to succeed in fields X and Y, and although the payoffs for specializing in these fields are the same, differential initial beliefs about underlying abilities drives the majority of simulated agents to specialize in field Y.
Thus, initial beliefs play a key role in specialization decisions.
\toedit{I plan on exploring how those beliefs are formed, and the consequences of forming those beliefs based on existing group outcomes.}


% Discuss: agents are risk-neutral across fields, but because of discounting they are risk averse across time (or something like that, I really wish I wrote that down!)

% Non-binary gender identities -
% do not have power to estimate parameters


%%%%%%%%%%%%%%%%%%%%%%%%%%%%%%%%%%%%%%%%%%%%%%%%%%%%%%%%%%%%%%%%%%%%%%%%%%%%%%%%
\section{Connection to statistical discrimination}\label{sec:stat_discrim}
%%%%%%%%%%%%%%%%%%%%%%%%%%%%%%%%%%%%%%%%%%%%%%%%%%%%%%%%%%%%%%%%%%%%%%%%%%%%%%%%

%!TEX root = outline.tex
This section discusses the above model through the lens of statistical discrimination.
I first briefly review the definition of statistical discrimination, and discuss some relevant literature.
I then connect the theory of statistical discrimination to the above model.

%%%%%%%%%%%%%%%%%%%%%%%%%%%%%%%%%%%%%%%%%%%%%%%%%%%%%%%%%%%%%%%%%%%%%%%%%%%%%%%%
\subsection{Statistical discrimination literature}
%%%%%%%%%%%%%%%%%%%%%%%%%%%%%%%%%%%%%%%%%%%%%%%%%%%%%%%%%%%%%%%%%%%%%%%%%%%%%%%%

Fundamentally, statistical discrimination is a theory whereby inequality results from rational agents forming expectations based on existing group characteristics.
To better understand this definition, it helps to briefly review models of taste-based discrimination that preceded it.
The canonical model of taste-based discrimination in labor markets, as formulated by \textcite{B57}, assumes prejudiced employers receive disutility from hiring employees belonging to a particular group.\footnote{
    For a review of the Becker model and details on testable implications, see \textcite{CG08}, who find empirical evidence for the existence of taste-based discrimination in the U.S. labor market.
} 
A key implication of the Becker model is that discriminatory firms will be less profitable than firms that do not discriminate. 
Thus, long-run neoclassical analysis suggests that discrimination will be driven out of the marketplace, an implication that appears incongruous with the persistence of unexplained wage differentials between groups of workers.\footnote{
    A number of authors have incorporated search frictions into the Becker model to explain the long-run persistence of racial wage gaps.
    See \textcite{LL12} for a review of the literature.
    It is worth noting that racial wage gaps cannot persist in a taste-based model with firm entry; discriminatory firms will always be less profitable. \nts{(this is worth checking at some point)}
}


The theory of statistical discrimination, as first formulated by \textcite{A72} and \textcite{P72}, grew out of this critique.
The classical analysis formulated in \textcite{AC77} assumes that a job applicant's group type is one variable an employer uses for inference about their unknown true productivity.\footnote{
    Note that much of the canonical discrimination literature primarily focuses on labor market discrimination, whereby employers discriminate offer lower wages to employees of a particular group.
    These theories of discrimination can easily be extended to alternative contexts.
}
If groups have different aggregate characteristics, and these characteristics are controlled for by the employer-as-statistician, then individuals with the same ability from different group may have different expected productivities.
Statistical discrimination therefore presents a different view of inequality than the Becker model; unequal outcomes may not be the result of prejudice or distaste for certain group types, but rather the result of rational decision making.

\nts{Whether or not an unequal outcome arises from taste-based or statistical discrimination matters from a policy perspective.
It is worth emphasizing that whether a discriminatory outcome arises from explicit prejudice or statistical inference is often irrelevant in a legal sense.
As emphasized in \textcite{LS83}, statistical discrimination is still discrimination, and is often illegal.}

% For example, the aforementioned papers consider the case where employer's use a job applicant's race as one variable when inferring potential productivity.
% However, all of these papers note that discrimination theories apply in alternative contexts.
% A key concern associated with statistical discrimination in labor markets are self-fulfilling prophecies \parencite{LS83,CL93}. 
% \toedit{As discussed in section TBD, this model in this paper is closely tied to the theory of statistical discrimination}.
% From statistical discrimination literature, know that self-fulfilling prophecies matter:
% \begin{blist}
% \item \textcite{CL93}
% \end{blist}
% Matters from an affirmative action point of view. Estimate inefficiencies across communities (compare to \textcite{AL16})
% % beliefs and preferences do appear to matter. 
% % some literature suggests that preferences are key, more so than beliefs.
% % know from statistical discrimination literature that self-fulfilling prophecies matter.
% % this paper ties these ideas together

% \nts{Question: is statistical discrimination just Bayesian linear regression?} \nts{How does this fit in with the \textcite{L98} idea that }

% % Here, it's not so much that labor market discrimination leads to inefficient outcomes. 

%%%%%%%%%%%%%%%%%%%%%%%%%%%%%%%%%%%%%%%%%%%%%%%%%%%%%%%%%%%%%%%%%%%%%%%%%%%%%%%%
\subsection{Connection to model}
%%%%%%%%%%%%%%%%%%%%%%%%%%%%%%%%%%%%%%%%%%%%%%%%%%%%%%%%%%%%%%%%%%%%%%%%%%%%%%%%

Statistical discrimination is often discussed through the lens of labor market discrimination. In the labor market example, employers do not know the true productivity of their potential employee. Unequal outcomes arise from employers-as-statisticians using group-based information about prospective employees for inference.

Fundamentally, the model outlined in sections \ref{sec:model} and \ref{sec:analytic_results} is a model of statistical discrimination. 
To convince the reader that this inequality of outcomes should indeed be classified as statistical discrimination, consider the definition set forth in \textcite{LS83}:
\begin{quote}
Economic discrimination exists when groups with equal average initial endowments of productive ability do not receive equal average compensation in equilibrium.
\end{quote}
This definition is employed to explicitly account for the fact that the existence of unequal outcomes may impact pre-labor market human capital investment decisions. 
% \item Discriminatory outcome: agents with the same outcome sort into different fields
% Human capital investment decisions endogenously respond to heterogeneous outcomes. 
In the model outlined above, agents with equal levels of initial human capital may not make the same specialization decisions because of differences in initial beliefs, as shown in section \ref{sec:sims}.
% rational beliefs may be formed based existing aggregate group outcomes.
Students-as-statisticians do not know their true productivity, and they use group-based information for inference. 
The resulting inequality of outcomes can thus be classified as statistical discrimination. 

Connecting the above model to statistical discrimination is conceptually beneficial.
Recent discrimination literature presents variants of traditional statistical discrimination models, such as dynamic discrimination \parencite{BIR19}, or inaccurate statistical discrimination \parencite{BHIP19-wp}. 
Dynamic belief formation, or the influence of inaccurate belief formation, are both extremely relevant to my model.
Thinking about my model as statistical discrimination allows me to draw from this nascent literature. 
Additionally, thinking about this model as statistical discrimination lays the groundwork to connect this model to the literature on affirmative action. 
Traditional theoretical literature on affirmative action, such as \textcite{CL93}, begin with a model of statistical discrimination.
The belief updating mechanism in my model still allows for self-fulfilling prophecies, but using a different framework.
Thus, this model presents an alternative framework for evaluating affirmative action policies, a goal I elaborate on in section \ref{sec:future_work}.

Although there seems to be an intuitive connection between the model outlined above and theories of statistical discrimination, I have not formalized it yet.
This is because my model does not currently consider how these beliefs dynamically change over time. 
Understanding this is key for, say, understanding how this model connects to theories of self-fulfilling prophecies.
Thus, my next goal is to build a dynamic version of the above model, a step I outline in section \ref{sec:dynamic_model}.

%%%%%%%%%%%%%%%%%%%%%%%%%%%%%%%%%%%%%%%%%%%%%%%%%%%%%%%%%%%%%%%%%%%%%%%%%%%%%%%%
\section{Future work: Identification and quantitative exercises}\label{sec:future_work}
%%%%%%%%%%%%%%%%%%%%%%%%%%%%%%%%%%%%%%%%%%%%%%%%%%%%%%%%%%%%%%%%%%%%%%%%%%%%%%%%

% %!TEX root = outline.tex
%%%%%%%%%%%%%%%%%%%%%%%%%%%%%%%%%%%%%%%%%%%%%%%%%%%%%%%%%%%%%%%%%%%%%%%%%%%%%%%%
\section{Future work: Identification and quantitative exercises}\label{sec:future_work}
%%%%%%%%%%%%%%%%%%%%%%%%%%%%%%%%%%%%%%%%%%%%%%%%%%%%%%%%%%%%%%%%%%%%%%%%%%%%%%%%

I would like to use the remainder of this paper to outline my research agenda. 
My first research goal is to use this model to explain the evolution of college major choices by gender over time, as seen in figure \ref{fig:ipeds_a}.
To do that, I first need to embed the above model in a dynamic framework that describes how aggregate group-based beliefs change over time. 
I detail that goal in section \ref{sec:dynamic_model}.
To fully characterize the evoluation of major choices, I will need to identify model parameters.
In section \ref{sec:identification_overview}, I discuss identification concerns, outline my goals for identifying model parameters, and briefly review potential data sources.
Once I have properly calibrated model parameters, I can test whether my model is effective at re-creating the dynamics in major choice seen in figure \ref{fig:ipeds_a}.
I will then be able to use my model in additional quantitative exercises.
In section \ref{sec:extensions}, I briefly discuss potential counterfactual exercises, as well as possible model extensions, before concluding. 

%%%%%%%%%%%%%%%%%%%%%%%%%%%%%%%%%%%%%%%%%%%%%%%%%%%%%%%%%%%%%%%%%%%%%%%%%%%%%%%%
\subsection{Dynamic model}\label{sec:dynamic_model}

There are several ways in which I plan on expanding my model.
First, and most importantly, I want to incorporate intergenerational learning into the above framework.
Additionally, I want to carefully discuss the role of preferences. I elaborate on these extensions below.

In the aggregate, the model presented above is static; while I do model each agent's decision over the course of their education, their initial beliefs are given and unchanging.
To understand how beliefs change over time, I need to embed this model in a dynamic framework, and specify how group-based beliefs change over time.

\toedit{\textcite{F13} provides a road map for approaching this modeling problem.
Her paper attempts to ascertain the role that shifting cultural norms played in the expansion of female labor force participation.
She does this by building a model of labor decisions and intergenerational learning.
Although the source of uncertainty in her model is disutility from working, not ability, the process of learning from aggregate outcomes is similar. 
I need to build my model to fit in a model of intergenerational learning and cultural change, similar to \textcite{F13}.}

Relatedly, I plan on better accounting for preferences in future versions of this model, particularly once I have a model of cultural change. 
As discussed in the literature review in section \ref{sec:intro}, preferences and norms are a key motivator of college major choice.
Much of the evidence for the primacy of preferences relies on subjective expectations evidence \parencite{WZ18,AHMR20}.
My results focus on beliefs about ability, a separate factor that determines college major choice.
Although there is evidence that beliefs about ability are important for college major choice \parencite{O20}, I want to be able to clearly articulate how my results compare to the tastes from the subjective expectations literature; it's not necessarily clear how the ``taste'' residuals from the subjective expectations literature will compare to my belief parameters.
The interplay of beliefs and preferences in this context can be theoretically important \parencite{BG02}, and I want to remain cognizant of that moving forward.

%%%%%%%%%%%%%%%%%%%%%%%%%%%%%%%%%%%%%%%%%%%%%%%%%%%%%%%%%%%%%%%%%%%%%%%%%%%%%%%%
\subsection{Identification overview}\label{sec:identification_overview}
%%%%%%%%%%%%%%%%%%%%%%%%%%%%%%%%%%%%%%%%%%%%%%%%%%%%%%%%%%%%%%%%%%%%%%%%%%%%%%%%

I would now like to return to the static version of the model outlined in section \ref{sec:model} and discuss calibration.\footnote{
    Thus, for the purpose of this discussion, I want to focus on calibrating a version of the model for a single cohort.
}
The parameters driving agent behavior in the model are summarized in table \ref{tab:parameter_descriptions}.
Potential data resources are outlined in section \ref{sec:data_sources}.
\begin{table}[!ht]
\centering
\caption{Model parameters}{}
\label{tab:parameter_descriptions}
\begin{tabular}{cl}
\hline \hline{}
Parameter & Description %& Source
\\ \hline{}
$J$ & Number of fields %& ACS \& IPEDS
\\
$\delta$ & Discount factor
\\ \hline
$h_{j0}$ & Initial field-$j$ human capital
\\ 
$(\alpha_{j0}, \beta_{j0})$ & Initial field-$j$ ability beliefs
\\ \hline \hline
\end{tabular}
\end{table}
The first part of table \ref{tab:parameter_descriptions} lists aggregate parameters that impact all agents.
Assuming a standard value is used for the discount rate $\delta$, the key aggregate parameter is the number of fields of study, $J$. 
I have developed taxonomies of fields using Integrated Postsecondary Education Data System (IPEDS) and American Community Survey (ACS) data; see reviews of these data resources in section \ref{sec:data_sources}.

Parameters that govern individual agent heterogeneity are in the second half of table \ref{tab:parameter_descriptions}.
To determine initial human capital levels, $h_{j0}$, I need information on students' initial field-specific human capital when they begin college.
Datasets such as the Beginning Postsecondary Students (BPS) or National Longitudinal Survey of Youth (NLS97) panels are thus ideal, as they are surveys of college performance and major choice that collect information on high school performance. 
For details on these resources, see section \ref{sec:data_sources}. 

The primary identification problem for this analysis is uncovering the group-based belief parameters $(\alpha_{j0}^g, \beta_{j0}^g)$.
Assuming we have data on the agent's choice of field at time $t$, we can utilize some type of conditional logit method and find the parameters that maximize the following likelihood:
\begin{equation*}
    \log \mathcal{L} 
    = 
    \log \sum_{i = 1}^n \sum_{t=1}^\infty \sum_{j=1}^J m_{ijt} 
    \log P(m_{ijt} = 1 \vert \overline{m}_{ijt}, \overline{s}_{ijt}, \alpha_{j0}^{g(i)}, \beta_{j0}^{g(i)}, h_{ij0}, \theta_{ji})
\end{equation*}
Thus, identifying the model entails specifying the appropriate choice probability.

One immediate concern associated with this process is time endogeneity; an agent's choice of a field at time $t$ is related to their choice at time $t+1$. A simple way to avoid this endogeneity is to identify model parameters from time $t=0$ (i.e. the beginning of the agent's education).
More sophisticated methods should take advantage of the problem's recursive structure to utilize more of an agent's panel. 

% Work in section \ref{sec:solving_index} suggests that estimation is possible. 
Let $G_j (x)$ denote the CDF of the index $j$. Recall from section \ref{sec:solving_index} that the agent's expected lifetime payoff associated with field $j$ is a function of their expected time in school, denoted $N$:
\begin{alignat*}{3}
    G_j(x) =& \PP \pr{\cls{\mathcal{I}_{jt} < x}{\pstates}} \span \span
    \\
    =&
    \PP \bigg(
        &&\frac{1}{1 - \delta} w_j
        \bigg(
            \EE \ce{\delta^N}{\cdot} 
            \pr{h_{j0} + \nu_j + N \ks}
            + 
            \EE \ce{\delta^N N}{\cdot} 
            \frac{\alpha_{j0} + \ks}{\alpha_{j0} + \beta_{j0} + \tilde{\study}_{j0}}
        \bigg) \\
    &&&< x \bigg\vert \pstates \bigg)
\end{alignat*}
The agent's probability distribution over possible stopping times is solved in section \ref{sec:solving_index}.
The choice probability can then be written as:
\begin{align*}
    \PP \pr{\cls{
        m_{jt} = 1
    }{\tilde{\study}_t, \tilde{\pass}_t, \psi_0}}
    =& 
    \PP \pr{\cls{
        \mathcal{I}_{jt} > \mathcal{I}_{kj}
    }{\tilde{\study}_t, \tilde{\pass}_t, \psi_0, \forall k \neq j}}
    \\
    =& 
    \int \prod_{k \neq j} 
    G_k \pr{
        \crs{x}{\tilde{\study}_{kt}, \tilde{\pass}_{kt}, \psi_{k0}}
    } d G_j \pr{
        \crs{x}{\tilde{\study}_{jt}, \tilde{\pass}_{jt}, \psi_{j0}}
    }
\end{align*}
The solution to $G_j(x)$ will depend on the distribution of $h_{j0}$ and the distribution of $\theta_j$ in society.
I'm not certain of the best way to incorporate individual heterogeneity into the model. 
But that will be key for analytically characterizing the CDF $G_j (x)$ and estimating $(\alpha_{j0}^g, \beta_{j0}^g)$ from the data.  
% My goal is to analytically characterize the CDF $G_j (x)$, and find helpful constraints on the distributions of $h_{j0}$ and $\theta_j$ such that I can identify $(\alpha_{j0}^g, \beta_{j0}^g)$ from the data. 


%%%%%%%%%%%%%%%%%%%%%%%%%%%%%%%%%%%%%%%%%%%%%%%%%%%%%%%%%%%%%%%%%%%%%%%%%%%%%%%
\subsection{Future quantitative exercises and concluding remarks}\label{sec:extensions}
%%%%%%%%%%%%%%%%%%%%%%%%%%%%%%%%%%%%%%%%%%%%%%%%%%%%%%%%%%%%%%%%%%%%%%%%%%%%%%%

My immediate goal is to see if the model outlined above can re-create the dynamics of gender major choice seen in figure \ref{fig:ipeds_a}.
Ultimately, I want to utilize this model in counterfactual analysis.
This section briefly discusses some potential applications.

First, I believe a dynamic version of the above model would shed some light on how long it would take for gender convergence across majors to occur. 
Assuming that participation in some fields follows an S-shaped pattern, as in \textcite{F13}, would parity across majors ever be achieved as cultural beliefs change?
Or will there always be a gap without additional interventions?
Interesting counterfactual analysis would also be possible with a dynamic version of the model. 
For instance, suppose that women were given men's belief distributions.
How long would it then take for men and women to then make the same specialization choices?

Relatedly, I believe this model would be valuable for evaluating affirmative action policies.
Suppose that affirmative action policies can impact group-based beliefs.
How long would it take for these policies to result in gender convergence?
And would full gender convergence across fields of study ever occur?
% Theoretical evaluations of affirmative action policies, beginning with \textcite{CL93}, often consider how long-term beliefs of employers respond to new information.
% I think this model presents the first part of a framework that allows for beliefs to change over time.
This may play an important role in explaining the efficacy of affirmative action policies.
% \toedit{What happens if I remove discrimination; how long would it take for women's beliefs to converge to the truth?}
% Discuss the role of affirmative action. Can affirmative action address these biases?

Finally, I am interested in the impact of this model on the aggregate misallocation of talent in the economy. 
As noted in \textcite{HHJK19}, barriers to human capital accumulation impact aggregate economic productivity.
As discussed above, if men and women have the different beliefs, they may make different specialization decisions.
If the underlying true ability distributions are the same across genders, this may represent a misallocation of talent, and have aggregate economic effects. 
I am interested in applying the framework from \textcite{HHJK19} to estimate what the aggregate effects of this misallocation of talent might be. 
 % the impact of barriers to human capital accumulation on aggregate productivity using a \textcite{R51} model of occupational choice. 
% I'm also interested in measuring the productivity impacts of differential group-based beliefs. 

% \toedit{
% % start of actual section.
% This model can then be used for counterfactual analysis. Possible quantitative exercises:
% \begin{outline}

% \item Productivity exercise: does the misallocation of talent due to group-based beliefs affect agregate productivity.
% % There are several key differences between the model outlined below and the one from \textcite{HHJK19} that warrant attention.
% %
% % First, they explicitly model barriers to field-specific human capital attainment in monetary terms. 
% % As such, field specialization in their model is reduced to picking the occupation with the highest indirect expected utility.
% %
% % Their model allows for sorting on by preferences or ability. 
% % Their benchmark scenario considers sorting on ability, and thus I assume the same.
% % 
% \item
% \item 
% \end{outline}

% } % end \toedit

Overall, I think the model of group-based beliefs and human capital specialization outlined in this prospectus can address a number of interesting economic questions. 
I have extensive work to do before I can fully address the topics outlined in this section.
But I believe this framework will allow me to answer these questions in a novel way.

Under construction

\printbibliography

%%%%%%%%%%%%%%%%%%%%%%%%%%%%%%%%%%%%%%%%%%%%%%%%%%%%%%%%%%%%%%%%%%%%%%%%%%%%%%
\section*{Appendix}
\setcounter{section}{0}
\setcounter{subsection}{0}
\renewcommand{\thesubsection}{A.\arabic{subsection}}
% \setcounter{table}{0}
% \renewcommand{\thetable}{A\arabic{table}}
%%%%%%%%%%%%%%%%%%%%%%%%%%%%%%%%%%%%%%%%%%%%%%%%%%%%%%%%%%%%%%%%%%%%%%%%%%%%%%

%!TEX root = outline.tex
%%%%%%%%%%%%%%%%%%%%%%%%%%%%%%%%%%%%%%%%%%%%%%%%%%%%%%%%%%%%%%%%%%%%%%%%%%%%%%%
\subsection{Solving the index}\label{sec:solving_index}
%%%%%%%%%%%%%%%%%%%%%%%%%%%%%%%%%%%%%%%%%%%%%%%%%%%%%%%%%%%%%%%%%%%%%%%%%%%%%%%

This section discusses the analytical solution to the index \eqref{eq:index_monotonicity_general}.
Specifically, I describe how to evaluate the expected value of discounted human capital accumulation, conditional on initial states:
\begin{equation}\label{eq:expected_discounted_hc_accumulation}
    \EE_t \ce{
        \delta^{\study_j^* - \tilde{\study}_{jt}}
        h_{j,t + (\study_j^* - \tilde{\study}_{jt})}
    }{\states}.
\end{equation}
To solve this, I first show how this expectation can be re-written as a function of expected time remaining in school.
Computing the index therefore requires finding the conditional probability distribution of stopping times. 
The remainder of section is devoted to finding this distribution. 
This is done by first bounding the stopping times, and then recursively defining the probability distribution. 

%%%%%%%%%%%%%%%%%%%%%%%%%%%%%%%%%%%%%%%%%%%%%%%%%%%%%%%%%%%%%%%%%%%%%%%%%%%%%%%
\subsubsection*{Index in terms of expected time in school}
%%%%%%%%%%%%%%%%%%%%%%%%%%%%%%%%%%%%%%%%%%%%%%%%%%%%%%%%%%%%%%%%%%%%%%%%%%%%%%%

To simplify notation, let $\psi_{j0}$ denote the initial belief parameters and human capital levels:
\begin{equation*}
    \psi_{j0} = \pr{\alpha_{j0}, \beta_{j0}, h_{j0}}.
\end{equation*}
the agent's state when evaluating field $j$ at time $t$ is now determined by $\pr{\pstates}$.

Ignoring other fields, an agents expects to study $j$ for $\study_j^* - \tilde{\study}_{jt}$ additional periods before beginning work as a field-$j$ specialist.
Substituting in the human capital accumulation function \eqref{eq:hc_accumulation} into \eqref{eq:expected_discounted_hc_accumulation}:
\begin{align*}
    \EE_t &\ce{
        \delta^{\study_j^* - \tilde{\study}_{jt}}
        h_{j,t + (\study_j^* - \tilde{\study}_{jt})}
    }{\pstates}
    \\
    &\quad\quad=
    \EE_t \ce{
        \delta^{\study_j^* - \tilde{\study}_{jt}}
        \pr{
            h_{j0} 
            + \nu_j \ks 
            + \sum_{x=0}^{\study_j^* - \tilde{\study}_{jt}} 
            \pass_{j, t + x}}
    }{\pstates}
    \\
    &\quad\quad=
    \EE_t \ce{
        \delta^{\study_j^* - \tilde{\study}_{jt}}        
    }{\pstates} \pr{h_{j0} + \nu_j \ks}
    + 
    \EE_t \ce{
        \delta^{\study_j^* - \tilde{\study}_{jt}}
        \sum_{x=0}^{\study_j^* - \tilde{\study}_{jt}} \pass_{j, t + x}
    }{\pstates}.
\end{align*}
Two expectations are key. 
The first is the expected value of discounting the next $\study_j^* - \tilde{\study}_{jt}$ additional periods. 
The second is the expected value of the discounted term times the number of of times an agent successfully passes their field-$j$ courses during those $\study_j^* - \tilde{\study}_{jt}$ periods. 
To simplify the second probability, use the law of iterated expectations:
\begin{align*}
    \EE_t &\ce{
        \delta^{\study_j^* - \tilde{\study}_{jt}}
        \sum_{x=0}^{\study_j^* - \tilde{\study}_{jt}} \pass_{j, t + x}
    }{\pstates}
    \\
    &\quad\quad=
    \EE_t \ce{
        \EE_t \ce{
            \delta^{\study_j^* - \tilde{\study}_{jt}}
            \sum_{x=0}^{\study_j^* - \tilde{\study}_{jt}} \pass_{j, t + x}
        }{\study_j^*, \pstates}
    }{\pstates}
    \\
    &\quad\quad=
    \EE_t \ce{
        \delta^{\study_j^* - \tilde{\study}_{jt}}
        \EE_t \ce{
            \sum_{x=0}^{\study_j^* - \tilde{\study}_{jt}} \pass_{j, t + x}
        }{\study_j^*, \pstates}
    }{\pstates}
\end{align*}
The number of times the agent successfully passes their field-$j$ courses over the next $\study_j^* - \tilde{\study}_{jt}$ periods is a series of $\study_j^* - \tilde{\study}_{jt}$ Bernoulli trials with probability $\theta_j$. 
The agent's expected value of this random variable is given by the sample size multiplied by their ability parameter $\theta_j$. 
Therefore the previous equation can be written as:\footnote{
    At this point, I must note that the following section may contain an error. 
}
\begin{align}
    \nonumber
    \EE_t &\ce{
        \delta^{\study_j^* - \tilde{\study}_{jt}}
        \EE_t \ce{
            \sum_{x=0}^{\study_j^* - \tilde{\study}_{jt}} \pass_{j, t + x}
        }{\study_j^*, \pstates}
    }{\pstates}
    \\
    \nonumber
    &\quad\quad=
    \EE_t \ce{
        \delta^{\study_j^* - \tilde{\study}_{jt}}
        \pr{\study_j^* - \tilde{\study}_{jt}}
        \EE_t \ce{
            \theta_j
        }{\study_j^*, \pstates}
    }{\pstates}
    \\
    \nonumber
    &\quad\quad=
    \EE_t \ce{
        \delta^{\study_j^* - \tilde{\study}_{jt}}
        \pr{\study_j^* - \tilde{\study}_{jt}}
        \frac{\alpha_{j0} + \ks}{\alpha_{j0} + \beta_{j0} + \tilde{\study}_{jt}}
    }{\pstates}    
    \\
    \nonumber
    &\quad\quad=
    \EE_t \ce{
        \delta^{\study_j^* - \tilde{\study}_{jt}}
        \pr{\study_j^* - \tilde{\study}_{jt}}
    }{\pstates}
    \frac{\alpha_{j0} + \ks}{\alpha_{j0} + \beta_{j0} + \tilde{\study}_{jt}}
\end{align}
The third line follows from the agent's expected value of $\theta_j$, conditional on their states and their time until completion, $\study_j^*$, according to their belief distribution.
We can use this result to further simplify \eqref{eq:expected_discounted_hc_accumulation} as:
\begin{equation*}
    \EE_t \ce{
        \delta^{\study_j^* - \tilde{\study}_{jt}}        
    }{\pstates} \pr{h_{j0} + \nu_j \ks}
    + 
    \EE_t \ce{
        \delta^{\study_j^* - \tilde{\study}_{jt}}
        \pr{\study_j^* - \tilde{\study}_{jt}}
    }{\pstates}
    \frac{\alpha_{j0} + \ks}{\alpha_{j0} + \beta_{j0} + \tilde{\study}_{jt}}.
\end{equation*}
Thus, the key expected value \eqref{eq:expected_discounted_hc_accumulation} is really the expected value of a function of time remaining in school.

Before proceeding, it's helpful to introduce some simplifying notation, and to re-scale the problem to start at time $t=0$. 
To simplify notation, let $N = \study_j^* - \tilde{\study}_{jt}$ denote the time remaining in school after $\tilde{\study}_{jt}$.
The variable $N$ is capitalized to emphasize the fact that $N$ is a random quantity. 
Next, note that for any agent evaluating field $j$ at time $t$ with states $\pr{\pstates}$, we can always define:
\begin{alignat*}{3}
    \hat{\alpha}_{j0} =& \alpha_{j0} + \tilde{\pass}_{jt},
    \quad \quad
    &\hat{\alpha}_{j0} + \hat{\beta}_{j0} 
    =& \alpha_{j0} + \beta_{j0} + \tilde{\study}_{jt},
    \\
    \hat{h}_{j0} 
    =& h_{j0} + \nu_j \tilde{\pass}_{jt}, 
    \quad \quad
    &\hat{\psi}_{j0} 
    =& 
    \pr{\hat{\alpha}_{j0}, \hat{\beta}_{j0}, \hat{h}_{j0}}.
\end{alignat*}
Therefore, instead of evaluating how many courses an agent has remaining after completing $\tilde{\study}_{jt}$ courses, we can re-define the agent's states and evaluate the agent's total expected time in school from $t=0$, before the agent has taken any courses in $j$.
In that vein, I will only condition on the initial states $\psi_{j0} = \pr{\alpha_{j0}, \beta_{j0}, h_{j0}}$, and assume $\tilde{\pass}_{jt} = \tilde{\study}_{jt} = 0$.
Recall that $\ks[,N]$ is the number of times the student successfully passes their courses after matriculating $N$ times. 
Using the above notation, the monotonic stopping condition \eqref{eq:stop_montonicity} is given by:
\begin{equation}\label{eq:stop_monotonicity_re_scaled}
    N \geq \frac{\delta}{1 - \delta}
    \pr{
        \frac{\nu_j \alpha_{j0} + \nu_j \ks[,N]}{h_{j0} + \nu_j \ks[,N]}
    } - \alpha_{j0} - \beta_{j0},
\end{equation}
The goal in subsequent sections is to evaluate the following conditional expectations:
\begin{align}    
    \nonumber
    \EE_0 \ce{\delta^N}{\alpha_{j0}, \beta_{j0}, h_{j0}}
    =&
    \sum_{z = 0}^\infty
    \delta^z
    \PP
    \pr{\crs{
        N = z
    }{\alpha_{j0}, \beta_{j0}, h_{j0}}}
    \\
    \label{eq:index_ce_n}
    \EE_0 \ce{\delta^N N}{\alpha_{j0}, \beta_{j0}, h_{j0}}
    =&
    \sum_{z = 0}^\infty
    \delta^z z
    \PP
    \pr{\crs{
        N = z
    }{\alpha_{j0}, \beta_{j0}, h_{j0}}}
\end{align}
The following section discusses the bounds on stopping times, so the above summation is finite.  





%%%%%%%%%%%%%%%%%%%%%%%%%%%%%%%%%%%%%%%%%%%%%%%%%%%%%%%%%%%%%%%%%%%%%%%%%%%%%%%
\subsubsection*{Bounds on stopping times}
%%%%%%%%%%%%%%%%%%%%%%%%%%%%%%%%%%%%%%%%%%%%%%%%%%%%%%%%%%%%%%%%%%%%%%%%%%%%%%%

This section starts by defining a lower bound larger than zero for all stopping times.
I then discuss stopping times with positive probability, and use this discussion to motivate the upper bound for all stopping times.

\begin{lemma}
Define the positive integer $\underline{n}$ as:
\begin{align}\label{eq:lower_n}
     \underline{n} = \min_N \left\{ 
     N \geq
        \frac{\delta}{1 - \delta}
    - \alpha_{j0} - \beta_{j0}
     \right\}
     =
     \ceil{\frac{\delta}{1 - \delta}} - \alpha_{j0} - \beta_{j0}.
 \end{align}
Then $\underline{n}$ is a lower bound for stopping times that satisfy the monotonic stopping condition \eqref{eq:stop_monotonicity_re_scaled}.
\end{lemma}
\begin{proof}
For all possible stopping times $N$ and all possible stochastic outcomes $\ks[,N]$, $\frac{\nu_j \alpha_{j0} + \nu_j \ks[N]}{h_{j0} + \nu_j \ks[N]} \geq 1$ under the initial monotonicity condition \eqref{eq:h_leq_alpha_v}.\footnote{
    Define $f(s) = \frac{\nu_j \alpha_{j0} + \nu_j s}{h_{j0} + \nu_j s} = \frac{\nu_j \alpha_{j0} - h_{j0}}{h_{j0} + \nu_j s} + 1$.
    Because $\nu_j \alpha_{j0} \geq h_{j0}$ and $s \geq 0$, $f(s) \geq 1$.
}
Therefore, $\underline{n}$ is a lower bound.
\end{proof}





\noindent
Before defining the upper bound of $N$, it is helpful to discuss the stopping condition for different potential values of $N$.
Given the stopping condition \eqref{eq:stop_monotonicity_re_scaled}, an agent will decide to stop studying in period $\underline{n} + x$ if: 
\begin{alignat}{3}
    &&
    \ddelta - \alpha_{j0} - \beta_{j0} + x
    \geq&
    \frac{\delta}{1-\delta}
    \pr{\frac{\nu_j \alpha_{j0} + \nu_j \ks[,\underline{n} + x]}{h_{j0} + \nu_j \ks[,\underline{n} + x]}}
    -\alpha_{j0} - \beta_{j0}
    \nonumber \\
    &\implies &
    \frac{\delta}{1 - \delta}
    + \epsilon_\delta
    + x
    \geq&
    \frac{\delta}{1 - \delta}
    \pr{
       \frac{\nu_j \alpha_{j0} + \nu_j \ks[,\underline{n} + x]}{h_{j0} + \nu_j \ks[,\underline{n} + x]}
    }
    \nonumber
    \\
    &\implies &
    \epsilon_\delta 
    + x
    \geq&
    \frac{\delta}{1 - \delta}
    \pr{
        \frac{\nu_j \alpha_{j0} - h_{j0}}{h_{j0} + \nu_j \ks[,\underline{n} + x]}
    }
    \nonumber
    \\
    &\implies &
    \pr{
        \epsilon_\delta  + x
    }
    \pr{h_{j0} + \nu_j \ks[,\underline{n} + x]}
    \geq&
    \frac{\delta}{1 - \delta}
    \pr{\alpha_{j0} \nu_j - h_{j0}}
    \label{eq:stop_with_error}
\end{alignat}
where the rounding error $\epsilon_\delta \in [0, 1)$ equals the difference between the ceiling of the discount factor $\frac{\delta}{1 - \delta}$ and its true value.\footnote{
    Specifically, if $\frac{\delta}{1 - \delta}$ is not an integer, then $\epsilon_\delta = \cmf{\frac{\delta}{1 - \delta}} - \left\{ \frac{\delta}{1 - \delta} \right\} = 1 - \left\{ \frac{\delta}{1 - \delta} \right\}$ is the difference between the ceiling of $\frac{\delta}{1 - \delta}$ and $\frac{\delta}{1 - \delta}$. 
    If $\frac{\delta}{1 - \delta}$ is an integer, then $\epsilon = 0$.
    Recall that the ceiling of any real number $x$, denoted $\ceil{x}$, is the smallest integer greater than or equal to $x$.
    The floor of a $x$, $\floor{x}$, is the largest integer less than or equal to $x$. 
    The fractional part of $x$, denoted $\{x\}$, is defined by $\{x\} = x - \floor{x}$.
}
% Using set-notation to eliminate the explicit rounding error \eqref{eq:stop_with_error}, an agent will stop studying at time $\underline{n} + x$ if:
% \begin{equation}
%     \label{eq:stop_no_error}
%     \left\{ x \pr{h_{j0} + \nu_j \tilde{\pass}_{\underline{n} + x}} \geq \frac{\delta}{1 - \delta} \pr{\alpha_{j0} \nu_j - h_{j0}} \right\}.
% \end{equation}


Now let's consider cases where agents would only take $N=\underline{n}$ courses, meaning that $x = 0$.
Using the stopping condition \eqref{eq:stop_with_error} with $x=0$, this only happens if:
\begin{alignat*}{3}
    &&
    \epsilon_\delta
    \pr{h_{j0} + \nu_j \ks[\underline{n}]} 
    \geq&
    \frac{\delta}{1 - \delta} 
    (\alpha_{j0} \nu_j - h_{j0})
    \\
    &\implies&
    \epsilon_\delta
    \nu_j \ks[\underline{n}]
    \geq&
    \frac{\delta}{1 - \delta} \nu_j \alpha_{j0} - \ddelta h_{j0}.
\end{alignat*}
The agent will only study for $\underline{n}$ periods if this inequality is satisfied for all possible stochastic outcomes. 
The only stochastic part of this inequality is $\ks[\underline{n}]$, which make take on values between 0 and $\underline{n}$.
Therefore, agents will only study for $\underline{n}$ periods if:
\begin{alignat*}{3}
    &&
    0
    \geq&
    \frac{\delta}{1 - \delta} \nu_j \alpha_{j0} - \ddelta h_{j0}.
    \\
    &\implies&
    \ddelta h_{j0} 
    \geq&
    \frac{\delta}{1 - \delta} \alpha_{j0} \nu_j.
 \end{alignat*}
 Combining the above with the initial monotonicity condition \eqref{eq:h_leq_alpha_v} implies that an agent will only study for exactly $N = \underline{n}$ periods if:
\begin{equation*}
    1 
    \leq 
    \frac{\nu_j \alpha_{j0}}{h_{j0}} 
    \leq
    \frac{\ddelta}{\frac{\delta}{1 - \delta}}.
\end{equation*}
Because the ratio of the ceiling of the discount factor to its true value will be close to 1, this inequality effectively states that the agent will only study for $N = \underline{n}$ periods if $h_{j0} = \nu_j \alpha_{j0}$ (with some adjustment for rounding error). 
This is the tractable case evaluated in \textcite{AF20}.
Specifically, assuming the slightly stronger initial condition, $h_{j0} = \nu_j \alpha_{j0}$, implies time spent in school $N$ equals $\underline{n}$; an agent who specializes in field $j$ will take exactly $N$ courses in field $j$.
Therefore, the the optimal number of field-$j$ courses is a deterministic function of the agent's initial beliefs.
However, for reasons discussed in the overview of section \ref{sec:analytic_results}, this assumption is not necessarily appropriate for this evaluation. 
Thus, we now turn to evaluating the upper bound of possible stopping times. 


\begin{lemma}
Define the positive integer $\overline{n}$ as:
\begin{equation}\label{eq:upper_n}
    \overline{n} = \ceil{\frac{\delta}{1 - \delta} \frac{\alpha_{j0} \nu_j}{h_{j0}}} - \alpha_{j0} - \beta_{j0}
\end{equation}
Then $\overline{n}$ is an upper bound for stopping times.
\end{lemma}

\begin{proof}

$\overline{n}$ is an upper bound for stopping times if, for all stochastic outcomes $\ks[\overline{n}]$:
\begin{equation*}
    \overline{n} \geq 
    \frac{\delta}{1 - \delta} 
    \pr{\frac{\alpha_{j0} \nu_j + \nu_j \ks[\overline{n}]}{h_{j0} + \nu_j \ks[\overline{n}]}} - \alpha_{j0} - \beta_{j0}
\end{equation*}
Because $\frac{\alpha_{j0} \nu_j + \nu_j \ks[\overline{n}]}{h_{j0} + \nu_j \ks[\overline{n}]}$ is decreasing in $\ks[\overline{n}]$,\footnote{
    Define $f(s) = \frac{\nu_j \alpha_{j0} + \nu_j s}{h_{j0} + \nu_j s} = \frac{\nu_j\alpha_{j0} - h_{j0}}{h_{j0} + \nu_j s} + 1$. 
    Note that $f'(s) = -\frac{\nu_j (\nu_j \alpha_{j0} - h_{j0})}{\pr{h_{j0} + \nu_j s}^2}$.
    This is nonpositive when $\nu_j \alpha_{j0} \geq h_{j0}$. 
}
$\overline{n}$ is an upper bound independent of stochastic outcomes only if the above inequality holds for $\ks[\overline{n}] = 0$ (i.e. when the agent has failed all of their field $j$ courses).
Therefore, $\overline{n}$ is an upper bound if:
\begin{alignat*}{3}
    &&
    \overline{n} 
    \geq& 
    \frac{\delta}{1 - \delta} 
    \pr{\frac{\alpha_{j0} \nu_j}{h_{j0}}} 
    - \alpha_{j0} - \beta_{j0}
    \\
    &\implies&
    \ceil{\frac{\delta}{1 - \delta} \frac{\alpha_{j0} \nu_j}{h_{j0}}}
    \geq&
     \frac{\delta}{1 - \delta} 
    \pr{\frac{\alpha_{j0} \nu_j}{h_{j0}}}.
\end{alignat*}
Therefore, $\overline{n}$ is an upper bound for stopping times.
\end{proof}








The details above can be used to bound the summations in \eqref{eq:index_ce_n}:
\begin{align*}
    \EE \ce{\delta^N}{\alpha_{j0}, \beta_{j0}, h_{j0}} 
    =&
    \sum_{z=\underline{n}}^{\overline{n}} 
    \delta^z
    \PP
    \pr{\crs{
        N = z
    }{\alpha_{j0}, \beta_{j0}, h_{j0}}}
    \\
    \EE \ce{\delta^N N}{\alpha_{j0}, \beta_{j0}, h_{j0}} 
    =&
    \sum_{z=\underline{n}}^{\overline{n}} 
    \delta^z z
    \PP
    \pr{\crs{
        N = z
    }{\alpha_{j0}, \beta_{j0}, h_{j0}}}
\end{align*}
The next section evaluates the above conditional probabilities.



\subsubsection*{Conditional probabilities}

Recall that the lower and upper bounds of $N$ are given by $\underline{n}$ and $\overline{n}$, respectively. 
The conditional probability that $N$ equals some integer $z$ can be evaluated as:
\begin{align}
    \PP \pr{\crs{
        N = z
    }{\psi_{j0}}}
    =&
    \nonumber
    \PP \pr{\cls{
        z 
        \geq 
        \frac{\delta}{1 - \delta}
        \frac{\alpha_{j0} \nu_j + \nu_j \ks[z]}{h_{j0} + \nu_j \ks[z]} -\alpha_{j0} - \beta_{j0}
    }{\psi_{j0}}}
    \\
    \nonumber
    =&
    \PP \pr{\cls{
        z - \pr{\underline{n} - \epsilon_\delta}
        \geq
        \frac{\delta}{1 - \delta}
        \frac{\alpha_{j0} \nu_j - h_{j0}}{h_{j0} + \nu_j \ks[z]}
    }{\psi_{j0}}} 
    \\
    =&
    \PP \pr{\cls{
        \pr{z - \underline{n} + \epsilon_\delta}
        \pr{h_{j0} + \nu_j \ks[z]}
        \geq
        \frac{\delta}{1 - \delta}
        \pr{\alpha_{j0} \nu_j - h_{j0}}
    }{\psi_{j0}}} 
    % \\
    % =&
    % \PP \pr{\cls{
    %     \pr{z - \underline{n} + \epsilon_\delta}
    %     \ks
    %     \geq
    %     \frac{\delta}{1 - \delta}
    %     \frac{\alpha_{j0} \nu_j - h_{j0}}{\nu_j}
    %     -
    %     \pr{z - \underline{n} + \epsilon_\delta}
    %     \frac{h_{j0}}{\nu_j}
    % }{\alpha_{j0}, \beta_{j0}, h_{j0}}}
    % \\
    % =&
    % \PP \pr{\cls{
    %     \pr{z - \underline{n} + \epsilon_\delta}
    %     \ks[z]
    %     \geq
    %     \frac{\delta}{1 - \delta} \alpha_{j0}
    %     -
    %     \pr{\frac{\delta}{1 - \delta} + z - \underline{n} + \epsilon_\delta}
    %     \frac{h_{j0}}{\nu_j}
    % }{\psi_{j0}}} 
    \\
    =&
    \PP \pr{\cls{
        \pr{z - \underline{n} + \epsilon_\delta}
        \ks[z]
        \geq
        \frac{\delta}{1 - \delta} \alpha_{j0}
        -
        \pr{\ddelta + z - \underline{n}}
        \frac{h_{j0}}{\nu_j}
    }{\psi_{j0}}} 
\end{align}
First consider the case where $N = \underline{n}$. Let $\PP_0$ denote this probability. Then, as discussed above:
\begin{align*}
    \PP_0 = \PP \pr{\crs{
        N = \underline{n}
    }{\psi_{j0}}}
    =&
    \begin{cases}
    1 &\text{ if } 1 \leq \frac{\alpha_{j0} \nu_j}{h_{j0}} \leq \frac{\ddelta}{\frac{\delta}{1 - \delta}}
    \\
    0 &\text{otherwise}.
    \end{cases}
\end{align*} 
% Note that if the fraction $\frac{\ddelta}{\frac{\delta}{1 - \delta}} = 1$, then $\epsilon_\delta = 0$, whereas if the fraction is greater than 1, $\epsilon_\delta > 0$. 
% For ease of notation, all probabilities below assume $\frac{\alpha_{j0} \nu_j}{h_{j0}} > 1$, and will evaluate cases for different values of $\epsilon$.

Before evaluating cases where $N > \underline{n}$, note the probability of stopping at some positive integer $z$ is always given by:
\begin{align*}
    \PP \pr{\cls{N = z}{\psi_{j0}}} 
    &=
    \PP \pr{\cls{N = z, N \neq z - 1}{\psi_{j0}}}
    +
    \PP \pr{\cls{N = z, N = z - 1}{\psi_{j0}}}
    \\
    &=
    \PP \pr{\cls{N = z}{N \neq z - 1, \psi_{j0}}}
    \PP \pr{\cls{N \neq z - 1}{\psi_{j0}}}
\end{align*}
The second line follows from Bayes' Rule, and the fact that an agent would never stop at time $z$ if they already stopped at time $z-1$.
In words, this states that the probability of stopping at time $z$ is given by the product of (1) the probability of stopping at time $z$ conditional on having not stopped at time $z-1$; and (2) the probability of having not stopped by $z-1$.

To evaluate the probability that $N = \underline{n} + 1$, first evaluate the conditional probability that $N = \underline{n} + 1$, conditional on the stopping time not equaling $\underline{n}$:
\begin{align*}
    \PP \pr{\cls{N = \underline{n} + 1}{N \neq \underline{n}, \psi_{j0}}}
    =&
    \PP \pr{\cls{
        \pr{1 + \epsilon_\delta}
        \pr{h_{j0} + \nu_j \ks[\underline{n} + 1]}
        \geq
        \frac{\delta}{1 - \delta}
        \pr{\alpha_{j0} \nu_j - h_{j0}}
    }{\psi_{j0}}}
    % \\
    % =&
    % \PP \pr{\cls{
    %     \pr{1 + \epsilon_\delta}
    %     \ks[\underline{n} + 1]
    %     \geq
    %     \frac{\delta}{1 - \delta} \alpha_{j0}
    %     -
    %     \pr{\ddelta + 1}
    %     \frac{h_{j0}}{\nu_j}
    % }{\psi_{j0}}}
    \\
    =&
    \PP \pr{\cls{
        \ks[\underline{n} + 1]
        \geq
        \frac{1}{\nu_j}
        \pr{        
            \frac{\delta}{1 - \delta} 
            \frac{1}{1 + \epsilon_\delta}
            \pr{\alpha_{j0} \nu_j - h_{j0}}
            - h_{j0}
        }
    }{\psi_{j0}}}
    \\
    =&
    \PP \pr{\cls{
        \ks[\underline{n} + 1]
        \geq
        \hat{k}_1
    }{\psi_{j0}}}
\end{align*}
The RHS of the above inequality is a function of initial parameters and can be treated as a constant.
The variable $\ks[\underline{n} + 1]$ is a random variable:
\begin{equation*}
    \ks[, \underline{n} + 1] = \sum_{t=0}^{\underline{n}} \pass_{jt} \sim \text{Binomial} (\underline{n} + 1, \theta_j)
\end{equation*}
Define $k_1$ as:
\begin{align*}
    k_1 &= \begin{cases}
    \hat{k}_1 - 1
    &\text{ if $\hat{k}_1$ an integer,}
    \\
    \floor{\hat{k}_1}
    &\text{ otherwise.}
    \end{cases}
\end{align*}
The conditional probability can be written as:
\begin{align*}
    \PP \pr{\cls{N = \underline{n} + 1}{N \neq \underline{n}, \psi_{j0}}}
    =&
    1 
    -
    \PP \pr{\cls{
        \ks[\underline{n} + 1]
        <
        k_1
    }{\psi_{j0}}} 
    \\
    =&
    1 
    - 
    \sum_{i=0}^{k_1}
    \binom{\underline{n} + 1}{i}
    \theta^i
    (1 - \theta)^{\underline{n} + 1 - i}
\end{align*}
Now we can fully evaluate the probability that $N = \underline{n} + 1$:
\begin{align*}
    \PP_1 
    = 
    \PP \pr{\cls{N = \underline{n} + 1}{\psi_{j0}}} 
    =& 
    \PP \pr{\cls{N = \underline{n} + 1}{N \neq \underline{n}, \psi_{j0}}}
    \PP \pr{\cls{N \neq \underline{n}}{\psi_{j0}}}
    \\
    =&
    \PP \pr{\cls{N = \underline{n} + 1}{N \neq \underline{n}, \psi_{j0}}} 
    \pr{1 - \PP_0}
    \\
    =&
    \pr{1 
        - 
        \sum_{i=0}^{k_1}
        \binom{\underline{n} + 1}{i}
        \theta^i
        (1 - \theta)^{\underline{n} + 1 - i}
    }
\pr{1 - \PP_0}.
\end{align*}

To evaluate the probability that $N = \underline{n} + x$ for some integer $x$, we have to evaluate:
\begin{align*}
    \PP_x = \PP \pr{\cls{N = \underline{n} + x}{\psi_{j0}}} 
    =&
    \PP  \pr{\crs{
        N = \underline{n} + x
    }{
    \bigcap_{i=0}^{x-1}
        \pr{N \neq \underline{n} + i}
    , \psi_{j0}}}
    \PP \pr{\cls{
        \bigcap_{i=0}^{x-1}
        \pr{N \neq \underline{n} + i}
    }{\psi_{j0}}}  
\end{align*} 
First consider the probability that $N$ is not equal any value between the lower bound $\underline{n}$ and $\underline{n} + x - 1$. By De Morgan's law and countability additivity:
\begin{align*}
    \PP \pr{\cls{
        \bigcap_{i=0}^{x-1}
        \pr{N \neq \underline{n} + i}
    }{\psi_{j0}}}
    =&
    1
    -
    \PP
    \pr{\cls{
        \bigcup_{i=0}^{x-1}
        \pr{N = \underline{n} + i}
    }{\psi_{j0}}}
    \\
    =&
    1
    -
    \sum_{i=0}^{x-1}
    \PP
    \pr{\cls{
        N = \underline{n} + i
    }{\psi_{j0}}}
    % \\
    =
    1
    -
    \sum_{i=0}^{x-1}
    \PP_i.
\end{align*}
Now we turn to the conditional probability of interest, which can be written as:
\begin{align}
    \nonumber
    \PP  &\pr{\crs{
        N = \underline{n} + x
    }{
    \bigcap_{i=0}^{x-1}
        \pr{N \neq \underline{n} + i}
    , \psi_{j0}}}
    \\ 
    \label{eq:prob_n_plus_x_long}
    &=
    \PP \left( 
        (x + \epsilon_\delta)
        \pr{h_{j0} + \nu_j \ks[, \underline{n} +x]
        }
        \geq
        \frac{\delta}{1 - \delta}
        \pr{\alpha_{j0} \nu_j - h_{j0}}
    \right.
    \\ \nonumber
    &\quad\quad\quad\left\vert
        \bigcap_{k=0}^{x - 1}
        (k + \epsilon_\delta)
        \pr{h_{j0} + \nu_j \ks[\underline{n} + k - 1]
        }
        <
        \frac{\delta}{1 - \delta}
        \pr{\alpha_{j0} \nu_j - h_{j0}}
    \right).
\end{align}
To simplify this conditional, recall that an agent will decide to keep studying at time $\underline{n} + y$ if:
\begin{alignat*}{3}
&&
\frac{\delta}{1 - \delta}
\pr{\alpha_{j0} \nu_j - h_{j0}}
>&
(y + \epsilon_\delta)
\pr{h_{j0} + \nu_j \ks[, \underline{n} + y]}
\\
&&
=&
(y - 1 + \epsilon_\delta)
\pr{h_{j0} + \nu_j \ks[, \underline{n} + y -1]}
+ 
\pr{h_{j0} + \nu_j \ks[, \underline{n} + y -1]}
+
(y + \epsilon_\delta)
\nu_j \pass_{j, \underline{n} + y - 1}
\\
&&
>&
(y - 1 + \epsilon_\delta)
\pr{h_{j0} + \nu_j \ks[, \underline{n} + y -1]}
\end{alignat*}
This is simply the monotonicity of the stopping problem in reverse; if an agent would decide to continue on at time $\underline{n} + y$, then they also would have wanted to continue on at time $\underline{n} + y - 1$.
This simplifies the conditional expression in equation \eqref{eq:prob_n_plus_x_long}:
\begin{align}
    \nonumber
    \PP  &\pr{\crs{
        N = \underline{n} + x
    }{
    \bigcap_{i=0}^{x-1}
        \pr{N \neq \underline{n} + i}
    , \psi_{j0}}}
    \\ 
    \nonumber
    &=
    \PP \left( 
        (x + \epsilon_\delta)
        \pr{h_{j0} + \nu_j \ks[, \underline{n} +x]
        }
        \geq
        \frac{\delta}{1 - \delta}
        \pr{\alpha_{j0} \nu_j - h_{j0}}
    \right.
    \\ \nonumber
    &\quad\quad\quad\left\vert
        (x - 1 + \epsilon_\delta)
        \pr{h_{j0} 
        + \nu_j \ks[\underline{n} + x - 1]
        }
        <
        \frac{\delta}{1 - \delta}
        \pr{\alpha_{j0} \nu_j - h_{j0}}
    \right)
\end{align}
This probability can be re-written to reflect the fact that $\ks[,\underline{n} + x] = \ks[,\underline{n} + x -1] + \pass_{j, \underline{n} + x - 1}$.
In words, the total number of successes seen by time $\underline{n} + x$ equals the total number of successes seen by time $\underline{n} + x - 1$ plus the course outcome during period $\underline{n} + x - 1$:
\begin{multline*}
    \PP \left( 
        \pr{x + \epsilon_\delta}
        \pr{h_{j0} 
        + \nu_j \ks[, \underline{n} + x - 1]}
        +
        \pr{x + \epsilon_\delta}
        \nu_j \pass_{j, \underline{n} + x - 1}
        \geq
        \frac{\delta}{1 - \delta}
        \pr{\alpha_{j0} \nu_j - h_{j0}}
    \right.
    \\
    \left\vert
        (x - 1 + \epsilon_\delta)
        \pr{h_{j0} 
        + \nu_j \ks[\underline{n} + x - 1]
        }
        <
        \frac{\delta}{1 - \delta}
        \pr{\alpha_{j0} \nu_j - h_{j0}}
    \right).
\end{multline*}
Define the random variables $Y$ and $Z$ and the constant $c$ as:
\begin{alignat*}{2}
    Y &= g(\ks[, \underline{n} + x - 1])  =
        \pr{x + \epsilon_\delta}
        \pr{h_{j0} 
        + \nu_j \ks[, \underline{n} + x - 1]},
    \quad \quad
    c &= \frac{\delta}{1 - \delta} \pr{\alpha_{j0} \nu_j - h_{j0}},
    \\
    Z &= h(\pass_{j, \underline{n} + x - 1}) = \pr{x + \epsilon_\delta}
        \nu_j \pass_{j, \underline{n} + x - 1}
\end{alignat*}
The conditional probability that $N = \underline{n} + x$ for $x > 1$ can now be written as:
\begin{align*}
    \PP \pr{\cls{Y + Z \geq c}{Y < c \frac{x + \epsilon_\delta}{x - 1 + \epsilon_\delta}}},
\end{align*}
where $Y$ and $Z$ are independent random variables whose distributions are one-to-one functions of binomial distributions:
\begin{align*}
    \PP (Y = y) 
    =& \PP \pr{ g\pr{\ks[,\underline{n} + x - 1]}= y}
    = \PP (\ks[,\underline{n} + x - 1] = g^{-1} (y)))
    \\
    =&
    \binom{\underline{n} + x - 2}{g^{-1} (y)}
    \theta_j^{g^{-1} (y)}
    (1 - \theta_j)^{\underline{n} + x - 2 - g^{-1} (y)},
    \\
    \PP
    (Z = z)
    =& \PP \pr{ h\pr{\pass_{j, \underline{n} + x - 1}} = z}
    = \PP \pr{\pass_{j, \underline{n} + x - 1} = h^{-1} (z)}
    \\
    =&
    \theta_j^{h^{-1}(z)} \pr{1 - \theta_j}^{1 - h^{-1}(z)}.
\end{align*}
The joint conditional distribution can be solved using Theorem 20.3 in \textcite[pg. 280]{B61}.


% Next, consider the case where $N > \underline{n}$.
% The probability of stopping at time $z$ becomes:
% \begin{align*}
%     \PP \pr{\crs{
%         n = z
%     }{\alpha_{j0}, \beta_{j0}, h_{j0}}}
%     =&
%     \PP \pr{\cls{
%         \ks[z]
%         \geq
%         \frac{\delta}{1 - \delta}
%         \frac{1}{z - \underline{n} + \epsilon_\delta}
%         \frac{\alpha_{j0} \nu_j - h_{j0}}{\nu_j}
%         -
%         \frac{h_{j0}}{\nu_j}
%     }{\alpha_{j0}, \beta_{j0}, h_{j0}}} 
%     \\
%     =&
% \end{align*}






% The smallest integer $z$ such that the stopping condition holds is given by:
% \begin{align*}
%     \min_z
%     \left\{ 
%     z
%     \geq
%     \frac{\delta}{1 - \delta}
%     \frac{\alpha_{j0} \nu_j - h_{j0}}{h_{j0} + \nu_j \ks}
%     + \frac{\delta}{1 - \delta} - \alpha_{j0} - \beta_{j0}
%     \right\} 
%     =&
%     \ceil{
%         \frac{\delta}{1 - \delta}
%         \frac{\alpha_{j0} \nu_j - h_{j0}}{h_{j0} + \nu_j \ks}
%         + \frac{\delta}{1 - \delta} - \alpha_{j0} - \beta_{j0}    
%     }
%     \\
%     =&
%     \ceil{
%         \frac{\delta}{1 - \delta}
%         \frac{\alpha_{j0} \nu_j - h_{j0}}{h_{j0} + \nu_j \ks}
%         +    
%         \frac{\delta}{1 - \delta}
%     }
%     - \alpha_{j0}
%     - \beta_{j0}
% \end{align*}





% \subsubsection*{Simplified index}


% If we further assume the analytic initial condition \eqref{eq:h_eq_alpha_v}, then \eqref{eq:stop_montonicity} simplifies to:\footnote{
%     The algebra here is simple, but aided by the following transformations. 
%     First, the updating rule \eqref{eq:beta_updating} and the human capital accumulation function \eqref{eq:hc_accumulation} imply that $\nu_j (\alpha_{jt} - \alpha_{j0}) = h_{jt} - h_{j0}$. Second, assuming \eqref{eq:h_eq_alpha_v} implies that $\alpha_{jt} \nu_j = h_{jt}$.
% }
% \begin{equation}\label{eq:stopping_func}
%     \frac{1 - \delta}{\delta} \geq 
%     \frac{
%         1
%     }{
%         \alpha_{jt}^g + \beta_{jt}^g
%     }.
% \end{equation}



% Then the stopping condition \eqref{eq:stopping_func} can be written as:
% \begin{equation*}
%     \frac{1 - \delta}{\delta} \geq 
%     \frac{
%         1
%     }{
%         h_{jt} (c_{jt} + \alpha_{j0}^g + \beta_{j0}^g)
%     }.
% \end{equation*}
% Now, the optimal number of field-$j$ courses is a deterministic function of the agent's initial beliefs:
%      c_j^* = \left\lceil \frac{\delta}{1 - \delta} \right\rceil - (\alpha_{j0} + \beta_{j0})
% \end{equation*}
% This means that an agent who specializes in field $j$ will take exactly $c_{j}^*$ courses in field $j$, where $c_{j}^*$ is a function of an agent's initial field-$j$ beliefs.% \begin{equation*}





% Therefore, assuming \eqref{eq:h_eq_alpha_v} and linear utility \eqref{eq:linear_utility} implies the optimal field-$j$ graduation region is given by:
% \begin{equation*}
%     \mathcal{G}_j (\alpha_{jt}, \beta_{jt}) = \left\{ 
%         \alpha_{jt}, \beta_{jt} 
%         \left\vert \frac{\delta}{1 - \delta} \leq \alpha_{jt} + \beta_{jt}
%     \right.\right\}
% \end{equation*}
% In this example, note that $\mathcal{G}_Y = \mathcal{G}_X$. Index in the graduation region given by $\frac{h_{jt}}{1 - \delta}$. 
% Index when not in graduation region given by Binomial distribution with parameters $\left(c_j^* - c_j, \frac{h_{jt}}{\nu(c_{jt} + \alpha_{j0} + \beta{j0})}\right)$.
% The index from equation \eqref{eq:index_general} can be simplified to:
% \begin{align*}
%     % \mathcal{I}_j (\alpha_{jt}, \beta_{jt}) = 
%     % \begin{cases}
%     %     \frac{w_{jt}}{1 - \delta} h_{jt} &\text{ if } 
%     %         \{\alpha_{jt}, \beta_{jt}\} \in \mathcal{G}_j \\ 
%     %     \frac{w_{jt}}{1 - \delta} 
%     %         \left(h_{jt} + \nu \mathbb{E}[\theta_j \vert \alpha_{jt}, \beta_{jt}]\right) &\text{ if } \{\alpha_{jt}, \beta_{jt}\} \notin \mathcal{G}_j
%     % \end{cases}
% \mathcal{I}_{jt} (h_{jt}, \alpha_{jt}, \beta_{jt}) = 
% \begin{cases}
% \frac{w_{jt} h_{jt}}{1 - \delta} & \text{if } \{\alpha_{jt}, \beta_{jt}\} \in \mathcal{G}_{j}, \\
% \frac{w_{jt} h_{jt}}{1 - \delta} \sbr{
%    \frac{
%       \left\lceil \frac{\delta}{1 - \delta} \right\rceil
%       \delta^{\left\lceil \frac{\delta}{1 - \delta} \right\rceil - c_{jt} - \alpha_{j0} - \beta_{j0}}}
%    {c_{jt} + \alpha_{j0} + \beta_{j0}}
%    } & \text{if } \{\alpha_{jt}, \beta_{jt}\} \notin \mathcal{G}_{j} \\
% \end{cases}
% \end{align*}

% % } % end \toedit{}
% % However, this assumption ties together beliefs and initial levels of human capital.
% \nts{Future versions of this project will relax this assumption.}
% \nts{May also want to add that using a simple version of this stopping problem highlights other mechanisms.}

%!TEX root = outline.tex
%%%%%%%%%%%%%%%%%%%%%%%%%%%%%%%%%%%%%%%%%%%%%%%%%%%%%%%%%%%%%%%%%%%%%%%%%%%%%%%%
% \subsubsection*{Data sources}
\subsection{Data sources}\label{sec:data_sources}

I have utilized a number of data resources for this project thus far. Several of them are outlined below. 

Characteristics of all U.S. postsecondary institutions are reported in the Integrated Postsecondary Education Data System (IPEDS) data.
These data are collected annually by the National Center for Educational Statistics and describe the universe of institutions that participate in federal student financial aid programs. 
The empirical motivation for this analysis relies on the IPEDS Completion Surveys, which describe all degrees and certificates awarded at postsecondary institutions by field of study, gender, and race.\footnote{
    For more details on the IPEDS series, please visit \url{https://nces.ed.gov/ipeds/}.
    IPEDS data are available from 1986 until the present, though I begin empirical analysis in 1990 due to changes in how fields of study are classified.
    For earlier data, such as those used in Figure \ref{fig:n_degrees}, I supplement the IPEDS series with data from \textcite{S93}.
    However, it is worth noting that the predecessor to IPEDS series is the Higher Education General Information Survey (HEGIS), available through the International Archive of Education Data at University of Michigan (\url{https://www.icpsr.umich.edu/web/ICPSR/series/00030}). As such, I refer the reader to the HEGIS series for a more detailed portrait of postsecondary education statistics than available in \textcite{S93}.
}
University-level graduation rates can be estimated using the IPEDS graduation surveys.
% Details on IPEDS graduation surveys are in the Appendix. 

The American Community Survey (ACS), conducted by the U.S. Census Bureau, collects detailed information on American households. 
These data include information on employment and income, demographic information, and, crucially, educational attainment. 
Specifically, beginning in 2009, the ACS began recording up to two undergraduate fields of study for household members.
This has allowed papers such as \textcite{SHB19} to examine the dynamics of human capital specialization decisions across cohorts. 
For detailed information on using ACS data, see \textcite{IPUMS}.
% Individual-level data on degree completions are from the American Community Survey (ACS), as accessed using \textcite{IPUMS}.
% I follow the data selection procedure outlined in Appendix A of \textcite{SHB19}.

Individual student data on beliefs and abilities are more difficult to come by.
An ideal resource is the 2012/17 Beginning Postsecondary Students Longitudinal Study, conducted by the National Center for Education Statistics, which can be used to identify initial levels of human capital and group-level beliefs. 
This longitudinal study collects education and employment data from a nationally representative sample of first-time beginning postsecondary students.
Respondents are initially surveyed in 2011-2012, the beginning of their postsecondary studies. 
Follow-up surveys were conducted three and six years after they began their studies.\footnote{
    For additional information, see \url{https://nces.ed.gov/pubsearch/pubsinfo.asp?pubid=2020504}.
    Restricted-use licenses are required for access to BPS microdata. 
    However, data aggregates and simple regression analysis are available through the NCES PowerStats DataLab. 
}
Key for this analysis are the BPS transcript studies, which contain information on first-year major choice, high school performance, and outcomes. 
An application for restricted-use BPS data is pending. 
% Used by: \textcite{SS14}

The National Longitudinal Survey of Youth (NLS97) provides additional data that can be used to identify student characteristic.
These data consist of a representative sample of approximately 9,000 U.S. men and women born between 1980 and 1984.
Respondent's educational and labor market experiences are recorded over time. 
In particular, these data contain transcript information on respondents.
Though BPS data would be preferable, as the data are more recent and more representative, NLS97 data provides a strong basis for beginning this analysis.  


% \input{degree_labels.tex}


\end{document}
