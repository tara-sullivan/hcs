%!TEX root = outline.tex
This section discusses the above model through the lens of statistical discrimination.
I first briefly review the definition of statistical discrimination, and discuss some relevant literature.
I then connect the theory of statistical discrimination to the above model.

%%%%%%%%%%%%%%%%%%%%%%%%%%%%%%%%%%%%%%%%%%%%%%%%%%%%%%%%%%%%%%%%%%%%%%%%%%%%%%%%
\subsection{Statistical discrimination literature}
%%%%%%%%%%%%%%%%%%%%%%%%%%%%%%%%%%%%%%%%%%%%%%%%%%%%%%%%%%%%%%%%%%%%%%%%%%%%%%%%

Fundamentally, statistical discrimination is a theory whereby inequality results from rational agents forming expectations based on existing group characteristics.
To better understand this definition, it helps to briefly review models of taste-based discrimination that preceded it.
The canonical model of taste-based discrimination in labor markets, as formulated by \textcite{B57}, assumes prejudiced employers receive disutility from hiring employees belonging to a particular group.\footnote{
    For a review of the Becker model and details on testable implications, see \textcite{CG08}, who find empirical evidence for the existence of taste-based discrimination in the U.S. labor market.
} 
A key implication of the Becker model is that discriminatory firms will be less profitable than firms that do not discriminate. 
Thus, long-run neoclassical analysis suggests that discrimination will be driven out of the marketplace, an implication that appears incongruous with the persistence of unexplained wage differentials between groups of workers.\footnote{
    A number of authors have incorporated search frictions into the Becker model to explain the long-run persistence of racial wage gaps.
    See \textcite{LL12} for a review of the literature.
    It is worth noting that racial wage gaps cannot persist in a taste-based model with firm entry; discriminatory firms will always be less profitable. \nts{(this is worth checking at some point)}
}


The theory of statistical discrimination, as first formulated by \textcite{A72} and \textcite{P72}, grew out of this critique.
The classical analysis formulated in \textcite{AC77} assumes that a job applicant's group type is one variable an employer uses for inference about their unknown true productivity.\footnote{
    Note that much of the canonical discrimination literature primarily focuses on labor market discrimination, whereby employers discriminate offer lower wages to employees of a particular group.
    These theories of discrimination can easily be extended to alternative contexts.
}
If groups have different aggregate characteristics, and these characteristics are controlled for by the employer-as-statistician, then individuals with the same ability from different group may have different expected productivities.
Statistical discrimination therefore presents a different view of inequality than the Becker model; unequal outcomes may not be the result of prejudice or distaste for certain group types, but rather the result of rational decision making.

\nts{Whether or not an unequal outcome arises from taste-based or statistical discrimination matters from a policy perspective.
It is worth emphasizing that whether a discriminatory outcome arises from explicit prejudice or statistical inference is often irrelevant in a legal sense.
As emphasized in \textcite{LS83}, statistical discrimination is still discrimination, and is often illegal.}

% For example, the aforementioned papers consider the case where employer's use a job applicant's race as one variable when inferring potential productivity.
% However, all of these papers note that discrimination theories apply in alternative contexts.
% A key concern associated with statistical discrimination in labor markets are self-fulfilling prophecies \parencite{LS83,CL93}. 
% \toedit{As discussed in section TBD, this model in this paper is closely tied to the theory of statistical discrimination}.
% From statistical discrimination literature, know that self-fulfilling prophecies matter:
% \begin{blist}
% \item \textcite{CL93}
% \end{blist}
% Matters from an affirmative action point of view. Estimate inefficiencies across communities (compare to \textcite{AL16})
% % beliefs and preferences do appear to matter. 
% % some literature suggests that preferences are key, more so than beliefs.
% % know from statistical discrimination literature that self-fulfilling prophecies matter.
% % this paper ties these ideas together

% \nts{Question: is statistical discrimination just Bayesian linear regression?} \nts{How does this fit in with the \textcite{L98} idea that }

% % Here, it's not so much that labor market discrimination leads to inefficient outcomes. 

%%%%%%%%%%%%%%%%%%%%%%%%%%%%%%%%%%%%%%%%%%%%%%%%%%%%%%%%%%%%%%%%%%%%%%%%%%%%%%%%
\subsection{Connection to model}
%%%%%%%%%%%%%%%%%%%%%%%%%%%%%%%%%%%%%%%%%%%%%%%%%%%%%%%%%%%%%%%%%%%%%%%%%%%%%%%%

Statistical discrimination is often discussed through the lens of labor market discrimination. In the labor market example, employers do not know the true productivity of their potential employee. Unequal outcomes arise from employers-as-statisticians using group-based information about prospective employees for inference.

Fundamentally, the model outlined in sections \ref{sec:model} and \ref{sec:analytic_results} is a model of statistical discrimination. 
To convince the reader that this inequality of outcomes should indeed be classified as statistical discrimination, consider the definition set forth in \textcite{LS83}:
\begin{quote}
Economic discrimination exists when groups with equal average initial endowments of productive ability do not receive equal average compensation in equilibrium.
\end{quote}
This definition is employed to explicitly account for the fact that the existence of unequal outcomes may impact pre-labor market human capital investment decisions. 
% \item Discriminatory outcome: agents with the same outcome sort into different fields
% Human capital investment decisions endogenously respond to heterogeneous outcomes. 
In the model outlined above, agents with equal levels of initial human capital may not make the same specialization decisions because of differences in initial beliefs, as shown in section \ref{sec:sims}.
% rational beliefs may be formed based existing aggregate group outcomes.
Students-as-statisticians do not know their true productivity, and they use group-based information for inference. 
The resulting inequality of outcomes can thus be classified as statistical discrimination. 

Connecting the above model to statistical discrimination is conceptually beneficial.
Recent discrimination literature presents variants of traditional statistical discrimination models, such as dynamic discrimination \parencite{BIR19}, or inaccurate statistical discrimination \parencite{BHIP19-wp}. 
Dynamic belief formation, or the influence of inaccurate belief formation, are both extremely relevant to my model.
Thinking about my model as statistical discrimination allows me to draw from this nascent literature. 
Additionally, thinking about this model as statistical discrimination lays the groundwork to connect this model to the literature on affirmative action. 
Traditional theoretical literature on affirmative action, such as \textcite{CL93}, begin with a model of statistical discrimination.
The belief updating mechanism in my model still allows for self-fulfilling prophecies, but using a different framework.
Thus, this model presents an alternative framework for evaluating affirmative action policies, a goal I elaborate on in section \ref{sec:future_work}.

\nts{Some notes on innaccuate statistical discrimination:
\begin{blist}
\item 
\end{blist}
}


Although there seems to be an intuitive connection between the model outlined 
above and theories of statistical discrimination, I have not formalized it yet.
This is because my model does not currently consider how these beliefs dynamically change over time. 
Understanding this is key for, say, understanding how this model connects to theories of self-fulfilling prophecies.
Thus, my next goal is to build a dynamic version of the above model.