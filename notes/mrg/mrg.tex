\documentclass[compress, 8pt]{beamer}
%\documentclass[handout, 8pt]{beamer}
\usepackage[utf8]{inputenc}



% Standard packages
\usepackage{hyperref} % Include links
\usepackage{setspace} % More control over local line spacing: http://felix11h.github.io/blog/line-spacing-beamer
% Math
\usepackage{amsthm, amsmath, mathtools}
\usepackage{appendixnumberbeamer}
% Multi=row and column
\usepackage{multirow,multicol}
% multiline columns
\usepackage{tabularx}


\usepackage{ifthen}

\usepackage{amsbsy}

%%%%%%%%%%%%%%
% Formatting %
%%%%%%%%%%%%%%

% Outer theme
\useoutertheme[subsection=false]{miniframes}
% Fix to display navigational dots; dots only shown for subsections
% Can comment out to have no dots in presentation
\AtBeginSection[]{\subsection{}}
% Font
\usepackage[default]{cantarell}

% Colors
\usepackage{xcolor}
% UCSD color palette: http://ucpa.ucsd.edu/brand/elements/color-palette/
\definecolor{DarkBlue}{RGB}{24, 43, 73}
\definecolor{MainBlue}{RGB}{0, 106, 150}
\definecolor{LightBlue}{RGB}{0, 198, 215}
\definecolor{Green}{RGB}{110 150 59}
\definecolor{Yellow}{RGB}{255 205 0}
\definecolor{LightYellow}{RGB}{243 29 0}
\definecolor{Gold}{RGB}{198 146 20}
\definecolor{Orange}{RGB}{252 137 0}
\definecolor{LightGrey}{RGB}{182, 177, 169}
\definecolor{DarkGray}{RGB}{116 118 120}
\definecolor{DarkPurple}{RGB}{25 0 51}
\definecolor{Purple}{RGB}{55 24 81}
\definecolor{Maroon}{HTML}{6A123D}


% Hyperlink colors
\hypersetup{
  colorlinks   = true, %Colours links instead of ugly boxes
  urlcolor     = Maroon, %Colour for external hyperlinks
  linkcolor    = DarkGray, %Colour of internal links
  citecolor   = MainBlue %Colour of citations
}


% Learned using: https://ramblingacademic.com/2015/12/how-to-quickly-overhaul-beamer-colors/
% To change additional sections: http://www.cpt.univ-mrs.fr/~masson/latex/Beamer-appearance-cheat-sheet.pdf
\setbeamercolor{palette primary}{bg=DarkBlue,fg=white}
\setbeamercolor{palette secondary}{bg=DarkBlue,fg=white}
\setbeamercolor{palette tertiary}{bg=DarkBlue,fg=white}
\setbeamercolor{palette quaternary}{bg=DarkBlue,fg=white}
\setbeamercolor{structure}{fg=DarkBlue} % itemize, enumerate, etc
\setbeamercolor{section in toc}{fg=DarkBlue} % TOC sections
\setbeamercolor{section in head/foot}{bg=DarkBlue,fg=LightGrey}
%\setbeamercolor{navigation symbols}{bg=Gold, fg=Gold}
\setbeamercolor{alerted text}{fg=Gold}
\beamertemplatenavigationsymbolsempty
%tem
%\setbeamertemplate{itemize items}

% Make bullet size smaller
\setbeamertemplate{itemize item}{\tiny\raise1.4pt\hbox{$\blacktriangleright$}}
\setbeamertemplate{itemize subitem}{--}
% make bullet closer to the text
\setlength{\labelsep}{.8ex}

% get rid of 'Figure: ' in caption
\setbeamertemplate{caption}{\raggedright\insertcaption\par}
% Caption flushed left
\usepackage{caption}
\captionsetup{%
    labelformat=empty,
    font=small,
    singlelinecheck=false,
    tableposition=top
}


%%%%%%%%%%%%%%%%
% New commands %
%%%%%%%%%%%%%%%%

\ifshowcomments
\newcommand{\nts}[2][DarkGray]{\setbeamercolor{taras comment}{fg=#1}{\usebeamercolor[fg]{taras comment}\setbeamercolor{item}{fg = taras comment.fg}\emph{#2}}}
\else 
\newcommand{\nts}[2][DarkGray]{}
\fi

% Timed color highlights
\newcommand<>{\blue}[1]{\textbf{\color#2{MainBlue}#1}}
\newcommand<>{\mblue}[1]{{\color#2{MainBlue}#1}}
\newcommand<>{\green}[1]{\textbf{\color#2{Green}#1}}
\newcommand<>{\mgreen}[1]{{\color#2{Green}#1}}

\newcommand{\EE}{\mathbb{E}}
\newcommand{\PP}{\mathbb{P}}
\newcommand{\llog}[1]{\log \left( #1 \right)}
\newcommand{\Cov}[1]{\text{Cov} \left( #1 \right)}

\newcommand{\br}[1]{\left\{ #1 \right\}}
\newcommand{\sbr}[1]{\left[ #1 \right]}
\newcommand{\pr}[1]{\left( #1 \right)}
\newcommand{\ce}[2]{\left[\left. #1 \right\vert #2 \right]}
\newcommand{\cls}[2]{\left. #1 \right\vert #2}
\newcommand{\crs}[2]{#1 \left\vert #2 \right.}
\newcommand{\ceil}[1]{\left\lceil #1 \right\rceil}

% Beamer buttons
% bottom buttom: \bbutton{name_of_link}{Text to print}
\newcommand{\bbutton}[2]{
    \begin{tikzpicture}[remember picture, overlay]
    \node[shift={(-1.4cm,0.5cm)}]() at (current page.south east){%
    \hyperlink{#1}{\beamergotobutton{#2}}};    
    \end{tikzpicture}
}    
% here button
\newcommand{\hbutton}[2]{
    \hyperlink{#1}{\beamergotobutton{#2}}
}
% Cite command: cc
\newcommand{\nn}[2][DarkGray]{{\small \color{#1} #2}}
\newcommand{\ccdoi}[2]{\href{https://doi.org/#1}{\nn{#2}}}
\newcommand{\ccurl}[2]{\href{#1}{\nn{#2}}}

% change slide width on a single slide
% source: https://tex.stackexchange.com/questions/160825/modifying-margins-for-one-slide/242073
\newcommand\Wider[2][3em]{%
\makebox[\linewidth][c]{%
  \begin{minipage}{\dimexpr\textwidth+#1\relax}
  \raggedright#2
  \end{minipage}%
  }%
}

% To remove sections from appendix
% source: https://tex.stackexchange.com/questions/37127/how-to-remove-some-pages-from-the-navigation-bullets-in-beamer
\makeatletter
\let\beamer@writeslidentry@miniframeson=\beamer@writeslidentry
\def\beamer@writeslidentry@miniframesoff{%
  \expandafter\beamer@ifempty\expandafter{\beamer@framestartpage}{}% does not happen normally
  {%else
    % removed \addtocontents commands
    \clearpage\beamer@notesactions%
  }
}
\newcommand*{\miniframeson}{\let\beamer@writeslidentry=\beamer@writeslidentry@miniframeson}
\newcommand*{\miniframesoff}{\let\beamer@writeslidentry=\beamer@writeslidentry@miniframesoff}
\makeatother

%%%%%%%%%%%%%%%%%%%%
% Figures and tikz %
%%%%%%%%%%%%%%%%%%%%

% \usepackage{pgfplots} % maybe for tikz axes?
\usepackage{graphicx} % Include figures
\usepackage{tikz} % Draw figures with tikz
% \usetikzlibrary{positioning} % to use right=of <node> syntax
% \usetikzlibrary{shapes} % for other node shapes; see http://www.texample.net/tikz/examples/node-shapes/
% \usetikzlibrary{shapes.multipart} % multiline nodes in tikz
% \usetikzlibrary{shapes.arrows} % arrow nodes in tikz
% \usetikzlibrary{shapes.misc} % miscellaneous nodes in tikz; for rounded rectangles
% \usetikzlibrary{shapes.geometric} % geometric nodes in tikz; fortriangles

% Presentation specific tikz
%\usetikzlibrary{tikzmark} % to use tikzmarkto add braces to itemize
% \usetikzlibrary{calc} % for widthof
% \tikzset{
%     above label/.style={
%         above = 5pt,
%         %font=\footnotesize,
%         text height = 1.5ex,
%         text depth = 1ex,
%     },
%     below label/.style={  
%         below=4pt,
%         %font=\footnotesize,
%         text height = 1.5ex,
%         text depth = 1ex
%     },
%     brace label/.style={
%         below = 4pt,
%         font=\footnotesize,
%         text height = 1.5ex,
%         text depth = 1ex
%     },
%     brace/.style={
%         decoration={brace, mirror},
%         decorate
%     }
% }

% For every picture that defines or uses external nodes, you'll have to
% apply the 'remember picture' style. To avoid some typing, we'll apply
% the style to all pictures.
%\tikzstyle{every picture}+=[remember picture]

% To avoid a warning
% \pgfplotsset{compat=1.7}

\usepackage[default]{cantarell}
%\usepackage{psfrag}
%\usepackage{longtable}
\usepackage{pdfpages}

% Avoid a warning on mac; can introduce a warning (and maybe a break) on windows
% \usepackage{etex}
% Avoid a warning
\let\Tiny=\tiny

\usepackage{bbm}

\usepackage{xparse} %NewDocumentCommand

% Notes to self: \nts[color option]{text that is a note}
\newif\ifshowcomments
% \showcommentstrue % uncomment to show
\ifshowcomments
\newcommand{\nts}[2][DarkGray]{\setbeamercolor{taras comment}{fg=#1}{\usebeamercolor[fg]{taras comment}\setbeamercolor{item}{fg = taras comment.fg}\emph{#2}}}
\else 
\newcommand{\nts}[2][DarkGray]{}
\fi

% Color highlights
\newcommand<>{\blue}[1]{\textbf{\color#2{MainBlue}#1}}
\newcommand<>{\mblue}[1]{{\color#2{MainBlue}#1}}
\newcommand<>{\green}[1]{\textbf{\color#2{Green}#1}}
\newcommand<>{\mgreen}[1]{{\color#2{Green}#1}}
\newcommand{\orange}[1]{\textbf{\color{Orange}#1}}
\newcommand{\gold}[1]{\textbf{\color{Gold}#1}}

\newcommand{\EE}[1]{\mathbb{E} \left[ #1 \right]}
% \NewDocumentCommand{\EE}{ O{} O{} m }{\mathbb{E}_{#1} #2 \left[ #3 \right]}
\newcommand{\llog}[1]{\log \left( #1 \right)}
\newcommand{\Cov}[1]{\text{Cov} \left( #1 \right)}

\newcommand{\br}[1]{\left\{ #1 \right\}}
\newcommand{\sbr}[1]{\left[ #1 \right]}
\newcommand{\pr}[1]{\left( #1 \right)}
\newcommand{\ce}[2]{\left[\left. #1 \right\vert #2 \right]}

% Beamer buttons
% bottom buttom: \bbutton{name_of_link}{Text to print}
\newcommand{\bbutton}[2]{
    \begin{tikzpicture}[remember picture, overlay]
    \node[shift={(-1.4cm,0.5cm)}]() at (current page.south east){%
    \hyperlink{#1}{\beamergotobutton{#2}}};    
    \end{tikzpicture}
}    
% here button
\newcommand{\hbutton}[2]{
    \hyperlink{#1}{\beamergotobutton{#2}}
}
% Cite command: cc
\newcommand{\nn}[2][DarkGray]{{\small \color{#1} #2}}
\newcommand{\ccdoi}[2]{\href{https://doi.org/#1}{\nn{#2}}}


%Information to be included in the title page:
\title{Group-based beliefs and human capital specialization}
\author[]{Tara Sullivan}
\institute{
Macro Lunch Presentation \\
Tara Sullivan \\
\textit{tasulliv@ucsd.edu}
}
\date{
\today
\nts{\\
\medskip
Notes/comments are in gray and will be excluded from the presentation.
}
}

\setbeamertemplate{caption}{\raggedright\insertcaption\par}

\usepackage{ifthen}
\usepackage{xifthen}% provides \isempty test

\usepackage{amsbsy}

% plot graphs with pgfplots
\usepackage{pgfplots}
\pgfplotsset{compat=newest}
\usepgfplotslibrary{groupplots}
\usepgfplotslibrary{dateplot}
\pgfplotsset{compat=newest,
    every axis/.style={
        axis y line*=left,
        axis x line*=bottom,
        % allows for multi-line legend entries
        legend style={cells={align=left}},
        % allows for multi-line titles
        title style={align=center},
    },
}

\makeatletter
\def\input@path{{../../img/}}
\makeatother

\usepackage{caption}
\captionsetup{%
    labelformat=empty,
   font=small,
   singlelinecheck=false,
   % tableposition=top
}

\usetikzlibrary{external}
\tikzexternalize[prefix=img/]
%%%%%%%%%%%%%%%%%%%%%%%%%%%%%%%%%%%%%%%%%%%%%%%%%%%%%%%%%%%%%%%%%%%%%%%%%%%%%%%%
\begin{document}    
%\beamertemplatenavigationsymbolsempty


%%%%%%%%%%%%%%%%%%%%%%%%%%%%%%%%%%%%%%%%%%%%%%%%%%%%%%%%%%%%%%%%%%%%%%%%%%%%%%%%
\begin{frame}
\titlepage
\end{frame}

% from johannes:

% identification: what moments are important for identification; justify importance of paper (instead of indfference graph)


%%%%%%%%%%%%%%%%%%%%%%%%%%%%%%%%%%%%%%%%%%%%%%%%%%%%%%%%%%%%%%%%%%%%%%%%%%%%%%%%
\section[Introduction]{Introduction}
%%%%%%%%%%%%%%%%%%%%%%%%%%%%%%%%%%%%%%%%%%%%%%%%%%%%%%%%%%%%%%%%%%%%%%%%%%%%%%%%

%%%%%%%%%%%%%%%%%%%%%%%%%%%%%%%%%%%%%%%%%%%%%%%%%%%%%%%%%%%%%%%%%%%%%%%%%%%%%%%%
\begin{frame}{Increased attainment of Bachelor's degrees} 

% Women are more educated; but what they are studying matters
% 450,000 men earned bachelor's degrees in 1970, compared to 340,000 women
% Highlight that we are discussing convergence in specific majors since the 1990s.}

\begin{figure}
% % This file was created by tikzplotlib v0.9.1.
\begin{tikzpicture}

\definecolor{color0}{rgb}{0.12156862745098,0.466666666666667,0.705882352941177}
\definecolor{color1}{rgb}{1,0.498039215686275,0.0549019607843137}

\begin{axis}[
legend cell align={left},
legend style={fill opacity=0.8, draw opacity=1, text opacity=1, at={(0.03,0.97)}, anchor=north west, draw=white!80!black},
tick align=outside,
tick pos=left,
title={Number of Bachelors Degrees awarded (millions)},
x grid style={white!69.0196078431373!black},
xlabel={year},
xmin=1988.6, xmax=2019.4,
xtick style={color=black},
y grid style={white!69.0196078431373!black},
ymin=0.4493241, ymax=1.2484059,
ytick style={color=black}
]
\addplot [semithick, color0]
table {%
1990 0.48564600944519
1991 0.490826010704041
1992 0.517989993095398
1993 0.532243013381958
1994 0.532928943634033
1995 0.528069019317627
1997 0.518990993499756
1998 0.522558927536011
1999 0.522891998291016
2000 0.533735036849976
2001 0.557978987693787
2002 0.579033017158508
2003 0.599171996116638
2004 0.629392027854919
2005 0.649704933166504
2006 0.66592800617218
2007 0.687217950820923
2008 0.703808069229126
2009 0.722702980041504
2010 0.750731945037842
2011 0.779560089111328
2012 0.814333915710449
2013 0.836575031280518
2014 0.850880980491638
2015 0.862040996551514
2016 0.871549010276794
2017 0.886856079101562
2018 0.897544026374817
};
\addlegendentry{Men}
\addplot [semithick, color1]
table {%
1990 0.555091023445129
1991 0.581097006797791
1992 0.614879012107849
1993 0.632375001907349
1994 0.638270020484924
1995 0.636955976486206
1996 0.644475936889648
1997 0.652374982833862
1998 0.666815042495728
1999 0.684229016304016
2000 0.708883047103882
2001 0.745826005935669
2002 0.779849052429199
2003 0.805441975593567
2004 0.849164962768555
2005 0.87644100189209
2006 0.905745983123779
2007 0.927672982215881
2008 0.947183012962341
2009 0.968661069869995
2010 1.00603902339935
2011 1.04807901382446
2012 1.09642803668976
2013 1.12388396263123
2015 1.15306401252747
2016 1.17018795013428
2017 1.19238698482513
2018 1.2120840549469
};
\addlegendentry{Women}
\end{axis}

\end{tikzpicture}

% This file was created by tikzplotlib v0.9.1.
\begin{tikzpicture}

\definecolor{color0}{rgb}{0.12156862745098,0.466666666666667,0.705882352941177}
\definecolor{color1}{rgb}{1,0.498039215686275,0.0549019607843137}

\begin{axis}[
legend cell align={left},
legend style={fill opacity=0.8, draw opacity=1, text opacity=1, at={(0.03,0.97)}, anchor=north west, draw=white!80!black},
tick align=outside,
tick pos=left,
title={Number of Bachelors Degrees awarded (millions)},
x grid style={white!69.0196078431373!black},
xlabel={year},
xmin=1988.6, xmax=2019.4,
xtick style={color=black},
y grid style={white!69.0196078431373!black},
ymin=0.4493241, ymax=1.2484059,
ytick style={color=black}
]
\addplot [semithick, color0]
table {%
1990 0.48564600944519
1991 0.490826010704041
1992 0.517989993095398
1993 0.532243013381958
1994 0.532928943634033
1995 0.528069019317627
1997 0.518990993499756
1998 0.522558927536011
1999 0.522891998291016
2000 0.533735036849976
2001 0.557978987693787
2002 0.579033017158508
2003 0.599171996116638
2004 0.629392027854919
2005 0.649704933166504
2006 0.66592800617218
2007 0.687217950820923
2008 0.703808069229126
2009 0.722702980041504
2010 0.750731945037842
2011 0.779560089111328
2012 0.814333915710449
2013 0.836575031280518
2014 0.850880980491638
2015 0.862040996551514
2016 0.871549010276794
2017 0.886856079101562
2018 0.897544026374817
};
\addlegendentry{Men}
\addplot [semithick, color1]
table {%
1990 0.555091023445129
1991 0.581097006797791
1992 0.614879012107849
1993 0.632375001907349
1994 0.638270020484924
1995 0.636955976486206
1996 0.644475936889648
1997 0.652374982833862
1998 0.666815042495728
1999 0.684229016304016
2000 0.708883047103882
2001 0.745826005935669
2002 0.779849052429199
2003 0.805441975593567
2004 0.849164962768555
2005 0.87644100189209
2006 0.905745983123779
2007 0.927672982215881
2008 0.947183012962341
2009 0.968661069869995
2010 1.00603902339935
2011 1.04807901382446
2012 1.09642803668976
2013 1.12388396263123
2015 1.15306401252747
2016 1.17018795013428
2017 1.19238698482513
2018 1.2120840549469
};
\addlegendentry{Women}
\end{axis}

\end{tikzpicture}

\caption{Source: IPEDS}
\end{figure}

\end{frame}

%%%%%%%%%%%%%%%%%%%%%%%%%%%%%%%%%%%%%%%%%%%%%%%%%%%%%%%%%%%%%%%%%%%%%%%%%%%%%%%%
\begin{frame}{Historically male-dominated fields}

% When we consider gender convergence across different fields

\begin{figure}
% % This file was created by tikzplotlib v0.9.1.
\begin{tikzpicture}

\definecolor{color0}{rgb}{0.12156862745098,0.466666666666667,0.705882352941177}
\definecolor{color1}{rgb}{1,0.498039215686275,0.0549019607843137}

\begin{axis}[
legend cell align={left},
legend style={fill opacity=0.8, draw opacity=1, text opacity=1, at={(0.03,0.97)}, anchor=north west, draw=white!80!black},
tick align=outside,
tick pos=left,
title={Number of Bachelors Degrees awarded (millions)},
x grid style={white!69.0196078431373!black},
xlabel={year},
xmin=1988.6, xmax=2019.4,
xtick style={color=black},
y grid style={white!69.0196078431373!black},
ymin=0.4493241, ymax=1.2484059,
ytick style={color=black}
]
\addplot [semithick, color0]
table {%
1990 0.48564600944519
1991 0.490826010704041
1992 0.517989993095398
1993 0.532243013381958
1994 0.532928943634033
1995 0.528069019317627
1997 0.518990993499756
1998 0.522558927536011
1999 0.522891998291016
2000 0.533735036849976
2001 0.557978987693787
2002 0.579033017158508
2003 0.599171996116638
2004 0.629392027854919
2005 0.649704933166504
2006 0.66592800617218
2007 0.687217950820923
2008 0.703808069229126
2009 0.722702980041504
2010 0.750731945037842
2011 0.779560089111328
2012 0.814333915710449
2013 0.836575031280518
2014 0.850880980491638
2015 0.862040996551514
2016 0.871549010276794
2017 0.886856079101562
2018 0.897544026374817
};
\addlegendentry{Men}
\addplot [semithick, color1]
table {%
1990 0.555091023445129
1991 0.581097006797791
1992 0.614879012107849
1993 0.632375001907349
1994 0.638270020484924
1995 0.636955976486206
1996 0.644475936889648
1997 0.652374982833862
1998 0.666815042495728
1999 0.684229016304016
2000 0.708883047103882
2001 0.745826005935669
2002 0.779849052429199
2003 0.805441975593567
2004 0.849164962768555
2005 0.87644100189209
2006 0.905745983123779
2007 0.927672982215881
2008 0.947183012962341
2009 0.968661069869995
2010 1.00603902339935
2011 1.04807901382446
2012 1.09642803668976
2013 1.12388396263123
2015 1.15306401252747
2016 1.17018795013428
2017 1.19238698482513
2018 1.2120840549469
};
\addlegendentry{Women}
\end{axis}

\end{tikzpicture}

% This file was created by tikzplotlib v0.9.1.
\begin{tikzpicture}

\definecolor{color0}{rgb}{0.266666666666667,0.466666666666667,0.666666666666667}
\definecolor{color1}{rgb}{0.933333333333333,0.4,0.466666666666667}
\definecolor{color2}{rgb}{0.133333333333333,0.533333333333333,0.2}
\definecolor{color3}{rgb}{0.8,0.733333333333333,0.266666666666667}
\definecolor{color4}{rgb}{0.4,0.8,0.933333333333333}
\definecolor{color5}{rgb}{0.666666666666667,0.2,0.466666666666667}

\begin{axis}[
height=207pt,
tick align=outside,
tick pos=left,
title={Ratio of women to men},
width=300pt,
x grid style={white!69.0196078431373!black},
xmin=1990, xmax=2030,
xtick style={color=black},
xtick={1990,1995,2000,2005,2010,2015},
xticklabels={\(\displaystyle 1990\),\(\displaystyle 1995\),\(\displaystyle 2000\),\(\displaystyle 2005\),\(\displaystyle 2010\),\(\displaystyle 2015\)},
ymajorgrids,
ymin=0.111868877532793, ymax=1.72181142557414,
ytick style={color=black}
]
\addplot [semithick, color0]
table {%
1990 0.187528491020203
1991 0.187889814376831
1992 0.18504810333252
1993 0.191165328025818
1994 0.198027372360229
1995 0.209525942802429
1996 0.219993829727173
1997 0.226323843002319
1998 0.230563402175903
1999 0.248593807220459
2000 0.259858369827271
2001 0.253314614295959
2002 0.266838073730469
2003 0.248422026634216
2004 0.258007287979126
2005 0.250089168548584
2006 0.243573904037476
2007 0.228035092353821
2008 0.227158188819885
2009 0.221451997756958
2011 0.231951236724854
2012 0.238436937332153
2013 0.240317106246948
2014 0.248236060142517
2015 0.251599311828613
2016 0.265729546546936
2017 0.274784326553345
2018 0.286244869232178
};
\addplot [semithick, color1]
table {%
1990 0.429780006408691
1991 0.419558525085449
1993 0.395862579345703
1994 0.4017493724823
1995 0.399999976158142
1996 0.381898045539856
1997 0.373727560043335
1998 0.368845701217651
1999 0.373373627662659
2000 0.390813827514648
2001 0.387295722961426
2002 0.384417414665222
2003 0.37157928943634
2004 0.335911631584167
2005 0.288139462471008
2006 0.261456251144409
2007 0.22909951210022
2008 0.216255187988281
2009 0.219069957733154
2010 0.223915338516235
2011 0.216599345207214
2012 0.225852251052856
2013 0.219969153404236
2014 0.223478555679321
2015 0.222921013832092
2016 0.235600471496582
2017 0.242461562156677
2018 0.257417798042297
};
\addplot [semithick, color2]
table {%
1990 0.460548996925354
1991 0.464974999427795
1992 0.4886314868927
1993 0.488829135894775
1994 0.512030601501465
1995 0.540966749191284
1996 0.565948724746704
1997 0.602615833282471
1998 0.629949331283569
1999 0.666009306907654
2000 0.686820149421692
2001 0.706353425979614
2002 0.736422896385193
2003 0.7142094373703
2004 0.72872519493103
2005 0.739134073257446
2006 0.725082635879517
2007 0.690871238708496
2008 0.687974691390991
2009 0.686181664466858
2010 0.688009262084961
2011 0.668954968452454
2012 0.67028284072876
2013 0.634251594543457
2014 0.645684003829956
2015 0.62436580657959
2016 0.629948854446411
2017 0.653665065765381
2018 0.67027759552002
};
\addplot [semithick, color3]
table {%
1990 0.843912363052368
1991 0.878591060638428
1992 0.892377018928528
1993 0.916429758071899
1994 0.941344499588013
1995 0.972469568252563
1996 1.01399052143097
1997 1.05340433120728
1998 1.07601177692413
1999 1.12696635723114
2000 1.16036009788513
2001 1.19738638401031
2002 1.19550442695618
2003 1.1779510974884
2004 1.14445388317108
2005 1.13158094882965
2006 1.10574865341187
2007 1.10396230220795
2008 1.07783782482147
2009 1.08280813694
2010 1.08092784881592
2011 1.07324647903442
2012 1.07712745666504
2013 1.0801340341568
2014 1.06191837787628
2015 1.04196679592133
2016 1.07565009593964
2017 1.0877937078476
2018 1.10102880001068
};
\addplot [semithick, color4]
table {%
1990 0.86025857925415
1991 0.899104356765747
1992 0.886856079101562
1993 0.900158047676086
1994 0.865336656570435
1995 0.88508152961731
1996 0.85411524772644
1997 0.864051580429077
1998 0.88809061050415
1999 0.93377161026001
2000 0.914776563644409
2001 0.881730198860168
2003 0.793915510177612
2004 0.801473140716553
2005 0.769950985908508
2006 0.776896238327026
2007 0.747289896011353
2008 0.764146089553833
2009 0.725896239280701
2010 0.73568868637085
2011 0.726636171340942
2012 0.726435899734497
2013 0.727647542953491
2014 0.723636150360107
2015 0.720515251159668
2016 0.709809184074402
2017 0.696071982383728
2018 0.709341764450073
};
\addplot [semithick, color5]
table {%
1990 0.889276146888733
1991 0.908068180084229
1992 0.908786058425903
1993 0.911298394203186
1995 0.937413215637207
1996 0.960581421852112
1997 0.962796211242676
1998 0.959450006484985
1999 0.985843420028687
2000 1.00777983665466
2001 1.00261080265045
2002 1.01744496822357
2003 1.02670252323151
2004 1.02088141441345
2005 1.00308656692505
2006 0.996885895729065
2007 0.972628951072693
2008 0.96298623085022
2009 0.959978342056274
2010 0.95364236831665
2011 0.953920364379883
2012 0.931203603744507
2013 0.921860694885254
2014 0.899493217468262
2015 0.900677680969238
2016 0.891088247299194
2017 0.886778354644775
2018 0.88598895072937
};
\addplot [semithick, white!73.3333333333333!black]
table {%
1990 1.04078304767609
1991 1.04256737232208
1992 1.07366561889648
1993 1.06894600391388
1994 1.05757308006287
1995 1.10440194606781
1996 1.11996006965637
1997 1.17428719997406
1998 1.23312425613403
1999 1.30454993247986
2000 1.40134906768799
2001 1.46844744682312
2002 1.54577028751373
2003 1.63139522075653
2004 1.64863216876984
2005 1.63168549537659
2006 1.602987408638
2007 1.51491057872772
2008 1.47168242931366
2009 1.46346211433411
2010 1.41208970546722
2011 1.44205784797668
2012 1.42999362945557
2013 1.42106962203979
2014 1.41722071170807
2015 1.44253933429718
2016 1.49894797801971
2017 1.56678104400635
2018 1.64683747291565
};
\draw (axis cs:2018.5,0.286244813278008) node[
  anchor=base west,
  text=color0,
  rotate=0.0
]{Engineering};
\draw (axis cs:2018.5,0.217417838961352) node[
  anchor=base west,
  text=color1,
  rotate=0.0
]{Computer services};
\draw (axis cs:2018.5,0.63027762382224) node[
  anchor=base west,
  text=color2,
  rotate=0.0
]{Physical sciences};
\draw (axis cs:2018.5,1.10102875591671) node[
  anchor=base west,
  text=color3,
  rotate=0.0
]{Social sciences};
\draw (axis cs:2018.5,0.709341764874964) node[
  anchor=base west,
  text=color4,
  rotate=0.0
]{Math and stats};
\draw (axis cs:2018.5,0.885989010989011) node[
  anchor=base west,
  text=color5,
  rotate=0.0
]{Business};
\draw (axis cs:2018.5,1.64683752645192) node[
  anchor=base west,
  text=white!73.3333333333333!black,
  rotate=0.0
]{Biological sciences};
\end{axis}

\end{tikzpicture}

\caption{Source: IPEDS}
\end{figure}
% In 1970, females were 30% of biology majors

\end{frame}

%%%%%%%%%%%%%%%%%%%%%%%%%%%%%%%%%%%%%%%%%%%%%%%%%%%%%%%%%%%%%%%%%%%%%%%%%%%%%%%%
\begin{frame}{Social Sciences}

\begin{figure}
% % This file was created by tikzplotlib v0.9.1.
\begin{tikzpicture}

\definecolor{color0}{rgb}{0.12156862745098,0.466666666666667,0.705882352941177}
\definecolor{color1}{rgb}{1,0.498039215686275,0.0549019607843137}

\begin{axis}[
legend cell align={left},
legend style={fill opacity=0.8, draw opacity=1, text opacity=1, at={(0.03,0.97)}, anchor=north west, draw=white!80!black},
tick align=outside,
tick pos=left,
title={Number of Bachelors Degrees awarded (millions)},
x grid style={white!69.0196078431373!black},
xlabel={year},
xmin=1988.6, xmax=2019.4,
xtick style={color=black},
y grid style={white!69.0196078431373!black},
ymin=0.4493241, ymax=1.2484059,
ytick style={color=black}
]
\addplot [semithick, color0]
table {%
1990 0.48564600944519
1991 0.490826010704041
1992 0.517989993095398
1993 0.532243013381958
1994 0.532928943634033
1995 0.528069019317627
1997 0.518990993499756
1998 0.522558927536011
1999 0.522891998291016
2000 0.533735036849976
2001 0.557978987693787
2002 0.579033017158508
2003 0.599171996116638
2004 0.629392027854919
2005 0.649704933166504
2006 0.66592800617218
2007 0.687217950820923
2008 0.703808069229126
2009 0.722702980041504
2010 0.750731945037842
2011 0.779560089111328
2012 0.814333915710449
2013 0.836575031280518
2014 0.850880980491638
2015 0.862040996551514
2016 0.871549010276794
2017 0.886856079101562
2018 0.897544026374817
};
\addlegendentry{Men}
\addplot [semithick, color1]
table {%
1990 0.555091023445129
1991 0.581097006797791
1992 0.614879012107849
1993 0.632375001907349
1994 0.638270020484924
1995 0.636955976486206
1996 0.644475936889648
1997 0.652374982833862
1998 0.666815042495728
1999 0.684229016304016
2000 0.708883047103882
2001 0.745826005935669
2002 0.779849052429199
2003 0.805441975593567
2004 0.849164962768555
2005 0.87644100189209
2006 0.905745983123779
2007 0.927672982215881
2008 0.947183012962341
2009 0.968661069869995
2010 1.00603902339935
2011 1.04807901382446
2012 1.09642803668976
2013 1.12388396263123
2015 1.15306401252747
2016 1.17018795013428
2017 1.19238698482513
2018 1.2120840549469
};
\addlegendentry{Women}
\end{axis}

\end{tikzpicture}

\input{cip42_rat.tex}
\caption{Source: IPEDS}
\end{figure}

\end{frame}

%%%%%%%%%%%%%%%%%%%%%%%%%%%%%%%%%%%%%%%%%%%%%%%%%%%%%%%%%%%%%%%%%%%%%%%%%%%%%%%%
\begin{frame}{Gender convergence across fields}

Why has gender convergence been different across fields?
\begin{itemize}
    \item Differences in preferences or abilities

    \item Misallocation of talent; may impact aggregate productivity growth
    \ccdoi{10.3982/ECTA11427}{(Hsieh, Hurst, Jones, and Klenow 2019)} 

    % \item College major choice matters for future income \ccdoi{10.1146/annurev-economics-080511-110908}{(Altonji, Blom, and Meghir 2012)}
\end{itemize}

% \vspace{2ex}
% Common explanations for why convergence might not occur:
% \begin{itemize}
%     \item Preferences
%     \item Abilities
%     \item Discrimination in labor market
%     \item <2-> Beliefs
% \end{itemize}

\pause
\vspace{2ex} 
\textbf{This paper:} attempts to answer this question using a model of human capital specialization:
\begin{itemize}
    \item Agents with unknown heterogeneous abilities
    \item Fields of study with heterogeneous wages
    \item Use courses to learn about underlying abilities
\end{itemize}
% present a model of human capital and learning ability
% beliefs result in different course taking characterisitcs

\end{frame}

%%%%%%%%%%%%%%%%%%%%%%%%%%%%%%%%%%%%%%%%%%%%%%%%%%%%%%%%%%%%%%%%%%%%%%%%%%%%%%%%
\begin{frame}{Limited Literature Review}


1. Human capital specialization
\begin{itemize}
    \item Build on model of gradual specialization from Alon and Fershtman (2019)
\end{itemize}

\vspace{4ex}
2. Gender gaps in college choice
\begin{itemize}
    \item Empirically motivated by Sloan, Hurst, and Black (2019)  % pre-labor market human capital specialization
%     % Use newly expanded ACS data that asks about major choice for college educated individuals
%     % Women choose college majors associated with lower potential wages than men
%     % Need to understand this better!!
% Gender gap in schooling and experience has shrunk
\end{itemize}

\vspace{4ex}
3. Beliefs and specialization decisions
\begin{itemize}
    \item Theoretically similar to Arcidiacono et al. (2015)
\end{itemize}

\vspace{3ex}




\end{frame}


%%%%%%%%%%%%%%%%%%%%%%%%%%%%%%%%%%%%%%%%%%%%%%%%%%%%%%%%%%%%%%%%%%%%%%%%%%%%%%%
\miniframesoff
\begin{frame}{Outline}
    \tableofcontents
\end{frame}
\miniframeson
%%%%%%%%%%%%%%%%%%%%%%%%%%%%%%%%%%%%%%%%%%%%%%%%%%%%%%%%%%%%%%%%%%%%%%%%%%%%%%%

%%%%%%%%%%%%%%%%%%%%%%%%%%%%%%%%%%%%%%%%%%%%%%%%%%%%%%%%%%%%%%%%%%%%%%%%%%%%%%%%
\section[Model]{Model}
%%%%%%%%%%%%%%%%%%%%%%%%%%%%%%%%%%%%%%%%%%%%%%%%%%%%%%%%%%%%%%%%%%%%%%%%%%%%%%%%

% begin with an Alon-Fershtman slide; review as quicly as possible ``because many people here are familar with this paper, I'll try to keep this short'' Maybe check with david that this is a good idea
% Then jump into my key contribution: how do we think about the priors

%%%%%%%%%%%%%%%%%%%%%%%%%%%%%%%%%%%%%%%%%%%%%%%%%%%%%%%%%%%%%%%%%%%%%%%%%%%%%%%%
%%%%%%%%%%%%%%%%%%%%%%%%%%%%%%%%%%%%%%%%%%%%%%%%%%%%%%%%%%%%%%%%%%%%%%%%%%%%%%%%
\begin{frame}{Model preliminaries}

% Discrete time
% Infinitely lived agents

% unknown success probability in j

% write that with an indicator function

% 

Individuals endowed with:
\begin{itemize}
    \item [$h_{j0}$:] Skill-$j$ specific human capital ($j=0,\dots,J$)
    \item [$\theta_j$:] Unknown probability of success in $j$
    \item [$P_{j0}$:] Prior beliefs about $\theta_j$
\end{itemize}

\vspace{2ex}
At each $t$, individuals can choose to either study or work in one skill-$j$:
\begin{itemize}
    \item Studying accumulates skill-$j$ human capital and reveals information about underlying probability of success in $j$
    % Endogenous enter labor market at time t as a skill j specialist to maximize 
    \item If you work, you receive wage $w_j$
% To keep things simple, I'll assume utility is linear, and that the value of entering the market in period t as a skill k specialist simply depends on lifetime earnings
\end{itemize}
% Time constraint:
% \begin{equation*}
%     \sum_{j=0}^J (s_{jt} + \ell_{jt}) = 1, \quad \quad s_{jt}, \ell_{jt} \in \{0, 1\}
% \end{equation*}

\vspace{2ex}
Enter labor market at time $t$ in skill-$j$ to maximize expected lifetime payoff:
\begin{equation*}
    \frac{\delta^t}{1 - \delta} U_j (w_{j}, h_{jt}) \ell_{jt}
    = \frac{\delta^t}{1 - \delta} w_{j} h_{jt} \ell_{jt}
    % Is it okay to say this? 
\end{equation*}
% \vspace{2ex}


\end{frame}

%%%%%%%%%%%%%%%%%%%%%%%%%%%%%%%%%%%%%%%%%%%%%%%%%%%%%%%%%%%%%%%%%%%%%%%%%%%%%%%%
% \begin{frame}{Student specialization decision}
\begin{frame}{Evolution of human capital accumulation and beliefs}

% Choose probability 

Students studying skill-$j$ at time $t$ pass the course with probability $\theta_j$:
\begin{equation*}
    a_{jt} \sim \text{Bernoulli} (\theta_j)
\end{equation*}
\vspace{-2.5ex}
\begin{itemize}
  \item 
  Accumulate human capital if they pass the course:
  \begin{align*}
  % h_{jt+1} =& H(h_{kt}, a_{kt}), \quad \quad
%   % a_{jt} \sim F_{\theta_j}
    h_{jt+1} = h_{jt} + \nu_{j} a_{jt}
  \end{align*}
  \item 
  Beliefs about $\theta_j$ evolve:
  \begin{equation*}
      \mgreen<2->{P_{j,t+1}} = \Pi_j (\mblue<2->{P_{jt}},a_{jt})
        % P_{j, t+1} 
  \end{equation*}
\end{itemize}

\pause
\vspace{5ex}
\textbf{Key:} How are  \blue<2->{priors} formed, and how are they \green{updated}? 

% Parameter we are trying to understand is probability of success
% Know that we want a prior distribution supported on [0, 1]

\end{frame}

%%%%%%%%%%%%%%%%%%%%%%%%%%%%%%%%%%%%%%%%%%%%%%%%%%%%%%%%%%%%%%%%%%%%%%%%%%%%%%%%
\begin{frame}[t]{Belief distribution}

Initial prior drawn from Beta distribution 
\begin{equation*}
    \mblue{P_{j0}} = \mathcal{B} (\alpha_{j0}, \beta_{j0})
\end{equation*}
% Beta distribution is appropriate 
% need a probability distribution supported on [0,1]
% Also has some desireable analytic properties
Update according to Bayes Rule $\implies$ posterior drawn from Beta distribution:
\begin{equation*}
    \mgreen{P_{j,t+1}} = \mathcal{B} (\alpha_{j,t+1}, \beta_{j,t+1}), \quad \quad 
    (\alpha_{j,t+1}, \beta_{j,t+1}) = 
    \begin{cases} 
        (\alpha_{jt} + 1, \beta_{jt}) &\text{ if } a_{jt} = 1 \\
        (\alpha_{jt}, \beta_{jt} + 1) &\text{ if } a_{jt} = 0
    \end{cases}
\end{equation*}

\vspace{3ex}
\begin{columns}[T] % align columns
\begin{column}{.51\textwidth}

\pause
\vspace{3ex}
Example: $\alpha_0 = 1$, $\beta_0 = 1$
\vspace{1.5ex}
\begin{itemize}
    \item <3-> success at $t=1$ $\implies$ $\alpha_1 = 2$, $\beta_1 = 1$

    \vspace{1.5ex}
    \item <4-> failure at $t=2$ $\implies$ $\alpha_1 = 2$, $\beta_1 = 2$
    
    \vspace{1.5ex}
    \item <5-> success at $t=3$ $\implies$ $\alpha_1 = 3$, $\beta_1 = 2$
\end{itemize}

\end{column}%
% \hfill%
\begin{column}{.39\textwidth}
\begin{figure}
\only<2>{\input{beta_example0.tex}}
\only<3>{\input{beta_example1.tex}}
\only<4>{\input{beta_example2.tex}}
\only<5>{\input{beta_example3.tex}}
\end{figure}
\end{column}%
\end{columns}
% \hypertarget<2>{beta_11_example}{\beamerbutton{I'm on the fourth slide}}
\hypertarget<2>{model_beta_11}{
  \hyperlink{simulate}{\beamerbutton{Return: simulation set-up}}
  \hyperlink{sim_default}{\beamerbutton{Return: baseline simulation}}
}
\hypertarget<4>{model_beta_22}{\hyperlink{sim_beliefs}{\beamerbutton{Return: simulation}}}
% \hyperlink{belief_effect}{\beamerbutton{Return: simulation}}


\end{frame}




%%%%%%%%%%%%%%%%%%%%%%%%%%%%%%%%%%%%%%%%%%%%%%%%%%%%%%%%%%%%%%%%%%%%%%%%%%%%%%%%
\begin{frame}{Group-based parametrization}



Consider group-based beliefs about abilities:
\begin{itemize}
    \item Each individual has a group type: $g \in \{m, f\}$

    \item Students form beliefs, $P_{j0}$, based on previously observed group successes
\end{itemize}

\vspace{2ex}
Simple parameterization:
\begin{itemize}
    \item [$\alpha_{j0}^g$: ] Number of type-$g$ students who have succeeded in $j$ 
    \item [$\beta_{j0}^g$: ] Number of type-$g$ students who have failed in $j$

    \item [$\implies$] Observed success rate:
    \begin{equation*}
    \mu_{j0}^g = 
      \frac{\alpha_{j0}^g}{\alpha_{j0}^g + \beta_{j0}^g}.
\end{equation*}
This average is based on a sample size of type $g$ students:
\begin{equation*}
    n_{j0}^g = \alpha_{j0}^g + \beta_{j0}^g
 \end{equation*}
\end{itemize}

\vspace{2ex}
Group-based prior beliefs about probability of success in skill-$j$ courses, $\theta_j$:
\begin{equation*}
    \mathcal{B} \pr{\alpha_{j0}^g, \beta_{j0}^g} \quad \implies \quad
    \mathcal{B} \pr{\mu_{j0}^g n_{j0}^g, (1 - \mu_{j0}^g) n_{j0}^g}
\end{equation*}



\end{frame}

%%%%%%%%%%%%%%%%%%%%%%%%%%%%%%%%%%%%%%%%%%%%%%%%%%%%%%%%%%%%%%%%%%%%%%%%%%%%%%%%
\begin{frame}{Group-based belief distribution}

% Suppose sample size of men is larger than that of women, but the observed success rate is the same for the two groups:
Suppose there are more men then women in field $j$: 
\begin{equation*}
  n_{j0}^m > n_{j0}^f
\end{equation*}
But the observed success rate is the same for the two groups:
\begin{equation*}
    \mu_{j0} = \mu_{j0}^m = \mu_{j0}^w
\end{equation*}

\begin{figure}
\input{beta_example_gender.tex}
\end{figure}


\end{frame}


%%%%%%%%%%%%%%%%%%%%%%%%%%%%%%%%%%%%%%%%%%%%%%%%%%%%%%%%%%%%%%%%%%%%%%%%%%%%%%%%

% \nts{Frame remaining simple parameterization}
\begin{frame}{Individual problem}


A policy $\pi: (h_t, P_t^g) \to (s_t, \ell_t)$ is optimal if it maximizes:
\begin{align*}
& \mathbb{E}^\pi \sbr{
   \sum_{t=0}^\infty \delta^t 
   \left. \pr{\sum_{j=1}^J h_{jt} w_j \ell_{jt} } \right\vert
   \pr{(h_{10}, P_{10}^g), \dots, (h_{10}, P_{J0}^g)}
} \\
\text{subject to} \quad& h_{jt+1} = h_{jt}+ a_{jt} s_{jt}, \quad \quad a_{jt} = 
   \begin{cases} 
      \nu_j, & \text{with prob. } \theta_j,  \\ 
      0, & \text{with prob. } 1 - \theta_j,
   \end{cases} 
   %\quad \quad h_{j0} = \alpha_{j0} \nu_j, 
   \\
\quad& P_{jt+1}^g = 
  \mathcal{B} (\alpha_{j,t+1}^g, \beta_{j,t+1}^g), 
  \quad \quad \theta_j \sim P_{j,0}^g \equiv \mathcal{B} (\alpha_{j0}^g, \beta_{j0}^g)
  \quad \quad \text{if $j$ selected,} \\
\quad& \sum_{j=1}^J (s_{jt} + \ell_{jt}) = 1, \quad \quad s_{jt}, \ell_{jt} \in \{0,1\} \\
   & h_{j0} \leq \nu \alpha_{j0}
\end{align*}

\end{frame}

%%%%%%%%%%%%%%%%%%%%%%%%%%%%%%%%%%%%%%%%%%%%%%%%%%%%%%%%%%%%%%%%%%%%%%%%%%%%%%%%
\begin{frame}{Optimal policy rule}

Define the skill $j$ index as the expected payoff if you committed to studying $j$:
%\gen{ 
\begin{equation*}
\mathcal{I}_j (h_j, P_j^g) = \sup_{\tau \geq 0} \mathbb{E}^\tau
\ce{
   \sum_{t=0}^\infty \delta^t \pr{h_{jt} w_j \ell_{jt} }}
   {(h_{j0}, P_{j0}^g) = (h_j, P_j^g)
}
% \mathcal{I}_{jt} (h_{jt}, \alpha_{jt}, \beta_{jt}) = 
% \begin{cases}
% \frac{h_{jt}}{1 - \delta} & \text{if } \{\alpha_{jt}, \beta_{jt}\} \in \mathcal{G}_{j}, \\
% \frac{h_{jt}}{1 - \delta} \sbr{
%    \frac{
%       \left\lceil \frac{\delta}{1 - \delta} \right\rceil
%       \delta^{\left\lceil \frac{\delta}{1 - \delta} \right\rceil - c_{jt} - \alpha_{j0} - \beta_{j0}}}
%    {c_{jt} + \alpha_{j0} + \beta_{j0}}
%    } & \text{if } \{\alpha_{jt}, \beta_{jt}\} \notin \mathcal{G}_{j} \\
% \end{cases}
\end{equation*}

Define the graduation region of skill $j$ as: 
\begin{equation*}
% \mathcal{G}_j =  \left\{ \alpha_{jt}, \beta_{jt} \left\vert c_{jt} \geq \left\lceil \frac{\delta}{1 - \delta} \right\rceil - (\alpha_{j0} + \beta_{j0}) \right. \right\}
\mathcal{G}_j = \left\{ (h_j, P_j^g) \left\vert
   \arg \max_{\tau \geq 0} 
   \mathbb{E}^\tau \ce{\sum_{t=0}^\infty \delta^t \pr{h_{jt} w_j \ell_{jt} }}
   {(h_j, P_j^g)} = 0
   \right. \right\}
\end{equation*}

The following policy $\pi: (h_t, P_t^g) \to (s_t, \ell_t)$ is optimal: 
\begin{enumerate}
    \item At each $t \geq 0$, choose skill $j^* = \arg \max_{j \in J} \mathcal{I}_j$, breaking ties according to any rule
    \item If $(h_{j^*}, P_{j^*}^g) \in \mathcal{G}_{j}$, then enter the labor market as a $j^*$ specialist. Otherwise, study $j^*$ for an additional period.  
\end{enumerate}

\end{frame}

%%%%%%%%%%%%%%%%%%%%%%%%%%%%%%%%%%%%%%%%%%%%%%%%%%%%%%%%%%%%%%%%%%%%%%%%%%%%%%%%

%%%%%%%%%%%%%%%%%%%%%%%%%%%%%%%%%%%%%%%%%%%%%%%%%%%%%%%%%%%%%%%%%%%%%%%%%%%%%%%




%%%%%%%%%%%%%%%%%%%%%%%%%%%%%%%%%%%%%%%%%%%%%%%%%%%%%%%%%%%%%%%%%%%%%%%%%%%%%%%
\miniframesoff
\section[Simulations]{Model Simulations}
\begin{frame}
    \tableofcontents[currentsection]
\end{frame}
\miniframeson

%%%%%%%%%%%%%%%%%%%%%%%%%%%%%%%%%%%%%%%%%%%%%%%%%%%%%%%%%%%%%%%%%%%%%%%%%%%%%%%%
\begin{frame}{Simulate agent behavior}\label{simulate}

% To consider how my model can explain human capital sepcialization decisions

How can the model explain different specialization outcomes?

\vspace{3ex}
Consider a world with two fields, X and Y
\begin{itemize}
    \item Wages are equal: $w_X = w_Y$
    \item The agent's probabilities of success are equal: $\theta_X = \theta_Y$
    \item Initial beliefs are equal to the uninformative prior: \hyperlink{model_beta_11}{\beamerbutton{PDF of beliefs}}
    \begin{equation*}
        (\alpha_{X0}, \beta_{X0}) = (\alpha_{Y0}, \beta_{Y0}) = (1, 1)
    \end{equation*}
\end{itemize}

\vspace{3ex}
Simulate agent's specialization decisions when choosing between X and Y
\begin{itemize}
    \item Model fraction of simulated agents choosing X or Y at time $t$ 
    \item Assume $h_{j0} = \nu \alpha_{j0}$ \hyperlink{sim_parameterization}{\beamerbutton{Details}}
\end{itemize}


\end{frame}

%%%%%%%%%%%%%%%%%%%%%%%%%%%%%%%%%%%%%%%%%%%%%%%%%%%%%%%%%%%%%%%%%%%%%%%%%%%%%%%%
\begin{frame}{Default parameterization}\label{sim_default}

% 
\begin{figure}
\centering
% This file was created by tikzplotlib v0.9.2.
\begin{tikzpicture}

\definecolor{color0}{rgb}{0.266666666666667,0.466666666666667,0.666666666666667}
\definecolor{color1}{rgb}{0.933333333333333,0.4,0.466666666666667}

\begin{axis}[
height=6.6314113761540705cm,
tick align=outside,
tick pos=left,
title={Baseline Simulation},
width=9.904475999999999cm,
x grid style={white!69.0196078431373!black},
xlabel={t},
xmin=-1.05, xmax=27,
xtick style={color=black},
xtick={0,5,10,15,20},
xticklabels={\(\displaystyle 0\),\(\displaystyle 5\),\(\displaystyle 10\),\(\displaystyle 15\),\(\displaystyle 20\)},
ylabel={Fraction students enrolled in field},
ymajorgrids,
ymin=0, ymax=1,
ytick style={color=black},
ytick={0,0.25,0.5,0.75,1},
yticklabels={\(\displaystyle 0\),\(\displaystyle 0.25\),\(\displaystyle 0.5\),\(\displaystyle 0.75\),\(\displaystyle 1\)}
]
\addplot [semithick, color0]
table {%
0 0.505500078201294
1 0.503099918365479
2 0.502699971199036
3 0.498600006103516
4 0.502599954605103
5 0.502599954605103
6 0.503499984741211
7 0.506099939346313
8 0.505100011825562
9 0.505399942398071
10 0.505300045013428
11 0.506900072097778
12 0.506399989128113
13 0.506500005722046
14 0.506999969482422
15 0.506700038909912
16 0.506799936294556
17 0.506500005722046
20 0.506500005722046
21 0.506500005722046
};
\addplot [semithick, color1]
table {%
0 0.494500041007996
1 0.496899962425232
2 0.497300028800964
3 0.501399993896484
4 0.497400045394897
5 0.497400045394897
6 0.496500015258789
7 0.493900060653687
8 0.494899988174438
9 0.494600057601929
10 0.494699954986572
11 0.493099927902222
12 0.493600010871887
13 0.493499994277954
14 0.493000030517578
15 0.493299961090088
16 0.493200063705444
17 0.493499994277954
20 0.493499994277954
21 0.493499994277954
};
\draw (axis cs:21.5,0.4265) node[
  anchor=base west,
  text=color0,
  rotate=0.0
]{Field X};
\draw (axis cs:21.5,0.5235) node[
  anchor=base west,
  text=color1,
  rotate=0.0
]{Field Y};
\draw (axis cs:17,0.03) node[
  anchor=base west,
  text=black,
  rotate=0.0
]{N simulations = 10,000};
\end{axis}

\end{tikzpicture}

\end{figure}

\hyperlink{model_beta_11}{\beamerbutton{PDF of beliefs}}
\end{frame}

%%%%%%%%%%%%%%%%%%%%%%%%%%%%%%%%%%%%%%%%%%%%%%%%%%%%%%%%%%%%%%%%%%%%%%%%%%%%%%%%
\begin{frame}{Zooming in}

% should be symmetric

\begin{figure}
\centering
% This file was created by tikzplotlib v0.9.2.
\begin{tikzpicture}

\definecolor{color0}{rgb}{0.266666666666667,0.466666666666667,0.666666666666667}
\definecolor{color1}{rgb}{0.933333333333333,0.4,0.466666666666667}

\begin{axis}[
height=6.6314113761540705cm,
tick align=outside,
tick pos=left,
title={Baseline Simulation (zoomed in)},
width=9.904475999999999cm,
x grid style={white!69.0196078431373!black},
xlabel={t},
xmin=-1.05, xmax=27,
xtick style={color=black},
xtick={0,5,10,15,20},
xticklabels={\(\displaystyle 0\),\(\displaystyle 5\),\(\displaystyle 10\),\(\displaystyle 15\),\(\displaystyle 20\)},
ylabel={Fraction students enrolled in field},
ymajorgrids,
ymin=0, ymax=1,
ytick style={color=black},
ytick={0,0.25,0.5,0.75,1},
yticklabels={\(\displaystyle 0\),\(\displaystyle 0.25\),\(\displaystyle 0.5\),\(\displaystyle 0.75\),\(\displaystyle 1\)}
]
\addplot [semithick, color0]
table {%
0 0.5
1 0.319999933242798
2 0.360000014305115
3 0.379999995231628
4 0.360000014305115
5 0.379999995231628
6 0.360000014305115
7 0.379999995231628
8 0.360000014305115
9 0.379999995231628
13 0.379999995231628
14 0.399999976158142
15 0.379999995231628
21 0.379999995231628
};
\addplot [semithick, color1]
table {%
0 0.5
1 0.680000066757202
2 0.639999985694885
3 0.620000004768372
4 0.639999985694885
5 0.620000004768372
6 0.639999985694885
7 0.620000004768372
8 0.639999985694885
9 0.620000004768372
13 0.620000004768372
14 0.600000023841858
15 0.620000004768372
21 0.620000004768372
};
\draw (axis cs:21.5,0.3) node[
  anchor=base west,
  text=color0,
  rotate=0.0
]{Field X};
\draw (axis cs:21.5,0.65) node[
  anchor=base west,
  text=color1,
  rotate=0.0
]{Field Y};
\draw (axis cs:18,0.03) node[
  anchor=base west,
  text=black,
  rotate=0.0
]{N simulations = 50};
\end{axis}

\end{tikzpicture}

\end{figure}


\end{frame}

%%%%%%%%%%%%%%%%%%%%%%%%%%%%%%%%%%%%%%%%%%%%%%%%%%%%%%%%%%%%%%%%%%%%%%%%%%%%%%%%
\begin{frame}{Wage effects}

% 

\begin{figure}
\centering
% This file was created by tikzplotlib v0.9.1.
\begin{tikzpicture}

\definecolor{color0}{rgb}{0.266666666666667,0.466666666666667,0.666666666666667}
\definecolor{color1}{rgb}{0.933333333333333,0.4,0.466666666666667}

\begin{axis}[
height=207pt,
tick align=outside,
tick pos=left,
title={Field selection and wages \\ Field X: w = 1.0; Field Y: w = 1.5},
width=240pt,
x grid style={white!69.0196078431373!black},
xlabel={\(\displaystyle t\)},
xmin=-1.05, xmax=25,
xtick style={color=black},
xtick={0,5,10,15,20},
xticklabels={\(\displaystyle 0\),\(\displaystyle 5\),\(\displaystyle 10\),\(\displaystyle 15\),\(\displaystyle 20\)},
ylabel={Fraction students enrolled in field},
ymajorgrids,
ymin=0, ymax=1,
ytick style={color=black},
ytick={0,0.2,0.4,0.6,0.8,1},
yticklabels={\(\displaystyle 0\),\(\displaystyle 0.2\),\(\displaystyle 0.4\),\(\displaystyle 0.6\),\(\displaystyle 0.8\),\(\displaystyle 1\)}
]
\addplot [semithick, color0, mark=x, mark size=3, mark options={solid}]
table {%
0 0
1 0
2 0.28
3 0.16
4 0.22
5 0.18
6 0.19
7 0.19
8 0.19
9 0.17
10 0.18
11 0.16
12 0.16
13 0.16
14 0.17
15 0.17
16 0.17
17 0.17
18 0.17
19 0.17
20 0.17
21 0.17
};
\addplot [semithick, color1, mark=x, mark size=3, mark options={solid}]
table {%
0 1
1 1
2 0.72
3 0.84
4 0.78
5 0.82
6 0.81
7 0.81
8 0.81
9 0.83
10 0.82
11 0.84
12 0.84
13 0.84
14 0.83
15 0.83
16 0.83
17 0.83
18 0.83
19 0.83
20 0.83
21 0.83
};
\draw (axis cs:20.5,0.2) node[
  anchor=base west,
  text=color0,
  rotate=0.0
]{Field X};
\draw (axis cs:20.5,0.86) node[
  anchor=base west,
  text=color1,
  rotate=0.0
]{Field Y};
\draw (axis cs:14,0.03) node[
  anchor=base west,
  text=black,
  rotate=0.0
]{N simulations = 100};
\end{axis}

\end{tikzpicture}

\end{figure}

\end{frame}

%%%%%%%%%%%%%%%%%%%%%%%%%%%%%%%%%%%%%%%%%%%%%%%%%%%%%%%%%%%%%%%%%%%%%%%%%%%%%%%%
\begin{frame}{Belief effects}\label{sim_beliefs}

% risk aspect is on human capital accumulation
% your expected probability of gaining human capital is the same in both of these
% there's risk in timing - risk aversion over time; probably from discounting with linear utility function

%%%%%%%%%%%%
% Start going into y because you have more information
% more quickly will converge into what you go at 

% If you'er going to switch, you know sooner

% In expectation 
% What drives length to degree?
% AFter two periods you have sunk benefit


% Inforamtion value outweight human capital
% risk neutral is human capital 

% spend more in uncertain major 

% benefit 

% why do you spend less time studying when you  have more certainty

% if distribution is tighter, initial 

% any signal makes you more likely to switch if you start in more dispersed prior
% swtiching is bad; want to minimize time you spend in school
% go with option for which you 

% c_j depends on alpha0 and beta0

% poteriro vs posterior predictive

% Y provides more 
% you want to accumulate the most human capital in the shortest amount of time
% want to choose the field to get you to 10 suc

\begin{figure}
\centering
% This file was created by tikzplotlib v0.9.2.
\begin{tikzpicture}

\definecolor{color0}{rgb}{0.266666666666667,0.466666666666667,0.666666666666667}
\definecolor{color1}{rgb}{0.933333333333333,0.4,0.466666666666667}

\begin{axis}[
height=5.101085673964669cm,
tick align=outside,
tick pos=left,
title={Field selection and initial beliefs \\ Field X: \(\displaystyle (\alpha_0, \beta_0)=(1, 1)\); Field Y: \(\displaystyle (\alpha_0, \beta_0)=(2, 2)\)},
width=8.25373cm,
x grid style={white!69.0196078431373!black},
xmin=-0.95, xmax=23.75,
xtick style={color=black},
xtick={0,5,10,15},
xticklabels={\(\displaystyle 0\),\(\displaystyle 5\),\(\displaystyle 10\),\(\displaystyle 15\)},
ymajorgrids,
ymin=-0.05, ymax=1.05,
ytick style={color=black},
ytick={0,0.2,0.4,0.6,0.8,1},
yticklabels={\(\displaystyle 0\),\(\displaystyle 0.2\),\(\displaystyle 0.4\),\(\displaystyle 0.6\),\(\displaystyle 0.8\),\(\displaystyle 1\)}
]
\addplot [semithick, color0]
table {%
0 0
1 0.483
2 0.25
3 0.25
4 0.281
5 0.266
6 0.249
7 0.236
8 0.233
9 0.247
10 0.241
11 0.24
12 0.238
13 0.237
14 0.237
15 0.237
16 0.236
17 0.237
18 0.237
19 0.237
};
\addplot [semithick, color1]
table {%
0 1
1 0.517
2 0.75
3 0.75
4 0.719
5 0.734
6 0.751
7 0.764
8 0.767
9 0.753
10 0.759
11 0.76
12 0.762
13 0.763
14 0.763
15 0.763
16 0.764
17 0.763
18 0.763
19 0.763
};
\draw (axis cs:19.5,0.237) node[
  anchor=base west,
  text=color0,
  rotate=0.0
]{Field X};
\draw (axis cs:19.5,0.763) node[
  anchor=base west,
  text=color1,
  rotate=0.0
]{Field Y};
\end{axis}

\end{tikzpicture}

\end{figure}
\hyperlink{model_beta_22}{\beamerbutton{PDF of beliefs}}
\hyperlink{sim_parameterization}{\beamerbutton{Parameterization}}
\hyperlink{app_ability_v_effect}{\beamerbutton{
    Let $\alpha_{X0} \nu_X = \alpha_{Y0} \nu_Y$
}}

\end{frame}

%%%%%%%%%%%%%%%%%%%%%%%%%%%%%%%%%%%%%%%%%%%%%%%%%%%%%%%%%%%%%%%%%%%%%%%%%%%%%%%%
\begin{frame}{Ability to succeed}\label{sim_ability}

\begin{figure}
\centering
% This file was created by tikzplotlib v0.9.2.
\begin{tikzpicture}

\definecolor{color0}{rgb}{0.266666666666667,0.466666666666667,0.666666666666667}
\definecolor{color1}{rgb}{0.933333333333333,0.4,0.466666666666667}

\begin{axis}[
height=5.101085673964669cm,
tick align=outside,
tick pos=left,
title={Field selection and ability to succeed \\ Field X: \(\displaystyle \theta=0.25\); Field Y: \(\displaystyle \theta=0.75\)},
width=8.25373cm,
x grid style={white!69.0196078431373!black},
xmin=-1.05, xmax=26.25,
xtick style={color=black},
xtick={0,5,10,15,20},
xticklabels={\(\displaystyle 0\),\(\displaystyle 5\),\(\displaystyle 10\),\(\displaystyle 15\),\(\displaystyle 20\)},
ymajorgrids,
ymin=0, ymax=1,
ytick style={color=black},
ytick={0,0.2,0.4,0.6,0.8,1},
yticklabels={\(\displaystyle 0\),\(\displaystyle 0.2\),\(\displaystyle 0.4\),\(\displaystyle 0.6\),\(\displaystyle 0.8\),\(\displaystyle 1\)}
]
\addplot [semithick, color0]
table {%
0 0.484
1 0.23
2 0.228
3 0.151
4 0.138
5 0.134
6 0.118
7 0.104
8 0.105
9 0.109
10 0.105
11 0.097
12 0.1
13 0.1
14 0.096
15 0.096
16 0.096
17 0.095
18 0.096
19 0.096
20 0.096
21 0.096
};
\addplot [semithick, color1]
table {%
0 0.516
1 0.77
2 0.772
3 0.849
4 0.862
5 0.866
6 0.882
7 0.896
8 0.895
9 0.891
10 0.895
11 0.903
12 0.9
13 0.9
14 0.904
15 0.904
16 0.904
17 0.905
18 0.904
19 0.904
20 0.904
21 0.904
};
\draw (axis cs:21.5,0.126) node[
  anchor=base west,
  text=color0,
  rotate=0.0
]{Field X};
\draw (axis cs:21.5,0.934) node[
  anchor=base west,
  text=color1,
  rotate=0.0
]{Field Y};
\end{axis}

\end{tikzpicture}

\end{figure}
\hyperlink{app_v_effects}{\beamerbutton{$\nu$ simulation}}

\end{frame}

%%%%%%%%%%%%%%%%%%%%%%%%%%%%%%%%%%%%%%%%%%%%%%%%%%%%%%%%%%%%%%%%%%%%%%%%%%%%%%%
\miniframesoff
%\section[Measurement]{Literature Review: measuring adaptation}
\section[Next steps]{Next Steps}
\begin{frame}
    \tableofcontents[currentsection]
\end{frame}
\miniframeson

%%%%%%%%%%%%%%%%%%%%%%%%%%%%%%%%%%%%%%%%%%%%%%%%%%%%%%%%%%%%%%%%%%%%%%%%%%%%%%%%
\begin{frame}{Identification}

\nts{This is just a sketch of what this slide will look like}

Identification:
\begin{itemize}
    \item $w$: Average expected lifetime wages by gender and field (ACS data)
    % \item $\theta:$ Underlying probability of success: I will make an assumption to make this simplify, but not sure if I want to:
    % \begin{itemize}
    %     \item Assume equal
    %     \item Assume matches aggregate beliefs (this is what I'm leaning towards)
    %     \item Assume matches group beliefs
    % \end{itemize} 
    \item $\nu$: Mincerian returns to education (ACS data) 
\end{itemize}
Initial beliefs:  
\begin{itemize}
    \item NCES Transcript study to estimate $\mu$
    \item IPEDS data for $\alpha_{j0}$
\end{itemize}

\vspace{2ex}
Note: assuming genders have the same ability and same success rates reduces the parameters I need to estimate
% Initial human capital....?

\vspace{4ex}
\pause
Important to consider how discrimination can impact model through:
\begin{itemize}
    \item Wages
    \item Probability of success
\end{itemize}
In general, worth considering how this model connects to statistical discrimination

\end{frame}



% %%%%%%%%%%%%%%%%%%%%%%%%%%%%%%%%%%%%%%%%%%%%%%%%%%%%%%%%%%%%%%%%%%%%%%%%%%%%%%%%
% \begin{frame}{Calibration}

% American Community Survey
% \begin{itemize}
%     \item Mincerian returns to education

%     \item 
% \end{itemize}

% HEGIS/IPEDs

% \end{frame}


%%%%%%%%%%%%%%%%%%%%%%%%%%%%%%%%%%%%%%%%%%%%%%%%%%%%%%%%%%%%%%%%%%%%%%%%%%%%%%%%
% \begin{frame}{}

% \nts{Placeholder: will be expanding on this}

% Calibrate model 

% \vspace{3ex}
% Expand upon analytical results
% \begin{itemize}
%     \item I assume initial beliefs are distributed as:
%     \begin{equation*}
%     \mathcal{B} \pr{\mu_{j0} n_{j0}^g, (1 - \mu_{j0}) n_{j0}^g} 
% \end{equation*}
%     Assuming men and women share a common beliefs about mean abilities ($\mu_{j0}$) allows for some interesting analytical results
% \end{itemize}

% \end{frame}

% %%%%%%%%%%%%%%%%%%%%%%%%%%%%%%%%%%%%%%%%%%%%%%%%%%%%%%%%%%%%%%%%%%%%%%%%%%%%%%%%
% \begin{frame}{Further extensions}

% \nts{Placeholder: will be expanding on this}

% \vspace{3ex}
% Efficiency losses from talent mis-match

% \vspace{3ex}
% Adjsuted gender wage gap

% \vspace{3ex}
% Connection to statistical discrimination

% \vspace{3ex}
% Affirmative action

% % unexplained part of gap might understate discrimination

% % think about this in terms of oaxaca decomposition

% \end{frame}

% When we look at gender wage gap, we see that 
% If we assumet this is true, selection into certain occupations is not exogenous; can possibly create adjusted gender-wage gap






%%%%%%%%%%%%%%%%%%%%%%%%%%%%%%%%%%%%%%%%%%%%%%%%%%%%%%%%%%%%%%%%%%%%%%%%%%%%%%%%
%\begin{frame}{}
%
%\begin{figure}
%\centering
%\includegraphics[width=\textwidth]{img/fig5.pdf}
%\end{figure}
%
%\end{frame}



%%%%%%%%%%%%%%%%%%%%%%%%%%%%%%%%%%%%%%%%%%%%%%%%%%%%%%%%%%%%%%%%%%%%%%%%%%%%%%%%
% \section[Results]{Results}
%%%%%%%%%%%%%%%%%%%%%%%%%%%%%%%%%%%%%%%%%%%%%%%%%%%%%%%%%%%%%%%%%%%%%%%%%%%%%%%%





%%%%%%%%%%%%%%%%%%%%%%%%%%%%%%%%%%%%%%%%%%%%%%%%%%%%%%%%%%%%%%%%%%%%%%%%%%%%%%%%
% Appendix
%%%%%%%%%%%%%%%%%%%%%%%%%%%%%%%%%%%%%%%%%%%%%%%%%%%%%%%%%%%%%%%%%%%%%%%%%%%%%%%%
\miniframesoff
\begin{frame}[noframenumbering]
\begin{beamercolorbox}[sep=11pt,center]{title}
Appendix
\end{beamercolorbox}
\end{frame}
%%%%%%%%%%%%%%%%%%%%%%%%%%%%%%%%%%%%%%%%%%%%%%%%%%%%%%%%%%%%%%%%%%%%%%%%%%%%%%%

% %%%%%%%%%%%%%%%%%%%%%%%%%%%%%%%%%%%%%%%%%%%%%%%%%%%%%%%%%%%%%%%%%%%%%%%%%%%%%%%%
% \begin{frame}{Parametric example}\label{app_parameterization}

% Assuming $h_{j0} = \nu_j \alpha_{j0}$ and letting $c_{jt}$ be the number of periods and individual spends studying $j$, the stopping condition in this problem simplifies to:
% \begin{equation*}
%     \frac{1- \delta}{\delta} \geq \frac{1}{c_{jt} + \alpha_{j0} + \beta_{j0}} \implies 
%     c_j^* = \left\lceil \frac{\delta}{1 - \delta} \right\rceil - (\alpha_{j0} + \beta_{j0})
% \end{equation*}
% This yields the index function:
% \begin{align*}
% \mathcal{I}_{jt} (h_{jt}, \alpha_{jt}, \beta_{jt}) = 
% \begin{cases}
% \frac{w_{jt} h_{jt}}{1 - \delta} & \text{if } \{\alpha_{jt}, \beta_{jt}\} \in \mathcal{G}_{j}, \\
% \frac{w_{jt} h_{jt}}{1 - \delta} \sbr{
%    \frac{
%       \left\lceil \frac{\delta}{1 - \delta} \right\rceil
%       \delta^{\left\lceil \frac{\delta}{1 - \delta} \right\rceil - c_{jt} - \alpha_{j0} - \beta_{j0}}}
%    {c_{jt} + \alpha_{j0} + \beta_{j0}}
%    } & \text{if } \{\alpha_{jt}, \beta_{jt}\} \notin \mathcal{G}_{j} \\
% \end{cases}
% \end{align*}
% And graduation region:
% \begin{equation*}
%    \mathcal{G}_j =  \left\{ \alpha_{jt}, \beta_{jt} \left\vert c_{jt} \geq \left\lceil \frac{\delta}{1 - \delta} \right\rceil - (\alpha_{j0} + \beta_{j0}) \right. \right\}
% \end{equation*}

% % \hyperlink{simulate}{\beamergotobutton{Return: Simulation}} 
% \bbutton{simulate}{Return: Simulation}

% \end{frame}

%%%%%%%%%%%%%%%%%%%%%%%%%%%%%%%%%%%%%%%%%%%%%%%%%%%%%%%%%%%%%%%%%%%%%%%%%%%%%%%%
\begin{frame}{Parametric example}\label{sim_parameterization}

% Monotonicity of stopping problem (ensured by $h_{j0} \leq \nu \alpha_{j0}$) 
% \begin{itemize}
%     \item [$\implies$] Optimality of one-step-look-ahead, i.e. comparing
%     \begin{enumerate}
%         \item Value of entering labor market as skill $j$ specialist today, to
%         \item Value of entering labor market as a skill $j$ specialist tomorrow
%     \end{enumerate}
% \end{itemize}
Assuming $h_{j0} = \nu \alpha_{j0}$ and letting $c_{jt}$ be time spent studying $j$:
\begin{itemize}
    \item [$\implies$] Deterministic stopping function
\end{itemize}
\begin{equation*}
    \frac{1- \delta}{\delta} \geq \frac{1}{c_{jt} + \alpha_{j0} + \beta_{j0}} \implies 
    c_j^* = \left\lceil \frac{\delta}{1 - \delta} \right\rceil - (\alpha_{j0} + \beta_{j0})
\end{equation*}
Graduation regions given by:
\begin{equation*}
    \mathcal{G}_j (\alpha_{jt}, \beta_{jt}) = \left\{ 
        \alpha_{jt}, \beta_{jt} 
        \left\vert \frac{\delta}{1 - \delta} \leq \alpha_{jt} + \beta_{jt}
    \right.\right\}
\end{equation*}

In this example, note that $\mathcal{G}_Y = \mathcal{G}_X$. Index in the graduation region given by $\frac{h_{jt}}{1 - \delta}$. Index when not in graduation region given by Binomial distribution with parameters $\left(c_j^* - c_j, \frac{h_{jt}}{\nu(c_{jt} + \alpha_{j0} + \beta{j0})}\right)$:
    % \mathcal{I}_j (\alpha_{jt}, \beta_{jt}) = 
    % \begin{cases}
    %     \frac{w_{jt}}{1 - \delta} h_{jt} &\text{ if } 
    %         \{\alpha_{jt}, \beta_{jt}\} \in \mathcal{G}_j \\ 
    %     \frac{w_{jt}}{1 - \delta} 
    %         \left(h_{jt} + \nu \mathbb{E}[\theta_j \vert \alpha_{jt}, \beta_{jt}]\right) &\text{ if } \{\alpha_{jt}, \beta_{jt}\} \notin \mathcal{G}_j
    % \end{cases}
\begin{align*}
\mathcal{I}_{jt} (h_{jt}, \alpha_{jt}, \beta_{jt}) = 
\begin{cases}
\frac{w_{jt} h_{jt}}{1 - \delta} & \text{if } \{\alpha_{jt}, \beta_{jt}\} \in \mathcal{G}_{j}, \\
\frac{w_{jt} h_{jt}}{1 - \delta} \sbr{
   \frac{
      \left\lceil \frac{\delta}{1 - \delta} \right\rceil
      \delta^{\left\lceil \frac{\delta}{1 - \delta} \right\rceil - c_{jt} - \alpha_{j0} - \beta_{j0}}}
   {c_{jt} + \alpha_{j0} + \beta_{j0}}
   } & \text{if } \{\alpha_{jt}, \beta_{jt}\} \notin \mathcal{G}_{j} \\
\end{cases}
\end{align*}

\hyperlink{simulate}{\beamergotobutton{Return: simulation set-up}}
\hyperlink{sim_beliefs}{\beamerbutton{Return: belief simulation}}

\end{frame}

%%%%%%%%%%%%%%%%%%%%%%%%%%%%%%%%%%%%%%%%%%%%%%%%%%%%%%%%%%%%%%%%%%%%%%%%%%%%%%%%
\begin{frame}{}\label{app_ability_v_effect}

If 
$\nu_X = 
\frac{\alpha_{X0} + \beta_{X0}}{\alpha_{Y0} + \beta_{Y0}} 
\cdot \frac{\alpha_{Y0}}{\alpha_{X0}}
\cdot \delta^{\alpha_{X0} + \beta_{X0} - \alpha_{Y0} - \beta_{Y0}}$, then:
\begin{itemize}
     \item $h_{X0} = h_{Y0}$, and
     \item Agents randomly choose between fields $X$ and $Y$ at $t=0$
 \end{itemize}  
\begin{figure}
\centering
% This file was created by tikzplotlib v0.9.2.
\begin{tikzpicture}

\definecolor{color0}{rgb}{0.266666666666667,0.466666666666667,0.666666666666667}
\definecolor{color1}{rgb}{0.933333333333333,0.4,0.466666666666667}

\begin{axis}[
height=6.6314113761540705cm,
tick align=outside,
tick pos=left,
title={Field X: \(\displaystyle \nu=1.09\); Field Y: \(\displaystyle \nu=1.0\)},
width=9.904475999999999cm,
x grid style={white!69.0196078431373!black},
xlabel={t},
xmin=-0.95, xmax=27,
xtick style={color=black},
xtick={0,5,10,15},
xticklabels={\(\displaystyle 0\),\(\displaystyle 5\),\(\displaystyle 10\),\(\displaystyle 15\)},
ylabel={Fraction students enrolled in field},
ymajorgrids,
ymin=0, ymax=1,
ytick style={color=black},
ytick={0,0.2,0.4,0.6,0.8,1},
yticklabels={\(\displaystyle 0\),\(\displaystyle 0.2\),\(\displaystyle 0.4\),\(\displaystyle 0.6\),\(\displaystyle 0.8\),\(\displaystyle 1\)}
]
\addplot [semithick, color0]
table {%
0 0.531000018119812
1 0.51800000667572
2 0.388999938964844
3 0.371999979019165
4 0.434000015258789
5 0.375999927520752
6 0.363999962806702
7 0.378999948501587
8 0.375999927520752
9 0.370000004768372
10 0.371999979019165
11 0.371000051498413
12 0.373000025749207
14 0.373000025749207
15 0.371999979019165
18 0.371999979019165
19 0.371000051498413
};
\addplot [semithick, color1]
table {%
0 0.468999981880188
1 0.48199999332428
2 0.611000061035156
3 0.628000020980835
4 0.565999984741211
5 0.624000072479248
6 0.635999917984009
7 0.621000051498413
8 0.624000072479248
9 0.629999995231628
10 0.628000020980835
11 0.628999948501587
12 0.626999974250793
14 0.626999974250793
15 0.628000020980835
18 0.628000020980835
19 0.628999948501587
};
\draw (axis cs:19.5,0.371) node[
  anchor=base west,
  text=color0,
  rotate=0.0
]{Field X};
\draw (axis cs:19.5,0.629) node[
  anchor=base west,
  text=color1,
  rotate=0.0
]{Field Y};
\draw (axis cs:18,0.03) node[
  anchor=base west,
  text=black,
  rotate=0.0
]{N simulations = 1000};
\end{axis}

\end{tikzpicture}

\end{figure}
\hyperlink{app_v_effects}{\beamerbutton{$\nu$ effects}}
\hyperlink{sim_beliefs}{\beamerbutton{Return: belief simulation}}

\end{frame}


%%%%%%%%%%%%%%%%%%%%%%%%%%%%%%%%%%%%%%%%%%%%%%%%%%%%%%%%%%%%%%%%%%%%%%%%%%%%%%%%
\begin{frame}{$\nu$ effects}\label{app_v_effects}

\begin{figure}
\centering
% This file was created by tikzplotlib v0.9.2.
\begin{tikzpicture}

\definecolor{color0}{rgb}{0.266666666666667,0.466666666666667,0.666666666666667}
\definecolor{color1}{rgb}{0.933333333333333,0.4,0.466666666666667}

\begin{axis}[
height=5.101085673964669cm,
tick align=outside,
tick pos=left,
title={Field selection and human capital updating \\ Field X: \(\displaystyle \nu=1.09\); Field Y: \(\displaystyle \nu=1.0\)},
width=8.25373cm,
x grid style={white!69.0196078431373!black},
xmin=-1.05, xmax=26.25,
xtick style={color=black},
xtick={0,5,10,15,20},
xticklabels={\(\displaystyle 0\),\(\displaystyle 5\),\(\displaystyle 10\),\(\displaystyle 15\),\(\displaystyle 20\)},
ymajorgrids,
ymin=-0.05, ymax=1.05,
ytick style={color=black},
ytick={0,0.2,0.4,0.6,0.8,1},
yticklabels={\(\displaystyle 0\),\(\displaystyle 0.2\),\(\displaystyle 0.4\),\(\displaystyle 0.6\),\(\displaystyle 0.8\),\(\displaystyle 1\)}
]
\addplot [semithick, color0]
table {%
0 1
1 0.506
2 0.764
3 0.507
4 0.639
5 0.606
6 0.653
7 0.6
8 0.618
9 0.621
10 0.622
11 0.617
12 0.621
13 0.62
14 0.621
15 0.621
16 0.621
17 0.621
18 0.621
19 0.621
20 0.621
21 0.621
};
\addplot [semithick, color1]
table {%
0 0
1 0.494
2 0.236
3 0.493
4 0.361
5 0.394
6 0.347
7 0.4
8 0.382
9 0.379
10 0.378
11 0.383
12 0.379
13 0.38
14 0.379
15 0.379
16 0.379
17 0.379
18 0.379
19 0.379
20 0.379
21 0.379
};
\draw (axis cs:21.5,0.651) node[
  anchor=base west,
  text=color0,
  rotate=0.0
]{Field X};
\draw (axis cs:21.5,0.409) node[
  anchor=base west,
  text=color1,
  rotate=0.0
]{Field Y};
\end{axis}

\end{tikzpicture}

\end{figure}
\hyperlink{app_ability_v_effect}{\beamerbutton{Return: $\alpha_{X0} \nu_X = \alpha_{Y0} \nu_Y$}}
\hyperlink{sim_beliefs}{\beamerbutton{Return: belief simulation}}
\hyperlink{sim_beliefs}{\beamerbutton{Return: ability simulation}}

\end{frame}

\end{document}