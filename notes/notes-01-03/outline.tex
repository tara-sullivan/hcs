\documentclass[10 pt]{article}

% Format
\usepackage[T1]{fontenc}
\usepackage[utf8]{inputenc}
\usepackage[margin=1in]{geometry} % 1-inch margins
\usepackage[english]{babel} % English hyphenation, etc.	
\usepackage{setspace} % Set spacing 
%\usepackage{enumerate} % Use different types of enumerate options
\usepackage{enumitem}
\usepackage{csquotes} % Block quotes
\usepackage[dvipsnames]{xcolor} % colors: https://en.wikibooks.org/wiki/LaTeX/Colors#The_68_standard_colors_known_to_dvips
% Math
\usepackage{amsmath, mathrsfs, amsfonts, amssymb, amsthm}
% Figures
\usepackage{graphicx} % Include figures
\usepackage{float} % Improved control over floats
\usepackage{tikz} % Draw figures with tikz

% Colors
\definecolor{Indigo}{HTML}{3C6478}
\definecolor{DarkBrown}{HTML}{45281B}
\definecolor{Brown}{HTML}{161402}
\definecolor{DarkGreen}{HTML}{325101}
\definecolor{LeafGreen}{HTML}{4A6F01}
\definecolor{DarkAlice}{HTML}{107896}
\definecolor{Alice}{HTML}{1496BB}
\definecolor{DarkGray}{RGB}{116 118 120}
\definecolor{DarkBlue}{HTML}{022C36}
\definecolor{MainBlue}{HTML}{132881}
\definecolor{Maroon}{HTML}{6A123D}
\definecolor{DarkPurple}{HTML}{2C033A}
\definecolor{Orange}{HTML}{F18312}

% Hyperlinks
\usepackage{hyperref} % Include hyperlinks
\hypersetup{
  colorlinks   = true, %Colours links instead of ugly boxes
  urlcolor     = Maroon, %Colour for external hyperlinks
  linkcolor    = DarkGray, %Colour of internal links
  citecolor   = MainBlue %Colour of citations
}


% Macro Shortcuts
\newcommand{\R}{\mathbb{R}}
\newcommand{\Q}{\mathbb{Q}}
\newcommand{\Z}{\mathbb{Z}}
\newcommand{\N}{\mathbb{N}}
\newcommand{\EE}{\mathbb{E}}
\newcommand{\PP}{\mathbb{P}}
\newcommand{\BB}{\mathscr{B}}
\newcommand{\e}{\text{e}}
\newcommand{\dd}{\text{d}}


% Theorems
\newtheorem{prop}{Proposition}[section]
\newtheorem{thm}{Theorem}
\theoremstyle{remark}
\newtheorem{claim}{Claim}[section]
\newtheorem{remark}{Remark}
\theoremstyle{definition}
\newtheorem{defn}{Definition}[section]
\newtheorem{lemma}{Lemma}
\newtheorem{ass}{Assumption}


\newif\ifnts
\ntstrue % uncomment to show 
% Notes to self 
\ifnts
  \newcommand{\nts}[1]{{\color{gray}#1}}
\else
  \newcommand{\nts}[1]{}
\fi

%%%%%%%%%%
% Sections that have:
%   (A) Roman numerals 
%   (B) fixed width = fixw
%   (C) coloring

% (A) Roman numeral for section and subsection
% \renewcommand{\thesection}{\Roman{section}} 
% \renewcommand{\thesubsection}{\roman{subsection}}

% (B) Each section has fixed width = fixw 

% Define fixw
\newcommand{\fixw}{28pt}
\newcommand{\fixwh}{14pt}

% (C) Define colors
\newcommand{\secc}[1]{{\color{DarkGreen}#1}} % section color
\newcommand{\sectc}{DarkGreen} % section text color
\newcommand{\subsecc}[1]{{\color{LeafGreen}#1}} % subsection color
\newcommand{\subsectc}{LeafGreen} % subsection text color
\newcommand{\numc}{DarkAlice}

% Set each section width and color
\usepackage{titlesec}
\titleformat{\section}{\normalfont\Large\bfseries\color{\sectc}}
	{\makebox[\fixw][l]{\secc{\thesection.}}}{0pt}{} 
\titleformat{\subsection}{\normalfont\large\bfseries\color{\subsectc}}
	{\makebox[{\fixw}][l]{\subsecc{\thesubsection.}}}{0pt}{} 
\titleformat{\subsubsection}{\normalfont\bfseries}
	{}{0pt}{} %{\makebox[{\fixw}][l]{}}{0pt}{} 

% Highlight certain items
\newcommand{\hitem}[2][DarkAlice]{\color{#1} \item #2 \color{black}}

%%%%%%%%%%
% Lists that start at fixw (see section above)
\newlist{outline}{enumerate}{2}
\setlist[outline,1]{label=\arabic*.,left=0pt .. \fixw}
\setlist[outline,2]{label=\alph*.,left=0pt .. \fixw}

\newlist{blist}{itemize}{2}
\setlist[blist,1]{label=\textbullet,left=0pt .. \fixw}
\setlist[blist,2]{label=\textendash,left=0pt .. \fixw}

%%%%%%%%%%
% Enumerate in footnote
\newlist{footcount}{enumerate}{1}
\setlist[footcount]{label=(\alph*),left=0pt .. \fixw}

%%%%%%%%%%
% Foodnote Edits
% Bottom package ensures that footnote won't be above a figure
\usepackage[bottom]{footmisc}

% No indent in footnotes
% NOTHING SEEMS TO WORK
% \usepackage{scrextend}
% \deffootnote[\fixw]{\fixw}{.195in}{\makebox[\fixw][r]{\thefootnotemark.\hspace{.2in}}}
% \usepackage[flushmargin, hang]{footmisc} % flush footnote mark to left margin
% \renewcommand{\footnotelayout}{\doublespacing\raggedright}
% \usepackage[flushmargin,hang]{footmisc}
% \usepackage[hang, flushmargin]{footmisc}
% \setlength{\footnotemargin}{0.5in}


% % \usepackage[marginal]{footmisc}
% \setlength\footnotemargin{0pt}  % default value: 1.8em

% \usepackage[flushmargin,hang]{footmisc}
% % \setlength{\footnotemargin}{1em} % just to show clearly equal output

% % \usepackage[marginal]{footmisc}
% \setlength{\footnotemargin}{10em} % just to show clearly equal output

% \renewcommand{\footnotelayout}{\raggedright}


%%%%%%%%%%
% Skip line between paragraphs, set indent to \fixw
\usepackage[parfill, indent=\fixw]{parskip}

%%%%%%%%%%
% format caption
% get rid of 'Figure: ' in caption
\usepackage{caption}
% \captionsetup[table]{labelsep=space}
\captionsetup{%
    % labelformat=empty,
    % font=small,
    labelsep=quad,
    tableposition=top,
    labelsep=period,
    margin=\fixw,
}



\newcommand{\cc}{\mathbf{c}}
\newcommand{\xx}{\mathbf{x}}

\addbibresource{../bibliography.bib}

\begin{document}
\title{Structural Approaches to Agricultural Climate Change Adaptation}
\author{Tara Sullivan}

\maketitle
\onehalfspacing

\noindent\nts{Please note that gray text are notes/comments}

%%%%%%%%%%%%%%%%%%%%%%%%%%%%%%%%%%%%%%%%%%%%%%%%%%%%%%%%%%%%%%%%%%%%%%%%%%%%%%%%
\section{Introduction}
%%%%%%%%%%%%%%%%%%%%%%%%%%%%%%%%%%%%%%%%%%%%%%%%%%%%%%%%%%%%%%%%%%%%%%%%%%%%%%%%

\begin{outline}

\item Estimating the impact of climate change on food production is not straightforward.
Overall, a warming climate appears to have a negative impact on yields of maize, wheat, rice, and soybeans \parencite{IPCC5WG2C7}.
However, increasing temperature is not the sole metric of climate change; a number of other changes in climatic systems have also been observed, including shifts in precipitation conditions, extreme weather, and atmospheric composition \parencite{IPCC5WG1}. 
The interaction of these dynamics will have heterogeneous impacts on agricultural production. 
Furthermore, the response of economic agents to these physical processes will shift the overall impact of climate change on food systems.

\item This paper discusses how economists have looked at adaptation to climate change \nts{(overview of paper to follow)}

%Economic forces are also important; institutional context, dynamic market conditions, and endogenous production decisions all impact output. 

\end{outline}

%%%%%%%%%%%%%%%%%%%%%%%%%%%%%%%%%%%%%%%%%%%%%%%%%%%%%%%%%%%%%%%%%%%%%%%%%%%%%%%%
\section{Definition of climate change adaptation}
%%%%%%%%%%%%%%%%%%%%%%%%%%%%%%%%%%%%%%%%%%%%%%%%%%%%%%%%%%%%%%%%%%%%%%%%%%%%%%%%

\begin{outline}

%Adaptation defined differently in different areas. 

\item IPCC definition

Adaptation is broadly defined by the IPCC as ``the process of adjustment to actual or expected climate and its effects'' \parencite{IPCC5WG2}.  %AR5 WG2 Chp 14
Adaptation thus encompasses a host of different activities.
\textcite{IPCC5WG2C14} broadly divide these adaptation options into three categories:\footnote{
		For a full list of adaptation options, see \href{https://www.ipcc.ch/site/assets/uploads/2018/02/WGII_AR5_Table14-1.jpg}{Table 14-1}.
		Examples relevant to agricultural adaptation are included below. 
		Footnotes contain section references from \textcite{IPCC5WG2}.
	} \nts{[Agricultural examples in brackets]}
	\begin{blist}
		\item Structural/physical \nts{[
		%%% Engineered and built environment 
			adapting transport systems;\footnote{
				See 8.3.3.6., 8.3.3.2.; Table 8-6; 10.4.4.
			} % super important: 8.3.3.6!!! costs of transport impact food!!!
			% From the urban chapter: need to be connected to global food distribution centers, and need to be able to deal with extreme events (8.3.3.2; table at the end ) Ports are vulnerable to weather events (10.4.4)
		%%% Technological: 
			adjusting crop and livestock varieties and practices;\footnote{
				Adjusting crop practices includes cultivar adjustments, planting date adjustments, fertilizer optimization, irrigation optimization (7.5.1.1.1., Table 7-2, Table 9-7, and 7.5.1.1.3.), 
				hydroponics
				% CULTIVATING VEGETABLES AND OTHER CROPS ON FLOADING GARDENS
				%https://doi.org/10.3200/ENV.51.4.22-31
				%https://link.springer.com/chapter/10.1007/978-3-642-31110-9_44)
				%\textcite{CMC12}: Global look at sea level rise and the global rice market 
				%IPCCWG2C5 "The combination of rice yield reduction induced by climate change and inundation of lands by seawater causes an important reduction in production" Important to look at Inundation (cited in chp. 5)
				%\item Infusion of saltwater into Bangladesh household ponds
				%https://www.mdpi.com/2071-1050/5/4/1510 
				% I believe this recommends that farmers produce saline-tolerate crops 
			}
				% land cover change + climate change has negative impact on water available for irrigation https://link.springer.com/article/10.1007/s10113-013-0462-2
				% Table 9-7
			water resource management;\footnote{
				Water resource management includes efficient irrigation (22.4.5.7.) and other water saving technologies (24.4.1.5.; 10.3.8.)
			}
			better food storage practices;\footnote{22.4.5.7.}
		%%% Ecosystem-based
			ecosystem based adaptations;
			%Vulnerability of coastal regions; ecosystem based-adaptation; risk management 
			%https://www.sciencedirect.com/science/article/pii/S0964569112002207
		%%% Services
			%Social safety nets and social protection (Box 13-2; 8.3; 17.5.1. 22.4.5.2	);
			% food banks
			trade\footnote{9.3 (especially 9.3.3.3.2.); 9.4}
			]}
		\item Social \nts{[education; 
			informational;\footnote{
				Community based adaptation plans (5.5.1.4.; 24.4.6.5.). See Table 5-4 for coastal community-based adaptation measures.
			}
			behavioral\footnote{
			For example, changing crop practices (7.5.1.1.1.)
		}
		]}
		\item Institutional \nts{[economic incentives and services; laws and regulations; government policies and programs]}
	\end{blist}
	
	% Not all adaptation needs will be met (16.7.1)
	%Definition from IPCC AR5 7.3.1.2. (pp. 497): Adaptation occurs on a range of time scales and by a range of actors. Incremental adaptation... most commonly assessed in the impacts literature,... Sections 7.3 and 7.4. Systemic and transformational adaptations are discussed in Section 7.5. Methods exist to examine impacts and adaptation together in the context of nonclimatic drivers (Mandryk et al., 2012), but conclusions are %difficult to measure
	
	% How does the IPCC currently take into account adaptation/production decisions

\item Definitions by economists and policymakers

\begin{blist}

%Why do these actors have slightly different definitions than, say, the IPCC

\item Adaptation is a vague economic concept; \textcite{BL10} define adaptation as ``any response that improves an outcome.''
% Also discusses difference between autonomous and planned adaptation (central for policymaking), and defines the three phases of autonomous adaptation
\textcite{ZZH12} define adaptation as ``changes in public and private decision making and resource allocation in expecting or responding to the prospect or reality of large-scale and long-lasting changes.''

\item Adaptation sometimes typified along different strata than the IPCC. 
For instance, economists often distinguish between ``private adaptation'' and ``public adaptation.''\nts{\footnote{
		\textcite{BL10} distinguishes ``autonomous'' and ``planned'' adaptation, while \textcite{M12} uses the terms ``private'' and ``public'' adaptation. \textcite{F17} takes a stronger stance on terminology, holding that ``[autonomous adaptation] is a misnomer, as no adaptation actions happen autonomously. They are always the result of deliberate, sometimes complex decisions taken by the actors involved.''
	}
}

\item Often policymakers and economists define adaptation in terms of resilience

\begin{blist}
	\item \textcite{F17} defines adaptation as climate resilience, though he notes that there can be subtle differences between those terms.

	\nts{\item Kritee: there are pillars of adaptation involving resilience in farming practices, resistance of the ecosystem itself, and resilience of the community itself.}

	\nts{\item Richie Ahuja: Farmers are always adapting. For them, it's risk mitigation}

	\nts{\item Doria Gordon and \textcite{WS06}: resilience different than adaptation, and comprises resistance, adaptation and transformation.}
\end{blist}

\end{blist}

\nts{\item Be clear about your definition of adaptation and context. Make sure it's different than the resilience definition, and clarify that distinction.}

\item In the academic literature, the specific definition of adaptation is often closely tied to the methodological strategy. 
As such, I will try to highlight different definitions in the literature below. 

\end{outline}


%%%%%%%%%%%%%%%%%%%%%%%%%%%%%%%%%%%%%%%%%%%%%%%%%%%%%%%%%%%%%%%%%%%%%%%%%%%%%%%%
\section{Basic methods/Ricardian model}
%%%%%%%%%%%%%%%%%%%%%%%%%%%%%%%%%%%%%%%%%%%%%%%%%%%%%%%%%%%%%%%%%%%%%%%%%%%%%%%%

\begin{blist}

\item Adaptation can be a production choice or consumption choice; a result of government intervention or an optimization in a free market. 
Below, I focus on adaptation in agricultural production decisions. 

\item \nts{Clean this up/ better examples:} Different adaptation strategies can be explicitly or implicitly measured. 
Economics is best suited to account for implicit adaptation. %Not a perfect delineation, but I think this broadly works.

\begin{blist}

\item The number of households using fertilizer and hybrid seeds in a region is an explicit measurement of adaptation.

\item Estimates of the impact of climate change on outcomes that take into account farmer optimization decision are implicitly accounting for adaptation. 
Measuring adaptation from these estimates often requires some functional form assumptions.

\end{blist}

\item Baseline approach: Ricardian approach, started with \textcite{MNS94} 

\begin{blist}

\item Estimates impact of climate change on agricultural outcomes

\item Takes farmer optimization seriously

\end{blist}

\item Note: how it's closely tied to empirical estimates of climate change impacts, I follow the econometric techniques set forth in \textcite{H16} for evaluating climate change impact.

%\item Explicit adaptation: microeconometric approaches, discrete choices, technology adoption

%\item Implicit adaptation: structural, macroeconomic approaches
% Schneider et al: explicit vs. implicit adaptation

\item \textcite{SEM00}: dumb-farmer approach vs clairvoyant farmer approach

\end{blist}

% If I were writing a book: scientific adaptation/ how is adaptation used in policy applications or climate models

%%%%%%%%%%%%%%%%%%%%%%%%%%%%%%%%%%%%%%%%%%%%%%%%%%%%%%%%%%%%%%%%%%%%%%%%%%%%%%%%
\subsection{Ricardian framework}

\begin{outline}

\hitem{Overview}
% Responses to climate change look like responses to beliefs about the climate or responses to weather 

\textcite{MNS94} use cross-sectional data to estimate the impact of temperature on the value of farmland (as opposed to say, yields).
This implicitly accounts for adaptation by allowing for different production decisions, substitution of inputs, etc. 



%%%%%%%%%%%%%%%%%%%%%%%%%%%%%%%%%%%%%%%%%%%%%%%%%%%%%%%%%%%%%%%%%%%%%%%%%%%%%%%%
\hitem{Econometric estimates}

%Authors estimate how some agricultural outcome will change with climate variables in a particular region.\footnote{
%	Agricultural choices studied include crop choice in Africa \parencite{KM08} and South America \parencite{SM08}; livestock choice in South America \parencite{SMM10}; and irrigation decisions in Africa \parencite{KKM11}. Climate variables include seasonal temperature and rainfall (5-year averages) and soil data (hydrology and elevation controlled for in irrigation paper). The authors control for various household characteristics and prices. 
%}
In practice, the authors estimate the impact of climatic variables (the vector $\cc_i$) and other controls (the vector $\xx_i$) on agricultural outcomes ($y_i$) in a particular region $i$ using a cross-sectional regression:
\begin{equation}
	y_i = \alpha + \beta \cc_i + \gamma \xx_i + \epsilon_i.
\end{equation}
Here, $\alpha$ is a constant, and $\epsilon_i$ is the unobserved prediction error. 

The identifying assumption is that difference in output between areas with the same characteristics/inputs $\xx$ are soley driven by climate. 
\textcite{H16} describes this as ``unit homogeneity.''
This is a difficult assumption to make. 
For instance, there is evidence that temperature may be correlated with unobserved characteristics like institutional quality, leading to omitted variable bias \parencite{AJR02}. 
This method also relies on accurate measurements of land values, which may lead to measurement error \parencite{DG07}.
To address this, economists may use time series methods to control for unobserved characteristics.

\nts{Variation in climate comes from variation in weather. \nts{Elaborate on this and cite \textcite{H16}.}}
%%%%%%%%%%%%%%%%%%%%%%%%%%%%%%%%%%%%%%%%%%%%%%%%%%%%%%%%%%%%%%%%%%%%%%%%%%%%%%%%
\hitem{Structural underpinnings}

Underlying this equation are several structural assumptions.
The economic agents are farmers, land renters, and consumers. 
Markets are perfectly competitive.
Each crop has an implicit profit function. 
Farmers will chose the crop that maximizes profits. 
We assume that there is a perfectly competitive agricultural land rental market. 
The land rent will determined by the most valuable use of the land; this is represented by the line in Figure 1 from \textcite{MNS94}. %that will ensure that the land is attributed to the use that extracts the largest rents. 
We also assume unit elastic demand. % Demand side issues are not taken into account. 

%The structural adaptation decision underlying this type of model is made explicit in later work. 
%For example, \textcite{SM08} assume there are $j = 0, \dots, J$ crop varieties.
%Then total potential revenue from producing alternative $j$ given by:
%\begin{equation*}
	%y_{ij}^* = \beta \cc_{ij} + \gamma \xx_{i} + u_{ij}
%\end{equation*}
%Farmers produce the crop that yields the highest revenue:
%\begin{equation*}
	%y_{ij} = \arg \max_{j=0,\dots,J} \left\{ y_{i0}^*,\dots,y_{iJ}^* \right\}
%\end{equation*}
%The adaptation decision is modeled by a multinomial choice, and can be estimated using a conditional logit model. 

%\begin{blist}

% Does not allow for intercropping!

%\item Structural estimation strategy is a multinomial choice; before I talked about fixed effects. 

%\item Demand side implicitly unit-elastic

%\end{blist}




\end{outline}


%%%%%%%%%%%%%%%%%%%%%%%%%%%%%%%%%%%%%%%%%%%%%%%%%%%%%%%%%%%%%%%%%%%%%%%%%%%%%%%%
\subsection{Time-series and Panel approaches}

\begin{outline}

%%%%%%%%%%%%%%%%%%%%%%%%%%%%%%%%%%%%%%%%%%%%%%%%%%%%%%%%%%%%%%%%%%%%%%%%%%%%%%%%
\hitem{Overview}

%%%%%%%%%%%%%%%%%%%%%%%%%%%%%%%%%%%%%%%%%%%%%%%%%%%%%%%%%%%%%%%%%%%%%%%%%%%%%%%%
\hitem{Econometric specifications} 

\begin{blist}

\item Time-series: see \textcite{H16} for empirical specification.
\begin{equation}\label{ts}
	\nts{Placeholder}
\end{equation}
%Not different structurally than cross-sectional approach

\item Panel approaches: combine elements of the cross-sectional and panel approaches.
Long-differences were first used in growth equations in \textcite{DJO12}. 
\begin{equation}
	\overline{y}_{i2} - \overline{y}_{i1} = \alpha + \beta_i (\overline{\cc}_{i2} - \overline{\cc}_{i1}) + \gamma (\overline{\xx}_{i2} - \overline{\xx}_{i1}) + \epsilon_i
\end{equation}

\end{blist}

%%%%%%%%%%%%%%%%%%%%%%%%%%%%%%%%%%%%%%%%%%%%%%%%%%%%%%%%%%%%%%%%%%%%%%%%%%%%%%%%
\hitem{Structural underpinnings}

\begin{blist}

\item Using the method set forth in \textcite{SR09}, you can use this to empirically estimate adaptation, as in \textcite{BE16} (adaptation = $1 - \beta_{FE}/\beta_{LD}$)

\item Clarify that type of adaptation is crop choice

\item Important takeaway: as noted in \textcite{H16}, you can impose structural assumptions to push this methodology \parencite{CDS16,DR15}. This will be elaborated  upon in the following section. \nts{This point should perhaps be made in a different place; the adaptation interpretation of the regressions in \textcite{BE16} doesn't necessarily need further structural assumptions, I think. But this transition is nice. }
\end{blist}

\end{outline}

%%%%%%%%%%%%%%%%%%%%%%%%%%%%%%%%%%%%%%%%%%%%%%%%%%%%%%%%%%%%%%%%%%%%%%%%%%%%%%%%
\subsection{To cite}

\textcite{SR09,BS17,DG07,SL10,ZDMZ18}
\begin{blist}

\item \textcite{C07}: uses Ricardian approach to incorporate local adaptation into agricultural impact estimates

\item \textcite{SR09}: need to estimate the impact of temperature on yields in order to determine subsequent economic decisions
\end{blist}

%%%%%%%%%%%%%%%%%%%%%%%%%%%%%%%%%%%%%%%%%%%%%%%%%%%%%%%%%%%%%%%%%%%%%%%%%%%%%%%%
\section{Extending Ricardian approach through trade models}
%%%%%%%%%%%%%%%%%%%%%%%%%%%%%%%%%%%%%%%%%%%%%%%%%%%%%%%%%%%%%%%%%%%%%%%%%%%%%%%%

\begin{blist}

\item Impose structure on the basic Ricardian framework in a way that allows for further econometric insight

\end{blist}

%%%%%%%%%%%%%%%%%%%%%%%%%%%%%%%%%%%%%%%%%%%%%%%%%%%%%%%%%%%%%%%%%%%%%%%%%%%%%%%%
\subsection{Trade as adaptation}

\begin{outline}

\item There is a brief literature on international trade as an adaptation strategy.\footnote{See review in \textcite{ZZH12} and section 9.3.3.3.2. in \textcite{IPCC5WG2}).}
More recently, authors have been using spatial data and gravity trade models to estimate the impact of climate change.
In this context, adaptation is defined by changes in crop production choices and international trade. 
\begin{blist}
	\item \textcite{CDS16}: use  FAO-GAEZ data on crop suitability to measure the comparative advantage of particular “fields” (i.e. grid cells) in producing certain crops, both under current conditions and IPCC projections. They then use a trade model to estimate changes in production under these two scenarios. A trade model is useful for counterfactual analysis because it allows you to estimate bilateral flows (i.e. the production and consumption of certain crops) in a computational general equilibrium framework across different locations. And while trade models are obviously most commonly used to measure international trade, they are helpful in regional spatial contexts as well.

	\item \textcite{DMH19-wp}: follows a similar approach above, with two key differences: first, they employ a different trade framework that incorporates spatial correlation of productivities.\footnote{
		For details on trade models that incorporate correlation in productivities, see \textcite{ACR12} and \textcite{LR18-wp}.
	} Second, they do not use FAO-GAEZ agronomic forecasts, but historical responses to El Ni\~{n}o. 
\end{blist}

\item Trade models have also been used to describe the development of agriculture in the U.S \parencite{DH16,CD16-wp}.

\end{outline}

%%%%%%%%%%%%%%%%%%%%%%%%%%%%%%%%%%%%%%%%%%%%%%%%%%%%%%%%%%%%
\subsection{Trade as a framework}

\begin{outline}

\hitem{About trade models}

\begin{blist}

	% Remember: models are one of our primary tools for conducting economic analysis. the other are empirical methods. There's tons of overalap. 

	\item Trade models are quantitative general equilibrium models of bilateral interactions.
	Though they are most often used in international economics, trade frameworks are employed in a number of other settings. See, for example, the use of trade models in a regional setting to study intranational pricing \parencite{AD15-wp}.
	Trade models are particularly useful for counterfactual analysis. 

	\item I use the term \textbf{trade model} to refer to a \textbf{{\color{Orange}Ricardian} {\color{Alice}gravity} trade model}.
	In a {\color{Orange}Ricardian} model of trade, differences in technology drive trading patterns, and countries specialize in producing goods according to their comparative advantage. 
	In a {\color{Alice}gravity} trade model, size and distance impact the bilateral flow multiplicatively. 
	This captures two key empirical facts often seen in international trade data: (1) trade is proportional to size, and (2) trade is inversely proportional to distance \textcite{HM14}. 
	Thus, a {\color{Orange}Ricardian} {\color{Alice}gravity} trade model captures how comparative advantage and geographic barriers govern trade patterns.

	\item An important type of trade model is developed in \textcite{EK02} framework. A significant advantage of the \textcite{EK02} model over other types of quantitative general equilibrium models is parsimony; counterfactual analysis can be conducted with only one structural parameter estimate \parencite{ACD17}.


\end{blist}

\hitem{Problems with trade models}

\begin{blist}

\item For parsimony, need strong functional form assumptions. 
For instance, we assume that the production possibilities frontier follows a Fr\'{e}chet distribution. 
As discussed in \textcite{EK02}, this assumption is reasonable if: (1) new technologies become available according to a Poisson process, and (2) production is based on the best technology.
It's worth noting whether or not this is a reasonable assumption.

\item Results describe steady state, but not transitions.

\end{blist}

\hitem{Areas for further research}

\begin{blist}

\item Can apply these models in a regional setting, akin to, to look at regional transport networks \parencite{DH16,CD16-wp}.

\item Use updated productivity data, such as remote sensing yield predictions \parencite{BL17} (though keep in mind \cite{B18}).

\item Create a more explicit trade model that considers how robust global food systems are to extreme events. 

\end{blist}

\end{outline}

%%%%%%%%%%%%%%%%%%%%%%%%%%%%%%%%%%%%%%%%%%%%%%%%%%%%%%%%%%%%%%%%%%%%%%%%%%%%%%%%
\section{Innovation diffusion and adaptation}
%%%%%%%%%%%%%%%%%%%%%%%%%%%%%%%%%%%%%%%%%%%%%%%%%%%%%%%%%%%%%%%%%%%%%%%%%%%%%%%%

%%%%%%%%%%%%%%%%%%%%%%%%%%%%%%%%%%%%%%%%%%%%%%%%%%%%%%%%%%%%%%%%%%%%%%%%%%%%%%%%
\subsection{Innovation in economics}

\begin{outline}

\hitem{Definition}

\begin{blist}

\item Define innovation. Mention that it is trying to define increases in productivity/technology (be clear about these definitions).

\item Cite:  \textcite{PNJ10,ZZH12}

\item Measured using trade models; these can be used to understand how adaptation ``ideas'' spread \parencite{BO-wp}.
\end{blist}

%%%%%%%%%%%%%%%%%%%%%%%%%%%%%%%%%%%%%%%%%%%%%%%%%%%%%%%%%%%%%%%%%%%%%%%%%%%%%%%%
\hitem{Questions about innovation and adaptation}

\begin{blist}
\item  %Create a more explicit trade model that considers how robust global food systems are to extreme events. 
What is the optimal level of self-sufficiency? %How does trade in agricultural goods develop?
\end{blist}

\end{outline}
%%%%%%%%%%%%%%%%%%%%%%%%%%%%%%%%%%%%%%%%%%%%%%%%%%%%%%%%%%%%%%%%%%%%%%%%%%%%%%%%
\subsection{Innovation and adaptation}

\begin{blist}

\item We have concepts that use trade models to understand how ideas spread \parencite{BO-wp}
We can use these models to understand how adaptation ``ideas'' spread 

\nts{\item Possibly prove: reduced form innovation common in macroeconomics has similar structural underpinnings to adaptation definitions in climate change}

\end{blist}












%\nts{\item Question: under what circumstances can adaptation be viewed as the adoption of technology, capital embodied or otherwise?
%Adaptation, in particular private adaptation, is the result of discrete decision making. In this light, we can view adaptation in the light of technology adoption and/or multinomial decision making, as noted in, say \textcite{ZZH12}. However this Ricardian-type approach is not necessarily accurate for understanding adaptation. 

% My questions

% Trade
% Self-sufficiency


% Future research:

% Richie: measuring adaptation; an adaptation index

\printbibliography

\end{document}
