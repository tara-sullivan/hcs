\documentclass[10 pt]{article}

% Format
\usepackage[T1]{fontenc}
\usepackage[utf8]{inputenc}
\usepackage[margin=1in]{geometry} % 1-inch margins
\usepackage[english]{babel} % English hyphenation, etc.	
\usepackage{setspace} % Set spacing 
%\usepackage{enumerate} % Use different types of enumerate options
\usepackage{enumitem}
\usepackage{csquotes} % Block quotes
\usepackage[dvipsnames]{xcolor} % colors: https://en.wikibooks.org/wiki/LaTeX/Colors#The_68_standard_colors_known_to_dvips
% Math
\usepackage{amsmath, mathrsfs, amsfonts, amssymb, amsthm}
% Figures
\usepackage{graphicx} % Include figures
\usepackage{float} % Improved control over floats
\usepackage{tikz} % Draw figures with tikz

% Colors
\definecolor{Indigo}{HTML}{3C6478}
\definecolor{DarkBrown}{HTML}{45281B}
\definecolor{Brown}{HTML}{161402}
\definecolor{DarkGreen}{HTML}{325101}
\definecolor{LeafGreen}{HTML}{4A6F01}
\definecolor{DarkAlice}{HTML}{107896}
\definecolor{Alice}{HTML}{1496BB}
\definecolor{DarkGray}{RGB}{116 118 120}
\definecolor{DarkBlue}{HTML}{022C36}
\definecolor{MainBlue}{HTML}{132881}
\definecolor{Maroon}{HTML}{6A123D}
\definecolor{DarkPurple}{HTML}{2C033A}
\definecolor{Orange}{HTML}{F18312}

% Hyperlinks
\usepackage{hyperref} % Include hyperlinks
\hypersetup{
  colorlinks   = true, %Colours links instead of ugly boxes
  urlcolor     = Maroon, %Colour for external hyperlinks
  linkcolor    = DarkGray, %Colour of internal links
  citecolor   = MainBlue %Colour of citations
}


% Macro Shortcuts
\newcommand{\R}{\mathbb{R}}
\newcommand{\Q}{\mathbb{Q}}
\newcommand{\Z}{\mathbb{Z}}
\newcommand{\N}{\mathbb{N}}
\newcommand{\EE}{\mathbb{E}}
\newcommand{\PP}{\mathbb{P}}
\newcommand{\BB}{\mathscr{B}}
\newcommand{\e}{\text{e}}
\newcommand{\dd}{\text{d}}


% Theorems
\newtheorem{prop}{Proposition}[section]
\newtheorem{thm}{Theorem}
\theoremstyle{remark}
\newtheorem{claim}{Claim}[section]
\newtheorem{remark}{Remark}
\theoremstyle{definition}
\newtheorem{defn}{Definition}[section]
\newtheorem{lemma}{Lemma}
\newtheorem{ass}{Assumption}


\newif\ifnts
\ntstrue % uncomment to show 
% Notes to self 
\ifnts
  \newcommand{\nts}[1]{{\color{gray}#1}}
\else
  \newcommand{\nts}[1]{}
\fi

%%%%%%%%%%
% Sections that have:
%   (A) Roman numerals 
%   (B) fixed width = fixw
%   (C) coloring

% (A) Roman numeral for section and subsection
% \renewcommand{\thesection}{\Roman{section}} 
% \renewcommand{\thesubsection}{\roman{subsection}}

% (B) Each section has fixed width = fixw 

% Define fixw
\newcommand{\fixw}{28pt}
\newcommand{\fixwh}{14pt}

% (C) Define colors
\newcommand{\secc}[1]{{\color{DarkGreen}#1}} % section color
\newcommand{\sectc}{DarkGreen} % section text color
\newcommand{\subsecc}[1]{{\color{LeafGreen}#1}} % subsection color
\newcommand{\subsectc}{LeafGreen} % subsection text color
\newcommand{\numc}{DarkAlice}

% Set each section width and color
\usepackage{titlesec}
\titleformat{\section}{\normalfont\Large\bfseries\color{\sectc}}
	{\makebox[\fixw][l]{\secc{\thesection.}}}{0pt}{} 
\titleformat{\subsection}{\normalfont\large\bfseries\color{\subsectc}}
	{\makebox[{\fixw}][l]{\subsecc{\thesubsection.}}}{0pt}{} 
\titleformat{\subsubsection}{\normalfont\bfseries}
	{}{0pt}{} %{\makebox[{\fixw}][l]{}}{0pt}{} 

% Highlight certain items
\newcommand{\hitem}[2][DarkAlice]{\color{#1} \item #2 \color{black}}

%%%%%%%%%%
% Lists that start at fixw (see section above)
\newlist{outline}{enumerate}{2}
\setlist[outline,1]{label=\arabic*.,left=0pt .. \fixw}
\setlist[outline,2]{label=\alph*.,left=0pt .. \fixw}

\newlist{blist}{itemize}{2}
\setlist[blist,1]{label=\textbullet,left=0pt .. \fixw}
\setlist[blist,2]{label=\textendash,left=0pt .. \fixw}

%%%%%%%%%%
% Enumerate in footnote
\newlist{footcount}{enumerate}{1}
\setlist[footcount]{label=(\alph*),left=0pt .. \fixw}

%%%%%%%%%%
% Foodnote Edits
% Bottom package ensures that footnote won't be above a figure
\usepackage[bottom]{footmisc}

% No indent in footnotes
% NOTHING SEEMS TO WORK
% \usepackage{scrextend}
% \deffootnote[\fixw]{\fixw}{.195in}{\makebox[\fixw][r]{\thefootnotemark.\hspace{.2in}}}
% \usepackage[flushmargin, hang]{footmisc} % flush footnote mark to left margin
% \renewcommand{\footnotelayout}{\doublespacing\raggedright}
% \usepackage[flushmargin,hang]{footmisc}
% \usepackage[hang, flushmargin]{footmisc}
% \setlength{\footnotemargin}{0.5in}


% % \usepackage[marginal]{footmisc}
% \setlength\footnotemargin{0pt}  % default value: 1.8em

% \usepackage[flushmargin,hang]{footmisc}
% % \setlength{\footnotemargin}{1em} % just to show clearly equal output

% % \usepackage[marginal]{footmisc}
% \setlength{\footnotemargin}{10em} % just to show clearly equal output

% \renewcommand{\footnotelayout}{\raggedright}


%%%%%%%%%%
% Skip line between paragraphs, set indent to \fixw
\usepackage[parfill, indent=\fixw]{parskip}

%%%%%%%%%%
% format caption
% get rid of 'Figure: ' in caption
\usepackage{caption}
% \captionsetup[table]{labelsep=space}
\captionsetup{%
    % labelformat=empty,
    % font=small,
    labelsep=quad,
    tableposition=top,
    labelsep=period,
    margin=\fixw,
}



% General problem, not bernoulli-beta example
\newif\ifgen
%\gentrue % uncomment to show 
\ifgen
  \newcommand{\gen}[1]{#1}
\else
  \newcommand{\gen}[1]{}
\fi

% Include introduction or not
\newif\ifintro
%\introtrue % uncomment to show 
\ifintro
  \newcommand{\intro}[1]{#1}
\else
  \newcommand{\intro}[1]{}
\fi

\newcommand{\cc}{\mathbf{c}}
\newcommand{\xx}{\mathbf{x}}

\newcommand{\br}[1]{\left\{ #1 \right\}}
\newcommand{\sbr}[1]{\left[ #1 \right]}
\newcommand{\pr}[1]{\left( #1 \right)}
\newcommand{\ce}[2]{\left[\left. #1 \right\vert #2 \right]}

\addbibresource{../bibliography.bib}

\begin{document}
\title{Notes on Human Capital Specialization}
\author{Tara Sullivan}

\maketitle
\onehalfspacing

\noindent\nts{Please note that gray text are notes/comments}

\intro{

%%%%%%%%%%%%%%%%%%%%%%%%%%%%%%%%%%%%%%%%%%%%%%%%%%%%%%%%%%%%%%%%%%%%%%%%%%%%%%%%
\section{Introduction}
%%%%%%%%%%%%%%%%%%%%%%%%%%%%%%%%%%%%%%%%%%%%%%%%%%%%%%%%%%%%%%%%%%%%%%%%%%%%%%%%

% In a model of human capital specialization with unknown ability 
% priors informed by existin group outcomes
% formalizes self-fulfiling prohpecies 
% provides a framework for quantifiying the inefficiencies associated with human capital specialization 
% 

\begin{outline}

\item Current literature on discrimination/affirmative action in higher education: \nts{(goal is to find a statement that encapsulates the following in a relevant way)}

\begin{blist}

\item Whether or not it helps or hurts the minority students it's meant to help \parencite{AL16}

\item Whether or not employing these policies is biased against students in the majority category\footnote{
   Maybe cite something about the Harvard trial here; we read a nice piece in ECON128 in Week 8. 
}

\item Specifically consider choice of college major and the focus on student preparedness \textcite{AAH16} 




\end{blist}

\item Shortcomings of the literature I want to address: \nts{(still working through this)}

\begin{blist}

%\item %Empirical studies rely on very precise mechanisms; I want something flexble.

\item \nts{I think the focus on student preparedness and outcomes misses a couple of things. First, measuring preparedness by GPA and SAT score, as is often the case, misses a lot. I don't know if I can address this, but I think it's important to mention. Second, misallocation of human capital has aggregate welfare effects. Empirical results can't account for that. This is something I can get at. }

\nts{\item I also don't like assuming that the college process is the same for different groups. The cost function for particular majors may be very different for one group; note that this is where the \textcite{CL93} framework might be useful.}

\end{blist}

\item This paper:

\begin{blist}

\item I consider the relationship between discrimination theories \nts{and existing group outcomes} and human capital specialization
% How do different theories of discrimination impact human capital specialization? 

\item Augment model of gradual human capital specialization in \textcite{AF20}

\item Flexible framework that accommodates mechanisms that are not generally accounted for in this literature. \

\item Existing grout outcomes have an impact on human capital specialization decisions. 

\item Shocks received impact human capital specialization decisions. Key: shocks impact groups differently \nts{(I think)}.

\item Incorporates statistical discrimination 

\item limitations of empirical analyses when evaluating the efficacy of affirmative action programs. 

\item Counterfactual analysis: misallocation of human capital specialization

\item Counterfactual analysis: the effectiveness of affirmative action policies. 

\end{blist}

% Another application: how do peopl specialize after recessions

% prior determined by existing outcomes
% prior determined by statistical discrimination 
% how you interpret shocks 

\end{outline}

%%%%%%%%%%%%%%%%%%%%%%%%%%%%%%%%%%%%%%%%%%%%%%%%%%%%%%%%%%%%%%%%%%%%%%%%%%%%%%%%
\subsection*{Literature}

\begin{blist}

\item Similar to \textcite{D08} in that I'm interested in efficiency loss from human capital misallocation \nts{(there's commentary on this in \textcite{AL16} that's worth reflecting on)}

\end{blist}


} % end of intro flag


%%%%%%%%%%%%%%%%%%%%%%%%%%%%%%%%%%%%%%%%%%%%%%%%%%%%%%%%%%%%%%%%%%%%%%%%%%%%%%%%
\section{Model}
%%%%%%%%%%%%%%%%%%%%%%%%%%%%%%%%%%%%%%%%%%%%%%%%%%%%%%%%%%%%%%%%%%%%%%%%%%%%%%%%

%%%%%%%%%%%%%%%%%%%%%%%%%%%%%%%%%%%%%%%%%%%%%%%%%%%%%%%%%%%%%%%%%%%%%%%%%%%%%%%%
\subsection{Alon and Fershtman (2020)} \nocite{AF20}

%%%%%%%%%%%%%%%%%%%%%%%%%%%%%%%%%%%%%%%%%%%%%%%%%%%%%%%%%%%%%%%%%%%%%%%%%%%%%%%%
\subsubsection{Preliminaries}

\begin{tabular}{@{}lll}
\textbf{General}  & \textbf{Description} \\
\multicolumn{2}{@{}l}{\emph{Endowments}} \\
$t$                                       & indivisible unit of time per period \\
$h_0    = \{ h_{10} \dots h_{N0} \}$      & initial human capital \\
$\theta = \{\theta_1, \dots, \theta_N \}$ & unknown abilities \\
$P_{j0}$                                  & independent initial beliefs on $\theta_j$ \\
\multicolumn{2}{@{}l}{\emph{Actions}} \\
$s_t    = \{ s_{1t}, \dots, s_{Nt} \}$    & study time \\
$\ell_t = \{ l_{1t}, \dots, \ell_{Nt} \}$ & work time \\
\multicolumn{2}{@{}l}{\emph{Technology}} \\
$a_{it}   \sim F_{\theta_i}$      & effective study time \\
$h_{it+1} = H_i (h_{it}, a_{it})$ & accumulation by $t+1$ \\
\multicolumn{2}{@{}l}{\emph{Beliefs}} \\
$P_{it+1} = \Pi (P_{it}, a_{it})$ & evolution of beliefs over $\theta_i$
\end{tabular}

%%%%%%%%%%%%%%%%%%%%%%%%%%%%%%%%%%%%%%%%%%%%%%%%%%%%%%%%%%%%%%%%%%%%%%%%%%%%%%%%
\subsubsection{Problem}

\gen{
A policy $\pi: (h_t, P_t) \to (s_t, \ell_t)$ is optimal if it maximizes: 
\begin{align*}
& \mathbb{E}^\pi \sbr{
   \sum_{t=0}^\infty \delta^t 
   \left. \pr{\sum_{i=1}^N \ell_{it} U_i(w_i, h_{it})} \right\vert
   \pr{(h_{10}, P_{10}), \dots, (h_{10}, P_{N0})}
} \\
\text{subject to} \quad& h_{it+1} = H_i (h_{it}, a_{it}), \quad \quad
P_{it+1} = \Pi (P_{it}, a_{it}), 
\quad \quad \text{if $i$ selected,} \\
\quad& \sum_{i=1}^N (s_{it} + \ell_{it}) = 1, \quad \quad s_{it}, \ell_{it} \in \{0,1\}
\end{align*}
It is often useful to parameterize this problem.}
A policy $\pi: (h_t, P_t) \to (s_t, \ell_t)$ is optimal if it maximizes:
\begin{align*}
& \mathbb{E}^\pi \sbr{
   \sum_{t=0}^\infty \delta^t 
   \left. \pr{\sum_{i=1}^N \ell_{it} w_i h_{it}} \right\vert
   \pr{(h_{10}, P_{10}), \dots, (h_{10}, P_{N0})}
} \\
\text{subject to} \quad& h_{it+1} = h_{it}+ a_{it} s_{it}, \quad \quad a_{it} = 
   \begin{cases} 
      \nu_i, & \text{with prob. } \theta_i,  \\ 
      0, & \text{with prob. } 1 - \theta_i,
   \end{cases} 
   \quad \quad h_{i0} = \alpha_{i0} \nu_i, \\
\quad& P_{it+1} = \mathcal{B} (\alpha_{i,t+1}, \beta_{i,t+1}), 
   \quad \quad \theta_i \sim P_{i,0} \equiv \mathcal{B} (\alpha_{i0}, \beta_{i0})
   \quad \quad \text{if $i$ selected,} \\
\quad& \sum_{i=1}^N (s_{it} + \ell_{it}) = 1, \quad \quad s_{it}, \ell_{it} \in \{0,1\}
\end{align*}

% Notes/questions
\nts{
\begin{itemize}
\item Shouldn't I either index $\pi$ by $t$, or say that a policy at time $t$ is optimal? 
\item Shouldn't this be if $s_{it} = 1$, meaning that you study $i$ during period $t$, rather than ``if $i$ is selected?''. I'm pretty sure that, because you could choose to work in skill $i$, you need to specify if you're studying.
\item Also, shouldn't the transition laws be $h_{it+1} = H_i (h_{it}, a_{it}, s_{it})$, since it depends on whether you study? This could also be fixed by changing the ``if $i$ selected line.'' I would also call this transition law ``course outcome'' or ``studying outcome.''
\end{itemize}
} 

%%%%%%%%%%%%%%%%%%%%%%%%%%%%%%%%%%%%%%%%%%%%%%%%%%%%%%%%%%%%%%%%%%%%%%%%%%%%%%%%
\subsubsection{Optimal Policy}

Let $\tau$ be an optimal stopping rule defined over $\{ a_{j1}, a_{j2}, \dots \}$. 
%%%%%%%%%%%%%%%
% Skill j index
Define the skill $j$ index as the expected payoff if you committed to studying $j$:
%\gen{ 
\begin{equation*}
\mathcal{I}_j (h_j, P_j) = \sup_{\tau \geq 0} \mathbb{E}^\tau
\ce{
   \sum_{t=0}^\infty \delta^t U_j (w_j, h_{jt}) \ell_{jt}}
   {(h_{j0}, P_{j0}) = (h_j, P_j))
}
\end{equation*}
%%%%%%%%%%%%%%%%%%%
% Graduation region
Define the graduation region of skill $j$ as: 
\begin{equation*}
\mathcal{G}_j = \left\{ (h_j, P_j) \left\vert
   \arg \max_{\tau \geq 0} 
   \mathbb{E}^\tau \ce{\sum_{t=0}^\infty \delta^t U_j (w_j, h_j) \ell_{jt}}
   {(h_j, P_j)} = 0
   \right. \right\}
\end{equation*}
%%%%%%%%%%%%%%%%%%%
% Parameterizations
Given the parameterization of the problem, and letting $m_{jt} = \alpha_{jt} + \beta_{jt}$, these objects can be written as: 
\begin{align*}
\mathcal{I}_{jt} (h_{jt}, \alpha_{jt}, \beta_{jt}) = &
\begin{cases}
\frac{h_{jt}}{1 - \delta} & \text{if } \{\alpha_{jt}, \beta_{jt}\} \in \mathcal{G}_{j}, \\
\frac{h_{jt}}{1 - \delta} \sbr{
   \frac{
      \left\lceil \frac{\delta}{1 - \delta} \right\rceil
      \delta^{\left\lceil \frac{\delta}{1 - \delta} \right\rceil - m_{jt}}}
   {m_{jt}}
   } & \text{if } \{\alpha_{jt}, \beta_{jt}\} \notin \mathcal{G}_{j} \\
\end{cases} \\
   \mathcal{G}_j = & \left\{ \alpha_{jt}, \beta_{jt} \left\vert m_{jt} \geq \left\lceil \frac{\delta}{1 - \delta} \right\rceil \right. \right\}
\end{align*}
%%%%%%%%%%%%%%%%%%%%%%%%
% Optimal policy theorem
The following policy $\pi: (h_t, P_t) \to (s_t, \ell_t)$ is optimal: 
\begin{enumerate}
	\item At each $t \geq 0$, choose skill $j^* = \arg \max_{i \in N} \mathcal{I}_i$, breaking ties according to any rule
	\item If $(h_{j^*}, P_{j^*}) \in \mathcal{G}_{j}$, then enter the labor market as a $j^*$ specialist. Otherwise, study $j^*$ for an additional period.  
\end{enumerate}

% Notes/questions
\nts{
\begin{itemize}
   \item Shouldn't it be $\mathcal{G}_{j^*}$?
   \item Shouldn't $\tau$ be indexed by $j$?
   \item Is graduation region based on OSLA or initial conditions? I think that when I have some of my ECON200B notes around, I should think about how I could more cleanly write the optimal stopping rule. May it would be clearer to write $\tau_j$?
\end{itemize}

}

%%%%%%%%%%%%%%%%%%%%%%%%%%%%%%%%%%%%%%%%%%%%%%%%%%%%%%%%%%%%%%%%%%%%%%%%%%%%%%%%
\subsubsection*{Math notes}

Math theorem: If stopping time problem monotonic, then OSLA (one-step look-ahead) stopping rule optimal. 
\begin{blist}

\item Monotonic stopping time: If the optimal stopping rule for skill $j$ says stop today, then the optimal stopping rule would say stop tomorrow, regardless of the stochastic outcome. This would be built into the human capital problem. 

\item Monotonicity of the stopping problem means that stopping at $t$ implies stopping at $t+1$. Stopping at time $t$ implies: \nts{(I think)}
\begin{equation*}
 U_j(w_j, h_{jt}) > \delta \mathbb{E} U_j(w_j, h_{jt+1})
\end{equation*}
If the stoping problem is monotonic, then this means: 
\begin{equation*}
 U_j(w_j, h_{jt+1}) > \delta \mathbb{E} U_j(w_j, h_{jt+2})
\end{equation*}
This holds for all realizations of $a$. 
This condition then implies the optimality of OSLA.

Given the utility function in the parametric problem, if you are studying skill $j$, you would stop at time $t$ if:
\begin{equation}
h_{jt} > \delta \mathbb{E} \sbr{h_{jt+1}}
\end{equation}
The optimal stopping time is monotonic if the following holds for all $j$, for all realizations of $a_{jt}$:
\begin{equation}
h_{jt+1} > \delta \mathbb{E} \sbr{h_{jt+2}}
\end{equation}
Recall:
\begin{align*}
\mathbb{E} \sbr{h_{jt+1}} = \delta (h_{jt} + \mathbb{E} \sbr{a_{it}} s_{it})
\end{align*}
You stop studying skill $j$ at time $t$ ($s_{jt}=0$) if the utility associated with stopping is greater than the expected utility associated with studying ($s_{jt}=1$):
\begin{align*}
h_{jt} >& \delta \mathbb{E} \sbr{h_{jt+1}} \\
=& \delta (h_{jt} + \mathbb{E} \sbr{a_{jt}} s_{jt}) \\
=& \delta h_{jt} + \delta \pr{\nu_j \mathbb{E} \sbr{\theta_j} + 0 (1 - \mathbb{E} \sbr{\theta_j})} \\
=& \delta h_{jt} + \delta \pr{\nu_j \frac{\alpha_{jt}}{\alpha_{jt} + \beta_{jt}}} \\
\implies h_{jt} >& \frac{\delta}{1 - \delta} \pr{\frac{\nu_j \alpha_{jt}}{\alpha_{jt} + \beta_{jt}}}
\end{align*}
Using the fact that $h_{jt} = h_{j0} + \nu (\alpha_{jt} - \alpha_0)$ and the assumption that $h_{j0} = \alpha_{j0} h_{j0}$, the above implies:
\begin{equation*}
\nu \alpha_{jt} > \frac{\delta}{1 - \delta} \pr{\frac{\nu_j \alpha_{jt}}{\alpha_{jt} + \beta_{jt}}} \iff \frac{1-\delta}{\delta} > \alpha_{jt} + \beta_{jt}
\end{equation*} 
\nts{I'm not finishing this right now. But it's worth noting that, in the notes that Titan gave me, a sufficient condition for monotonicity in the beta-Bernoulli problem is that $h_{j0} = \alpha_{j0} h_{j0}$. So I know that you can prove that, under this assumption, the problem is monotonic. I might need to relax that assumption for my model, but it doesn't need to be a problem.}
\end{blist}



%%%%%%%%%%%%%%%%%%%%%%%%%%%%%%%%%%%%%%%%%%%%%%%%%%%%%%%%%%%%%%%%%%%%%%%%%%%%%%%%
\subsection{Augmented model}
%%%%%%%%%%%%%%%%%%%%%%%%%%%%%%%%%%%%%%%%%%%%%%%%%%%%%%%%%%%%%%%%%%%%%%%%%%%%%%%%

%%%%%%%%%%%%%%%%%%%%%%%%%%%%%%%%%%%%%%%%%%%%%%%%%%%%%%%%%%%%%%%%%%%%%%%%%%%%%%%%
%\subsubsection{Preliminaries}

I now consider how at student's membership in a particular group influences specialization decisions. 
For example, let $g \in \{ m, f \}$ denote gender,\nts{\footnote{\nts{I think a cool thing to do would be to prove this for countably infinite types, and see what that means. That could provide ways of valuing diversity? Or something?}}}  
and assume that the distributions of underlying abilities, $\theta_j$, are the same for men and women.
However, initial beliefs about underlying abilities, $P_{j0}^g \equiv \mathcal{B} \pr{\alpha_{j0}^g, \beta_{j0}^g}$ are different for men and women. 
\nts{Be specific about what this will do.}

\begin{blist}

\item The parameter $\theta_j$ is the probability that a student successfully increases their human capital if they take a course. 
This can be thought of the student's probability of success or failure in a particular field $j$.\footnote{
   It is worth highlighting that the following assumes that probability of success or failure is independent and stationary throughout time.}
Thus, the beta distribution $\mathcal{B} \pr{\alpha_{j0}^g, \beta_{j0}^g}$ represents a distribution of beliefs about probabilities.\footnote{For a nice, intuitive understanding of this, see \url{https://stats.stackexchange.com/q/47782}} 
\item 
Assume observed success rates in skill $j$ are equal for men and women:

Suppose $\pr{\alpha_{j0}^g, \beta_{j0}^g}$ are formed based on previously observed data (i.e. existing group outcomes). 

% https://www.users.csbsju.edu/~mgass/robert.pdf 
% pg. 82 (70)

\item A reasonable way to form a prior is to think of $\alpha$ and $\beta$ as previously observed data.

\end{blist}

% bayes rule requires us to specify liklihood function. This is something i think we can connct to Loury


% Rose talked about Texas A&M data on grades 

\printbibliography

\end{document}
