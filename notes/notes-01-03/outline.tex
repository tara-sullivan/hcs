\documentclass[10 pt]{article}

% Format
\usepackage[T1]{fontenc}
\usepackage[utf8]{inputenc}
\usepackage[margin=1in]{geometry} % 1-inch margins
\usepackage[english]{babel} % English hyphenation, etc.	
\usepackage{setspace} % Set spacing 
%\usepackage{enumerate} % Use different types of enumerate options
\usepackage{enumitem}
\usepackage{csquotes} % Block quotes
\usepackage[dvipsnames]{xcolor} % colors: https://en.wikibooks.org/wiki/LaTeX/Colors#The_68_standard_colors_known_to_dvips
% Math
\usepackage{amsmath, mathrsfs, amsfonts, amssymb, amsthm}
% Figures
\usepackage{graphicx} % Include figures
\usepackage{float} % Improved control over floats
\usepackage{tikz} % Draw figures with tikz

% Colors
\definecolor{Indigo}{HTML}{3C6478}
\definecolor{DarkBrown}{HTML}{45281B}
\definecolor{Brown}{HTML}{161402}
\definecolor{DarkGreen}{HTML}{325101}
\definecolor{LeafGreen}{HTML}{4A6F01}
\definecolor{DarkAlice}{HTML}{107896}
\definecolor{Alice}{HTML}{1496BB}
\definecolor{DarkGray}{RGB}{116 118 120}
\definecolor{DarkBlue}{HTML}{022C36}
\definecolor{MainBlue}{HTML}{132881}
\definecolor{Maroon}{HTML}{6A123D}
\definecolor{DarkPurple}{HTML}{2C033A}
\definecolor{Orange}{HTML}{F18312}

% Hyperlinks
\usepackage{hyperref} % Include hyperlinks
\hypersetup{
  colorlinks   = true, %Colours links instead of ugly boxes
  urlcolor     = Maroon, %Colour for external hyperlinks
  linkcolor    = DarkGray, %Colour of internal links
  citecolor   = MainBlue %Colour of citations
}


% Macro Shortcuts
\newcommand{\R}{\mathbb{R}}
\newcommand{\Q}{\mathbb{Q}}
\newcommand{\Z}{\mathbb{Z}}
\newcommand{\N}{\mathbb{N}}
\newcommand{\EE}{\mathbb{E}}
\newcommand{\PP}{\mathbb{P}}
\newcommand{\BB}{\mathscr{B}}
\newcommand{\e}{\text{e}}
\newcommand{\dd}{\text{d}}


% Theorems
\newtheorem{prop}{Proposition}[section]
\newtheorem{thm}{Theorem}
\theoremstyle{remark}
\newtheorem{claim}{Claim}[section]
\newtheorem{remark}{Remark}
\theoremstyle{definition}
\newtheorem{defn}{Definition}[section]
\newtheorem{lemma}{Lemma}
\newtheorem{ass}{Assumption}


\newif\ifnts
\ntstrue % uncomment to show 
% Notes to self 
\ifnts
  \newcommand{\nts}[1]{{\color{gray}#1}}
\else
  \newcommand{\nts}[1]{}
\fi

%%%%%%%%%%
% Sections that have:
%   (A) Roman numerals 
%   (B) fixed width = fixw
%   (C) coloring

% (A) Roman numeral for section and subsection
% \renewcommand{\thesection}{\Roman{section}} 
% \renewcommand{\thesubsection}{\roman{subsection}}

% (B) Each section has fixed width = fixw 

% Define fixw
\newcommand{\fixw}{28pt}
\newcommand{\fixwh}{14pt}

% (C) Define colors
\newcommand{\secc}[1]{{\color{DarkGreen}#1}} % section color
\newcommand{\sectc}{DarkGreen} % section text color
\newcommand{\subsecc}[1]{{\color{LeafGreen}#1}} % subsection color
\newcommand{\subsectc}{LeafGreen} % subsection text color
\newcommand{\numc}{DarkAlice}

% Set each section width and color
\usepackage{titlesec}
\titleformat{\section}{\normalfont\Large\bfseries\color{\sectc}}
	{\makebox[\fixw][l]{\secc{\thesection.}}}{0pt}{} 
\titleformat{\subsection}{\normalfont\large\bfseries\color{\subsectc}}
	{\makebox[{\fixw}][l]{\subsecc{\thesubsection.}}}{0pt}{} 
\titleformat{\subsubsection}{\normalfont\bfseries}
	{}{0pt}{} %{\makebox[{\fixw}][l]{}}{0pt}{} 

% Highlight certain items
\newcommand{\hitem}[2][DarkAlice]{\color{#1} \item #2 \color{black}}

%%%%%%%%%%
% Lists that start at fixw (see section above)
\newlist{outline}{enumerate}{2}
\setlist[outline,1]{label=\arabic*.,left=0pt .. \fixw}
\setlist[outline,2]{label=\alph*.,left=0pt .. \fixw}

\newlist{blist}{itemize}{2}
\setlist[blist,1]{label=\textbullet,left=0pt .. \fixw}
\setlist[blist,2]{label=\textendash,left=0pt .. \fixw}

%%%%%%%%%%
% Enumerate in footnote
\newlist{footcount}{enumerate}{1}
\setlist[footcount]{label=(\alph*),left=0pt .. \fixw}

%%%%%%%%%%
% Foodnote Edits
% Bottom package ensures that footnote won't be above a figure
\usepackage[bottom]{footmisc}

% No indent in footnotes
% NOTHING SEEMS TO WORK
% \usepackage{scrextend}
% \deffootnote[\fixw]{\fixw}{.195in}{\makebox[\fixw][r]{\thefootnotemark.\hspace{.2in}}}
% \usepackage[flushmargin, hang]{footmisc} % flush footnote mark to left margin
% \renewcommand{\footnotelayout}{\doublespacing\raggedright}
% \usepackage[flushmargin,hang]{footmisc}
% \usepackage[hang, flushmargin]{footmisc}
% \setlength{\footnotemargin}{0.5in}


% % \usepackage[marginal]{footmisc}
% \setlength\footnotemargin{0pt}  % default value: 1.8em

% \usepackage[flushmargin,hang]{footmisc}
% % \setlength{\footnotemargin}{1em} % just to show clearly equal output

% % \usepackage[marginal]{footmisc}
% \setlength{\footnotemargin}{10em} % just to show clearly equal output

% \renewcommand{\footnotelayout}{\raggedright}


%%%%%%%%%%
% Skip line between paragraphs, set indent to \fixw
\usepackage[parfill, indent=\fixw]{parskip}

%%%%%%%%%%
% format caption
% get rid of 'Figure: ' in caption
\usepackage{caption}
% \captionsetup[table]{labelsep=space}
\captionsetup{%
    % labelformat=empty,
    % font=small,
    labelsep=quad,
    tableposition=top,
    labelsep=period,
    margin=\fixw,
}



\newcommand{\cc}{\mathbf{c}}
\newcommand{\xx}{\mathbf{x}}

\newcommand{\br}[1]{\left\{ #1 \right\}}
\newcommand{\sbr}[1]{\left[ #1 \right]}
\newcommand{\pr}[1]{\left( #1 \right)}
\newcommand{\ce}[2]{\left[\left. #1 \right\vert #2 \right]}

\addbibresource{../bibliography.bib}

\begin{document}
\title{Notes on Human Capital Specialization}
\author{Tara Sullivan}

\maketitle
\onehalfspacing

\noindent\nts{Please note that gray text are notes/comments}

%%%%%%%%%%%%%%%%%%%%%%%%%%%%%%%%%%%%%%%%%%%%%%%%%%%%%%%%%%%%%%%%%%%%%%%%%%%%%%%%
\section{Introduction}
%%%%%%%%%%%%%%%%%%%%%%%%%%%%%%%%%%%%%%%%%%%%%%%%%%%%%%%%%%%%%%%%%%%%%%%%%%%%%%%%

\begin{outline}

\item \nts{Motivating question} Debate surrounding affirmative action. Does not focus on endogenous group outcomes, cannot get data on that. 

\item This paper presents a model of gradual human capital specialization, whereby existing group outcomes influence \textbf{something}.

\item To accomplish this, I augment the model of human capital specialization developed in \textcite{AF20}. Their model of gradual specialization in human capital is used to study misallocation in U.S. education system. 

% Another application: how do peopl specialize after recessions

\end{outline}

%%%%%%%%%%%%%%%%%%%%%%%%%%%%%%%%%%%%%%%%%%%%%%%%%%%%%%%%%%%%%%%%%%%%%%%%%%%%%%%%
\section{Model}
%%%%%%%%%%%%%%%%%%%%%%%%%%%%%%%%%%%%%%%%%%%%%%%%%%%%%%%%%%%%%%%%%%%%%%%%%%%%%%%%

Below is the general model from \textcite{AF20}

%%%%%%%%%%%%%%%%%%%%%%%%%%%%%%%%%%%%%%%%%%%%%%%%%%%%%%%%%%%%%%%%%%%%%%%%%%%%%%%%
\subsection{Alon and Fershtman (2020)}

%%%%%%%%%%%%%%%%%%%%%%%%%%%%%%%%%%%%%%%%%%%%%%%%%%%%%%%%%%%%%%%%%%%%%%%%%%%%%%%%
\subsubsection{Preliminaries}

\begin{tabular}{@{}lll}
\textbf{General}  & \textbf{Description} \\
\multicolumn{2}{@{}l}{\emph{Endowments}} \\
$t$                                       & indivisible unit of time per period \\
$h_0    = \{ h_{10} \dots h_{N0} \}$      & initial human capital \\
$\theta = \{\theta_1, \dots, \theta_N \}$ & unkown abilities \\
$P_{j0}$                                  & independent initial beliefs on $\theta_j$ \\
\multicolumn{2}{@{}l}{\emph{Actions}} \\
$s_t    = \{ s_{1t}, \dots, s_{Nt} \}$    & study time \\
$\ell_t = \{ l_{1t}, \dots, \ell_{Nt} \}$ & work time \\
\multicolumn{2}{@{}l}{\emph{Technology}} \\
$a_{it}   \sim F_{\theta_i}$      & effective study time \\
$h_{it+1} = H_i (h_{it}, a_{it})$ & accumulation by $t+1$ \\
\multicolumn{2}{@{}l}{\emph{Beliefs}} \\
$P_{it+1} = \Pi (P_{it}, a_{it})$ & evolution of beliefs over $\theta_i$
\end{tabular}

%%%%%%%%%%%%%%%%%%%%%%%%%%%%%%%%%%%%%%%%%%%%%%%%%%%%%%%%%%%%%%%%%%%%%%%%%%%%%%%%
\subsubsection{Problem}

A policy $\pi: (h_t, P_t) \to (s_t, \ell_t)$ is optimal if it maximizes: \nts{Shouldn't I either index $\pi$ by $t$, or say that a policy at time $t$ is optimal? }
\begin{align*}
& \mathbb{E}^\pi \sbr{
   \sum_{t=0}^\infty \delta^t 
   \left. \pr{\sum_{i=1}^N \ell_{it} U_i(w_i, h_{it})} \right\vert
   \pr{(h_{10}, P_{10}), \dots, (h_{10}, P_{N0})}
} \\
\text{subject to} \quad& h_{it+1} = H_i (h_{it}, a_{it}), \quad \quad
P_{it+1} = \Pi (P_{it}, a_{it}), 
\quad \quad \text{if $i$ selected,} \\
\quad& \sum_{i=1}^N (s_{it} + \ell_{it}) = 1, \quad \quad s_{it}, \ell_{it} \in \{0,1\}
\end{align*}
\nts{Shouldn't this be if $s_{it} = 1$, meaning that you study $i$ during period $t$, rather than ``if $i$ is selected?''. I'm pretty sure that, because you could choose to work in skill $i$, you need to specify if you're studying. Also, shouldn't the transition laws be $h_{it+1} = H_i (h_{it}, a_{it}, s_{it})$, since it depends on whether you study? This could also be fixed by chaning the ``if $i$ selected line.'' I would also call this transition law ``course outcome'' or ``studying outcome.''}

%%%%%%%%%%%%%%%%%%%%%%%%%%%%%%%%%%%%%%%%%%%%%%%%%%%%%%%%%%%%%%%%%%%%%%%%%%%%%%%%
\subsubsection{Optimal Policy}

Let $\tau$ be an \textbf{optimal stopping rule} defined over $\{ a_{j1}, a_{j2}, \dots \}$. Skill $j$ index: \nts{expected payoff if you commit to $j$}
\begin{equation*}
\mathcal{I}_j (h_j, P_j) = \sup_{\tau \geq 0} \mathbb{E}^\tau
\ce{
   \sum_{t=0}^\infty \delta^t U_j (w_j, h_{jt}) \ell_{jt}}
   {(h_{j0}, P_{j0}) = (h_j, P_j))
}
\end{equation*}
Graduation region of skill $j$: \nts{Is graduation region based on OSLA or initial conditions? I think that when I have some of my ECON200B notes around, I should think about how I could more cleanly write the optimal stopical rule. May it would be clearer to write $\tau_j$?}
\begin{equation*}
\mathcal{G}_j = \left\{\left. (h_j, P_j) \right\vert 
   \arg \max_{\tau \geq 0} 
   \mathbb{E}^\tau \ce{\sum_{t=0}^\infty \delta^t U_j (w_j, h_j) \ell_{jt}}
   {(h_j, P_j)} = 0
   \right\}
\end{equation*}
% Theory is that existing distribution for a group impacts the underlying priors, probably just alpha_0 and beta_0
Optimal Policy Theorem: The following policy $\pi: (h_t, P_t) \to (s_t, \ell_t)$ is optimal:
\begin{enumerate}
	\item At each $t \geq 0$, choose skill $j^* = \arg \max_{i \in N} \mathcal{I}_i$, breaking ties according to any rule
	\item If $(h_{j^*}, P_{j^*}) \in \mathcal{G}_{j}$\nts{ (shouldn't it be $\mathcal{G}_{j^*}$?)}, then enter the labor market as a $j^*$ specialist. Otherwise, study $j^*$ for an additional period.  
\end{enumerate}

\subsubsection*{Math notes}
Math theorem: If stopping time problem monotonic, then OSLA (one-step look-ahead) stopping rule optimal. 
\begin{blist}

\item Monotonic stopping time: If the optimal stopping rule for skill $j$ says stop today, then the optimal stopping rule would say stop tomorrow, regardless of the stochastic outcome. This would be built into the human capital problem. 

\item Monoticity of the stopping problem means that stopping at $t$ implies stopping at $t+1$. Stopping at time $t$ implies: \nts{(I think)}
\begin{equation*}
 U_j(w_j, h_{jt}) > \delta \mathbb{E} U_j(w_j, h_{jt+1})
\end{equation*}
If the stoping problem is monotonic, then this means: 
\begin{equation*}
 U_j(w_j, h_{jt+1}) > \delta \mathbb{E} U_j(w_j, h_{jt+2})
\end{equation*}
This holds for all realizations of $a$. 
This condition then implies the optimality of OSLA.
\end{blist}


%%%%%%%%%%%%%%%%%%%%%%%%%%%%%%%%%%%%%%%%%%%%%%%%%%%%%%%%%%%%%%%%%%%%%%%%%%%%%%%%
\subsection{Beta-Bernoulli Model}

\begin{outline}

\hitem{Problem}

A policy $\pi: (h_t, P_t) \to (s_t, \ell_t)$ is optimal if it maximizes:
\begin{align*}
& \mathbb{E}^\pi \sbr{
   \sum_{t=0}^\infty \delta^t 
   \left. \pr{\sum_{i=1}^N \ell_{it} w_i h_{it}} \right\vert
   \pr{(h_{10}, P_{10}), \dots, (h_{10}, P_{N0})}
} \\
\text{subject to} \quad& h_{it+1} = h_{it}+ a_{it} s_{it}, \quad \quad a_{it} = 
   \begin{cases} 
      \nu_i, & \text{with prob. } \theta_i,  \\ 
      0, & \text{with prob. } 1 - \theta_i,
   \end{cases} 
   \quad \quad h_{i0} = \alpha_{i0} \nu_i, \\
\quad& P_{it+1} = \mathcal{B} (\alpha_{i,t+1}, \beta_{i,t+1}), 
   \quad \quad \theta_i \sim P_{i,0} \equiv \mathcal{B} (\alpha_{i0}, \beta_{i0})
   \quad \quad \text{if $i$ selected,} \\
\quad& \sum_{i=1}^N (s_{it} + \ell_{it}) = 1, \quad \quad s_{it}, \ell_{it} \in \{0,1\}
\end{align*}

\hitem{Monotonicity of stopping problem}

Given the utility function, if you are studying skill $j$, you would stop at time $t$ if:
\begin{equation}
h_{jt} > \delta \mathbb{E} \sbr{h_{jt+1}}
\end{equation}
The optimal stopping time is monotonic if the following holds for all $j$, for all realizations of $a_{jt}$:
\begin{equation}
h_{jt+1} > \delta \mathbb{E} \sbr{h_{jt+2}}
\end{equation}
Recall:
\begin{align*}
\mathbb{E} \sbr{h_{jt+1}} = \delta (h_{jt} + \mathbb{E} \sbr{a_{it}} s_{it})
\end{align*}
You stop studying skill $j$ at time $t$ ($s_{jt}=0$) if the utility associated with stopping is greater than the expected utility associated with studying ($s_{jt}=1$):
\begin{align*}
h_{jt} >& \delta \mathbb{E} \sbr{h_{jt+1}} \\
=& \delta (h_{jt} + \mathbb{E} \sbr{a_{jt}} s_{jt}) \\
=& \delta h_{jt} + \delta \pr{\nu_j \mathbb{E} \sbr{\theta_j} + 0 (1 - \mathbb{E} \sbr{\theta_j})} \\
=& \delta h_{jt} + \delta \pr{\nu_j \frac{\alpha_{jt}}{\alpha_{jt} + \beta_{jt}}} \\
\implies h_{jt} >& \frac{\delta}{1 - \delta} \pr{\frac{\nu_j \alpha_{jt}}{\alpha_{jt} + \beta_{jt}}}
\end{align*}
Using the fact that $h_{jt} = h_{j0} + \nu (\alpha_{jt} - \alpha_0)$ and the assumption that $h_{j0} = \alpha_{j0} h_{j0}$, the above implies:
\begin{equation*}
\nu \alpha_{jt} > \frac{\delta}{1 - \delta} \pr{\frac{\nu_j \alpha_{jt}}{\alpha_{jt} + \beta_{jt}}} \iff \frac{1-\delta}{\delta} > \alpha_{jt} + \beta_{jt}
\end{equation*} 
\nts{I'm not finishing this right now. But it's worth noting that, in the notes that Titan gave me, a sufficient condition for monotoncity in the beta-bernoulli problem is that $h_{j0} = \alpha_{j0} h_{j0}$. So I know that you can prove that, under this assumption, the problem is monotonic. I might need to relax that assumption for my model, but it doesn't need to be a problem.}

Skill $j$ index given by: 
\begin{equation*}
\mathcal{I}_{it} (h_{it}, \alpha_{it}, \beta_{it})
\end{equation*}

\end{outline}

%%%%%%%%%%%%%%%%%%%%%%%%%%%%%%%%%%%%%%%%%%%%%%%%%%%%%%%%%%%%%%%%%%%%%%%%%%%%%%%%
\subsection{Augmented model}
%%%%%%%%%%%%%%%%%%%%%%%%%%%%%%%%%%%%%%%%%%%%%%%%%%%%%%%%%%%%%%%%%%%%%%%%%%%%%%%%



\printbibliography

\end{document}
